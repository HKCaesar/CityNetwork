%%%%%%%%%%%%%%%%%%%%%%%%%%%%%
% Standard header for working papers
%
% WPHeader.tex
%
%%%%%%%%%%%%%%%%%%%%%%%%%%%%%

\documentclass[11pt]{article}

%%%%%%%%%%%%%%%%%%%%
%% Include general header where common packages are defined
%%%%%%%%%%%%%%%%%%%%

% general packages without options
\usepackage{amsmath,amssymb,bbm}




%%%%%%%%%%%%%%%%%%%%
%% Idem general commands
%%%%%%%%%%%%%%%%%%%%
%% Commands

\newcommand{\noun}[1]{\textsc{#1}}


%% Math

% Operators
\DeclareMathOperator{\Cov}{Cov}
\DeclareMathOperator{\Var}{Var}
\DeclareMathOperator{\E}{\mathbb{E}}
\DeclareMathOperator{\Proba}{\mathbb{P}}

\newcommand{\Covb}[2]{\ensuremath{\Cov\!\left[#1,#2\right]}}
\newcommand{\Eb}[1]{\ensuremath{\E\!\left[#1\right]}}
\newcommand{\Pb}[1]{\ensuremath{\Proba\!\left[#1\right]}}
\newcommand{\Varb}[1]{\ensuremath{\Var\!\left[#1\right]}}

% norm
\newcommand{\norm}[1]{\| #1 \|}


% amsthm environments
\newtheorem{definition}{Definition}



%% graphics

% renew graphics command for relative path providment only ?
%\renewcommand{\includegraphics[]{}}








% geometry
\usepackage[margin=2cm]{geometry}

% layout : use fancyhdr package
\usepackage{fancyhdr}
\pagestyle{fancy}

\makeatletter

\renewcommand{\headrulewidth}{0.4pt}
\renewcommand{\footrulewidth}{0.4pt}
%\fancyhead[RO,RE]{\textit{Working Paper}}
\fancyhead[RO,RE]{\textit{ECTQG 2015}}
%\fancyhead[LO,LE]{G{\'e}ographie-Cit{\'e}s/LVMT}
\fancyhead[LO,LE]{An Algorithmic Systematic Review}
\fancyfoot[RO,RE] {\thepage}
\fancyfoot[LO,LE] {\noun{J. Raimbault}}
\fancyfoot[CO,CE] {}

\makeatother


%%%%%%%%%%%%%%%%%%%%%
%% Begin doc
%%%%%%%%%%%%%%%%%%%%%

\begin{document}







\title{A Theory of co-evolutive networked territorial systems : Exemplification of Network Necessity\bigskip\\
\textit{Working Paper}
}
\author{\noun{Juste Raimbault}}
\date{Date}


\maketitle

\justify


\begin{abstract}
Second part of theoretical paper developing a theory of co-evolutive networked territorial systems : application to simple models of urban growth for systems of cities.
\end{abstract}



\section{Context and Objective}

\subsection{Literature review}

\cite{bretagnolle2000long} already propose a spatial extension of the Gibrat model (\textit{detail})

\cite{favaro2011gibrat} is a more refined extension with economic cycles

% marius as an extension of Gibrat ? in the deterministic case.


\subsection{Exemplifying Network Necessity}

% what plan to do
% how does goes in the direction of theory confirmation ?


\section{Model Description}

\subsection{From Gibrat to Marius : the dilemma of formulation}

% stochastic vs deterministic formulation ? precise possible formulations and what choice means.

%  - Gabaix \cite{gabaix1999zipf} details stationary distribution of Gibrat model - link with Zipf law. why we do not accept this "explanation" : NOT stationary. depends on time scales to reach stationarity ?
%  - bayesian iterative formulation ? (mcmc)
%  - how does formulation influence ? equivalence in certain cases between stoch-cov and interdependent expectancies ?




\subsection{Model description}






\section{Results}

\subsection{Implementation}


\subsection{Model Exploration}

% qualitative behavior : hierarchy inversions/ trajectories of normalized populations (cf papier Denise Anne etc)
% need of synthetic data ?


\subsection{Model Calibration}

% here put empirical AIC ?



\section{Discussion}

% how this example illustrates well the theory






\section{Supplementary Materials}


\subsection{Integrating Gibrat}

Analytical resolution is possible for some aspects of the Gibrat model. We detail here the computation for some.

\paragraph{Expectancies} If working with expectancies, it makes no sense to proceed to Monte Carlo simulation as a direct resolution gives a deterministic recurrence relation on expectancies. Let $\mu_t = \Eb{P(t)}$

\paragraph{Covariance}


\paragraph{Distribution}



\subsection{A Bayesian iterative approach}





%%%%%%%%%%%%%%%%%%%%
%% Biblio
%%%%%%%%%%%%%%%%%%%%

\bibliographystyle{apalike}
\bibliography{/Users/Juste/Documents/ComplexSystems/CityNetwork/Biblio/Bibtex/CityNetwork}


\end{document}
