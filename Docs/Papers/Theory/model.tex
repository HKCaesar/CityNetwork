%%%%%%%%%%%%%%%%%%%%%%%%%%%%%
% Standard header for working papers
%
% WPHeader.tex
%
%%%%%%%%%%%%%%%%%%%%%%%%%%%%%

\documentclass[11pt]{article}

%%%%%%%%%%%%%%%%%%%%
%% Include general header where common packages are defined
%%%%%%%%%%%%%%%%%%%%

% general packages without options
\usepackage{amsmath,amssymb,bbm}




%%%%%%%%%%%%%%%%%%%%
%% Idem general commands
%%%%%%%%%%%%%%%%%%%%
%% Commands

\newcommand{\noun}[1]{\textsc{#1}}


%% Math

% Operators
\DeclareMathOperator{\Cov}{Cov}
\DeclareMathOperator{\Var}{Var}
\DeclareMathOperator{\E}{\mathbb{E}}
\DeclareMathOperator{\Proba}{\mathbb{P}}

\newcommand{\Covb}[2]{\ensuremath{\Cov\!\left[#1,#2\right]}}
\newcommand{\Eb}[1]{\ensuremath{\E\!\left[#1\right]}}
\newcommand{\Pb}[1]{\ensuremath{\Proba\!\left[#1\right]}}
\newcommand{\Varb}[1]{\ensuremath{\Var\!\left[#1\right]}}

% norm
\newcommand{\norm}[1]{\| #1 \|}


% amsthm environments
\newtheorem{definition}{Definition}



%% graphics

% renew graphics command for relative path providment only ?
%\renewcommand{\includegraphics[]{}}








% geometry
\usepackage[margin=2cm]{geometry}

% layout : use fancyhdr package
\usepackage{fancyhdr}
\pagestyle{fancy}

\makeatletter

\renewcommand{\headrulewidth}{0.4pt}
\renewcommand{\footrulewidth}{0.4pt}
%\fancyhead[RO,RE]{\textit{Working Paper}}
\fancyhead[RO,RE]{\textit{ECTQG 2015}}
%\fancyhead[LO,LE]{G{\'e}ographie-Cit{\'e}s/LVMT}
\fancyhead[LO,LE]{An Algorithmic Systematic Review}
\fancyfoot[RO,RE] {\thepage}
\fancyfoot[LO,LE] {\noun{J. Raimbault}}
\fancyfoot[CO,CE] {}

\makeatother


%%%%%%%%%%%%%%%%%%%%%
%% Begin doc
%%%%%%%%%%%%%%%%%%%%%

\begin{document}







\title{A Theory of co-evolutive networked territorial systems : Exemplification of Network Necessity\bigskip\\
\textit{Working Paper}
}
\author{\noun{Juste Raimbault}}
\date{Date}


\maketitle

\justify


\begin{abstract}
This paper is the application of a theoretical paper developing a theory of co-evolutive networked territorial systems. We apply simple models of urban growth for systems of cities, which include in particular the role of physical networks.
\end{abstract}



\section{Context and Objective}



% Do not mention that in the model paper - that will be cited by the theoretical paper.

%\subsection{Exemplifying Network Necessity}

% what plan to do
% how does goes in the direction of theory confirmation ?



%We propose to support our hypothesis that \textit{physical transportation networks are necessary to explain the morphogenesis of territorial systems} (aka \textit{Network Necessity}) by showing on a relatively simple case that the integration of physical networks into some models effectively increase their explanative power (being careful on the precise definition of model improvement to avoid overfitting). We work on simple territorial systems that are country-wide city systems, and more particularly French cities, on a time scale corresponding to that spatial scale, i.e. two last centuries. Taking into account physical networks can improve the understanding of city growth within that system in two ways : a qualitative one, for which the extended model would exhibit qualitative features corresponding to stylized facts empirically observed but that more basic models do not manage to reproduce, and a quantitative way, in the sense that model extension improves explained variance further than the mechanic improvement due to the introduction of supplementary degrees of freedom. If at least one of these is unveiled in our particular case, the evidence will support the theory at these scale and in this context.


\subsection{Model context}

\cite{bretagnolle2000long} already proposed a spatial extension of the Gibrat model. Later, \cite{favaro2011gibrat} developed a more refined extension with economic cycles and innovation waves, yielding a system dynamics version of the core of Simpop models.

% marius as an extension of Gibrat ? in the deterministic case.


\section{Model Description}

\subsection{From Gibrat to Marius : the dilemma of formulation}

% stochastic vs deterministic formulation ? precise possible formulations and what choice means.

%  - Gabaix \cite{gabaix1999zipf} details stationary distribution of Gibrat model - link with Zipf law. why we do not accept this "explanation" : NOT stationary. depends on time scales to reach stationarity ?
%  - bayesian iterative formulation ? (mcmc)
%  - how does formulation influence ? equivalence in certain cases between stoch-cov and interdependent expectancies ?



%%%%%%%%%%%%%%%%%%%%%%
\subsection{Model description}

We choose to work on a deterministic extension of the Gibrat model, what is equivalent to consider only expectancies in time as detailed before. Let $\vec{P}(t)=(P_i(t))_i$ be the population of cities in time. Under Gibrat independence assumptions, we have $\Covb{P_i(t)}{P_j(t)}=0$. If $\vec{P}(t+1)=\mathbf{R}\cdot \vec{P}(t)$ where $\mathbf{R}$ is also independent, then $\Eb{\vec{P}(t+1)}=\mathbf{R}\cdot\Eb{\vec{P}}(t)$. With $\vec{\mu}(t)=\Eb{\vec{P}(t)}$, we generalize this approach by taking $\vec{\mu}(t+1)=f(\vec{\mu}(t))$. In our case, we take

\[f(\vec{\mu}) = r_0\cdot \mathbf{Id}\cdot \vec{\mu} + \mathbf{G}\cdot \mathbf{1} + \mathbf{N}\cdot \]

with 
\begin{itemize}
\item $G_{ij} = w_G\cdot \frac{V_{ij}}{<V_{ij}>}$ and $V_{ij} = \left(\frac{\mu_i\mu_j}{\sum{\mu_k}^2}\right)^{\gamma_G} \exp{(-d_{ij}/d_G)}$
\item $N_{i} = w_N \cdot \sum_{kl} \left(\frac{\mu_k\mu_l}{\sum\mu}\right)^{\gamma_N}\exp{(-d_{kl,i})/d_N}$ where $d_{kl,i}$ is distance to shortest path between $k,l$ computed with slope impedance ($Z=\left(1+\alpha/\alpha_0\right)^{n_0}$ with $\alpha_0\simeq 3$)
\end{itemize}

The first component is the pure Gibrat model, that we obtain by setting the weights $w_G = w_N = 0$. The second component captures interdependencies between 



\section{Results}

\subsection{Data}


\paragraph{Population data}


\paragraph{Physical flows}

As stated above, this modeling exercise focuses on exploring the role of physical flows, whatever the effective shape of the network. We do not need for this reason network data which is furthermore not easily available at different time periods, and physical flows are assumed to take the geographical shortest path that include terrain slope (to avoid geographical absurdities such as cities with a difficult access having an overestimated growth rate). Using the 1km resolution Digital Elevation Model openly available from IGN~\cite{}%cite bdalti
, we construct an impedance field of the form
\[
Z = \left(1 + \frac{\alpha}{\alpha_0}\right)^{n_0}
\]

We took fixed parameter values $\alpha_0 = 3$ (corresponding to approximatively a 5\% slope).


\subsection{Implementation}

Data preprocessing, result processing and models profiling are implemented in R. For performances reasons and an easier integration into the OpenMole software for model exploration~\cite{reuillon2013openmole}, a \texttt{scala} version was also developed. The typical question of trade-off between implementation performance and interoperability appeared quickly as an issue, as a blind exploration and calibration can difficultly provide useful thematic conclusions for that kind of model. Finding an improvement in model fit among one parameter dimension is significant if the geographical situation is visualized and the improvement is confirmed as reasonable and not an absurdity.



\subsection{Model Exploration}

% qualitative behavior : hierarchy inversions/ trajectories of normalized populations (cf papier Denise Anne etc)
% need of synthetic data ?


\subsection{Model Calibration}

% here put empirical AIC ?



\section{Discussion}

% how this example illustrates well the theory






\section{Supplementary Materials}


\subsection{Integrating Gibrat}

Analytical resolution is possible for some aspects of the Gibrat model. We detail here the computation for some.

\paragraph{Expectancies} If working with expectancies, it makes no sense to proceed to Monte Carlo simulation as a direct resolution gives a deterministic recurrence relation on expectancies. Let $\mu_t = \Eb{P(t)}$

\paragraph{Covariance}


\paragraph{Distribution}



\subsection{A Bayesian iterative approach}

Readers familiar with Bayesian Signal Processing techniques will have remarked that similarities exist between the described models and iterative filters such as the Kalman filter, more precisely particle filters.



\subsection{Model explanative power : propositions for an empirical AIC}




%%%%%%%%%%%%%%%%%%%%
%% Biblio
%%%%%%%%%%%%%%%%%%%%

\bibliographystyle{apalike}
\bibliography{/Users/Juste/Documents/ComplexSystems/CityNetwork/Biblio/Bibtex/CityNetwork}


\end{document}
