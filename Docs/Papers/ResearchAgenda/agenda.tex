%%%%%%%%%%%%%%%%%%%%%%%%%%%%%
% Standard header for working papers
%
% WPHeader.tex
%
%%%%%%%%%%%%%%%%%%%%%%%%%%%%%

\documentclass[11pt]{article}

%%%%%%%%%%%%%%%%%%%%
%% Include general header where common packages are defined
%%%%%%%%%%%%%%%%%%%%

% general packages without options
\usepackage{amsmath,amssymb,bbm}




%%%%%%%%%%%%%%%%%%%%
%% Idem general commands
%%%%%%%%%%%%%%%%%%%%
%% Commands

\newcommand{\noun}[1]{\textsc{#1}}


%% Math

% Operators
\DeclareMathOperator{\Cov}{Cov}
\DeclareMathOperator{\Var}{Var}
\DeclareMathOperator{\E}{\mathbb{E}}
\DeclareMathOperator{\Proba}{\mathbb{P}}

\newcommand{\Covb}[2]{\ensuremath{\Cov\!\left[#1,#2\right]}}
\newcommand{\Eb}[1]{\ensuremath{\E\!\left[#1\right]}}
\newcommand{\Pb}[1]{\ensuremath{\Proba\!\left[#1\right]}}
\newcommand{\Varb}[1]{\ensuremath{\Var\!\left[#1\right]}}

% norm
\newcommand{\norm}[1]{\| #1 \|}


% amsthm environments
\newtheorem{definition}{Definition}



%% graphics

% renew graphics command for relative path providment only ?
%\renewcommand{\includegraphics[]{}}








% geometry
\usepackage[margin=2cm]{geometry}

% layout : use fancyhdr package
\usepackage{fancyhdr}
\pagestyle{fancy}

\makeatletter

\renewcommand{\headrulewidth}{0.4pt}
\renewcommand{\footrulewidth}{0.4pt}
%\fancyhead[RO,RE]{\textit{Working Paper}}
\fancyhead[RO,RE]{\textit{ECTQG 2015}}
%\fancyhead[LO,LE]{G{\'e}ographie-Cit{\'e}s/LVMT}
\fancyhead[LO,LE]{An Algorithmic Systematic Review}
\fancyfoot[RO,RE] {\thepage}
\fancyfoot[LO,LE] {\noun{J. Raimbault}}
\fancyfoot[CO,CE] {}

\makeatother


%%%%%%%%%%%%%%%%%%%%%
%% Begin doc
%%%%%%%%%%%%%%%%%%%%%

\begin{document}







\title{Towards a dynamic modeling of coevolution processes between transportation and land-use : Construction of a research agenda\bigskip\\
\textit{Working Paper}
}
\author{\noun{Juste Raimbault}}
\date{Friday October 9th}


\maketitle

\justify


\begin{abstract}
\textit{TBW}
\end{abstract}


%%%%%%%%%%%%%%%%%%%%
\section{Introduction}
%%%%%%%%%%%%%%%%%%%%


\subsection{Context}

% small paragraph on issues related to interaction land-use/transportation.



\subsection{Scientific positioning}

% now take a detour within sci context

Within an evolutive urban theory~\cite{pumain2006evolutionary}, a considerable body of knowledge on urban systems self-organisation has recently been built through the construction, the exploration and the calibration of thematic-based models of simulation, of which the serie of Simpop models is emblematic~\cite{pumain2012multi}. The elaboration of an integrated platform for the construction and the evaluation of geographical models, including the development of the user-friendly, yet powerful by the transparent access to grid computation ressources, Model Experiment software OpenMole~\cite{reuillon2013openmole}, but also an epistemological framework and associated meta-heuristics for model validation~\cite{rey2015plateforme}, was central for the establishment of evidence-based thematic conclusions, which differentiation with the consequent previous amount of geographical research lead by similar methods of agent-based modeling and simulation was indeed the introduction of novel methods and tools going a step further for the validation stage. A illustrative example is the application of the Calibration Profile algorithm (which reveals a single parameter influence on model performance within the whole parameter space) to the sufficient and necessary parameters to reproduce existing urban systems patterns on a long time scale by the SimpopLocal model~\cite{schmitt2014half}, and other methods such as PSE algorithm aimed to detect rare outputs of a model, were successively applied, to the Marius model in that case~\cite{10.1371/journal.pone.0138212}.

At first sight this methodological and scientific context seems rather disconnected from our geographical objects of study which are the processes of coevolution between transportation network and urban growth, in a generic form (i.e. at any scale temporal and spatial scales, and in any geographical context) in a first approach and of course geographically contextualized once the entreprise of this paper will have been completed. These works are indeed the giants on which shoulder we intend to stand on. We rely on Bretagnolle concluding considerations in~\cite{bretagnolle:tel-00459720}, insisting on the need to pursue the various empirical findings on long-time network and cities interactions, by modeling approaches which should shed light on underlying coevolution processes. We propose to explore that paradigm which has been poorly tackled and has many obstacles associated with. Theoretically, Bretagnolle's work is positioning precisely within the evolutionnary urban framework, which assets include the compliance with complexity approaches which allow to take into account the particularities of urban systems such as their non-ergodicity~\cite{pumain2012urban}. Methodologically, it seems intuitively suitable to our purpose, what will be confirmed further. Indeed, many of the issues we will identify, espacially related to modeling, should be efficiently tackled building on it.


\subsection{Modeling the coevolution : overview of the scientific landscape}

% brief overview, as in ecqtg paper.

Many disciplines have found interest in studying city growth, transportation network growth, or both including some of their interactions. Each has its own problematics, corresponding models and time scales. We propose an overview of the scientific landscape on the subject, to better understand typical issues that can arise.

\paragraph{Land-Use Transportation Interaction Models}

A wide range of models that have been developed mainly in an aim of transportation planning are the so-called Land-Use Transportation Interaction Models (LUTI). Transportation planners (historically beginning in the US for road infrastructure planning) mainly propose that kind of model to evaluate the impact of a new infrastructure on the evolution of the urban system through the impact on land-use.
% Building on Alonso-Mills urban economy model % -> check first luti ; CA ?
Recent reviews give an broad idea of existing approaches and capabilities of such models~\cite{chang2006models,iacono2008models,wegener2004land}. Different scales can be considered (e.g. from the scale of the metropolitan area
% TODO find cit. luti meso scale
to the regional scale in the frame of regional planning).
% idem reg scale ? !! Sleuth model is not a luti.
Contrary to most common ideas, these models do not necessarily rely on equilibrium assumptions~\cite{kryvobokov2013comparison}. Furthermore, operationnal models are still developped today and state-of-the-art models do provide accurate forecasts on an intermediate time-scale. Various features can be included, such as in~\cite{delons:hal-00319087} where a detailed structure of the metropolitan housing market for Paris region gives targeted high-resolution forecasts. The main feature of interest for our question is the fact that these models consider transportation network as static, and simulates the evolution of a dynamic land-use. It means that the characteristic time-scale is smaller than the time scale of infrastructure evolution.
% TODO : diagram to show space and time scales - from long-time continetal scale to meso-time metropolitan scale.


\paragraph{Network Growth Modeling}


\paragraph{Coevolution Approaches}



% -> conclude : few models, different scales, objects, purposes.
% : refine the question at this point .


\subsection{Proposing a research agenda}

% justification of vrious axes ? ! not exhaustive at all.

% rely on approaches having tried ? -> what was considered, what lacked ?

The purpose of this paper is therefore to identify crucial obstacles to a dynamic modeling of coevolution, notions and methods associated, and to propose a research agenda making a synthesis of key problems that in our perspective need to be tackled to be able to construct such models.


% announce plan

The rest of the paper is organised as follows : we analyse first separately each axis proposed before, developing for each a specific problematic and proposing associated research projects. We make then a synthesis of these various aspects, establishing the research agenda.



%%%%%%%%%%%%%%%%%%%%
\section{Analysis}
%%%%%%%%%%%%%%%%%%%%



%%%%%%%%%%%%%%%%%%%%
\subsection{Giere's Deamon, or when disciplinary compartimentation narrows perspectives}
%%%%%%%%%%%%%%%%%%%%

% First track : Applied epistemology to understand the complementarity yet diversification of sci approaches. // Caruso and Krugman reconcile urban economics with ABM ? 

% results of ecqtg, completed with citation network [scholar api etc -> cit repo ? doi ? ]

% perspectives on datamining approach

As Laplace had its Deamon that was the embodiement of a determinist and reductivist approach to science
% \cite{laplace deamon} ?
, we could imagine Giere's Deamon, the embodiement of perspectivism. Whatever its initially attributed tasks, the poor Deamon would surely desesperately try to reconcile conflicting disciplines and viewpoints.
% small comment on difficulty of interdisciplinarity ?




%%%%%%%%%%%%%%%%%%%%
\subsection{Empirical analysis : ``Lost in the Smog''}
%%%%%%%%%%%%%%%%%%%%

% Parallel with scaling : different conclusions, different objects, different scales etc.

% -> need of precise empirical constatations within model context - NO GENERALITIES
% here evoque a general form linking Barthelemy and Raimbault ? -> machine learning based approaches to estimate transition matrix ?

% -> Essential question of scale AND ontology
% question of scale : systematic large scale correlations study





%%%%%%%%%%%%%%%%%%%%
\subsection{Methodological Foundations, need for concrete}
%%%%%%%%%%%%%%%%%%%%

% -> behavior of coupled models : example of difficult ?
% -> framework for model coupling (ebimm Clem Paul etc) : extending to other modeling tools ?
% -> Statistical control, synthetic data and -- Space Matters --

The modeling of coevolution of instrisically a problem of coupling in a complex way (in the sense of Varenne~\cite{varenne2010framework})
% TODO check citation
complementary geographical processes. The question of coupling models is shady and subtle, and no generalized approach have been proposed up to date. Indeed, this would partially consist in a meta-modeling entreprise, what could be the reason of its absence.
% develop on the impossibility of meta-modeling -- CIT. // beware, may be conatrdictory with mentionnaing and devoloping ebbimm, simfamily, simpuzzle.

A recent methodological progress within the context of agent-based modeling has been the proposition of a modular framework for model construction~\cite{cottineau2015incremental}. A 3 dimensionnal space in which a model is embedded, allows to modularly construct a model, at different levels of complexity for each dimension. Note that the concrete application of this framework in~\cite{cottineau2014evolution} underlies that the next spatial complexity levels to build are indeed the modeling of the coevolution (our problem) and the inclusion of governance structures at a next level (in transition with the next explored axis).

Let formulate a generalisation of the incremental framework, that would be compatible with other modeling approaches (e.g. dynamical systems, statistical modeling), would allow a formal definition of model coupling and would also be compatible with the ontological vision of agent-based modeling of Livet et al.~\cite{livet2010}.

Let $(M_i)_{1\leq i \leq N}$ the models to be compared. We define the ontological model space by
\[
\mathcal{O} = \bigcup_{i=1}^{N} O[M_i]
\]

where $O$ is the definition of model ontology.
% TODO : does any type of model has an ontology ?
%  clarify that. -- find a definition general enough ?
Models extension and coupling will be defined by equivalence and order relations in a partially ordered space. We say that $M_0$ extends $M_1$ \emph{within the same ontological dimension} if and only if $O[M_1] \subset O[M_0]$ and we write $M_1 \leq M_0 $. We define an equivalence class within the ontological space by
\[
M_1 \sim M_2 \iff M_1 \leq M_2 \& M_2 \leq M_1  
\]


The coupling of $M_1$ and $M_2$ can not be a unique model as many thematic and technical choice are made when concretely coupling models. In our framework, a generic model coupling will be an equivalence class of models, denoted $M_1 \times M_2$, such that 
% TODO refine coupling def.
The degree of coupling can be defined through Kolmogorov complexity


We can also add a ``meta-dimension'' to our framework, i.e. an other aspect needed to be able to compare models, is the performance on data. To compare two models on that aspect, at least two points have to be verified :
\begin{itemize}
\item Models can be run on the same dataset for the same purpose.
% TODO if pretreatment that do not alter information in data ; then can plus a model with a filter
\end{itemize}



%%%%%%%%%%%%%%%%%%%%
\subsection{Modeling the Governance : the Grand Pari(s)}
%%%%%%%%%%%%%%%%%%%%

% bring in by examples (Offner book ?) the question of gvernance.
% why necessary, why diificult
% first prospects with lutecia ? -- Quote Metropolsim and transmodyn transition.



%%%%%%%%%%%%%%%%%%%%
\section{Synthesis}
%%%%%%%%%%%%%%%%%%%%


\subsection{Proposition of a research agenda}

% assemble various tracks




\subsection{Towards calibrated dynamic models of coevolution}



% proposition of projects, concrete tracks.





%%%%%%%%%%%%%%%%%%%%
\section{Conclusion}
%%%%%%%%%%%%%%%%%%%%










%%%%%%%%%%%%%%%%%%%%
%% Biblio
%%%%%%%%%%%%%%%%%%%%

\bibliographystyle{apalike}
\bibliography{/Users/Juste/Documents/ComplexSystems/CityNetwork/Biblio/Bibtex/CityNetwork,biblio}


\end{document}
