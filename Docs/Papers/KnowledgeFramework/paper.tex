
%%%%%%%%%%%%%%%%%%%%%%% file typeinst.tex %%%%%%%%%%%%%%%%%%%%%%%%%
%
% This is the LaTeX source for the instructions to authors using
% the LaTeX document class 'llncs.cls' for contributions to
% the Lecture Notes in Computer Sciences series.
% http://www.springer.com/lncs       Springer Heidelberg 2006/05/04
%
% It may be used as a template for your own input - copy it
% to a new file with a new name and use it as the basis
% for your article.
%
% NB: the document class 'llncs' has its own and detailed documentation, see
% ftp://ftp.springer.de/data/pubftp/pub/tex/latex/llncs/latex2e/llncsdoc.pdf
%
%%%%%%%%%%%%%%%%%%%%%%%%%%%%%%%%%%%%%%%%%%%%%%%%%%%%%%%%%%%%%%%%%%%


\documentclass[runningheads,a4paper]{llncs2e/llncs}

\usepackage{amssymb,amsmath,bbm}
\setcounter{tocdepth}{3}
\usepackage{graphicx}

\usepackage{url}
\urldef{\mailjr}\path|juste.raimbault@polytechnique.edu|
\urldef{\mailjk}\path|julien.keutchayan@polytechnique.edu|
  
\newcommand{\keywords}[1]{\par\addvspace\baselineskip
\noindent\keywordname\enspace\ignorespaces#1}

\newcommand{\noun}[1]{\textsc{#1}}



\begin{document}

\mainmatter  % start of an individual contribution




% first the title is needed
\title{Title}

% a short form should be given in case it is too long for the running head
\titlerunning{}

% the name(s) of the author(s) follow(s) next
%
% NB: Chinese authors should write their first names(s) in front of
% their surnames. This ensures that the names appear correctly in
% the running heads and the author index.
%
\author{\noun{Juste Raimbault}$^{1,2}$}
%
\authorrunning{}
% (feature abused for this document to repeat the title also on left hand pages)

% the affiliations are given next; don't give your e-mail address
% unless you accept that it will be published
\institute{$^{1}$ UMR CNRS 8504 G{\'e}ographie-Cit{\'e}s, Paris, France\\
$^{2}$ UMR-T IFSTTAR 9403 LVMT, Champs-sur-Marne, France\\
%$^{3}$ D{\'e}partement de math{\'e}matiques et de g{\'e}nie industriel,\\
%Ecole Polytechnique de Montr{\'e}al, Montr{\'e}al, Canada\\\medskip
\mailjr\\
%\mailjk
}

\toctitle{Lecture Notes in Computer Science}
\tocauthor{Authors' Instructions}
\maketitle


\begin{abstract}

\keywords{kw1, kw2}
\end{abstract}



%%%%%%%%%%%%%%%%
\section{Introduction}
%%%%%%%%%%%%%%%%

% 2p

% https://arxiv.org/pdf/1704.01407.pdf : framework in AI (not knowledge framework)




%%%%%%%%%%%%%%%%
\section{Case Studies}
%%%%%%%%%%%%%%%%

% - th evolutive : 4p (incl. figure nw)
% - ingeneering example : 2p



%%%%%%%%%%%%%%%%
\section{Formalization}
%%%%%%%%%%%%%%%%

% 3p (precise defs etc ; incl. figure synthesis)



%%%%%%%%%%%%%%%%
\section{Discussion}
%%%%%%%%%%%%%%%%

% 1p


%%%%%%%%%%%%%%%%
%% Conclusion
%%%%%%%%%%%%%%%%
\section*{Conclusion}





%\section*{Acknowledgments}




%%%%%%%%%%%%%%%%
%% Biblio
%%%%%%%%%%%%%%%%

\bibliographystyle{unsrt}
%\bibliography{biblio/biblio}





\end{document}
