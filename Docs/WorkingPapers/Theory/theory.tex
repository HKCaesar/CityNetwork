%%%%%%%%%%%%%%%%%%%%%%%%%%%%%
% Standard header for working papers
%
% WPHeader.tex
%
%%%%%%%%%%%%%%%%%%%%%%%%%%%%%

\documentclass[11pt]{article}

%%%%%%%%%%%%%%%%%%%%
%% Include general header where common packages are defined
%%%%%%%%%%%%%%%%%%%%

% general packages without options
\usepackage{amsmath,amssymb,bbm}




%%%%%%%%%%%%%%%%%%%%
%% Idem general commands
%%%%%%%%%%%%%%%%%%%%
%% Commands

\newcommand{\noun}[1]{\textsc{#1}}


%% Math

% Operators
\DeclareMathOperator{\Cov}{Cov}
\DeclareMathOperator{\Var}{Var}
\DeclareMathOperator{\E}{\mathbb{E}}
\DeclareMathOperator{\Proba}{\mathbb{P}}

\newcommand{\Covb}[2]{\ensuremath{\Cov\!\left[#1,#2\right]}}
\newcommand{\Eb}[1]{\ensuremath{\E\!\left[#1\right]}}
\newcommand{\Pb}[1]{\ensuremath{\Proba\!\left[#1\right]}}
\newcommand{\Varb}[1]{\ensuremath{\Var\!\left[#1\right]}}

% norm
\newcommand{\norm}[1]{\| #1 \|}


% amsthm environments
\newtheorem{definition}{Definition}



%% graphics

% renew graphics command for relative path providment only ?
%\renewcommand{\includegraphics[]{}}








% geometry
\usepackage[margin=2cm]{geometry}

% layout : use fancyhdr package
\usepackage{fancyhdr}
\pagestyle{fancy}

\makeatletter

\renewcommand{\headrulewidth}{0.4pt}
\renewcommand{\footrulewidth}{0.4pt}
%\fancyhead[RO,RE]{\textit{Working Paper}}
\fancyhead[RO,RE]{\textit{ECTQG 2015}}
%\fancyhead[LO,LE]{G{\'e}ographie-Cit{\'e}s/LVMT}
\fancyhead[LO,LE]{An Algorithmic Systematic Review}
\fancyfoot[RO,RE] {\thepage}
\fancyfoot[LO,LE] {\noun{J. Raimbault}}
\fancyfoot[CO,CE] {}

\makeatother


%%%%%%%%%%%%%%%%%%%%%
%% Begin doc
%%%%%%%%%%%%%%%%%%%%%

\begin{document}







\title{
% alternative title : A Model-oriented Perspectivist Theory of [Urban ?] [Complex ?] Systems
A Reflexive Theory for the Study of Socio-Technical Systems
\bigskip\\
\textit{Working Paper}
}
\author{\noun{Juste Raimbault}}
\date{18th November 2015}


\maketitle

\justify


\begin{abstract}

\end{abstract}



\section*{Introduction}

The structural misunderstandings between Social Sciences and Humanities on one side, and so-called Exact Sciences on the other side, far from being a generality, seems to have however a significant impact on the structure of scientific knowledge~\cite{2015arXiv151103981H}. In particular, the place of theory (and indeed the signification of this term itself) in the elaboration of knowledge has a totally different place, partly because of the different \emph{perceived complexities}\footnote{We used the term \emph{perceived} as most of systems studied by physics might be described as simple whereas they are intrinsically complex and indeed not well understood~\cite{laughlin2006different}.} of studied objects : for example, mathematical constructions and by extent theoretical physics are \emph{simple} in the sense that they are mostly entierely analytically solvable, whereas Social Science subjects such as humans or society (to give a \emph{clich{\'e}} exemple) are \emph{complex} in the sense of complex systems\footnote{for which no unified definition exists but of which fields of application range broadly from neuroscience to quantitative finance, incuding e.g. quantitative sociology, quantitative geography, integrative biology, etc.~\cite{newman2011complex}, and for which study various complementary approaches may be applied, such as Dynamical Systems, Agent-based Modeling, Random Matrix Theory}, thus a stronger need of a constructed theoretical (generally empirically based) framework to identify and define the objects of research that are necessarily more arbitrary in the framing of their boundaries, relations and processes. % ref on that claim ?
These differences in backgrounds are naturally desirable in the caleidogram of science, but things can get nasty when playing on ``common'' terrains, typically complex systems problematics as already detailed, as the exemple of geographical urban systems has recently shown~\cite{dupuy2015sciences}. Complex System Science\footnote{that we deliberately call that way} is seen by some as 



\section*{Objectives}




\section*{Construction of the theory}




\section*{Application : co-evolution of subsystems}



\section*{Discussion}



\section*{Conclusion}





%%%%%%%%%%%%%%%%%%%%
%% Biblio
%%%%%%%%%%%%%%%%%%%%

\bibliographystyle{apalike}
\bibliography{/Users/Juste/Documents/ComplexSystems/CityNetwork/Biblio/Bibtex/CityNetwork}


\end{document}
