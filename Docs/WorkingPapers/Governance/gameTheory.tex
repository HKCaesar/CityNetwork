%%%%%%%%%%%%%%%%%%%%%%%%%%%%%
% Standard header for working papers
%
% WPHeader.tex
%
%%%%%%%%%%%%%%%%%%%%%%%%%%%%%

\documentclass[11pt]{article}

%%%%%%%%%%%%%%%%%%%%
%% Include general header where common packages are defined
%%%%%%%%%%%%%%%%%%%%

% general packages without options
\usepackage{amsmath,amssymb,bbm}




%%%%%%%%%%%%%%%%%%%%
%% Idem general commands
%%%%%%%%%%%%%%%%%%%%
%% Commands

\newcommand{\noun}[1]{\textsc{#1}}


%% Math

% Operators
\DeclareMathOperator{\Cov}{Cov}
\DeclareMathOperator{\Var}{Var}
\DeclareMathOperator{\E}{\mathbb{E}}
\DeclareMathOperator{\Proba}{\mathbb{P}}

\newcommand{\Covb}[2]{\ensuremath{\Cov\!\left[#1,#2\right]}}
\newcommand{\Eb}[1]{\ensuremath{\E\!\left[#1\right]}}
\newcommand{\Pb}[1]{\ensuremath{\Proba\!\left[#1\right]}}
\newcommand{\Varb}[1]{\ensuremath{\Var\!\left[#1\right]}}

% norm
\newcommand{\norm}[1]{\| #1 \|}


% amsthm environments
\newtheorem{definition}{Definition}



%% graphics

% renew graphics command for relative path providment only ?
%\renewcommand{\includegraphics[]{}}








% geometry
\usepackage[margin=2cm]{geometry}

% layout : use fancyhdr package
\usepackage{fancyhdr}
\pagestyle{fancy}

\makeatletter

\renewcommand{\headrulewidth}{0.4pt}
\renewcommand{\footrulewidth}{0.4pt}
%\fancyhead[RO,RE]{\textit{Working Paper}}
\fancyhead[RO,RE]{\textit{ECTQG 2015}}
%\fancyhead[LO,LE]{G{\'e}ographie-Cit{\'e}s/LVMT}
\fancyhead[LO,LE]{An Algorithmic Systematic Review}
\fancyfoot[RO,RE] {\thepage}
\fancyfoot[LO,LE] {\noun{J. Raimbault}}
\fancyfoot[CO,CE] {}

\makeatother


%%%%%%%%%%%%%%%%%%%%%
%% Begin doc
%%%%%%%%%%%%%%%%%%%%%

\begin{document}







\title{Game-theory Based Behavioral Rules for Governing Agents in a Hybrid Land-Use Transportation Model \bigskip\\
\textit{Working Paper}
}
\author{\noun{Juste Raimbault}}
\date{Friday 27th March}


\maketitle

\justify

\begin{abstract}
We briefly describe a simple game-theory based framework which aims to be integrated as behavioral rules for governing agents in a hybrid model introduced in~\cite{le2010approche} and formalized then explored in~\cite{lenechet2012}. This model couples land-use dynamics with transportation infrastructure evolution and aims to endogeneize transportation infrastructure development at different levels. The framework proposed extends it by allowing cooperation and fusion between governing entities.
\end{abstract}


\subsection*{Thematic Description}

As detailed in~\cite{lenechet2012}, a conceptual city system with local administrative boundaries and corresponding governing agents (mayors), and a global governor (state) is the foundation of the model. A land-use evolution (residences and employments localisations) and transportation (gravital flows) are the first step of an iteration. The transportation infrastructure (road network) is then evolved by contructing a new road. First level of decision (global or local) is chosen randomly according to a fixed probability, and in the case of a local decision, the richest mayor will build the new road. The road is then build optimizing the marginal accessibility for the area corresponding to the builder in charge (all world if global, commun if local).

One thematic aspect lacking in the model and that would be interesting to study is the emergence of larger administrative zones, i.e. the emergence of new levels of governance in poycentric metropolitan areas. The reality is of course not as simple, as bottom-up initiatives such as collaboration between neighbor cities are entrelaced with top-down decisions such as e.g. the ``M{\'e}tropole du Grand Paris'' which is a new administrative structure for Paris Area decided at the state level~\cite{gilli2009paris}. It would be however interesting to test conditions for emergence of governance patterns from the bottom-up in a conceptual way by extending the model and adding interactions and fusion between administrative entities.

The extension shall consist in relaxing the assumption of a single road segment built at each time step and attribute one segment to the $N$ richest mayors. That leads to situation where neighor towns may want to construct both a new road. As they are likely to communicate with each other, we assume that negociations take place and that they consider eventually to build in common, in which case they merge after (rough simplifying but stylized assumption). Such negociations may be interpreted as a game in the sense of Game Theory, which as already been widely applied for modeling in social and political sciences for questions dealing with cognitive interacting agents with individual interests~\cite{ordeshook1986game}
. Such a framework as already been used in transportation investment studies, as e.g. in~\cite{Roumboutsos2008209} where choices of operators (public and privates) to integrate their system in a global consistent commuter system is explored through the notion of Nash equilibrium.




\subsection*{Formalization}

\textit{We refer to \cite{lenechet2012} for notations.}
\medskip

The workflow for transportation network development is the following :

\begin{itemize}

\item At each time step, $N$ new road segments are built. Choice between local and global is still done through uniform drawing with probability $\xi$. In the case of local building, roads are attributed successively to mayors with probabilities $\xi_i$, what means that richer areas may get many roads. It stays consistent with the thematic assumption than each road correspond to the allocation of one public market which are done independantly (with $N$ becoming greater, this assumption should be relaxed as attribution of subventions to local areas is of course not proportional to wealth, but we assume that it stays true with small $N$ values). 

\item Areas building a road without neighbors doing it follow the standard procedure to develop the road network.

\item Neighbor areas building a road will enter negociations. We assume in this first simple version of the model that ony bilateral negociations may occur. Therefore, in the case of clusters with more than two areas, pairing is done at random (uniform drawing) between neighbors until all areas are paired.

\item Possible strategies for players (negociating areas, $i=1,2$) are : staying alone ($A$) and collaborating ($C$). Strategies are chosen simultaneously (non-cooperative game) as detailed after. For $(C,A)$ and $(A,C)$ couples, the collaborating agent loose its investment and cannot build a road whereas the other continues his buiseness alone. For $(A,A)$ both act as alone, and for $(C,C)$ a common development is done. We denote $Z^{\ast}_i(S_1,S_2)$ the optimal infrastructure for area $i$ with $(S_1,S_2)\in \{(A,C),(C,A),(A,A)\}$ which are determined the standard way in each zone separately, and $Z^{\ast}_C$ the optimal common infrastructure computed with a 2 segments infrastructure on the union of both areas, which corresponds to the case where both strategies are $C$. Marginal accessibilities for area $i$ and infrastucture $Z$ is defined as $\Delta X_i(Z)=X^Z_i - X_i$. We introduce the costs of construction which are necessary to build the payoff matrix. They are assumed spatially uniform and noted $I$ for a road segment, whereas a 2 road segment will cost $2\cdot I - \delta I$ ($\delta I > 0$ cost gain of common technical means, assumed to be equally shared). An interesting generalisation would be to divise costs proportionnaly to wealth in the case of a collaboration. The payoff matrix of the game is the following, with $\kappa$ a normalization constant (``price of acessibility'') :

\medskip
\hfill
\begin{tabular}{ |c|c|c| } 

 \hline
 1 $|$ 2  & C & A \\ \hline
 C & $U_i = \kappa \cdot \Delta X_i(Z^{\ast}_C) - I + \frac{\delta I}{2}$
   & $\begin{cases}U_1 = -I + \frac{\delta I}{2} \\U_2 = \kappa \cdot \Delta X_2(Z^{\ast}_2)-I + \frac{\delta I}{2}\end{cases}$ \\ \hline
 A & $\begin{cases}U_1 = \kappa \cdot \Delta X_1(Z^{\ast}_1)-I + \frac{\delta I}{2}\\U_2 = -I + \frac{\delta I}{2}\end{cases}$
   & $U_i = \kappa \cdot \Delta X_i(Z^{\ast}_i) - I$ \\
 \hline
\end{tabular}
\hfill\hfill
\medskip

We have a typical coordination game for which it is clear that no strategy is dominant for any player. In a probabilistic mixed-strategy case, there always exists a Nash equilibrium that we can easily determine in our case. It is reasonable to make such an assumption since negociations take generally some time during which agents are able to find the way to optimize rationnaly their expected utility. If $\Pb{S_1=C} = p_1$ and $\Pb{S_2=C} = p_2$, we have

\[
\begin{split}
\Eb{U_1} & =p_1 p_2 U_1(C,C) + p_1\cdot (1-p_2) U_1(C,A) + p_2 \cdot (1-p_1) U_1(A,C) + (1-p_1)(1-p_2) U_1(A,A)\\
& = p_1 \cdot \left[ p_2 \cdot \left(\kappa \cdot \Delta X_1(Z^{\ast}_C) - \frac{\delta I}{2} \right) - \kappa \cdot \Delta X_1(Z^{\ast}_1) + I\right] + p_2\cdot\frac{\delta I}{2} + \kappa\cdot\Delta X_1(Z^{\ast}_1)-I
\end{split}
\]

Optimizing the expected utility along $p_1$ (the variable on which agent 1 has control) imposes the condition on $p_2$

\[
\frac{\partial \Eb{U_1}}{\partial p_1} = 0 \iff p_2 = \frac{\Delta X_1(Z^{\ast}_1)-\frac{I}{\kappa}}{\Delta X_1(Z^{\ast}_C) - \frac{\delta I}{2\cdot \kappa}}
\]

We obtain the same way

\[
p_1 = \frac{\Delta X_2(Z^{\ast}_2)-\frac{I}{\kappa}}{\Delta X_2(Z^{\ast}_C) - \frac{\delta I}{2\cdot\kappa}}
\]

Note that we can directly interpret these expressions, as a player chances to cooperate will decrease with the potential gain of the other player, what is intuitive for a competetive game. It also forces feasability conditions on $I$ and $\delta I$ to keep a probability, that are $I \leq \kappa\cdot \min(\Delta X_1(Z^{\ast}_1),\Delta X_2(Z^{\ast}_2))$ (binary positive cost-benefit conditions) and $I-\delta I > \kappa \cdot \max_i (\Delta X_i(Z^{\ast}_i)-\Delta X_i(Z^{\ast}_C))$. As soon as accessibility difference stay relatively small, both shall be compatible when $\delta I \ll I$, giving corresponding boundaries for $I$.

\item Agents make choice of strategy following uniform drawings with probability computed above. Corresponding infrastructures are built, and in the case of choices $(C,C)$, towns merge in a single one with new corresponding variables (employment, actives, etc. ).


\end{itemize}



\subsection*{Notes for the implementation}

To adapt an existing implementation, one just has to add the negociation stage if conditions are met, using probabilities given above. The accessibility-dimensioned parameters $\alpha = \frac{I}{\kappa}$ and $\delta \alpha = \frac{\delta I}{\kappa}$ should be more simple to deal with.



\subsection*{Further work}

Implementation, sensitivity analysis and exhaustive exploration of the model. Determination of conditions for emergence of large collaborative matropolitan structures.


%%%%%%%%%%%%%%%%%%%%
%% Biblio
%%%%%%%%%%%%%%%%%%%%

\bibliographystyle{apalike}
\bibliography{/Users/Juste/Documents/ComplexSystems/CityNetwork/Biblio/Bibtex/CityNetwork}


\end{document}
