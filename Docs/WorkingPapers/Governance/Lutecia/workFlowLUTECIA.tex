

\title{Step-by-step validation of LUTECIA model}

\begin{itemize}
\item LU stands for Land Use module
\item T stands for Transport module
\item EC stands for Evaluation of Cooperation module
\item I stands for Infrastructure provision module
\item A stands for Agglomeration economies module
\end{itemize}

\section{Validation of the LU-T joint modules}

List of key processes implemented, and associated parameters (\textit{ref tableau qui indique l'ensemble des param�tres du mod�le}):
\begin{itemize}
\item Production of flows, parameters ?
\item Relocation of inhabitants, parameters ?
\item Relocation of activities, parameters ?
\end{itemize}


Stylized facts to retrieve : 
\begin{itemize}
	\item Stability of monocentric structure with one CBD cell and descending density gradient
	\item Response to a new spider network (forced) : sprawl that keep the mathematical form of the descending density gradient
	\item Response to a change in cost of energy (parameter \lambda) : sprawl of inhabitants bigger than the sprawl of jobs
\end{itemize}

\section{Validation of the LU-T-I joint modules}

List of key processes implemented, and associated parameters (\textit{ref tableau qui indique l'ensemble des param�tres du mod�le}):
\begin{itemize}
\item Creation of a new infrastructure to maximize new accessibility of a given stakeholder
\end{itemize}

Stylized facts to retrieve :
\begin{itemize}
	\item Reasonable path dependancy : form follow network, but not too rapidly
	\item Optimal routes correspond to physical least effort law which network better link the four points of a square \rightarrow not a cross, but a \verbatim{>-<} shape
\end{itemize}

\section{Validation of the LU-T-I-A joint modules}


List of key processes implemented, and associated parameters (\textit{ref tableau qui indique l'ensemble des param�tres du mod�le}):
\begin{itemize}
\item Creation of new jobs in the region to account for the increase of overall accessibility
\item Spread of this jobs in a few places with exceptional amenities
\item Adaptation of number of inhabitants and relocation
\end{itemize}


Stylized facts to retrieve : 
\begin{itemize}
	\item Increase of accessibility of a monocentric region shall fall within reasonable interval (see litterature such as Desjardin in Paris region) 
	\item Resulting increase of number of jobs shall fall within reasonable interval (see litterature on agglomeration economies)
\end{itemize}

\section{Validation of the LU-T-EC-I-A joint modules}

List of key processes implemented, and associated parameters (\textit{ref tableau qui indique l'ensemble des param�tres du mod�le}):
\begin{itemize}
	\item Choice between local stakeholders developments of integrated regional development
	
\end{itemize}

Stylized facts to retrieve : 
\begin{itemize}
	\item For a region with obvious no need to collaborate (e.g. cities far away from each other) : collaboration never occurs
	\item For a region with obvious need to collaborate (e.g. large elasticity of jobs regarding to overall accessibility and urban form such as better link for local purpose and regional purpose are self-evident \textbb{and} different
	\item For a intermediate region, the relative share of cooperation / non-cooperation outcome is quite stable among replications of the simulation
\end{itemize}


\section{Sensitivity analysis}

Quand on en sera l� ce sera super, mais pour m�moire, je trouve qu'on a pas assez travaill� l'aspect forme urbaine initiale.