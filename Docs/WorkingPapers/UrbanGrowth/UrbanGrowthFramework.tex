%%%%%%%%%%%%%%%%%%%%%%%%%%%%%
% Standard header for working papers
%
% WPHeader.tex
%
%%%%%%%%%%%%%%%%%%%%%%%%%%%%%

\documentclass[11pt]{article}

%%%%%%%%%%%%%%%%%%%%
%% Include general header where common packages are defined
%%%%%%%%%%%%%%%%%%%%

% general packages without options
\usepackage{amsmath,amssymb,bbm}




%%%%%%%%%%%%%%%%%%%%
%% Idem general commands
%%%%%%%%%%%%%%%%%%%%
%% Commands

\newcommand{\noun}[1]{\textsc{#1}}


%% Math

% Operators
\DeclareMathOperator{\Cov}{Cov}
\DeclareMathOperator{\Var}{Var}
\DeclareMathOperator{\E}{\mathbb{E}}
\DeclareMathOperator{\Proba}{\mathbb{P}}

\newcommand{\Covb}[2]{\ensuremath{\Cov\!\left[#1,#2\right]}}
\newcommand{\Eb}[1]{\ensuremath{\E\!\left[#1\right]}}
\newcommand{\Pb}[1]{\ensuremath{\Proba\!\left[#1\right]}}
\newcommand{\Varb}[1]{\ensuremath{\Var\!\left[#1\right]}}

% norm
\newcommand{\norm}[1]{\| #1 \|}


% amsthm environments
\newtheorem{definition}{Definition}



%% graphics

% renew graphics command for relative path providment only ?
%\renewcommand{\includegraphics[]{}}








% geometry
\usepackage[margin=2cm]{geometry}

% layout : use fancyhdr package
\usepackage{fancyhdr}
\pagestyle{fancy}

\makeatletter

\renewcommand{\headrulewidth}{0.4pt}
\renewcommand{\footrulewidth}{0.4pt}
%\fancyhead[RO,RE]{\textit{Working Paper}}
\fancyhead[RO,RE]{\textit{ECTQG 2015}}
%\fancyhead[LO,LE]{G{\'e}ographie-Cit{\'e}s/LVMT}
\fancyhead[LO,LE]{An Algorithmic Systematic Review}
\fancyfoot[RO,RE] {\thepage}
\fancyfoot[LO,LE] {\noun{J. Raimbault}}
\fancyfoot[CO,CE] {}

\makeatother


%%%%%%%%%%%%%%%%%%%%%
%% Begin doc
%%%%%%%%%%%%%%%%%%%%%

\begin{document}







\title{A Unified Framework for Stochastic Models of Urban Growth\bigskip\\
\textit{Working Paper}
}
\author{\noun{Juste Raimbault}}
\date{Tuesday 28th April}


\maketitle

\justify


\begin{abstract}
Various stochastic models aiming to reproduce population patterns on large temporal and spatial scales (city systems) have been discussed across various fields of the litterature, from economics to geography, including models proposed by physicists. We propose a general framework that allows to include different famous models (in particular Gibrat, Simon and Preferential Attachment model) within an unified vision. It brings first an insight into epistemological debates on the relevance of models. Furthermore, bridges between models lead to the possible transfer of analytical results to some models that are not directly tractable.
\end{abstract}




%%%%%%%%%%%%%%%%%%%%
\section{Introduction}
%%%%%%%%%%%%%%%%%%%%

%%%%%%%%%%%%%%%%%%%%
\subsection{Context}

General biblio.

Precise type of models : mathematical models ; stay to a certain level of tractability as essence of our approach is link between models. No clear definition, includes all models that can be linked in the sense of \emph{Generalization/Particularization/Limit case/?}.


%%%%%%%%%%%%%%%%%%%%
\subsection{Notations}



%%%%%%%%%%%%%%%%%%%%
\section{Framework}
%%%%%%%%%%%%%%%%%%%%


%%%%%%%%%%%%%%%%%%%%
\subsection{Formulation}

\subsubsection{Presentation}
What we propose as a framework can be understood as a meta-model in the sense of~\cite{cottineau2015incremental}, i.e. an modular general modeling process within each model can be understood as a limit case or as a specific case of another model. More simply it shoud be a diagram of formal relations between models. The ontological aspect is also tackled by embedding the diagram into an ontological state space (which discretization corresponds to the ``bricks'' of the incremental construction of~\cite{cottineau2015incremental}). It constructs a sort of model classification.

\subsubsection{Models Included}

The following models are included in our framework. The list is arbitrary but aims to offer a broad view of disciplines concerned

\subsubsection{Thematic Classification}


\subsubsection{Framework Formulation}

Diagram linking various models ; first embedded into time/population plane, cases Discrete/Continous. Other aspects more sparse (ex. spatialization) ; how represent it ?

%%%%%%%%%%%%%%%%%%%%
\subsection{Models formulation}



%%%%%%%%%%%%%%%%%%%%
\subsection{Derivations}

\subsubsection{Generalization of Preferential Attachment}

See \cite{yamasaki2006preferential}.

\subsubsection{Link between Gibrat and Preferential Attachment Models}

Let consider a stricly positive growth Gibrat model given by $P_i(t)=R_i(t)\cdot P_{i}(t-1)$ with $R_i(t)>1$, $\mu_i(t)=\Eb{R_i(t)}$ and $\sigma_i(t)=\Eb{R_i(t)^2}$. On the other hand, we take a simple preferential attachment, with fixed attachment probability $\lambda \in [0,1]$ and new arrivants number $m>0$. We derive that Gibrat model can be statistically equivalent to a limit of the preferential attachment model, assuming that the moment-generating function of $R_i(t)$ exists. Classical distributions that could be used in that case, e.g. log-normal distribution, are entirely defined by two first moments, making this assumption reasonable.

\begin{lemma}
The limit of a Preferential Attachment model when $\lambda \ll 1$ is a linear-growth Gibrat model, with limit parameters $\mu_i(t)=1+\frac{\lambda}{m\cdot (t-1)}$.
\end{lemma}

\begin{proof}

Starting with first moment, we denote $\bar{P}_i(t)=\Eb{P_i(t)}$. Independance of Gibrat growth rate yields directly $\bar{P}_i(t)=\Eb{R_i(t)}\cdot \bar{P}_i(t-1)$. Starting for the preferential attachment model, we have $\bar{P}_i(t) = \Eb{P_i(t)} = \sum_{k=0}^{+\infty}{k\Pb{P_i(t)=k}}$. But
\[
\{P_i(t)=k\}=\bigcup_{\delta=0}^{\infty}{\left(\{P_i(t-1)=k-\delta\}\cap \{P_i\leftarrow P_i + 1\}^{\delta}\right)}
\]

where the second event corresponds to city $i$ being increased $\delta$ times between $t-1$ and $t$ (note that events are empty for $\delta \geq k$). Thus, being careful on the conditional nature of preferential attachment formulation, stating that $\Pb{\{P_i\leftarrow P_i + 1\} | P_i(t-1)=p} = \lambda\cdot\frac{p}{P(t-1)}$ (total population $P(t)$ assumed deterministic), we obtain

\begin{equation*}
\begin{split}
\Pb{\{P_i\leftarrow P_i + 1\}} & = \sum_{p}{\Pb{\{P_i\leftarrow P_i + 1\} | P_i(t-1)=p}\cdot \Pb{P_i(t-1)=p}}\\
&=\sum_{p}{\lambda\cdot\frac{p}{P(t-1)}\Pb{P_i(t-1)=p}}=\lambda\cdot\frac{\bar{P}_i(t-1)}{P(t-1)}\\
\end{split}
\end{equation*}

It gives therefore, knowing that $P(t-1)=P_0 + m\cdot (t-1)$ and denoting $q=\lambda\cdot\frac{\bar{P}_i(t-1)}{P_0 + m\cdot (t-1)}$

\[
\begin{split}
\bar{P}_i(t) & =\sum_{k=0}^{\infty}{\sum_{\delta=0}^{\infty}{k\cdot \left(\lambda\cdot\frac{\bar{P}_i(t-1)}{P_0 + m\cdot (t-1)}\right)^{\delta}\cdot \Pb{P_i(t-1)=k-\delta}}}\\
& = \sum_{\delta^{\prime}=0}^{\infty}{\sum_{k^{\prime}=0}^{\infty}{\left(k^\prime + \delta^{\prime}\right)\cdot q^{\delta^{\prime}} \cdot \Pb{P_i(t-1)=k^\prime}}}\\
& = \sum_{\delta^{\prime}=0}^{\infty}{q^{\delta^{\prime}}\cdot \left(\delta^{\prime} + \bar{P}_i(t-1)\right)} = \frac{q}{(1-q)^2} + \frac{\bar{P}_i(t-1)}{(1-q)} = \frac{\bar{P}_i(t-1)}{1-q}\left[1 + \frac{1}{\bar{P}_i(t-1)}\frac{q}{(1-q)}\right]
\end{split}
\]

%& = \bar{P}_i(t-1)\cdot \frac{1}{1-\lambda\cdot\frac{\bar{P}_i(t-1)}{P_0 + m\cdot (t-1)}} \left[1 + \frac{\lambda}{P_0 + m\cdot (t-1)}\cdot \frac{1}{1-\lambda\cdot\frac{\bar{P}_i(t-1)}{P_0 + m\cdot (t-1)}} \right]


As it is not expected to have $\bar{P}_i(t)\ll P(t)$ (fat tail distributions), a limit can be taken only through $\lambda$. Taking $\lambda \ll 1$ yields, as $0 < \bar{P}_i(t)/P(t) < 1$, that $q=\lambda\cdot\frac{\bar{P}_i(t-1)}{P_0 + m\cdot (t-1)} \ll 1$ and thus we can expand in first order of $q$, what gives $\bar{P}_i(t)=\bar{P}_i(t-1)\cdot \left[1 + \left(1+\frac{1}{\bar{P}_i(t-1)}\right)q + o(q))\right]$

\[
\bar{P}_i(t) \simeq \left[1 + \frac{\lambda}{P_0 + m\cdot (t-1)}\right]\cdot \bar{P}_i(t-1)
\]

It means that this limit is equivalent in expectancy to a Gibrat model with $\mu_i(t) = \mu(t)=1 + \frac{\lambda}{P_0 + m\cdot (t-1)}$.

For the second moment, we can do an analog computation. We have still $\Eb{P_i(t)^2} = \Eb{R_i(t)^2}\cdot \Eb{P_i(t-1)^2}$ and $\Eb{P_i(t)^2}=\sum_{k=0}^{+\infty}{k^2 \Pb{P_i(t)=k}}$. We obtain the same way 

\[
\begin{split}
\Eb{P_i(t)^2} & = \sum_{\delta^{\prime}=0}^{\infty}{\sum_{k^{\prime}=0}^{\infty}{\left(k^\prime + \delta^{\prime}\right)^2\cdot q^{\delta^{\prime}} \cdot \Pb{P_i(t-1)=k^\prime}}} = \sum_{\delta^{\prime}=0}^{\infty}{q^{\delta^{\prime}}\cdot \left(\Eb{P_i(t-1)^2}+2\delta^{\prime}\bar{P}_i(t-1) + {\delta^{\prime}}^2\right)}\\
& = \frac{\Eb{P_i(t-1)^2}}{1-q} + \frac{2 q \bar{P}_i(t-1)}{(1-q)^2} + \frac{q(q+1)}{(1-q)^3} = \frac{\Eb{P_i(t-1)^2}}{1-q}\left[1 + \frac{q}{\Eb{P_i(t-1)^2}}\left(\frac{2\bar{P}_i(t-1)}{1-q} + \frac{(1+q)}{(1-q)^2}\right)\right]
\end{split}
\]

\qed

\end{proof}




\subsubsection{Link between Simon and Preferential Attachment}
\label{subsubsec:gibrat-simon}


\subsubsection{Link between Favaro-Pumain and Gibrat}

\cite{favaro2011gibrat}

\subsubsection{Link between Bettencourt-West and Simon}

\cite{bettencourt2008large}


\subsubsection{Other Models}

\cite{gabaix1999zipf} : Economic model giving a Simon equivalent formulation. Finds that in upper tail, proportional growth process occurs. We find the same result as a consequence of~\ref{subsubsec:gibrat-simon}.




%%%%%%%%%%%%%%%%%%%%
\section{Application}
%%%%%%%%%%%%%%%%%%%%




%%%%%%%%%%%%%%%%%%%%
\section{Discussion}
%%%%%%%%%%%%%%%%%%%%






%%%%%%%%%%%%%%%%%%%%
\section*{Conclusion}
%%%%%%%%%%%%%%%%%%%%



%%%%%%%%%%%%%%%%%%%%
%% Biblio
%%%%%%%%%%%%%%%%%%%%

\bibliographystyle{apalike}
\bibliography{/Users/Juste/Documents/ComplexSystems/CityNetwork/Biblio/Bibtex/CityNetwork}


\end{document}