%% Commands

\newcommand{\noun}[1]{\textsc{#1}}

% command fort head of chapter citation
\newcommand{\headercit}[3]{
\begin{multicols}{2}
\phantom{}
\columnbreak
\textit{#1}

 - \noun{#2}~#3
\end{multicols}
}



%% Math

% Operators
\DeclareMathOperator{\Cov}{Cov}
\DeclareMathOperator{\Var}{Var}
\DeclareMathOperator{\E}{\mathbb{E}}
\DeclareMathOperator{\Proba}{\mathbb{P}}

\newcommand{\Covb}[2]{\ensuremath{\Cov\!\left[#1,#2\right]}}
\newcommand{\Eb}[1]{\ensuremath{\E\!\left[#1\right]}}
\newcommand{\Pb}[1]{\ensuremath{\Proba\!\left[#1\right]}}
\newcommand{\Varb}[1]{\ensuremath{\Var\!\left[#1\right]}}

% norm
\newcommand{\norm}[1]{\| #1 \|}

% independent
\newcommand{\indep}{\rotatebox[origin=c]{90}{$\models$}}


% amsthm environments
\newtheorem{definition}{Definition}
\newtheorem{proposition}{Proposition}
\newtheorem{assumption}{Assumption}
\newtheorem{lemma}{Lemma}

\newenvironment{proof}[1][Proof]{\begin{trivlist}
\item[\hskip \labelsep {\bfseries #1}]}{\end{trivlist}}




\newcommand{\qed}{\nobreak \ifvmode \relax \else
      \ifdim\lastskip<1.5em \hskip-\lastskip
      \hskip1.5em plus0em minus0.5em \fi \nobreak
      \vrule height0.75em width0.5em depth0.25em\fi}



%%%%%%%%%%%%%%%%%%%
%%  Additional packages
%%%%%%%%%%%%%%%%%%%

%\usepackage{subcaption}

\usepackage{amssymb}

\usepackage{multicol}

\usepackage{bbm}

% chinese
%\usepackage{ctex}
%\setCJKmainfont[BoldFont=FandolSong-Bold.otf,ItalicFont=FandolKai-Regular.otf]{FandolSong-Regular.otf}
%\setCJKsansfont[BoldFont=FandolHei-Bold.otf]{FandolHei-Regular.otf}
%\setCJKmonofont{FandolFang-Regular.otf}

%  --> needs XeLateX ? then needs to comment pdflatex options in class definition.

\usepackage{mdframed}

%%%

\renewcommand{\PrelimText}{%
  \footnotesize[\,\today\ at \thistime\ -- \texttt{Thesis}~\myVersion\,]}


%%%%%%%%
% bilingual version
\usepackage{xparse}
\usepackage{ifthen}


% biling paragraph

\newcommand{\bpar}[2]{
    \ifthenelse{\thelanguage=0}{#1}{}
    \ifthenelse{\thelanguage=1}{#2}{}
}


% Sectioning commands
% from http://tex.stackexchange.com/questions/109557/renewdefine-section-with-additional-argument


%\let\oldchapter\chapter
%
%\RenewDocumentCommand\chapter {s o o m m}
%   {
%    \IfBooleanTF{#1}                   
%        {
%        \ifthenelse{\thelanguage=0}{\oldchapter*{#4}}{}
%        \ifthenelse{\thelanguage=1}{\oldchapter*{#5}}{}
%          \IfNoValueF{#2}            % if TOC arg is given create a TOC entry
%            {          
%              \ifthenelse{\thelanguage=0}{\addcontentsline{toc}{chapter}{#2}}{}
%              \ifthenelse{\thelanguage=1}{\addcontentsline{toc}{chapter}{#3}}{}
%            }
%        }  
%      {                              % no star given 
%        \IfNoValueTF{#2}
%          {
%            \ifthenelse{\thelanguage=0}{\oldchapter{#4}}{}
%            \ifthenelse{\thelanguage=1}{\oldchapter{#5}}{}
%           }       % no TOC arg
%          { 
%            \ifthenelse{\thelanguage=0}{\oldchapter[#2]{#4}}{}
%            \ifthenelse{\thelanguage=1}{\oldchapter[#3]{#5}}{}
%           }
%      }   
%  }
%
%



\let\oldsection\section

\RenewDocumentCommand\section {s o o m m}
   {
    \IfBooleanTF{#1}                   
        {
        \ifthenelse{\thelanguage=0}{\oldsection*{#4}}{}
        \ifthenelse{\thelanguage=1}{\oldsection*{#5}}{}
          \IfNoValueF{#2}            % if TOC arg is given create a TOC entry
            {          
              \ifthenelse{\thelanguage=0}{\addcontentsline{toc}{section}{#2}}{}
              \ifthenelse{\thelanguage=1}{\addcontentsline{toc}{section}{#3}}{}
            }
        }  
      {                              % no star given 
        \IfNoValueTF{#2}
          {
            \ifthenelse{\thelanguage=0}{\oldsection{#4}}{}
            \ifthenelse{\thelanguage=1}{\oldsection{#5}}{}
           }       % no TOC arg
          { 
            \ifthenelse{\thelanguage=0}{\oldsection[#2]{#4}}{}
            \ifthenelse{\thelanguage=1}{\oldsection[#3]{#5}}{}
           }
      }   
  }

\let\oldsubsection\subsection

\RenewDocumentCommand\subsection {s o o m m}
   {
    \IfBooleanTF{#1}                   
        {
        \ifthenelse{\thelanguage=0}{\oldsubsection*{#4}}{}
        \ifthenelse{\thelanguage=1}{\oldsubsection*{#5}}{}
          \IfNoValueF{#2}            % if TOC arg is given create a TOC entry
             {          
              \ifthenelse{\thelanguage=0}{\addcontentsline{toc}{subsection}{#2}}{}
              \ifthenelse{\thelanguage=1}{\addcontentsline{toc}{subsection}{#3}}{}
            }
        }  
      {                              % no star given 
        \IfNoValueTF{#2}
          {
           \ifthenelse{\thelanguage=0}{\oldsubsection{#4}}{}
           \ifthenelse{\thelanguage=1}{\oldsubsection{#5}}{}
           }       % no TOC arg
          { 
           \ifthenelse{\thelanguage=0}{\oldsubsection[#2]{#4}}{}
           \ifthenelse{\thelanguage=1}{\oldsubsection[#3]{#5}}{}
           }
      }   
  }


\let\oldsubsubsection\subsubsection

\RenewDocumentCommand\subsubsection {s o o m m}
   {
    \IfBooleanTF{#1}   
        {
        \ifthenelse{\thelanguage=0}{\oldsubsubsection*{#4}}{}
        \ifthenelse{\thelanguage=1}{\oldsubsubsection*{#5}}{}
          \IfNoValueF{#2}            % if TOC arg is given create a TOC entry
             {          
              \ifthenelse{\thelanguage=0}{\addcontentsline{toc}{subsubsection}{#2}}{}
              \ifthenelse{\thelanguage=1}{\addcontentsline{toc}{subsubsection}{#3}}{}
            }
        }  
      {                              % no star given 
        \IfNoValueTF{#2}
          {
           \ifthenelse{\thelanguage=0}{\oldsubsubsection{#4}}{}
           \ifthenelse{\thelanguage=1}{\oldsubsubsection{#5}}{}
           }       % no TOC arg
          { 
           \ifthenelse{\thelanguage=0}{\oldsubsubsection[#2]{#4}}{}
           \ifthenelse{\thelanguage=1}{\oldsubsubsection[#3]{#5}}{}
           }
      }   
  }
  
  
  
\let\oldparagraph\paragraph

\RenewDocumentCommand\paragraph {s m m}
   {
    \IfBooleanTF{#1}
        {
        \ifthenelse{\thelanguage=0}{\oldparagraph*{#2}}{}
        \ifthenelse{\thelanguage=1}{\oldparagraph*{#3}}{}
        }  
      { 
           \ifthenelse{\thelanguage=0}{\oldparagraph{#2}}{}
           \ifthenelse{\thelanguage=1}{\oldparagraph{#3}}{}   
      }   
  }


\let\oldcaption\caption

\RenewDocumentCommand\caption {o o m m}
   {
        \IfNoValueTF{#1}
          {
           \ifthenelse{\thelanguage=0}{\oldcaption{#3}}{}
           \ifthenelse{\thelanguage=1}{\oldcaption{#4}}{}
           }       % no TOC arg
          { 
           \ifthenelse{\thelanguage=0}{\oldcaption[#1]{#3}}{}
           \ifthenelse{\thelanguage=1}{\oldcaption[#2]{#4}}{}
           }
      } 
      
      
        
%%%%%%%%%%
%  Citation

\let\oldcite\cite
\renewcommand{\cite}[1]{[\oldcite{#1}]}




%%%%%%%%%%
%  Drafting

% writing utilities

% comments	 and responses
%  -> use this comment to ask questions on what other wrote/answer questions with optional arguments (up to 4 answers)


\DeclareDocumentCommand{\comment}{m o o o o}
{\ifthenelse{\draft=1}{
    \textcolor{red}{\textbf{C : }#1}
    \IfValueT{#2}{\textcolor{blue}{\textbf{A1 : }#2}}
    \IfValueT{#3}{\textcolor{ForestGreen}{\textbf{A2 : }#3}}
    \IfValueT{#4}{\textcolor{red!50!blue}{\textbf{A3 : }#4}}
    \IfValueT{#5}{\textcolor{Aquamarine}{\textbf{A4 : }#5}}
 }{}
}


% todo
\newcommand{\todo}[1]{
    \ifthenelse{\draft=1}{\textcolor{red!50!blue}{\textbf{TODO : \textit{#1}}}}{}
}



% provisory part, removed if not draft


\newcommand{\provisory}[1]{
    \ifthenelse{\draft=1}{
    \color{blue} \textbf{\textit{PROVISORY}} #1 \color{black}
    }{}
}

%\newenvironment{provisory}{\par\color{blue}}{\par}














\title{Vers des Modèles Couplant Développement Urbain et Croissance des Réseaux de Transport\bigskip\\
\textit{Plan de Thèse}
}
\author{\noun{Juste Raimbault}}
%\date{Novembre 2016}


\maketitle

\justify


\begin{abstract}
\end{abstract}


\section*{Echéancier}

\paragraph{1 juillet 2017}

\begin{itemize}
\item Introduction, 1.1, 2.1 et 2.2 : complets
\item 2.3 : Méthodologie, résultats partiels
\item Chapitre 3 complet
\item Chapitre 5 complet
\item 5.1 : état courant du papier \cite{antelope2016interdisciplinary} (études d'épistmo quanti supplémentaires en cours)
\item 6.2 et 6.3 complets
\item 4.1 : Méthodologie, résultats partiels
\item 4.2 et 4.3 complets
\item Chapitre 9
\end{itemize}



\paragraph{1 aout 2017}

\begin{itemize}
\item 1.2 complet
\item Chapitre 2 complet (sauf interviews ; reste 2.3)
\item Chapitre 5 complet (reste 5.1)
\item Chapitre 6 complet (reste 6.1)
\item 8.3 complet (Lutecia)
\end{itemize}

\paragraph{1 septembre 2017 (fin terrain Chine)}

\begin{itemize}
\item Chapitre 1 complet (reste 1.3 : terrain qualitatif)
\item Chapitre 7 complet
\item Chapitre 8 complet (reste 8.1 et 8.2)
\item Chapitre 9 complet (finalisation)
\item Conclusion provisoire
\end{itemize}


\paragraph{Du 1er septembre au 15 décembre 2017}

\textit{Ajustements, Complément 2.3 par interviews, corrections, annexes, traduction.}


%\newpage

\vspace{2cm}

\hrule

\vspace{0.5cm}

{\hfill
\Huge \textbf{Plan}\hfill
}

\vspace{1.5cm}

\part*{Introduction}

\textit{Introduction du sujet par exemples concrets ; cadre scientifique ; interdisciplinarité ; sciences des systèmes complexes ; complexité en géographie}

%%%%%%%%%%%%%%%%%%%%%%%%%%
\part{Fondations}
%%%%%%%%%%%%%%%%%%%%%%%%%%


%%%%%%%%%%%%%%%%%%%%%%%%%%
\section{Interactions entre Réseaux et Territoires}

\subsection{Réseaux et Territoires}

\textit{Revue de Littérature thématique ; construction de la question de recherche : introduction progressive de la problématique de co-évolution, précision des objets (réseaux et territoires)}


\subsection{Etude de cas}

\textit{Collection de situations concrètes d'interactions entre réseaux et territoires ; les cas du Grand Paris et du Delta de la Rivière des Perles} \todo{rassembler et synthétiser fiches de lecture}


\subsection{Recherche Qualitative}

\textit{Une expérience en observation flottante : les transports en région parisienne et dans le Delta de la Rivière des Perles} \todo{finir terrains ; écrire compte-rendus/interprétation}



%%%%%%%%%%%%%%%%%%%%%%%%%%
\section{Modéliser les Interactions entre Réseaux et Territoires}

% chapitre état de l'art

\subsection{Etat de l'art}

\textit{Modéliser en Géographie Théorique et Quantitative ; revue inter-disciplinaire des modèles de croissance urbaine et de réseau}


\subsection{Une Approche Epistémologique}

\paragraph{Interviews} 

\textit{Analyse d'interviews d'acteurs académiques sur la question des modèles de co-évolution}


\paragraph{Revue systématique algorithmique}

\textit{Etude algorithmique du paysage scientifique sur les interactions entre réseaux et territoires~\cite{raimbault2015models} : des domaines très cloisonnés}


\paragraph{Bibliométrie indirecte par hyperréseau}


\textit{Raffinement de l'étude précédente par couplage du réseau de citation au réseau sémantique : méthode présentée dans~\cite{raimbault2016indirect}} \todo{Soumettre papier Cybergeo (Scientometrics) ; appliquer à corpus réseau-territoire ; traduire l'article en remplaçant les résultats. ETA 1w}




\subsection{Modélographie}

\textit{``Classification'' systématique des modèles existants : processus, échelles, cas d'application (restant à un niveau meta pour les types de modèles pour lesquels on se base déjà sur une revue, par exemple LUTI)}



%%%%%%%%%%%%%%%%%%%%%%%%%%
\section{Positionnements}


\subsection{Reproductibilité}

\textit{Etudes de cas sur la reproductibilité ; illustration concrète et leçons générales}

\subsection{Données massives, computation et exploration des modèles}

\textit{Pour un usage précautionneux des données massives et de la computation : rationnelle de~\cite{raimbault2016cautious}}

\textit{Pour une connaissance plus fine et systématique du comportement des modèles : utilisation de données synthétiques pour un contrôle sur les conditions initiales (projet Space Matters \cite{cottineau2015revisiting})}



\subsection{Positionnement épistémologique}

\textit{Pour une science anarchiste (Feyerabend) ; compatibilité avec le Perspectivisme de Giere et pourquoi celui-ci est particulièrement adapté aux paradigmes de la complexité ; multiplicité des lectures de la thèse (voir annexe réflexivité, au delà d'une lecture linéaire) $\rightarrow$ presentation JIG et papier CSDM 2017}














%%%%%%%%%%%%%%%%%%%%%%%%%%
\part{Matériaux}
%%%%%%%%%%%%%%%%%%%%%%%%%%

\comment{Cette partie est la plus délicate dans son organisation et articulation ; bien expliquer les liens et le cheminement - éventuellement déjà introduire de la reflexivité avec un diagramme d'interactions des domaines de connaissance.}


\section{Théorie Evolutive Urbaine}

{\color{blue}Premières preuves d'existence des interactions et de leur forme, ainsi que des processus concernés.}

\subsection{Correlations entre Forme Urbaine et Forme de Réseau}

\textit{Les correlations spatiales entre indicateurs de forme urbaine et de forme de réseau révèlent la non-stationnarité des interactions, qui peut être reliée à la non-ergodicité sous certaines hypothèses~\cite{raimbault2016cautious}}


\subsection{Causalités spatio-temporelles}


\textit{Exploration synthétique des régimes de causalité du modèle RBD~\cite{raimbault2014hybrid} par la méthode de granger étendue - application de la méthode aux données sud-africaines (presentation ECTQG)}


\subsection{Effets de Réseaux révélés par un modèle de croissance macroscopique}

\textit{Modèle de Gibrat étendu par interactions gravitaires au premier ordre, par retroaction des flux physiques au second ordre~\cite{raimbault2016models}, révèle effets de réseaux par validation du modèle étendu via critère d'Akaike empirique} \todo{finish papier ASAP, à soumettre à EPB}




\section{Interactions à l'Echelle Microscopique}

{\color{blue}Des difficultés sont rencontrées si les échelles et le système ne sont pas proprement choisis}

\subsection{Equilibre Utilisateur Statique}

\textit{Investigation de l'existence empirique de l'Equilibre Utilisateur Statique~\cite{raimbault2016investigating}}


\subsection{Transport Routier et déterminants des coûts}

\textit{Paper energy price : justify the presence of a hidden network. Unveils again non-stationarity, and modular structure of territorial systems}



\subsection{Transactions immobilières et Grand Paris}

\textit{Recherche de correlations et/ou causalités entre transactions immobilières (base BIEN) et tracé du réseau du métro du Grand Paris : Papier Sageo}

\comment{Problème : méthodo présentée en 6.2 - ok si lecture non linéaire : illustre faible intérêt d'une présentation linéaire (justifier la lecture par échelles ici)}





\section{Morphogenèse Urbaine}

{\color{blue}Une entrée de modélisation alternative par la morphogenèse, construction progressive de modèles.}

\subsection{Une approche interdisciplinaire de la Morphogenèse}

\textit{Construction épistémologique d'une définition unifiée de la morphogenèse~\cite{antelope2016interdisciplinary}}


\subsection{Morphogenèse Urbaine par Aggregation-Diffusion}

\textit{Modèle de croissance urbaine par processus d'aggregation diffusion, reproduit de manière fine l'ensemble des morphologies urbaines existantes en Europe} \todo{Article PlosOne à finaliser}


\subsection{Génération de systèmes corrélés}

\textit{Couplage faible du modèle précédent à une heuristique de génération de réseau, permet de générer des système couplés à la correlation contrôlée~\cite{raimbault2016generation}}





%%%%%%%%%%%%%%%%%%%%%%%%%%
\part{Co-évolution}
%%%%%%%%%%%%%%%%%%%%%%%%%%

\comment{partie agencée par degré de complexité des modèles.}

%%%%%%%%%%%%%%%%%%%%%%%%%%
\section{Co-evolution à l'échelle macroscopique}

\subsection{Exploration de SimpopNet}

\textit{Exploration systématique de SimpopNet : quelle connaissances supplémentaires tire-t-on ?}

\subsection{Extension du modèle d'interaction}


\textit{Extension co-évolutive du gibrat-interaction $\rightarrow$ Conférence medium mi-juin}

\todo{valider et tester sur réseau de train (base Thévenin) et réseau d'autoroutes (base à créer)}

\subsection{Modèle SimpopSino}

\textit{Adaptation du modèle pour le système de ville Chinois}





%%%%%%%%%%%%%%%%%%%%%%%%%%
\section{Co-evolution à l'échelle mesoscopique}

\subsection{Co-evolution des formes}

\paragraph{Modèle de Morphogenèse Urbaine}

\comment{Formulation très générique du modèle ?}

\textit{Couplage faible de~\cite{raimbault2016generation} rendu fort, induisant un modèle de morphogenèse incluant la co-évolution ; calibration du modèle $\rightarrow$ presentation @CCS17}

\textit{Régimes de causalité, calibration au second ordre (correlations)}

\subsection{Comparaison des heuristiques de réseau}

\textit{Comparaison de différentes heuristiques de réseau couplées au modèle précédent (par exemple génération de réseau biologique \cite{raimbault2015labex}}



\subsection{Lutetia : un modèle de co-évolution incluant la gouvernance des systèmes de transports}

\paragraph{Modèle}

\textit{Modèle de co-évolution sur le temps long, couplant un LUTI à un module de gouvernance des transports basé sur la théorie des jeux, pour le développement du réseau~\cite{le2015modeling}}

\paragraph{Application au Delta de la Rivière des Perles}

\textit{Application de Lutecia au cas réel de la Mega-city Region du PRD.}

\todo{Calibration et validation du modèle sur le Delta de la Rivière des Perles : objectif Article Transport Geography début mai}




%%%%%%%%%%%%%%%%%%%%%%%%%%
\section{Ouverture}



%%%%%%%%%%%%%%%%%%%%%%%%%%
% Constructions Théoriques 

\textit{Constructions théoriques successives, avec un niveau meta progressif}

\subsection{Une Théorie des Systèmes Territoriaux Co-évolutifs en Réseau}

\textit{Développement de la théorie géographique co-construite avec les autres domaines de la thèse, qui couple l'entrée morphogenétique avec la théorie évolutive des villes~\cite{raimbault:halshs-01422484}}


\subsection{Une Théorie abstraite pour modéliser les systèmes socio-techniques}

\textit{Méta-théorie pour formaliser des perspectives de modélisations multiples sur les systèmes socio-techniques} \todo{reste à développer action des modèles sur les données, y associer une structure d'action de monoïde}



\subsection{Un cadre de connaissances pour une géographie intégrée}

\textit{Précision du cadre de connaissances, basé sur les fondements épistémologiques introduits en 3.3. Mise en perspective de la connaissance produite par la thèse comme illustration de la co-évolution des connaissances en Géographie Théorique et Quantitative~\cite{raimbault2017theo}}

\comment{here integrate as chapter conclusion reflexion on reflexivity, types of complexities etc. ? (presentation discutant colloque Geodivercity)}


%%%%%%%%%%%%%%%%%%%%%%%%%%
\part*{Conclusion}
%%%%%%%%%%%%%%%%%%%%%%%%%%


\section*{Perspectives}

\paragraph{Développements Spécifiques}

\textit{Projets de recherche détaillés issus de divers développements (par exemple communication scientifique 
\cite{serra2016game} ; épistémologie quantitative~\cite{raimbault2016techno} ; science ouverte
\cite{cybergeo20})}

\comment{remarque : proposition de cours de modélisation, peut être évoqué ici.}

\paragraph{Vers un Programme de Recherche}

\textit{Synthèse des axes de recherche révélés tout au long de la thèse, proposition d'un programme de longue durée pour l'étude des systèmes territoriaux complexes}



\section*{Conclusion générale}



\part*{Annexes}



% not necessary
%\section{Une approche interdisciplinaire de la morphogenèse}

%\textit{texte complet de~\cite{antelope2016interdisciplinary}}




\section{Supplementary Information}


\subsection{Dérivations}

\textit{Dérivations Analytiques pour diverses parties de la thèse}


\subsection{Exploration des Modèles}

\textit{Explorations raffinées pour certains modèles ; applications compagnon d'exploration interactive}


\section{Développements Méthodologiques}

\subsection{A Unified Framework for Models of Urban Growth}

\textit{The various model we will develop could enter a unified framework ; derivation of the link between Gibrat and Simon models}


\subsection{Sensitivity of Urban Scaling to City Definition}

\textit{Analytical validation of the sensitivity of scaling exponents to city definition in a simple model or urban form}

\subsection{Quantifying Robustness through Discrepancy}

\textit{Complex systems are by nature multi-objective : in the particular case of multi-attribute evaluations, we introduce a framework to quantify robustness independently of the model, based on data discrepancy~\cite{raimbault2016discrepancy}}


%\subsection{Spatio-temporal correlations and causalities}

%\textit{Linking spatial and temporal correlations of geographical indicators in simple cases ; a granger-causality method to identify spatio-temporal causalities}
% --> integrated in paper Sageo



\section{Développements Thématiques}

\textit{Laïus introductif : approche unifiée des Systèmes Complexes, positionner chaque développement dans une vision synthétique globale.} \comment{maybe in conclusion / opening ?}

\subsection{Données synthétiques}

\textit{Développement de~\cite{raimbault2016generation} dans le champ de la Finance Quantitative}


\subsection{Epistémologie Quantitative}

\textit{CybergeoNetworks : détails méthodologiques, résultats sur Cybergeo. Résultats sur les Brevets.}

\subsection{Système de Transport en partage}

\textit{\cite{raimbault2015hybrid} montre l'hétérogénéité et la complexité des interactions à l'échelle microscopique}







\section{Reflexivité}


\textit{Application des outils d'épistémologie quantitative à la thèse elle-même ; statistiques détaillées des différents projets ; graphe des concepts et parties de la thèse (application compagnon ?) et proposition de pistes alternatives de lecture}


\section{Bases de Données}


\textit{Description des bases crées dans le cadre de la thèse : réseau routier simplifié prou l'Europe ; traffic routier en Ile de France ; Données VLib sur 3ans ; Autoroutes dynamiques} \todo{pour la base topologique OSM, data paper (Scientific Data) ; pour la base VLib, data paper Cybergeo Data Papers ?}


\section{Logiciels et Packages}

\textit{Packages réutilisables développés dans le cadre de la thèse : largeNetworkR ; Scientific Corpus Mining}


\section{Architecture et Source}

\textit{Architecture et Source des modèles de simulation et d'analyse de données}


\section{Productivité}

\textit{Outils ouverts pour une productivité scientifique améliorée}




%%%%
% n'a pas sa place dans une thèse -> ?
%  -- pas forcément, distiller pour les transitions etc --


%\section{Science et Art}

%\textit{Oeuvres d'art sérendipiteuses (composition graphique et poésie) produites dans le cadre de la thèse}


%\newpage

%%%%%%%%%%%%%%%%%%%%
%% Biblio
%%%%%%%%%%%%%%%%%%%%

\bibliographystyle{apalike}
\bibliography{biblio}


\end{document}
