%% Commands

\newcommand{\noun}[1]{\textsc{#1}}

% command fort head of chapter citation
\newcommand{\headercit}[3]{
\begin{multicols}{2}
\phantom{}
\columnbreak
\textit{#1}

 - \noun{#2}~#3
\end{multicols}
}



%% Math

% Operators
\DeclareMathOperator{\Cov}{Cov}
\DeclareMathOperator{\Var}{Var}
\DeclareMathOperator{\E}{\mathbb{E}}
\DeclareMathOperator{\Proba}{\mathbb{P}}

\newcommand{\Covb}[2]{\ensuremath{\Cov\!\left[#1,#2\right]}}
\newcommand{\Eb}[1]{\ensuremath{\E\!\left[#1\right]}}
\newcommand{\Pb}[1]{\ensuremath{\Proba\!\left[#1\right]}}
\newcommand{\Varb}[1]{\ensuremath{\Var\!\left[#1\right]}}

% norm
\newcommand{\norm}[1]{\| #1 \|}

% independent
\newcommand{\indep}{\rotatebox[origin=c]{90}{$\models$}}


% amsthm environments
\newtheorem{definition}{Definition}
\newtheorem{proposition}{Proposition}
\newtheorem{assumption}{Assumption}
\newtheorem{lemma}{Lemma}

\newenvironment{proof}[1][Proof]{\begin{trivlist}
\item[\hskip \labelsep {\bfseries #1}]}{\end{trivlist}}




\newcommand{\qed}{\nobreak \ifvmode \relax \else
      \ifdim\lastskip<1.5em \hskip-\lastskip
      \hskip1.5em plus0em minus0.5em \fi \nobreak
      \vrule height0.75em width0.5em depth0.25em\fi}



%%%%%%%%%%%%%%%%%%%
%%  Additional packages
%%%%%%%%%%%%%%%%%%%

%\usepackage{subcaption}

\usepackage{amssymb}

\usepackage{multicol}

\usepackage{bbm}

% chinese
%\usepackage{ctex}
%\setCJKmainfont[BoldFont=FandolSong-Bold.otf,ItalicFont=FandolKai-Regular.otf]{FandolSong-Regular.otf}
%\setCJKsansfont[BoldFont=FandolHei-Bold.otf]{FandolHei-Regular.otf}
%\setCJKmonofont{FandolFang-Regular.otf}

%  --> needs XeLateX ? then needs to comment pdflatex options in class definition.

\usepackage{mdframed}

%%%

\renewcommand{\PrelimText}{%
  \footnotesize[\,\today\ at \thistime\ -- \texttt{Thesis}~\myVersion\,]}


%%%%%%%%
% bilingual version
\usepackage{xparse}
\usepackage{ifthen}


% biling paragraph

\newcommand{\bpar}[2]{
    \ifthenelse{\thelanguage=0}{#1}{}
    \ifthenelse{\thelanguage=1}{#2}{}
}


% Sectioning commands
% from http://tex.stackexchange.com/questions/109557/renewdefine-section-with-additional-argument


%\let\oldchapter\chapter
%
%\RenewDocumentCommand\chapter {s o o m m}
%   {
%    \IfBooleanTF{#1}                   
%        {
%        \ifthenelse{\thelanguage=0}{\oldchapter*{#4}}{}
%        \ifthenelse{\thelanguage=1}{\oldchapter*{#5}}{}
%          \IfNoValueF{#2}            % if TOC arg is given create a TOC entry
%            {          
%              \ifthenelse{\thelanguage=0}{\addcontentsline{toc}{chapter}{#2}}{}
%              \ifthenelse{\thelanguage=1}{\addcontentsline{toc}{chapter}{#3}}{}
%            }
%        }  
%      {                              % no star given 
%        \IfNoValueTF{#2}
%          {
%            \ifthenelse{\thelanguage=0}{\oldchapter{#4}}{}
%            \ifthenelse{\thelanguage=1}{\oldchapter{#5}}{}
%           }       % no TOC arg
%          { 
%            \ifthenelse{\thelanguage=0}{\oldchapter[#2]{#4}}{}
%            \ifthenelse{\thelanguage=1}{\oldchapter[#3]{#5}}{}
%           }
%      }   
%  }
%
%



\let\oldsection\section

\RenewDocumentCommand\section {s o o m m}
   {
    \IfBooleanTF{#1}                   
        {
        \ifthenelse{\thelanguage=0}{\oldsection*{#4}}{}
        \ifthenelse{\thelanguage=1}{\oldsection*{#5}}{}
          \IfNoValueF{#2}            % if TOC arg is given create a TOC entry
            {          
              \ifthenelse{\thelanguage=0}{\addcontentsline{toc}{section}{#2}}{}
              \ifthenelse{\thelanguage=1}{\addcontentsline{toc}{section}{#3}}{}
            }
        }  
      {                              % no star given 
        \IfNoValueTF{#2}
          {
            \ifthenelse{\thelanguage=0}{\oldsection{#4}}{}
            \ifthenelse{\thelanguage=1}{\oldsection{#5}}{}
           }       % no TOC arg
          { 
            \ifthenelse{\thelanguage=0}{\oldsection[#2]{#4}}{}
            \ifthenelse{\thelanguage=1}{\oldsection[#3]{#5}}{}
           }
      }   
  }

\let\oldsubsection\subsection

\RenewDocumentCommand\subsection {s o o m m}
   {
    \IfBooleanTF{#1}                   
        {
        \ifthenelse{\thelanguage=0}{\oldsubsection*{#4}}{}
        \ifthenelse{\thelanguage=1}{\oldsubsection*{#5}}{}
          \IfNoValueF{#2}            % if TOC arg is given create a TOC entry
             {          
              \ifthenelse{\thelanguage=0}{\addcontentsline{toc}{subsection}{#2}}{}
              \ifthenelse{\thelanguage=1}{\addcontentsline{toc}{subsection}{#3}}{}
            }
        }  
      {                              % no star given 
        \IfNoValueTF{#2}
          {
           \ifthenelse{\thelanguage=0}{\oldsubsection{#4}}{}
           \ifthenelse{\thelanguage=1}{\oldsubsection{#5}}{}
           }       % no TOC arg
          { 
           \ifthenelse{\thelanguage=0}{\oldsubsection[#2]{#4}}{}
           \ifthenelse{\thelanguage=1}{\oldsubsection[#3]{#5}}{}
           }
      }   
  }


\let\oldsubsubsection\subsubsection

\RenewDocumentCommand\subsubsection {s o o m m}
   {
    \IfBooleanTF{#1}   
        {
        \ifthenelse{\thelanguage=0}{\oldsubsubsection*{#4}}{}
        \ifthenelse{\thelanguage=1}{\oldsubsubsection*{#5}}{}
          \IfNoValueF{#2}            % if TOC arg is given create a TOC entry
             {          
              \ifthenelse{\thelanguage=0}{\addcontentsline{toc}{subsubsection}{#2}}{}
              \ifthenelse{\thelanguage=1}{\addcontentsline{toc}{subsubsection}{#3}}{}
            }
        }  
      {                              % no star given 
        \IfNoValueTF{#2}
          {
           \ifthenelse{\thelanguage=0}{\oldsubsubsection{#4}}{}
           \ifthenelse{\thelanguage=1}{\oldsubsubsection{#5}}{}
           }       % no TOC arg
          { 
           \ifthenelse{\thelanguage=0}{\oldsubsubsection[#2]{#4}}{}
           \ifthenelse{\thelanguage=1}{\oldsubsubsection[#3]{#5}}{}
           }
      }   
  }
  
  
  
\let\oldparagraph\paragraph

\RenewDocumentCommand\paragraph {s m m}
   {
    \IfBooleanTF{#1}
        {
        \ifthenelse{\thelanguage=0}{\oldparagraph*{#2}}{}
        \ifthenelse{\thelanguage=1}{\oldparagraph*{#3}}{}
        }  
      { 
           \ifthenelse{\thelanguage=0}{\oldparagraph{#2}}{}
           \ifthenelse{\thelanguage=1}{\oldparagraph{#3}}{}   
      }   
  }


\let\oldcaption\caption

\RenewDocumentCommand\caption {o o m m}
   {
        \IfNoValueTF{#1}
          {
           \ifthenelse{\thelanguage=0}{\oldcaption{#3}}{}
           \ifthenelse{\thelanguage=1}{\oldcaption{#4}}{}
           }       % no TOC arg
          { 
           \ifthenelse{\thelanguage=0}{\oldcaption[#1]{#3}}{}
           \ifthenelse{\thelanguage=1}{\oldcaption[#2]{#4}}{}
           }
      } 
      
      
        
%%%%%%%%%%
%  Citation

\let\oldcite\cite
\renewcommand{\cite}[1]{[\oldcite{#1}]}




%%%%%%%%%%
%  Drafting

% writing utilities

% comments	 and responses
%  -> use this comment to ask questions on what other wrote/answer questions with optional arguments (up to 4 answers)


\DeclareDocumentCommand{\comment}{m o o o o}
{\ifthenelse{\draft=1}{
    \textcolor{red}{\textbf{C : }#1}
    \IfValueT{#2}{\textcolor{blue}{\textbf{A1 : }#2}}
    \IfValueT{#3}{\textcolor{ForestGreen}{\textbf{A2 : }#3}}
    \IfValueT{#4}{\textcolor{red!50!blue}{\textbf{A3 : }#4}}
    \IfValueT{#5}{\textcolor{Aquamarine}{\textbf{A4 : }#5}}
 }{}
}


% todo
\newcommand{\todo}[1]{
    \ifthenelse{\draft=1}{\textcolor{red!50!blue}{\textbf{TODO : \textit{#1}}}}{}
}



% provisory part, removed if not draft


\newcommand{\provisory}[1]{
    \ifthenelse{\draft=1}{
    \color{blue} \textbf{\textit{PROVISORY}} #1 \color{black}
    }{}
}

%\newenvironment{provisory}{\par\color{blue}}{\par}














\title{Vers des Modèles Couplant Développement Urbain et Croissance des Réseaux de Transport\bigskip\\
\textit{Proposition de Plan}
}
\author{\noun{Juste Raimbault}}
%\date{Novembre 2016}


\maketitle

\justify


\begin{abstract}
\end{abstract}


\part*{Introduction}

\textit{Introduction du sujet par exemples concrets ; cadre scientifique ; interdisciplinarité ; sciences des systèmes complexes ; complexité en géographie}

%%%%%%%%%%%%%%%%%%%%%%%%%%
\part{Fondations}
%%%%%%%%%%%%%%%%%%%%%%%%%%


%%%%%%%%%%%%%%%%%%%%%%%%%%
\section{Interactions entre Réseaux et Territoires}

\subsection{Réseaux et Territoires}

\textit{Revue de Littérature thématique ; construction de la question de recherche : introduction progressive de la problématique de co-évolution, précision des objets (réseaux et territoires)}

\subsection{Modélisation}

\textit{Modéliser en Géographie Théorique et Quantitative ; revue inter-disciplinaire des modèles de croissance urbaine et de réseau}


\subsection{Cas d'étude}

\textit{Collection de situations concrètes d'interactions entre réseaux et territoires ; les cas du Grand Paris et du Delta de la Rivière des Perles} \todo{rassembler fiches de lecture} \todo{rassembler et synthétiser fiches de lecture}


\subsection{Recherche Qualitative}

\textit{Une expérience en observation flottante : les transports en région parisienne et dans le Delta de la Rivière des Perles} \todo{finir terrains ; écrire compte-rendus/interprétation}

\subsection{Synthèse des connaissances}

\textit{Mise en perspective de la connaissance produite par la thèse (faisant office d'annonce de plan) comme illustration de la co-évolution des connaissances en Géographie Théorique et Quantitative~\cite{raimbault2017theo}}




%%%%%%%%%%%%%%%%%%%%%%%%%%
\section{Positionnements}


\subsection{Reproductibilité}

\textit{Etude de cas sur la reproductibilité}

\subsection{Données massives et computation}

\textit{Pour un usage précautionneux des données massives et de la computation : rationnelle de~\cite{raimbault2016cautious}}


\subsection{Positionnement épistémologique}






%%%%%%%%%%%%%%%%%%%%%%%%%%
\section{Méthodologie}









%%%%%%%%%%%%%%%%%%%%%%%%%%
\section{Epistémologie Quantitative}

\subsection{Revue systématique algorithmique}

\textit{Etude algorithmique du paysage scientifique sur les interactions entre réseaux et territoires~\cite{raimbault2015models} : des domaines très cloisonnés}


\subsection{Bibliométrie indirecte par hyperréseau}


\textit{Raffinement de l'étude précédente par couplage du réseau de citation au réseau sémantique : méthode présentée dans~\cite{raimbault2016indirect} ; application au sujet en cours} \todo{Soumettre papier Cybergeo (Scientometrics) ; appliquer à corpus réseau-territoire ; traduire l'article en remplaçant les résultats. ETA 1w}

%%%%%%%%%%%%%%%%%%%%%%%%%%
\part{Matériaux}
%%%%%%%%%%%%%%%%%%%%%%%%%%





%%%%%%%%%%%%%%%%%%%%%%%%%%
\part{Synthèse}
%%%%%%%%%%%%%%%%%%%%%%%%%%




%%%%%%%%%%%%%%%%%%%%%%%%%%
\part{Ouverture}
%%%%%%%%%%%%%%%%%%%%%%%%%%






\part*{Conclusion}

\textit{Conclusion générale}



\part*{Annexes}








%\newpage

%%%%%%%%%%%%%%%%%%%%
%% Biblio
%%%%%%%%%%%%%%%%%%%%

\bibliographystyle{apalike}
\bibliography{biblio}


\end{document}
