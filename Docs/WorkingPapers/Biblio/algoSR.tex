%%%%%%%%%%%%%%%%%%%%%%%%%%%%%
% Standard header for working papers
%
% WPHeader.tex
%
%%%%%%%%%%%%%%%%%%%%%%%%%%%%%

\documentclass[11pt]{article}

%%%%%%%%%%%%%%%%%%%%
%% Include general header where common packages are defined
%%%%%%%%%%%%%%%%%%%%

% general packages without options
\usepackage{amsmath,amssymb,bbm}




%%%%%%%%%%%%%%%%%%%%
%% Idem general commands
%%%%%%%%%%%%%%%%%%%%
%% Commands

\newcommand{\noun}[1]{\textsc{#1}}


%% Math

% Operators
\DeclareMathOperator{\Cov}{Cov}
\DeclareMathOperator{\Var}{Var}
\DeclareMathOperator{\E}{\mathbb{E}}
\DeclareMathOperator{\Proba}{\mathbb{P}}

\newcommand{\Covb}[2]{\ensuremath{\Cov\!\left[#1,#2\right]}}
\newcommand{\Eb}[1]{\ensuremath{\E\!\left[#1\right]}}
\newcommand{\Pb}[1]{\ensuremath{\Proba\!\left[#1\right]}}
\newcommand{\Varb}[1]{\ensuremath{\Var\!\left[#1\right]}}

% norm
\newcommand{\norm}[1]{\| #1 \|}


% amsthm environments
\newtheorem{definition}{Definition}



%% graphics

% renew graphics command for relative path providment only ?
%\renewcommand{\includegraphics[]{}}








% geometry
\usepackage[margin=2cm]{geometry}

% layout : use fancyhdr package
\usepackage{fancyhdr}
\pagestyle{fancy}

\makeatletter

\renewcommand{\headrulewidth}{0.4pt}
\renewcommand{\footrulewidth}{0.4pt}
%\fancyhead[RO,RE]{\textit{Working Paper}}
\fancyhead[RO,RE]{\textit{ECTQG 2015}}
%\fancyhead[LO,LE]{G{\'e}ographie-Cit{\'e}s/LVMT}
\fancyhead[LO,LE]{An Algorithmic Systematic Review}
\fancyfoot[RO,RE] {\thepage}
\fancyfoot[LO,LE] {\noun{J. Raimbault}}
\fancyfoot[CO,CE] {}

\makeatother


%%%%%%%%%%%%%%%%%%%%%
%% Begin doc
%%%%%%%%%%%%%%%%%%%%%

\begin{document}







\title{An Iterative Query Algorithm for Robust Systematic Review\bigskip\\
\textit{Working Paper}
}
\author{\noun{Juste Raimbault}}
\date{February 26th 2015}
%\revised{}


\maketitle

\begin{abstract}
Literature review is a crucial preliminary step for any scientific work and its quality and extent may have a dramatic impact on perspectives for research question and objectives. We propose an algorithm performing an automatized systematic review, to tackle in an original way this issue. Through iterative requests to a catalog and keyword extraction from the retrieven corpus, the final corpus ready for manual screening is built in a more robust way than with a single database request. We describe an implementation of the algorithm and show first results.
\end{abstract}



\section{Introduction}


\section{Description of the Algorithm}



\section{Implementation}

\paragraph{General Implementation}
Because of the heterogeneity of operations required by the algorithm (references organisation, catalog requests, text procesing), it was found a reasonable choice to implement it in Java. Source code and binaries are available on the Github repository of the project\footnote{at \texttt{}}.


\paragraph{Catalog Requests}

\cite{mendeley}


\paragraph{Natural Language Processing}
Keyword extraction is done through Natural Language Processing (NLP) techniques, following the workflow given in~\cite{chavalarias2013phylomemetic}. Although powerful and flexible libraries exist for current operations\footnote{see e.g. Java library by The Stanford Natural Language Processing Group at \texttt{http://nlp.stanford.edu/software/corenlp.shtml}, or Python library NLTK at \texttt{http://www.nltk.org/}.}, the elaborated workflow of the paper would be painful to implement and is furthermore already made available by the authors on the dedicated website of the \emph{CorText} project\footnote{\texttt{http://manager.cortext.net}}.

\section{Results}











%%%%%%%%%%%%%%%%%%%%
%% Biblio
%%%%%%%%%%%%%%%%%%%%

\bibliographystyle{apalike}
\bibliography{/Users/Juste/Documents/ComplexSystems/CityNetwork/Biblio/Bibtex/CityNetwork}


\end{document}
