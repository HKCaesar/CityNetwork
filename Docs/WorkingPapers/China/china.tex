%%%%%%%%%%%%%%%%%%%%%%%%%%%%%
% Standard header for working papers
%
% WPHeader.tex
%
%%%%%%%%%%%%%%%%%%%%%%%%%%%%%

\documentclass[11pt]{article}

%%%%%%%%%%%%%%%%%%%%
%% Include general header where common packages are defined
%%%%%%%%%%%%%%%%%%%%

% general packages without options
\usepackage{amsmath,amssymb,bbm}




%%%%%%%%%%%%%%%%%%%%
%% Idem general commands
%%%%%%%%%%%%%%%%%%%%
%% Commands

\newcommand{\noun}[1]{\textsc{#1}}


%% Math

% Operators
\DeclareMathOperator{\Cov}{Cov}
\DeclareMathOperator{\Var}{Var}
\DeclareMathOperator{\E}{\mathbb{E}}
\DeclareMathOperator{\Proba}{\mathbb{P}}

\newcommand{\Covb}[2]{\ensuremath{\Cov\!\left[#1,#2\right]}}
\newcommand{\Eb}[1]{\ensuremath{\E\!\left[#1\right]}}
\newcommand{\Pb}[1]{\ensuremath{\Proba\!\left[#1\right]}}
\newcommand{\Varb}[1]{\ensuremath{\Var\!\left[#1\right]}}

% norm
\newcommand{\norm}[1]{\| #1 \|}


% amsthm environments
\newtheorem{definition}{Definition}



%% graphics

% renew graphics command for relative path providment only ?
%\renewcommand{\includegraphics[]{}}








% geometry
\usepackage[margin=2cm]{geometry}

% layout : use fancyhdr package
\usepackage{fancyhdr}
\pagestyle{fancy}

\makeatletter

\renewcommand{\headrulewidth}{0.4pt}
\renewcommand{\footrulewidth}{0.4pt}
%\fancyhead[RO,RE]{\textit{Working Paper}}
\fancyhead[RO,RE]{\textit{ECTQG 2015}}
%\fancyhead[LO,LE]{G{\'e}ographie-Cit{\'e}s/LVMT}
\fancyhead[LO,LE]{An Algorithmic Systematic Review}
\fancyfoot[RO,RE] {\thepage}
\fancyfoot[LO,LE] {\noun{J. Raimbault}}
\fancyfoot[CO,CE] {}

\makeatother


%%%%%%%%%%%%%%%%%%%%%
%% Begin doc
%%%%%%%%%%%%%%%%%%%%%

\begin{document}







\title{Interactions between Networks and Territories\\
\textit{China Fieldwork within Medium Project}\\
\textit{Research Directions Proposal}
}
\author{\noun{Juste Raimbault}}
\date{Date}


\maketitle

\justify


\begin{abstract}
This note describe potential research directions for a thesis fieldwork in China in the frame of the Medium project. The focus on literature by the author resides in the fact that most of these papers are part of the thesis and serve as basis or preliminaries for the directions presented here.
\end{abstract}



%%%%%%%%%%%%%%%%%%%
\section{Context}

%%%%%%%%%%%%%%%%%%%
\subsection{Thesis Context}

In the current state of affairs, purposes of the thesis includes (among others) the following highlights (read provisory Memoire~\cite{} for more details) :
% TODO zenodo etc for releases of the thesis ?

\begin{enumerate}
\item Proceed to a quantitative epistemology study 
\item Extract stylized facts on relations between Networks and Territories, focusing on transportation networks, at different spatial and temporal scales, and on different case studies.
\item Propose toy and semi-parametrized models of Urban and Network growth, designed to be either elementary bricks of a larger family of models of co-evolution, or exploratory tools.
\item In co-construction with the previous objectives, construct a geographical theory of co-evolutive networked territorial systems.
\end{enumerate}




%%%%%%%%%%%%%%%%%%%
\subsection{Insertion within Medium Project}

The Medium project features several research directions for some of which a fieldwork focused on themes given before would be of a particular interest, namely :

\begin{itemize}
\item Macroscopic perspective on middle size cities % Denise/Celine part
\item % qualitative ?
\end{itemize}




%%%%%%%%%%%%%%%%%%%
\section{Research Directions}

%%%%%%%%%%%%%%%%%%%
\subsection{Urban Systems Modeling}

In consistence with Medium axis X, a first major project is the study of the Chinese city system within the Evolutive Urban Theory. This mature theory (for specific aspects and details, see e.g. % insert here eg from territorialization biblio
) has already proven powerful to understand cities as complex adaptive systems.



%
% towards a SimpopSino model = specific variation of simpop family to China.
%  Questions : 
%    * is the simpop family a real entity ? we may more do a marius ? an extended Gibrat ?
%    * focus on network/territory : maybe China good for that - new transportation projects etc.
%    * mention/cite SimpopSan here. \cite{raimbault2016simpopsan}
%
%  Tasks : stylized facts from Elfie db / network dbs ?


\paragraph{Stylized facts from Chinese cities dynamical database}


\paragraph{Modeling Chinese cities growth : towards a \textit{SimpopSino} model}


\paragraph{Co-evolution models}



%%%%%%%%%%%%%%%%%%%
\subsection{Metropolitan Governance Modeling}

% Lutecia model : application 

At the metropolitan scale, the question of governance processes for transportation planning is of particular interest as path-dependency is strong at such a scale : bad planning decisions can have disastrous impacts such 

An extension of this model, integrating game-theory-based decision rules 
\cite{lenechet:halshs-01272236}


%%%%%%%%%%%%%%%%%%%
\subsection{Mobility and Inter-modality : a Bike-sharing case study}

% City bikes

At a microscopic scale, relations between Transportation Networks and Territories are partly captured in mobility processes, the study of which complexity provides in consequence useful insights. For example, concerning the properties of the processes, we studied in~\cite{} % TODO cit User Equilibrium paper (will be inserted ?)
 spatio-temporal stationarity properties of traffic flows for Paris region, unveiling strong variabilities with the consequence of a reduced validity range of the Static user Equilibrium framework.






%%%%%%%%%%%%%%%%%%%
\section{Organisation}

% Objectives for each project and estimated timetable - thesis redaction.







%%%%%%%%%%%%%%%%%%%%
%% Biblio
%%%%%%%%%%%%%%%%%%%%

\bibliographystyle{apalike}
\bibliography{/Users/Juste/Documents/ComplexSystems/CityNetwork/Biblio/Bibtex/CityNetwork}


\end{document}

