%%%%%%%%%%%%%%%%%%%%%%%%%%%%%
% Standard header for working papers
%
% WPHeader.tex
%
%%%%%%%%%%%%%%%%%%%%%%%%%%%%%

\documentclass[11pt]{article}

%%%%%%%%%%%%%%%%%%%%
%% Include general header where common packages are defined
%%%%%%%%%%%%%%%%%%%%

% general packages without options
\usepackage{amsmath,amssymb,bbm}




%%%%%%%%%%%%%%%%%%%%
%% Idem general commands
%%%%%%%%%%%%%%%%%%%%
%% Commands

\newcommand{\noun}[1]{\textsc{#1}}


%% Math

% Operators
\DeclareMathOperator{\Cov}{Cov}
\DeclareMathOperator{\Var}{Var}
\DeclareMathOperator{\E}{\mathbb{E}}
\DeclareMathOperator{\Proba}{\mathbb{P}}

\newcommand{\Covb}[2]{\ensuremath{\Cov\!\left[#1,#2\right]}}
\newcommand{\Eb}[1]{\ensuremath{\E\!\left[#1\right]}}
\newcommand{\Pb}[1]{\ensuremath{\Proba\!\left[#1\right]}}
\newcommand{\Varb}[1]{\ensuremath{\Var\!\left[#1\right]}}

% norm
\newcommand{\norm}[1]{\| #1 \|}


% amsthm environments
\newtheorem{definition}{Definition}



%% graphics

% renew graphics command for relative path providment only ?
%\renewcommand{\includegraphics[]{}}








% geometry
\usepackage[margin=2cm]{geometry}

% layout : use fancyhdr package
\usepackage{fancyhdr}
\pagestyle{fancy}

\makeatletter

\renewcommand{\headrulewidth}{0.4pt}
\renewcommand{\footrulewidth}{0.4pt}
%\fancyhead[RO,RE]{\textit{Working Paper}}
\fancyhead[RO,RE]{\textit{ECTQG 2015}}
%\fancyhead[LO,LE]{G{\'e}ographie-Cit{\'e}s/LVMT}
\fancyhead[LO,LE]{An Algorithmic Systematic Review}
\fancyfoot[RO,RE] {\thepage}
\fancyfoot[LO,LE] {\noun{J. Raimbault}}
\fancyfoot[CO,CE] {}

\makeatother


%%%%%%%%%%%%%%%%%%%%%
%% Begin doc
%%%%%%%%%%%%%%%%%%%%%

\begin{document}







\title{Interactions between Networks and Territories\\\medskip
\textit{China Fieldwork within Medium Project}\\
\textit{Research Directions Proposal}
}
\author{\noun{Juste Raimbault}$^{1,2}$\\
$^1$ UMR CNRS 8504 G{\'e}ographie-cit{\'e}s\\
$^2$ UMR-T 9403 IFSTTAR LVMT
}
\date{August 19th 2016}


\maketitle

\justify




\begin{abstract}
This note describe potential research directions for a thesis fieldwork in China in the frame of the Medium project. The focus on literature by the author resides in the fact that most of these papers are part of the thesis and will serve as basis or preliminaries for the directions presented here.
\end{abstract}



%%%%%%%%%%%%%%%%%%%
\section{Context}

%%%%%%%%%%%%%%%%%%%
\subsection{Thesis Context}

Our thesis subject can be roughly summed up as gaining insights into relations between Networks and Territories from an empirical and modeling perspective. In the current state of affairs, purposes of the thesis includes (among others) the following highlights (read provisory Memoire~\cite{raimbault2016memoire} for more details) :

\begin{enumerate}
\item Proceed to a quantitative epistemology study to investigate the different point of views on the question of interactions between networks and territories, from a very broad range of disciplines (see \cite{raimbault2015models}).
\item Extract stylized facts on interactions between Networks and Territories, focusing on transportation networks, at different spatial and temporal scales, and on different case studies.
\item Propose toy and semi-parametrized models of Urban and Network growth, designed to be either elementary bricks of a larger family of models of co-evolution, or exploratory tools.
\item In co-construction with the previous objectives, construct a geographical theory of co-evolutive networked territorial systems.
\end{enumerate}




%%%%%%%%%%%%%%%%%%%
\subsection{Insertion within Medium Project}

The Medium project~\cite{aveline2016medium} features some research subprojects for some of which a fieldwork focused on themes given before would be of a particular interest, namely :

 % qualitative ?
\begin{itemize}
\item Macroscopic socio-economic perspective on three selected middle size cities (axis 2) % Denise/Celine part
\item Big data and Smart Cities (axis 4)
\item Urban Governance (axis 5)
\end{itemize}

The first and last points will be linked to modeling efforts at different scales, whereas the two first will be also studied through empirical analysis. The following section will detail possible projects and clarifies the links between thesis objectives and medium axes.





%%%%%%%%%%%%%%%%%%%
\section{Research Directions}

%%%%%%%%%%%%%%%%%%%
\subsection{Urban Systems Modeling}

In consistence with Medium axis 2, a first major project is the study of the Chinese city system within the Evolutive Urban Theory. This theory introduced in~\cite{pumain1997pour}, proposes in particular to understand system of cities as complex adaptive systems in which the hierarchical propagation of spatio-temporal material and immaterial waves in space implies non-ergodicity~\cite{pumain2012urban} and non-stationarity and explains stylized facts such as scaling laws~\cite{pumain2006evolutionary} or the existence of innovation cycles as cities are drivers of socio-economic change~\cite{pumain2010theorie}. The empirical study of socio-economic trajectories of medium-sized within this framework should help to clarify their role within the Chinese Urban system.



%
% towards a SimpopSino model = specific variation of simpop family to China.
%  Questions : 
%    * is the simpop family a real entity ? we may more do a marius ? an extended Gibrat ?
%    * focus on network/territory : maybe China good for that - new transportation projects etc.
%    * mention/cite SimpopSan here. \cite{raimbault2016simpopsan}
%
%  Tasks : stylized facts from Elfie db / network dbs ?


\paragraph{Stylized facts from Chinese cities dynamical database}

In collaboration with \noun{E. Swerts}, a first step will be to extend the statistical analyses done in~\cite{swerts2013systemes}, building on the new databases constructed in the same work (\textit{forthcoming data paper}). We will in particular focus on the trajectory on middle-size cities, investigating e.g. causal relations and spatio-temporal correlations (as done in~\cite{raimbault2016cautious} with correlations between Network measures and Urban form measures).



\paragraph{Modeling Chinese cities growth : towards a \textit{SimpopSino} model}

The preliminary analysis will guide the construction of a city dynamics model, in the spirit of the series of Simpop models~\cite{pumain2012multi}. The ontological and modeling choices are left open, and we can either go in the direction of a stochastic agent-based model \textit{{\`a}-la-simpop}, or a deterministic system dynamics model as the Gibrat model extensions done by \noun{Favaro} in~\cite{favaro2011gibrat} and more recently by \noun{Cottineau} in~\cite{cottineau2015incremental}, depending on the nature of stylized fact targeted and of the data to fit. For example, in~\cite{raimbault2016simpopsan}, in the frame of a collaborative modeling project, an economic agent-based model was constructed in order to reproduce the evolution of Japanese urban structure under external economic constraints. In \cite{raimbault2016system}, a Gibrat extension including physical network effects is calibrated on French cities. These modeling efforts can be taken as basis for the construction of the core of the \emph{SimpopSino} model. Such a simulation model will be of particular interest for Medium research questions, as it allows testing of hypothesis on dynamics of medium-sized cities that would not be possible otherwise (for example, how will future dynamics be impacted by a given policy impacting the medium-sized city ; how would have the system evolved without its contribution ; etc.).



\paragraph{Co-evolution models}

Part of the fieldwork will be dedicated to the construction and/or research of dynamical transportation network databases. Indeed, the recent quick development in China of new inter and intra-urban transportation networks \cite{kenworthy2002transport,lyu2016developing} suggests the inclusion of dynamical networks in city system models, making China an adequate case for the principal objective of our thesis. Previous versions of model making cities and networks coevolve such as the one studied in~\cite{raimbault2014hybrid} or more recently in~\cite{raimbault2016generation}, as existing works in the field (see e.g. \cite{schmitt2014modelisation}), are still simple and make decoupling assumptions that are not necessarily reasonable on real cases. We will go in the direction of more complex models, extending if possible first version constructed in the previous step.


%%%%%%%%%%%%%%%%%%%
\subsection{Metropolitan Governance Modeling}

% Lutecia model : application 

At the metropolitan scale, the question of governance processes for transportation planning is of particular interest as path-dependency is strong at this scale : bad planning decisions can have disastrous impacts such the examples related by~ \cite{hall1982great}. A toy-model aimed to determine sufficient conditions for the emergence of collaboration between governance stakeholders within a polycentric metropolitan area and for the transition to a \emph{Mega-city-region} (MCR, see~\cite{florida2008rise}) was described by~\cite{lenechet2012} An extension of this model, integrating game-theory-based decision rules was given with preliminary exploration results in~\cite{lenechet:halshs-01272236} (\emph{Lutecia} model). One of the fieldwork destinations, the city of Zhuhai, is contained in the MCR of Shenzen-Hong-Kong which is important in South-east Asia~\cite{swerts2015megacities}. In the spirit of extending and/or applying the model, we propose the following points :
\begin{itemize}
\item Empirical description of governance structure for transportation within the MCR, associated issues.
\item Application of the model to the real case (after refined exploration, sensitivity analysis and internal validation).
\item Focus on specific questions involving the medium-sized city of Zhuhai : position and influence within MCR processes ; specific governance configuration (the city is a Special Economic Zone e.g.).
\end{itemize}

This project will provide insights into Medium axis 5, studying quantitatively structures of governance.


%%%%%%%%%%%%%%%%%%%
\subsection{Mobility and Inter-modality : a Bike-sharing case study}

% City bikes

At a microscopic scale, relations between Transportation Networks and Territories are partly captured in mobility processes, the study of which complexity provides in consequence useful insights. For example, concerning the properties of the processes, we studied in~\cite{2016arXiv160805266R} spatio-temporal stationarity properties of traffic flows for Paris region, unveiling strong variabilities with the consequence of a reduced validity range of the Static user Equilibrium framework. As both visited cities have bike-sharing transportation systems (in Hangzhou being the first implementation of this type of system in China), and previous works elaborated agent-based models to investigate internal mechanisms of use of such a system (see \cite{raimbault2015user} for the first version of the model, and \cite{raimbault2015hybrid} for an extension including discrete choice behavioral model), we propose to collect data on Chinese systems and proceed to a statistical and modeling comparison with Paris system. As a ``smart'' transportation system, associated with the use of information technologies, this subproject will enter the frame of Medium axis 4.



%%%%%%%%%%%%%%%%%%%
\section{Organisation}

% Objectives for each project and estimated timetable - thesis redaction.

%%
\subsection{Dates and Locations}

\begin{enumerate}
\item Zhuhai ; autumn 2016 (25th September-12th December 2016 and 2nd January-20th January 2017). Focus on Governance subproject ; first part of SimpopSino project.
\item Hangzhou ; summer 2017 (June-September 2017). End of city system modeling ; mobility and Bike-sharing.
\end{enumerate}



%%
\subsection{Deliverables}

\begin{itemize}
\item Governance : Journal paper (target Transportation Geography) ; due 1st January 2017.
\item SimpopSino : Journal paper on the modeling part (target PLoS One or Scientific Reports) ; due 1st September 2017.
\item Bike-sharing : Journal paper (target Transportation Research Part A) ; due 1st September 2017.
\item Medium seminars and conference (November 2016 ; June 2017)
\end{itemize}




%%%%%%%%%%%%%%%%%%%%
%% Biblio
%%%%%%%%%%%%%%%%%%%%

\bibliographystyle{apalike}
\bibliography{/Users/Juste/Documents/ComplexSystems/CityNetwork/Biblio/Bibtex/CityNetwork}


\end{document}

