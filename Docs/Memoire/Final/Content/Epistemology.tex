


%----------------------------------------------------------------------------------------


\newpage



\section{Epistemological Positioning}{Positionnement Epistémologique}\label{sec:epistemo-position}

%----------------------------------------------------------------------------------------




\comment[JR]{Pour une science anarchiste (Feyerabend) ; compatibilité avec le Perspectivisme de Giere et pourquoi celui-ci est particulièrement adapté aux paradigmes de la complexité ; multiplicité des lectures de la thèse (voir annexe réflexivité, au delà d'une lecture linéaire)}



\comment[JR]{compatibilité avec Monod sur la majorité des points ; divergences propres aux sciences sociales par rapport à la bio ? - notamment sur la morphogenèse. on en prend une définition ``unifiée'' qui convient bien à nos problématiques.}







%----------------------------------------------------------------------------------------

\subsection{Types of Complexity and Knowledge Production}{Types de Complexité et Production de Connaissances}


% Contenu de la présentation au colloque Geodivercity







