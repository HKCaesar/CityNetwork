




%----------------------------------------------------------------------------------------

\newpage



\section{Big Data, Computation and Model Exploration}{Données Massives, Calcul Intensif et Exploration des Modèles}



%----------------------------------------------------------------------------------------



Nous nous positionnons à présent sur les questions liées à l'utilisation des données massives et du calcul intensif, ce qui induit par extension une réflexion sur les méthodes d'exploration de modèles. Il n'est pas évident que ces nouvelles possibilités soient nécessairement accompagnées de mutations épistémologiques profondes, et nous montrons au contraire que leur utilisation nécessite plus que jamais un dialogue avec la théorie. Implicitement, cette position préfigure le cadre épistémologique pour l'étude des Systèmes Complexes dont nous donnons le contexte à la section suivante~\ref{} et que nous formalisons en ouverture~\ref{}.



\subsection{For a cautious use of big data and computation}{Pour un usage raisonné des données massives et de la computation}

\bpar{
The so-called \emph{big data revolution} resides as much in the availability of large datasets of novel and various types as in the always increasing available computational power. Although the \emph{computational shift} (\cite{arthur2015complexity}) is central for a science aware of complexity and is undeniably the basis of future modeling practices in geography as \cite{banos2013pour} points out, we argue that both \emph{data deluge} and \emph{computational potentialities} are dangerous if not framed into a proper theoretical and formal framework. The first may bias research directions towards available datasets (as e.g. numerous twitter mobility studies) with the risk to disconnect from a theoretical background, whereas the second may overshadow preliminaries analytical resolutions essential for a consistent use of simulations. We argue that the conditions for most of results in this thesis are indeed the ones endangered by incautious big-data enthusiasm, concluding that a main challenge for future Geocomputation is a wise integration of novel practices within the existing body of knowledge.
}{
La soi-disante \emph{révolution des données massives} réside autant dans la disponibilité de grands jeux de données de nouveaux types variés, que dans la puissance de calcul potentielle toujours en augmentation. Même si le \emph{tournant computationnel} (\cite{arthur2015complexity}) est central pour une science consciente de la complexité et est sans douter la base des pratiques de modélisation futures en géographie comme \cite{banos2013pour} souligne, nous soutenons que à la fois le \emph{déluge de données} et les \emph{capacités de calcul} sont dangereuses si non cadrées dans un cadre théorique et formel propre. Le premier peut biaiser les directions de recherche vers les jeux de données disponibles %(comme par exemple les nombreuses étude de mobilité se basant sur twitter) % TODO find less rude examples, and of different types ?
 avec le risque de se déconnecter d'un fond théorique, tandis que le second peut occulter des résolutions analytiques préliminaires essentielles pour un usage cohérent des simulations. Nous avançons que les conditions pour la majorité des résultats dans cette thèse sont en effet ceux mis en danger par un enthousiasme inconsidéré pour les données massives, tirant la conclusion qu'un challenge majeur pour la géocomputation future est une intégration sage des nouvelles pratiques au sein du corpus existant de connaissances.
}


\bpar{
The computational power available seems to follow an exponential trend, as some kind of Moore's law. Both effective Moore's law for hardware, and improvement of softwares and algorithms, combined with a democratization of access to large scale simulation facilities, makes always more and more CPU time available for the social scientist (and to the scientist in general but this shift happened quite before in other fields, as e.g. CERN is leading in cloud computing and grid computation). About 10 years ago, \cite{gleyze2005vulnerabilite} concluded that network analysis, for the case of Parisian public transportation network, was ``limited by computation''. Today most of these analyses would be quickly done on a personal computer with appropriated software and coding: \cite{2015arXiv151201268L} is a witness of such a progress, introducing new indicators with a higher computational complexity, computed on larger networks. The same parallel can be done for the Simpop models: the first Simpop models at the beginning of the millenium~\cite{sanders1997simpop} were ``calibrated'' by hand, whereas \cite{cottineau2015modular} calibrates the multi-modeling Marius model and~\cite{schmitt2014half} calibrates very precisely the SimpopLocal model, both on grid with billions of simulations. A last example, the field of Space Syntax, witnessed a long path and tremendous progresses from its theoretical origins~\cite{hillier1989social} to recent large-scale applications~\cite{hillier2016fourth}.
}{
La puissance de calcul disponible semble suivre un tendance exponentielle, comme une sorte de loi de Moore. Grace à d'une part la loi de Moore effective pour le matériel, d'autre part l'amélioration des logiciels et algorithmes, conjointement avec une démocratisation de l'accès au infrastructures de simulation à grande échelle, permet à toujours plus de temps processeur d'être disponible pour le chercheur en sciences sociales (et pour le scientifique en général, mais cette mutation a déjà été opérée depuis plus longtemps dans d'autres domaines, puisque par exemple le CERN est à la pointe en terme de calcul distant et sur grille). Il y a environ une dizaine d'année, \cite{gleyze2005vulnerabilite} était forcé de conclure que les analyses de réseau, pour les transports publics parisiens, étaient ``limitées par le calcul''. Aujourd'hui la plupart des mêmes analyses seraient rapidement réglée sur un ordinateur personnel avec les logiciels et programmes appropriés : \cite{2015arXiv151201268L} est un témoin d'un tel progrès, introduisant des nouveaux indicateurs avec une plus grande complexité de calcul, qui sont calculés sur des réseaux à grande échelle. Le même parallèle peut être fait pour les modèles Simpop : les premiers modèles Simpop au début du millénaire~\cite{sanders1997simpop} étaient ``calibrés'' à la main, tandis que \cite{cottineau2015modular} calibre le modèle Marius en multi-modélisation et~\cite{schmitt2014half} calibre très précisément le modèle SimpopLocal, chacun sur la grille avec des milliards de simulations. Un dernier exemple, le champ de la \emph{Space Syntax}, a témoigné d'une longue route et de progrès considérables depuis ses origines théoriques~\cite{hillier1989social} jusqu'à ses récentes applications à grande échelle~\cite{hillier2016fourth}.
}



\bpar{
Concerning the new and ``big'' data available, it is clear that always larger dataset are available and always newer type of data are available. Numerous examples of fields of application can be given. For example, mobility can now be studied from various entries, such as new data from smart transportation systems~\cite{o2014mining}, from social networks~\cite{frank2014constructing}, or other more exotic data such as mobile phone data~\cite{de2016death}. In an other spirit, the opening of ``classic'' datasets (such as city dashboards, open data government initiatives) should allow ever more meta-analyses. New ways to do research and produce data are also raising, towards more interactive and crowd-sourced initiatives. For example, \cite{2016arXiv160606162C} describes a web-application aimed at presenting a meta-analysis of Zipf's law across numerous datasets, but in particular features an upload option, where the user can upload its own dataset and add it to the meta-analysis. Other applications allow interactive exploration of scientific literature for a better knowledge of a complex scientific landscape, as~\cite{cybergeo20} does.
}{
Concernant les nouvelles données ``massives'' qui sont disponibles, il est clair que des quantités toujours plus grandes et des types toujours nouveaux sont disponibles. De nombreux exemples de champs d'application peuvent être donnés. La mobilité en est typique, puisque étudiée selon divers points de vue, comme les nouvelles données issues des systèmes de transport intelligents~\cite{o2014mining}, des réseaux sociaux~\cite{frank2014constructing}, ou des données plus exotiques comme des données de téléphonie mobile~\cite{de2016death}. Dans un autre esprit, l'ouverture de jeux de données ``classiques'' (comme les applications synthétiques urbaines, les initiatives gouvernementales pour les données ouvertes) devrait pouvoir toujours plus de méta-analyses. De nouvelles façon de pratiquer la recherche et produire des données sont également en train d'émerger, vers des initiatives plus interactives et venant de l'utilisateur. Ainsi, \cite{2016arXiv160606162C} décrit une application web ayant pour but de présenter une méta-analyse de la loi de Zipf sur de nombreux jeux de données, mais en particulier inclut une option de dépôt, à travers laquelle l'utilisateur peut télécharger sont propre jeu de données et l'inclure dans la méta-analyse. D'autres applications permettent l'exploration interactive de la littérature scientifique pour une meilleure connaissance d'un horizon scientifique complexe, comme~\cite{cybergeo20} fait.
}


\bpar{
As always the picture is naturally not as bright as it seems to be at first sight, and the green grass that we try to go eating in the neighbor's field quickly turns into a sad reality. Indeed, the purpose and motivation are fuzzy and one can get lost. Some examples speaks for themselves. \cite{barthelemy2013self} introduces a new dataset and rather new methods to quantify road network evolution, but the results, on which the authors seem to be astonished, are that a transition occurred in Paris at the Haussmann period. Any historian of urbanism would be puzzled by the exact purpose of the paper, as in the end a vague and bizarre feeling of reinventing the wheel floats in the air. The use of computation can also be exaggerated, and in the case of agent-based modeling it can be illustrated by the example of~\cite{axtell2016120}, for which the aim at simulating the system at scale 1:1 seems to be far from initial motivations and justifications for agent-based modeling, and may even give arguments to mainstream economists that denigrate easily ABMS. Other anecdotes raise worries: \cite{robin_cura_2014_11415} is a web application that wastes computational ressources to simulate Gaussian distributions for a Gibrat model in order to compute their mean and variance, that are input parameters of the model. It basically checks the Central Limit Theorem, which is a priori well accepted among most scientists. Otherwise, the full distribution given by a Gibrat model is theoretically known as it was fully solved e.g. by \cite{gabaix1999zipf}. Recently on the French speaking diffusion list \emph{Geotamtam}, a sudden rush around \emph{Pokemon Go} data seemed to answer more to an urgent unexplained need to exploit this new data source before anyone else rather than an elaborated theoretical construction. Simple existing accurate datasets, such as historical cities population (for France the Pumain-INED database for example), are far from being fully exploited and it may be more important to focus on these already existing classic data. One must also be aware of the possible misleading applications of some results: \cite{louail2016crowdsourcing} makes a very good analysis of potential redistribution of bank card transactions within a city, but pushes the results as possible basis for social equity policy recommandation by acting on mobility, forgetting that urban form and function are coupled in a complex way and that moving transactions from one place to the other involves far more complex processes than policies (even in a country like China were policies are actually enforced with a hand of steel).
}{
Comme toujours la situation n'est naturellement pas aussi idyllique qu'elle semble être au premier abord, et l'herbe verte du pré du voisin que nous pouvons être tentés d'aller brouter se transforme rapidement en un triste fumier. En effet, les objectifs et motivations sont flous et on peut facilement s'y perdre. Des illustrations parleront d'elles-même. \cite{barthelemy2013self} introduit un nouveau jeu de données et des méthodes relativement nouvelles pour quantifier l'évolution du réseau de rues, mais les résultats, sur lesquels les auteurs semblent s'étonner, sont qu'une transition a eu lieu à Paris à l'époque d'Haussmann. Tout historien de l'urbanisme s'interrogerait sur le but exact de l'étude, puisque à la fin un sentiment étrange de réinvention de la roue flotte dans l'air. L'utilisation des ressources de calcul peut également être exagéré, et dans le cas de la modélisation multi-agent, on peut citer~\cite{axtell2016120}, pour lequel l'objectif de simuler le système à l'échelle 1:1 semble être loin des motivations et justifications originelles de la modélisation agent, et pourrait même donner des arguments aux économistes \emph{mainstream} qui dénigrent facilement les ABMS. D'autres anecdotes peuvent inquiéter : %\cite{robin_cura_2014_11415} % do not cite this shit
est une application web qui gâche des ressources de calcul  pour simuler des distributions Gaussiennes afin de calculer pour un modèle de Gibrat, afin de calculer leur moyenne et variance, qui sont des paramètres d'entrée du modèle. En résumé, cela revient à vérifier le Théorème de la Limite Centrale, qui est a priori assez accepté par la plupart des scientifiques. D'autre part, la distribution complète donnée par un modèle de Gibrat est entièrement connue théoriquement comme résolu e.g. par~\cite{gabaix1999zipf}. Récemment, sur la liste de diffusion de géographie francophone \emph{Geotamtam}, un soudain engouement autour des données issues de \emph{Pokemon Go} a semblé répondre plus à un besoin urgent et inexpliqué d'exploiter cette source de données avant tous les autres, plutôt qu'à des considérations théoriques élaborées. Des jeux de données existant et précis, comme la population historiques des villes (pour la France la base Pumain-INED par exemple), sont loin d'être entièrement exploités et il pourrait être plus pertinent de se concentrer sur ces jeux de données classiques qui existent déjà. De même, il faut être conscient des possibles applications de résultats basée sur des malentendus : \cite{louail2016crowdsourcing} fait une très bonne analyse de la redistribution potentielle des transactions de carte bancaire au sein d'une ville, mais présente les résultats comme la base possible de recommandations de politiques pour une équité sociale en agissant sur la mobilité, oubliant que la forme et les fonctions urbaines sont couplés de manière complexe et que déplacer des transactions d'un endroit à un autre implique des processus bien plus complexes que des régulations directes (même dans un pays comme la Chine ou les régulations sont effectivement mise en place et imposées d'une main de fer).
}




\bpar{
Our main claim here is that the computational shift and simulation practices will be central in geography, but may also be dangerous, for the reasons illustrated above, i.e. that data deluge may impose research subjects and elude theory, and that computation may elude model construction and solving. A stronger link is required between computational practices, computer science, mathematics, statistics and theoretical geography. Theoretical and Quantitative Geography is at the center of this dynamic, as it was its initial purpose that seems forgotten in some cases. It implies the need for elaborated theories integrated with conscious simulation practices. In other words we can answer complementary naive questions that have however to be tackled one and for once. If a theory-free quantitative geography would be possible, the answer if naturally no as it is close to the trap of black-box data-mining analysis. Whatever is done in that case, the results will have a very poor explanatory power, as they can exhibit relations but not reconstruct processes. On an other hand, the possibility of a purely computational quantitative geography is a dangerous vision: even gaining three orders of magnitudes in computational power does not solve the dimensionality curse. In our work here, without theory, we would not know which objects, measures and properties to look at (e.g. multi-scale and dynamical nature of processes), and without analytics, it would be sometimes difficult to draw conclusions from empirical analysis. Nothing is really new here but this position has to be stated and stood up, precisely because our work will use this kind of tools, trying to advance on a thin and fragile edge, with the void of the unfunded theoretical charlatanism on one side and the abyss of the technocratic blind drowning in foolish amounts of data. More than ever we need simple but powerful and funded theories {\`a}-la-Occam~\cite{batty2016theoretical}, to allow a wise integration of new techniques into existing knowledge.
}{
Notre principal argument est que le tournant computationnel et les pratiques de simulation seront centrales en géographie, mais peuvent également être dangereux, pour les raisons illustrées ci-dessus, i.e. que le déluge de données peut imposer les sujets de recherche et occulter la théorie, et que la computation peut éluder la construction et la résolution de modèles. Un lien plus fort est nécessaire entre les pratiques de calcul, l'informatique, les mathématiques, les statistiques et la géographie théorique. La Géographie Théorique et Quantitative est au centre de cette dynamique, puisqu'il s'agit de sa motivation initiale principale qui semble oubliée dans certains cas. Cela implique un besoin de recherche de théorie élaborées intégrées avec des pratiques de simulation conscientes. En d'autres mots, on peut répondre à des questions naïves complémentaires qui ont toutefois besoin d'être traitées une bonne fois pour toutes. Si une géographie quantitative libérée de la théorie serait possible, la réponse est naturellement non puisque cela se rapproche du piège de la fouille de données par boîte noire. Quoi qu'il soit fait par cette approche, les résultats auront un pouvoir explicatif très faible, puisqu'ils pourront mettre en valeur des relations mais pas reconstruire des processus. D'autre part, la possibilité d'une géographie quantitative purement basée sur le calcul est une vision dangereuse : même le gain de trois ordres de grandeur dans la puissance de calcul disponible ne résout pas le sort de la dimension. Dans notre travail ici l'absence de théorie impliquerait de ne pas connaitre les objets, mesures et propriétés à étudier (e.g. le caractère multi-scalaire ou dynamique des processus), et sans résolutions analytiques, il serait souvent difficile de tirer des conclusions à partir des analyses empiriques. Rien n'est vraiment nouveau ici mais cette position doit être affirmée et tenue, précisément car notre travail se base sur ce type d'outils, essayant d'avancer sur une arête fine et fragile, avec d'un côté le vide du charlatanisme théorique infondé et de l'autre l'abîme de l'overdose technocratique dans des quantités de données folles. Plus que jamais on a besoin de théories simples mais fondées et puissantes {\`a}-la-Occam~\cite{batty2016theoretical}, pour permettre une intégration saine des nouvelles techniques au sein des connaissances existantes.
}


 
% multi-modeling and model families (see Cottineau, Rey and Reuillon presentation) as one way to do that ?

% Interdisciplinarity (and Nexus ?!) necessary to achieve that.



\comment[JR]{develop the case study presented at RGS, briefly, to illustrate both big data and dependency of computation, analytics and theory ?}










%----------------------------------------------------------------------------------------


\newpage


%  section : synthetic data control - introduces rochebrune paper, feasible correlation space etc, and forthcoming applications ?


\subsection{Statistical Control on Initial Conditions by Synthetic Data Generation}{Contrôle statistique pour les conditions initiales par génération de données synthétiques}

\subsubsection{Context}{Contexte}


\bpar{
When evaluating data-driven models, or even more simple partially data-driven models involving simplified parametrization, an unavoidable issue is the lack of control on ``underlying system parameters'' (what is a ill-defined notion but should be seen in our sense as parameters governing system dynamics). Indeed, a statistics extracted from running the model on enough different datasets can become strongly biased by the presence of confounding in the underlying real data, as it is impossible to know if result is due to processes the model tries to translate or to a hidden structure common to all data.
}{
Lors de l'évaluation de modèle basés sur les données, \comment{(Florent)  data driven models : sens de cette expression ?} ou même de modèle plus simples partiellement basés sur les données impliquant une paramétrisation simplifiée, une issue inévitable est le manque de contrôle sur les ``paramètres implicites du systèmes'' (ce qui n'est pas une notion stricte mais doit être vu dans notre sens comme les paramètres régissant la dynamique). En effet, une statistique issue d'executions du modèle sur un nombre suffisamment grand de jeu de données peut devenir fortement biaisé par la présence de \emph{confunding} dans les données réelles sous-jacentes, comme il est impossible de savoir si les résultats sont dus aux processus que le modèle cherche à traduire ou à une structure commune à toutes les données.
\comment{(Florent) a quelle question cherches-tu ici à répondre ?}
}


%Let illustrate the issue with a simple example.

We formalize briefly a proposition of method that would allow to add controls on meta-parameters, in the sense of parameters driving the represented system at a higher temporal and spatial scale, for a model of simulation. We make the hypothesis that such method is valid under constraints of disjonction for scales and/or ontologies between the model of simulation and the domain of meta-parameters.


\subsubsection{Description}{Description}

An advanced knowledge of the behavior of computational models on their parameter space is a necessary condition for deductions of thematic conclusions \comment{(Florent) pas nécessaire si les comportements sont bien calibrés par des observations dans le monde réel, il n'est pas besoin d'une compréhension si fine du modèle}
 or their practical application~\cite{banos2013pour}. But the choice of varying parameters is always subjective, as some may be fixed by a real-world parametrization, or other may be interpreted as arbitrarily fixed initial conditions. It raises methodological and epistemological issues for the sensitivity analysis, as the scope of the model may become ill-defined. \comment{(Florent) sens ?}

Let consider the concrete example of the Schelling Segregation model~\cite{schelling1971dynamic}. \comment{(Florent) Schelling hors sujet ?}
One of its crucial features on which the literature has been rather controversial is the influence of the spatial structure of the container on which agents evolve.%~(\textit{Biblio Marion}). 
 The thematic aim of the project developed in~\cite{cottineau2015revisiting} is to clarify this point through a systematic model exploration. A methodological contribution is the construction of a framework allowing the analysis of the sensitivity of models to \emph{meta-parameters}, i.e. to parameters considered as fixed initial conditions (e.g. the spatial structure for the Schelling model), or to parameters of another model generating an initial configuration%, as detailed in Fig.~\ref{} \textit{[insert scheme describing the approach]}, 
 yielding thus a \emph{simple coupling} between models (serial coupling). The benefits of such an approach are various but include for example the knowledge of model behavior in an extended frame, the possibility of statistical control when regressing model outputs, a finer exploration of model derivatives than with a naive approach. Some remarks can be made on the approach :
\begin{itemize}
\item What knowledge are brought by adding the upstream model, rather than for example in the Schelling case exploring a large set of initial geometries ? 

$\rightarrow$ \textit{to obtain a sufficiently large set of initial configuration, one quickly needs a model to generate them ; in that case a quasi-random generation followed by a filtering on morphological constraint will be a morphogenesis model, which parameters are the ones of the generation and the filtering methods. Furthermore, as detailed further, the determination of the derivative of the downstream model is made possible by the coupling and knowledge of the upstream model.}
\comment{(Florent) est-ce important pour la thèse ?}


\item Statistical noise is added by coupling models

$\rightarrow$ \textit{Repetitions needed for convergence are indeed larger as the final expectance has to be determined by repeating on the first times the second model ; but it is exactly the same as exploring directly many configuration, to obtain statistical robustness in that case one must repeat on similar configurations.}

\item Complexity is added by coupling models

% check Varenne citation
$\rightarrow$ \textit{In the sense of Varenne~\cite{varenne2010framework} , coupling is simple and no complexity is thus added.}
\end{itemize}
 
%\paragraph{Context}

%Let $M_{m}$ a stochastic model of simulation, which inputs are to simplify initial conditions $D_0$ and parameters $\vec{\alpha}$, and output $M_{m}\left[\vec{\alpha},D_0\right](t)$ at a given time $t$. We assume that it is partially data-driven in the sense that $D_0$ is supposed to represent a real situation at a given time, and model performance is measured by the distance of its output at final time to the real situation at the corresponding time, i.e. error function is of the form $\norm{\Eb{\vec{g}(M_{m}\left[\vec{\alpha},D_0\right](t_f))}-\vec{g}(D_f)}$ where $\vec{g}$ is a deterministic field corresponding to given indicators.

%\paragraph{Position of the Problem}

%Evaluating the model on real data is rapidly limited in control possibilities, being restricted to the search of datasets allowing natural control groups. Furthermore, statistical behaviors are generally poorly characterized because of the small number of realizations. Working with synthetic data first allows to solve this issue of robustness of statistics, and then gives possibilities of control on some ``meta-parameters'' in the sense described before.




%%%%%%%%%%%%%%%%%%%%
\subsubsection{Formal Description}{Description Formelle}

%\subsubsection{Deterministic Formulation}

One has the composition of the derivative along the meta-parameter

\[
\partial_{\alpha}\left[M_u \circ M_d\right] = \left(\partial_{\alpha} M_u \circ M_d \right)\cdot \partial_{\alpha} M_d
\]

$\rightarrow$ \textit{the sensitivity of the downstream model (Schelling) can be determined by studying the serial coupling and the upstream model ; thematic knowledge : sensitivity to an implicit meta-parameter ; and computational gain : generation of controlled differentiates in the ``initial space'' is quasi impossible.}

%\subsubsection{Stochasticity}

The question of stochasticity in simply coupled models causes no additional issue as $\Eb{X}=\Eb{\Eb{X|Y}}$. It naturally multiplies the number of repetition needed for convergence what is the expected behavior.






