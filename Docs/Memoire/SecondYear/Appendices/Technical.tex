\chapter{Technical Developments} % Chapter title

\label{app:technical} % For referencing the chapter elsewhere, use \autoref{ch:name} 

%----------------------------------------------------------------------------------------






\section{Derivations for Urban Growth Models}





\begin{lemma}
The limit of a Preferential Attachment model when $\lambda \ll 1$ is a linear-growth Gibrat model, with limit parameters $\mu_i(t)=1+\frac{\lambda}{m\cdot (t-1)}$.
\end{lemma}

\begin{proof}

Starting with first moment, we denote $\bar{P}_i(t)=\Eb{P_i(t)}$. Independence of Gibrat growth rate yields directly $\bar{P}_i(t)=\Eb{R_i(t)}\cdot \bar{P}_i(t-1)$. Starting for the preferential attachment model, we have $\bar{P}_i(t) = \Eb{P_i(t)} = \sum_{k=0}^{+\infty}{k\Pb{P_i(t)=k}}$. But
\[
\{P_i(t)=k\}=\bigcup_{\delta=0}^{\infty}{\left(\{P_i(t-1)=k-\delta\}\cap \{P_i\leftarrow P_i + 1\}^{\delta}\right)}
\]

where the second event corresponds to city $i$ being increased $\delta$ times between $t-1$ and $t$ (note that events are empty for $\delta \geq k$). Thus, being careful on the conditional nature of preferential attachment formulation, stating that $\Pb{\{P_i\leftarrow P_i + 1\} | P_i(t-1)=p} = \lambda\cdot\frac{p}{P(t-1)}$ (total population $P(t)$ assumed deterministic), we obtain

\begin{equation*}
\begin{split}
\Pb{\{P_i\leftarrow P_i + 1\}} & = \sum_{p}{\Pb{\{P_i\leftarrow P_i + 1\} | P_i(t-1)=p}\cdot \Pb{P_i(t-1)=p}}\\
&=\sum_{p}{\lambda\cdot\frac{p}{P(t-1)}\Pb{P_i(t-1)=p}}=\lambda\cdot\frac{\bar{P}_i(t-1)}{P(t-1)}\\
\end{split}
\end{equation*}

It gives therefore, knowing that $P(t-1)=P_0 + m\cdot (t-1)$ and denoting $q=\lambda\cdot\frac{\bar{P}_i(t-1)}{P_0 + m\cdot (t-1)}$

\[
\begin{split}
\bar{P}_i(t) & =\sum_{k=0}^{\infty}{\sum_{\delta=0}^{\infty}{k\cdot \left(\lambda\cdot\frac{\bar{P}_i(t-1)}{P_0 + m\cdot (t-1)}\right)^{\delta}\cdot \Pb{P_i(t-1)=k-\delta}}}\\
& = \sum_{\delta^{\prime}=0}^{\infty}{\sum_{k^{\prime}=0}^{\infty}{\left(k^\prime + \delta^{\prime}\right)\cdot q^{\delta^{\prime}} \cdot \Pb{P_i(t-1)=k^\prime}}}\\
& = \sum_{\delta^{\prime}=0}^{\infty}{q^{\delta^{\prime}}\cdot \left(\delta^{\prime} + \bar{P}_i(t-1)\right)} = \frac{q}{(1-q)^2} + \frac{\bar{P}_i(t-1)}{(1-q)}\\
& = \frac{\bar{P}_i(t-1)}{1-q}\left[1 + \frac{1}{\bar{P}_i(t-1)}\frac{q}{(1-q)}\right]
\end{split}
\]

%& = \bar{P}_i(t-1)\cdot \frac{1}{1-\lambda\cdot\frac{\bar{P}_i(t-1)}{P_0 + m\cdot (t-1)}} \left[1 + \frac{\lambda}{P_0 + m\cdot (t-1)}\cdot \frac{1}{1-\lambda\cdot\frac{\bar{P}_i(t-1)}{P_0 + m\cdot (t-1)}} \right]


As it is not expected to have $\bar{P}_i(t)\ll P(t)$ (fat tail distributions), a limit can be taken only through $\lambda$. Taking $\lambda \ll 1$ yields, as $0 < \bar{P}_i(t)/P(t) < 1$, that $q=\lambda\cdot\frac{\bar{P}_i(t-1)}{P_0 + m\cdot (t-1)} \ll 1$ and thus we can expand in first order of $q$, what gives $\bar{P}_i(t)=\bar{P}_i(t-1)\cdot \left[1 + \left(1+\frac{1}{\bar{P}_i(t-1)}\right)q + o(q))\right]$

\[
\bar{P}_i(t) \simeq \left[1 + \frac{\lambda}{P_0 + m\cdot (t-1)}\right]\cdot \bar{P}_i(t-1)
\]

It means that this limit is equivalent in expectancy to a Gibrat model with $\mu_i(t) = \mu(t)=1 + \frac{\lambda}{P_0 + m\cdot (t-1)}$.

For the second moment, we can do an analog computation. We have still \[\Eb{P_i(t)^2} = \Eb{R_i(t)^2}\cdot \Eb{P_i(t-1)^2}\]
and
\[\Eb{P_i(t)^2}=\sum_{k=0}^{+\infty}{k^2 \Pb{P_i(t)=k}}\] 

We obtain the same way 

\[
\begin{split}
\Eb{P_i(t)^2} & = \sum_{\delta^{\prime}=0}^{\infty}{\sum_{k^{\prime}=0}^{\infty}{\left(k^\prime + \delta^{\prime}\right)^2\cdot q^{\delta^{\prime}} \cdot \Pb{P_i(t-1)=k^\prime}}}\\ 
& = \sum_{\delta^{\prime}=0}^{\infty}{q^{\delta^{\prime}}\cdot \left(\Eb{P_i(t-1)^2}+2\delta^{\prime}\bar{P}_i(t-1) + {\delta^{\prime}}^2\right)}\\
& = \frac{\Eb{P_i(t-1)^2}}{1-q} + \frac{2 q \bar{P}_i(t-1)}{(1-q)^2} + \frac{q(q+1)}{(1-q)^3}\\
& = \frac{\Eb{P_i(t-1)^2}}{1-q}\left[1 + \frac{q}{\Eb{P_i(t-1)^2}}\left(\frac{2\bar{P}_i(t-1)}{1-q} + \frac{(1+q)}{(1-q)^2}\right)\right]
\end{split}
\]



We have therefore an equivalence between the Gibrat model as a continuous formulation of a Preferential Attachment (or Simon model) in a certain limit. \qed

\end{proof}



