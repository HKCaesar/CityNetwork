


% Chapter 

\chapter{Theoretical Framework} % Chapter title

\label{ch:theory} % For referencing the chapter elsewhere, use \autoref{ch:name} 

%----------------------------------------------------------------------------------------



\headercit{Your theory is crazy, but not enough to be true.}{Niels Bohr}{}

\bigskip


\bpar{
Theory is a key element of any scientific construction, especially in Human Sciences in which object definition and questioning are more open but also determining for research directions. We develop in this chapter a self-consistent theoretical background. It naturally emerges from thematic considerations of previous chapter, empirical explorations done in chapter~\ref{ch:empirical} and modeling experiments conducted in chapter~\ref{ch:modeling}, as a linear structure of knowledge is not appropriate to translate the type of scientific entreprise we are conducting, typically in the spirit of \noun{Sanders} in~\cite{livet2010} for which the simultaneous conjonction of empirical, conceptual and modeling domains is necessary for the emergence of knowledge. This theoretical construction is however presented to be understood independently, and is used as a structuring skeleton for the rest of the thesis.
}{
La théorie est un élément essentiel de toute construction scientifique, en particulier en Sciences Humaines pour lesquelles la définition des objets et questions de recherche sont plus ouverts mais aussi plus déterminants des directions de recherche alors prises. Nous développons dans ce chapitre un cadre théorique autonome. Il émerge naturellement des considérations thématiques du chapitre précédent, des explorations empiriques faites dans le chapitre~\ref{ch:empirical} et des experiences de modélisation conduites dans le chapitre~\ref{ch:modeling}
}


\bpar{
We propose first to construct the \emph{geographical theory} that will pose the studied objects and their meaning in the real world (their ontology), with their interrelations. This yields precise assumptions that will be sought to be confirmed or proven false in the following. Staying at a thematic level appears however to be not enough to obtain general guidelines on the type of methodologies and the approaches to use. More precisely, even if some theories imply an more natural use of some tools\footnote{to give a rough example, a theory emphasizing the complexity of relations between agents in a system will conduct generally to use agent-based modeling and simulation tools, whereas a theory based on macroscopic equilibrium will favorise the use of exact mathematical derivations.}, at the subtler level of contextualization in the sense of the approach taken to implement the theory (as models or empirical analysis), the freedom of choice may mislead into unappropriated techniques or questionings (see \cite{raimbault2016cautious} on the example of incautious use of big data and computation). We develop therefore in a second section a theoretical framework at a meta-level, aiming to give a vision and framing for modeling socio-technical systems.
}{
Nous proposons d'abord de construire une \emph{Théorie Géographique} qui fixera les objets étudiés et leur nature réelle (leur ontologie), ainsi que leur interrelations. Celle-ci permettra de produire des hypothèses précises qu'on cherchera à confirmer ou infirmer par la suite.
}




%----------------------------------------------------------------------------------------

\newpage

% first section that develops elements of geographical theory : system of cities, territories, etc.
%  city morphogenesis ? -> link to other approaches of morphogenesis (Turing)
%  coevolution : give a precise theoretical meaning

% HERE precise exact definition, framed precisely (reference for the following).

\section{Geographical Theoretical Context}{Pour une Théorie Géographique}



%%%%%%%%%%%
\subsection{Foundation}



%%%%%%%%%%%%%%
\subsubsection{Networked Human Territories}


\bpar{
Our first pillar has already been constructed before in the thematic exploration of the research subject. We rely on the notion of \emph{Human Territory} elaborated by \noun{Raffestin} as the basis for a definition of territorial systems. It permits to capture complex human geographical systems in their concrete and abstract characteristics and representation. For example, a metropolitan territorial system can be apprehended simply by the functional extent of daily commuting, or by the perceived or lived space of different populations, the choice depending on the precise question asked. Note that this approach to territory is a position and that other (possibly compatible) entries could be taken~\cite{murphy2012entente}. The concrete of this pillar in reinforced by the territorial theory of networks of \noun{Dupuy}, yielding the notion of networked human territory, as a human territory in which a set of potential transactional networks have been realized, which is in accordance with vision of the territory as networked places~\cite{champollion:halshs-00999026}. We make therein the assumption that real networks are necessary elements of territorial systems.
}{
Notre premier pilier a déjà été construit précédemment lors de l'exploration thématique du projet de recherche. Nous nous basons sur la notion de \emph{Territoire Humain} élaborée par \noun{Raffestin} comme la base de la définition d'un système territorial. Elle permet de capturer les systèmes complexes géographiques humains dans l'ensemble de leur caractéristiques concrètes et abstraites, ainsi que dans leur représentations. Par exemple, un territoire métropolitain peut être appréhendé simplement par l'étendue fonctionnelle des flux pendulaires journaliers, ou par l'espace perçu ou vécu des différentes populations, le choix dépendant de la question précise à laquelle on cherche à répondre. Cette approche au territoire est bien sûr un choix délibéré et que d'autres entrées, possiblement compatible, peuvent bien sûr être prises~\cite{murphy2012entente}. Le ciment de ce pilier est renforcé par la théorie territoriale des réseaux de \noun{Dupuy}, fournissant la notion de territoire humain en réseau, comme un territoire humain dans lequel un ensemble de réseaux transactionnels potentiels ont été réalisés, ce qui s'accorde par ailleurs avec les visions du territoire comme un lieu des réseaux~\cite{champollion:halshs-00999026}. % TODO here beware confusion between place and space : lieu en réseau ≠ lieu des réseaux ≠ espace en réseau ≠ espace des réseaux ???
Nous ferons pour cela l'hypothèse fondamentale que les réseaux réels sont des éléments nécessaires des systèmes territoriaux.
}

%%%%%%%%%%%%%%%
\subsubsection{Evolutive Urban Theory}{Théorie Evolutive des Villes}

% development of Denise theory
%  -> extension with precision on coevolution ? (read Holland on coevolution) -- beware of biological //


\bpar{
The second pillar of our theoretical construction is the Evolutive Urban Theory of \noun{Pumain}, closely linked to the complexity approach we take. This theory was first introduced in~\cite{pumain1997pour} which argues for a dynamical vision of city systems, in which self-organization is key. Cities are interdependent evolutive spatial entities whose interrelations produces the macroscopic behavior at the scale of city system. The city system is also designed as a network of city what emphasizes its view as a complex system. Each city is itself a complex system in the spirit of~\cite{berry1964cities}, the multi-scale aspect being essential in this theory, since microscopic agents convey system evolution processus through complex feedbacks between scales. The positioning within Complex System Sciences was later confirmed~\cite{pumain2003approche}. It was shown that this theory provide an interpretation for the origin of pervasive scaling laws, resulting from the diffusion of innovation cycles between cities~\cite{pumain2006evolutionary}. The aspect of resilience of system of cities, induced by the adaptive character of these complex systems, implies that cities are drivers and adapters of social change~\cite{pumain2010theorie}. Finally, path dependance yield non-ergodicity within these systems, making ``universal'' interpretations of scaling laws developed by physicists incompatible with evolutive urban theory~\cite{pumain2012urban}. The Evolutive Urban Theory was elaborated conjointly with models of urban systems: for example the Simpop2 model is an agent-based model taking into account economic processes, that simulates growth patterns on long time scales for Europe and the United States~\cite{doi:10.1177/0042098010377366}. % TODO \cite{bretagnolle2006theory} introduction of simpop2 model // remove french versions ?
 The latest accomplishment of the evolutive theory relies in the output of the ERC project GeoDivercity, presented in~\cite{pumain2017urban}, that include both advanced technical (software OpenMole), thematic (knowledge from SimpopLocal and Marius models) and methodological (incremental modeling) progresses. We will interpret territorial systems following that idea of complex adaptive systems.
}{
Le second pilier de notre construction théorique est la théorie évolutive des villes de \noun{Pumain}, en relation étroite avec l'approche complexe que nous prenons de manière générale. Cette théorie a été introduite initialement dans~\cite{pumain1997pour} qui argumente pour une vision dynamique des systèmes de ville, au sein desquels l'auto-organisation est essentielle. Les villes sont des entités spatiales évolutives interdépendantes dont les interrelations font émerger le comportement macroscopique à l'échelle du système de villes. Le système de villes est aussi vu comme un réseau de villes, ce qui renforce sa vision en tant que système complexe. Chaque ville est elle-même un système complexe dans l'esprit de~\cite{berry1964cities}, l'aspect multi-scalaire étant essentiel dans cette théorie, puisque les agents microscopiques véhiculent les processus d'évolution du système à travers des rétroactions complexes entre les échelles. Le positionnement de cette théorie au regard des Sciences des Systèmes Complexes a plus tard été confirmé~\cite{pumain2003approche}. Il a été montré que la théorie évolutive fournit une interprétation des lois d'échelle qui sont omniprésentes dans les systèmes urbains, qui découleraient de la diffusion des cycles d'innovation entre les villes~\cite{pumain2006evolutionary}. La notion de résilience d'un système de villes, induit par le caractère adaptatif des ces systèmes complexes, implique que les villes sont les moteurs et les adaptateurs du changement social~\cite{pumain2010theorie}. Enfin, la dépendance au chemin est source de non-ergodicité au sein de ces systèmes, rendant les interprétations ``universelles'' des lois d'échelle développées par les physiciens incompatibles avec la théorie évolutive~\cite{pumain2010theorie}. La Théorie Evolutive des Villes a été élaborée conjointement avec des modèles de systèmes urbains : par exemple le modèle Simpop2 est un modèle basé agent qui prend en compte des processus économiques, et simule sur de longues échelles de temps les motifs de croissance urbaine pour l'Europe et les Etats-unis~\cite{doi:10.1177/0042098010377366}. L'accomplissement le plus récent de la théorie évolutive réside dans la production de l'ERC GeoDivercity, présentée dans~\cite{pumain2017urban}, qui inclut à la fois des progrès techniques avancés (logiciel OpenMole), thématiques (connaissances issues des modèles SimpopLocal et Marius) et méthodologiques (modélisation par incréments). Nous interpréterons les systèmes territoriaux à la lumière de cette idée des villes comme systèmes complexes adaptatifs.
}



%%%%%%%%%%%
\subsubsection{Urban Morphogenesis}{Morphogenèse Urbaine}

% -> make a link between city systems and urban form/cityscape / territorial configurations

% Why morphogenesis is important : linked with modularity and scale -> if a submodule can be explained independantly (ie morphigeneis process is isolated), then we have the characteristic scale. then when size grows and interaction within city system -> can not explain alone (or with externalities ?) -> need a change in scale. ex. influecne of city system for size, activities; posiiton of an airport in a metropoltian region ; emergence of MCR.  ==> Assimptions to be tested with models ?.

% Alexander and Salingaros

% include transportation network, hierarchy and congestion in transport : Remy vs Benjamin (paper ? -> see with René)



\bpar{
The idea of morphogenesis was particularly underlined by \noun{Turing} in~\cite{turing1952chemical} % not exactly, already introduced before
 when trying to isolate simple chemical rules that could lead to the emergence of the embryo and its form. The morphogenesis of a system consists in self-consistent evolution rules that produce the emergence of its successives states, i.e. the precise definition of self-organization, with the additional property that an emergent architecture exists, in the sense of relations between form and function. Progresses towards the understanding of embryo morphogenesis (in particular the isolation of processes producing the differentiation of cells from an unique cell) has been made only recently with the use of Complexity Approaches in integrative biology~\cite{delile2016chapitre}. In the case of urban systems, the idea of urban morphogenesis, i.e. of self-consistent mechanisms that would produce the urban form, is more used in the field of architecture and urban design~\cite{hachi2013master} (as \noun{Alexander} generative grammar ``Pattern Language'' e.g.), in relation with theories of Urban Form~\cite{moudon1997urban}. This idea can be pushed into very small scales such as the building~\cite{whitehand1999urban} but we will use it more at a mesoscopic scale, in terms of land-use changes within an intermediate scale territorial system, in the same ontologies as Urban morphogenesis modeling literature (for example \cite{bonin2012modele} describes a model of urban morphogenesis with qualitative differentiation, whereas \cite{makse1998modeling} give a model of urban growth based on a mono-centric population distribution perturbed with correlated noises). The notion of morphogenesis will be important in our theory in link with modularity and scale. Modularity of a complex system consists in its decomposition into relatively independent sub-modules, and modular decomposition of a system can be seen as a way to disentangle non-intrinsic correlations~\cite{2015arXiv150904386K} (think of a block diagonalisation of a first order dynamical system). In the context of large-scale cyber-physical systems design and control, similar issues naturally raise and specific techniques are needed to scale up simple system control methods~\cite{2017arXiv170105880W}. The isolation of a subsystem yields a corresponding characteristic scale. Isolating possible morphogenesis processes imply a controlled isolation (controlled boundary conditions e.g.) of the considered system, corresponding to a modularity level and thus a scale. When self-consistent processes are not enough to explain the evolution of the system (with reasonable action on boundary conditions), a change of scale is necessary, caused by an underlying phase transition in modularity. The example of metropolitan growth is a good example: complexity of interactions within the metropolitan region will grow with size and diversity of functions leading to a change in scale necessary to understand processes. The emergence of an international airport will strongly influence local development, what corresponds to the significant integration within a larger system. The characteristic scales and processes for which these change occur will be precise questions to be investigated through modeling. It is interesting to remark that a territorial subsystem in which morphogenesis has a sense can be seen as an \emph{autopoietic system} in the extended sense of \noun{Bourgine} in~\cite{bourgine2004autopoiesis}, as a network of auto-reproducing processes\footnote{which are however not cognitive, making this auto-organized systems fortunately not alive in the sense of autopoietic and cognitive systems} regulating their boundary conditions, what emphasizes boundaries on which we will last insist.
}{
La notion de morphogenèse a été particulièrement soulignée par \noun{Turing} dans~\cite{turing1952chemical} lorsqu'il proposait d'isoler des règles chimiques élémentaires qui pourraient mener à l'émergence de l'embryon et à sa forme. La morphogenèse d'un système consiste en des règles d'évolution auto-cohérentes qui produisent l'émergence de ses états successifs, i.e. la définition précise de l'auto-organisation, avec la propriété supplémentaire qu'une architecture émergente existe, au sens de relations entre la forme et la fonction. Les progrès vers la compréhension de la morphogenèse de l'embryon (en particulier l'isolation de processus particuliers induisant la différentiation de cellules à partir d'une unique) sont relativement récents grâce à l'application des approches complexes en biologie intégrative~\cite{delile2016chapitre}. Dans le cas des systèmes urbains, l'idée de morphogenèse urbaine, i.e. de mécanismes auto-cohérents qui produiraient la forme urbaine, est plutôt utilisé dans les champs de l'architecture et de l'urbanisme~\cite{hachi2013master} (comme e.g. la grammaire générative du ``Pattern Language'' d'\noun{Alexander}), en relation avec des théories de la forme urbaine~\cite{moudon1997urban}. Cette idée peut être poussée jusqu'à de très petites échelles comme celle du bâtiment~\cite{whitehand1999urban} mais nous l'utiliserons plus à une échelle mesoscopique, en termes de changements d'usage du sol à une échelle intermédiaire des systèmes territoriaux, avec des ontologies similaires à la littérature de modélisation de la morphogenèse urbaine (par exemple \cite{bonin2012modele} décrit un modèle de morphogenèse urbaine avec différentiation qualitative, tandis que \cite{makse1998modeling} donne un modèle de croissance urbaine basé sur une distribution monocentrique de la population perturbée par des bruits corrélés). La notion de morphogenèse sera importante dans notre théorie en lien avec la modularité et l'échelle. La modularité d'un système complexe consiste en sa décomposition en sous-modules relativement indépendants, et la décomposition modulaire d'un système peut être vue comme un moyen de supprimer les correlations non intrinsèques~\cite{2015arXiv150904386K} (pour donner une image, penser à une diagonalisation par blocs d'un système dynamique du premier ordre). Dans le cadre de la conception et du contrôle de systèmes cyber-sociaux à grande échelle, des problèmes similaires surgissent naturellement et des techniques spécifiques sont nécessaires pour le passage à l'échelle des techniques simple de contrôle~\cite{2017arXiv170105880W}. L'isolation d'un sous-système fournit une échelle caractéristique correspondante. Isoler des processus de morphogenèse possibles implique une extraction contrôlée (conditions au bord contrôlées par exemple) du système considéré, ce qui correspond à un niveau de modularité et donc à une échelle. Quand des processus auto-cohérents ne sont pas suffisants pour expliquer l'évolution d'un système (dans des variations raisonnables des conditions initiales), un changement d'échelle est nécessaire, causé par une transition de phase implicite dans la modularité. L'exemple de la croissance métropolitaine en est une très bonne illustration : la complexité des interactions au sein de la région métropolitaine sera croissante avec sa taille et la diversité des fonctions urbaines, ce qui conduit à un changement de l'échelle nécessaire pour comprendre les processus. L'émergence d'un aéroport international influencera fortement le développement local, ce qui correspondant à une intégration significative dans un système plus vaste. Les échelles caractéristiques et la nature des processus pour lesquels ces changements ont lieu peuvent être des questions précisément approchées par l'angle de la modélisation. Il est intéressant de noter qu'un système territorial dans lequel la morphogenèse prend sens peut être vu comme un \emph{système auto-poiétique} % TODO no distinction ?
 au sens étendu de \noun{Bourgine} dans~\cite{bourgine2004autopoiesis}, comme un réseau de processus qui s'auto-reproduisent\footnote{qui ne sont toutefois pas cognitifs, ne rendant heureusement pas ces systèmes auto-organisés vivant au sens de auto-poiétique et cognitif} en régulant leur conditions aux bords, ce qui souligne la notion de frontière sur laquelle nous allons finalement nous attarder.
}


\cite{desmarais1992premisses}
\cite{levy2005formes}


% transition : Bourgine autopoiesis -> importance of boundaries -> link to Holland.

%%%%%%%%%%%
\subsubsection{Co-evolution}

% other insight : Holland Signal and Boundaries, ecological niche etc. : contextualize within this framework, clarify definition of co-evolution


\bpar{
Our last pillar is a clarification of the notion of \emph{co-evolution}, on which \noun{Holland} shed light through an approach of complex adaptive systems by a theory of CAS as signal processing agents operating thanks to their boundaries~\cite{holland2012signals}. In this theory, complex adaptive systems form aggregates at diverse hierarchical levels, that correspond to different level of self-organization, and boundaries are vertically and horizontally intricate in a complex way. That approach introduces the notion of \emph{niche} as a relatively independent subsystem in which ressources circulate (the same way as network communities): numerous illustrations are given such as economical niches or ecological niches. Agents within a niche are said to be \emph{co-evolving}. Co-evolution thus means strong interdependences (implying circular causal processes) and a certain independence regarding the exterior of the niche. The notion is naturally flexible as it will depend on ontologies, resolution, thresholds etc. considered to define the system. This concept is easily transmissible to the evolutive urban theory and converges with the notion of co-evolution described by \noun{Pumain}: co-evolving agents in a system of cities consist in a niche with its flows, signals and boundaries and thus co-evolving entities in the sense of \noun{Holland}. This notion will be important for us in the definition of territorial subsystems and their coupling.
}{
Notre dernier pilier consiste en une clarification de la notion de \emph{co-evolution}, sur laquelle \noun{Holland} apporte un éclairage pertinent à travers son approche des systèmes complexes adaptatifs (CAS) par une théorie des CAS comme agents traitant des signaux grâce à leur frontières~\cite{holland2012signals}. Dans cette théorie, les systèmes complexes adaptatifs forment des agrégats à différents niveaux hiérarchiques, qui correspondent à différents niveaux d'auto-organisation, et les frontières sont intriquées horizontalement et verticalement de manière complexe. Cette approche introduit la notion de \emph{niche} comme un sous-système relativement indépendant au sein duquel les ressources circulent (de la même façon que des communautés dans un réseau) : de nombreuses illustrations telles les niches écologiques ou économiques peuvent être données. Les agents au sein d'une niche sont dits en \emph{co-évolution}. La co-évolution implique ainsi de fortes interdépendances (impliquant des processus causaux circulaires) et une certaine indépendance au regard de l'extérieur de la niche. La notion est naturellement flexible puisqu'elle dépendra des ontologies, de la résolution, des seuils, etc. que l'on considère pour définir le système. Ce concept se transmet assez aisément à la théorie évolutive urbaine et correspond à la notion de co-évolution décrite par \noun{Pumain} : des agents co-évolutifs dans un système de villes consistent en une niche et ses flots, signaux et limites et sont donc des entités co-évolutives au sens de \noun{Holland}. Cette notion sera importante pour nous dans la définition des sous-systèmes territoriaux et de leur couplage.
}





%%%%%%%%%%%
%\subsection{Requirements}
% RQ : no requirements for the theory, contained within pillars : requirement is the presence of these pillars ?



%%%%%%%%%%%
\subsection{Synthesis : an theory of co-evolutive networked territorial systems}

% put different elements together and construct the geographical theory
% give here precise definitions

\bpar{
We synthesize our pillars as a short self-consistent geographical theory of territorial systems in which networks play a central role in the co-evolution of components of the system. See the foundation subsection for definitions and references. The formulation is intended to be minimalistic.
}{
Nous synthétisons les différents piliers en une théorie géographique autonome des systèmes territoriaux pour lesquels les réseaux jouent un rôle central pour la co-évolution des composantes du système. Pour les définitions des termes et les références, se référer à la section précédente. La formulation ici est voulue minimaliste.
}


\medskip


\bpar{
\begin{definition}
\textbf{ - Territorial System.} A territorial system is a set of networked human territories, i.e. human territories in and between which real networks exist.
\end{definition}
}{
\begin{definition}
\textbf{ - Système Territorial.} Un système territorial est un ensemble de territoires humains en réseau, c'est à dire des territoires humains au sein desquels et entre lesquels des réseaux réels existent.
\end{definition}
}



\medskip


\bpar{
At this step complexity and dynamical evolutive characters of territorial systems are implied but not an explicit part of the theory. We will assume to simplify a discrete definition of temporal, spatial and ontological dimensions under modularity and local stationarity assumptions.
}{
A cette étape la complexité et le caractère évolutif et dynamique des systèmes territoriaux sont impliqués par les partis pris mais pas une partie explicite de la théorie. We supposerons pour simplifier une définition discrète des dimensions temporelles, spatiales et ontologiques, sous des hypothèses de modularité et de stationnarité locale.
}
    
    
% definition of scale and stationarity
%\textit{Equivalence between existence of discrete scales and discrete stationarity levels ?}

\medskip


\bpar{
\begin{proposition}
\textbf{ - Discrete scales.} Assuming a discrete modular decomposition of a territorial system, the existence of a discrete set $(\tau_i,x_i)$ of temporal and functional scales for the territorial system is equivalent to the local temporal stationarity of a random dynamical system specification of the system.
\end{proposition}
}{
\begin{proposition}
\textbf{ - Echelle discrètes.} Supposant une décomposition modulaire discrète d'un système territorial, l'existence d'un ensemble discret $(\tau_i,x_i)$ d'échelles temporelles et fonctionnelles pour le système territorial est équivalent à la stationnarité temporelle locale d'une spécification par système dynamique stochastique du système. % Q here : does the master eq needs to be stochastic ?
\end{proposition}
}


\begin{proof}
\textbf{(Sketch of).} We underlie that any territorial system can be represented by random variables, what is equivalent to have well defined objects and states and use the Transfer Theorem on events of successive states. If $X=(X_j)$ is the modular decomposition, we have necessarily quasi-independence of components in the sense that $\Covb{dX_j}{dX_{j'}}\simeq 0$ at any time. General stationarity transitions induce modular transitions that are kept or not depending if they correspond to an effective transition within the subsystem, what provide temporal scales as characteristic times of sub-dynamics. Functional scales are the corresponding extent in the state space.\qed
\end{proof}

% assumption : existence of scales

\medskip


\bpar{
This proposition induce a discrete representation of system dynamics in time. Note that even in the case of no modular representation, the system as a whole will verify the property. This definition of scales allows to explicitly introduce feedback loops and thus emergence and complexity, making our theory compatible with the evolutive urban theory.
}{
Cette proposition induit une représentation des dynamiques du système dans le temps. On peut noter que même en l'absence de représentation modulaire, le système dans son ensemble vérifiera la propriété. Cette définition des échelles permet d'introduire explicitement des boucles de rétroaction et ainsi l'émergence et la complexité, rendant la théorie compatible avec la théorie évolutive urbaine.
}



\bpar{
\begin{assumption}
\textbf{ - Scales and Subsystems intrication. } Complex networks of feedbacks exist both between and inside scales, what impose the existence of weak emergence~\cite{bedau2002downward}. Furthermore a horizontal and vertical hierarchical imbrication of boundaries is not the rule.
\end{assumption}
}{
\begin{assumption}
\textbf{ - Imbrication des échelles et des sous-systèmes. } Des réseaux complexes de retroaction existent à la fois entre et à l'intérieur des échelles~\cite{bedau2002downward}. De plus, un emboîtement horizontal et vertical des limites ne sera généralement pas hiérarchique.
\end{assumption}
}

% co-evolution

\bpar{
Within these complex subsystems intrications we can isolate co-evolving components using morphogenesis. The following proposition is a consequence of the equivalence between the independence of a niche and its morphogenesis. Morphogenesis provides the modular decomposition (local stationarity assumed) needed for the existence of scale, giving minimal vertically (scale) and horizontally (space) independent subsystems.
}{
Au sein de ces imbrications de sous-systèmes nous pouvons isoler des composantes en co-évolution en utilisant la morphogenèse. La proposition suivante est une conséquence de l'équivalence entre l'indépendance d'une niche et sa morphogenèse. La morphogenèse fournit la décomposition modulaire (sous hypothèse de stationnarité locale) nécessaire pour l'existence de l'échelle, donnant des sous-systèmes minimaux indépendants de manière verticale (échelle) et horizontale (espace).
}


\bpar{
\begin{proposition}
\textbf{ - Co-evolution of components. } Morphogenesis processes of a territorial system are an equivalent formulation of the existence of co-evolutive subsystems.
\end{proposition}
}{
\begin{proposition}
\textbf{ - Co-évolution des composantes. } Les processus morphogénétiques d'un système territorial sont une formulation équivalente de l'existence de sous-systèmes co-évolutifs.
\end{proposition}
}



% importance of nws as necessary subcomponents
%  maybe where we diverge from Denise theory ?


\bpar{
Finally we make a key assumption putting real networks at the center of co-evolutive dynamics, introducing their necessity to explain dynamical processes of territorial systems.
}{
Nous formulons finalement une hypothèse clé qui met les réseaux réels au centre des dynamiques co-évolutives, introduisant leur nécessité pour expliquer les processus dynamiques des systèmes territoriaux.
}


\bpar{
\begin{assumption}
\textbf{ - Necessity of Networks. } Network evolution can not be explained only by the dynamics of other territorial components and reciprocally, i.e. co-evolving territorial subsystems include real networks. They can thus be at the origin of regime changes (transition between stationarity regimes) or more dramatic bifurcations in dynamics of the whole territorial system.
\end{assumption}
}{
\begin{assumption}
\textbf{ - Nécessité des réseaux. } L'évolution des réseaux ne peut pas être expliquée simplement par la dynamique des autres composantes territoriales et réciproquement, i.e. les sous-systèmes territoriaux co-évolutifs contiennent les réseaux réels. Ceux-ci peuvent ainsi être à l'origine de changements de régime (transitions entre régimes stationnaires) ou de bifurcations plus conséquentes dans les dynamiques de l'ensemble du système territorial.
\end{assumption}
}


\bpar{
On long time scale, an overall co-evolution has been shown for the French railway network by~\cite{bretagnolle:tel-00459720}. At smaller scales it is less evident (debate on structural effects) but we postulate that co-evolution effects are present at any scale. Regional examples may illustrate that : Lyon has not the same dynamical relations with Clermont than with Saint-Etienne and network connectivity has necessarily a role in that (among intrinsic interaction dynamics and distance). At a smaller scale, we think that effects are even less observable, but precisely because of the fact that co-evolution is stronger and local bifurcations will occur with stronger amplitude and greater frequency than in macroscopic systems where attractors are more stable and stationarity scales greater. We will try to identify bifurcation or phase transitions in toy models, hybrid models and empirical analysis, at different scales, on different case studies and with different ontologies.
}{
Sur de longues échelles temporelles, une co-évolution globale a été montrée pour le systèmes ferroviaire français par~\cite{bretagnolle:tel-00459720}. A de plus petites échelles celle-ci est moins évidente (débat sur les effets structurants) mais nous supposons la présence d'effets co-évolutifs à toutes les échelles. Des exemples régionaux peuvent illustrer ce fait : Lyon n'a pas les mêmes relations dynamiques avec Clermont qu'avec Saint-Etienne, et la connectivité de réseau a nécessairement un rôle à y jouer (parmi les effets des dynamiques intrinsèques des interactions, et de la distance par example). A une plus petite échelle encore, nous partons du principe que les effets sont encore moins observables, mais précisément à cause du fait que la co-évolution est plus forte et les bifurcations locales se produisent avec une plus grande amplitude et une plus grande fréquence que dans les systèmes macroscopiques où les attracteurs sont plus stables et les échelles de stationnarité plus grandes. Nous essayerons d'identifier des bifurcations ou des transitions de phase dans des modèles jouets, des modèles hybrides, et des analyses empiriques, à différentes échelles, sur différents cas d'études et avec différentes ontologies. 
}



\bpar{
One difficulty in our construction is the stationarity assumption. Even if it seems a reasonable assumption on large scales and has already been observed in empirical data~\cite{sanders1992systeme}, we shall verify it in our empirical studies. Indeed, this question is at the center of current research efforts to apply deep learning techniques to geographical systems: \noun{Bourgine} has recently developed a framework to extract patterns of Complex Adaptive Systems\footnote{Using a representation theorem~\cite{knight1975predictive}, any discrete stationary process is a \emph{Hidden Markov Model}. Given the definition of a causal state as $\Pb{future | A} = \Pb{future | B}$, the partition of system states induced by the corresponding equivalence relations allows to derive a \emph{Recurrent Network} that is sufficient to determine the next state of the system, as it is a \emph{deterministic} function of previous state and hidden states~\cite{shalizi2001computational}: $(x_{t+1},s_{t+1}) = F\left[(x_t,s_t)\right]$. The estimation of Hidden States and of the Recurrent Function thus captures through deep learning entirely dynamical patterns of the system, i.e. full information on its dynamics and internal processes.}. The issues are then if the stationarity assumption be tackled through augmentation of system states, and if heterogeneous and asynchronous data can be used to bootstrap long time-series necessary for a correct estimation of the neural network. These issue are related to the stationarity assumption for the first and to non-ergodicity for the second.
}{
Une difficulté dans notre construction est l'hypothèse de stationnarité. Même si cela paraît une hypothèse raisonnable à de grandes échelles et a déjà été observé des des données empiriques~\cite{sanders1992systeme}, nous devrons le vérifier dans nos études empiriques. En effet, cette question est au centre des efforts de recherche courants pour appliquer les techniques d'apprentissage profond aux systèmes géographiques : \noun{Bourgine} a récemment développé un cadre pour extraire des motifs des systèmes complexes adaptatifs\footnote{En utilisant un théorème de représentation~\cite{knight1975predictive}, tout processus stationnaire discret est un \emph{Modèle de Markov Caché}. Etant donné la définition d'un état causal comme $\Pb{future | A} = \Pb{future | B}$, la partition des états du système par la relation d'équivalence correspondantes permet de produire un \emph{Réseau Récurrent} qui est suffisant pour déterminer l'état suivant du système, puisqu'il s'agit d'une fonction \emph{déterministe} des états précédents et des états cachés~\cite{shalizi2001computational} : $(x_{t+1},s_{t+1}) = F\left[(x_t,s_t)\right]$. L'estimation des états cachés et de la fonction récurrente capture ainsi entièrement par apprentissage profond le comportement dynamique du système, i.e. l'information complète sur ses dynamiques et les processus internes.}. Les questions sont ensuite si les hypothèses de stationnarité peuvent être réglés par augmentation des états du système, et si des données hétérogènes et asynchrones peuvent être utilisées pour initialiser des séries temporelles assez longues pour une estimation correcte du réseau de neurones. Ces questions sont reliées à l'hypothèse de stationnarité pour la première et à la non-ergodicité pour la seconde.
}





%----------------------------------------------------------------------------------------

\newpage

\section{A theoretical Framework for the Study of Socio-technical Systems}{Un Cadre Théorique pour l'Etude des Systèmes Sociaux-techniques}

After having set up the thematic theoretical framework, we develop a more general framework in which the previous can enter. At an epistemological level, it is essential to frame generally our directions of research.


\subsection{Introduction}

\subsubsection*{Scientific Context}


\bpar{
The structural misunderstandings between Social Sciences and Humanities on one side, and so-called Exact Sciences on the other side, far from being a generality, seems to have however a significant impact on the structure of scientific knowledge~\cite{2015arXiv151103981H}. In particular, the place of theory (and indeed the signification of this term itself) in the elaboration of knowledge has a totally different place, partly because of the different \emph{perceived complexities}\footnote{We used the term \emph{perceived} as most of systems studied by physics might be described as simple whereas they are intrinsically complex and indeed not well understood~\cite{laughlin2006different}.} of studied objects: for example, mathematical constructions and by extent theoretical physics are \emph{simple} in the sense that they are generally analytically solvable (or at least semi-analytically), whereas Social Science subjects such as humans or society (to give a \emph{clich{\'e}} exemple) are \emph{complex} in the sense of complex systems\footnote{for which no unified definition exists but of which fields of application range broadly from neuroscience to quantitative finance, including e.g. quantitative sociology, quantitative geography, integrative biology, etc.~\cite{newman2011complex}, and for which study various complementary approaches may be applied, such as Dynamical Systems, Agent-based Modeling, Random Matrix Theory}, thus a stronger need of a constructed theoretical (generally empirically based) framework to identify and define the objects of research that are necessarily more arbitrary in the framing of their boundaries, relations and processes, because of the multitude of possible viewpoints: \noun{Pumain} suggests indeed in~\cite{pumain2005cumulativite} a new approach to complexity deeply rooted in social sciences that ``would be measured by the diversity of disciplines needed to elaborate a notion''. These differences in backgrounds are naturally desirable in the spectrum of science, but things can get nasty when playing on overlapping terrains, typically complex systems problematics as already detailed, as the exemple of geographical urban systems has recently shown~\cite{dupuy2015sciences}. Complex System Science\footnote{that we deliberately call that way although there is a running debate on wether it can be seen as a Science in itself or more as a different way to do Science.} is presented by some as a ``new kind of Science''~\cite{wolfram2002new}, and would at least be a symptom of a shift in scientific practices, from analytical and ``exact'' approaches to computational and evidence-based approaches~\cite{arthur2015complexity}, but what is sure is that it brings, together with new methodologies, new scientific fields in the sense of converging interests of various disciplines on transversal questions or of integrated approaches on a particular field~\cite{2009arXiv0907.2221B}.
}{
Les malentendus structurels entre les Sciences Sociales et Humanités d'une part, et les dénommées Sciences Exactes d'autre part, loin d'être une généralité, semble toutefois avoir un impact conséquent sur la structure de la connaissance scientifique~\cite{2015arXiv151103981H}. Plus particulièrement, le rôle de la théorie (et en fait la signification elle-même du terme) dans l'élaboration de la connaissance a une place complètement différente, en partie à cause de différentes \emph{complexités perçues}\footnote{Nous utilisons le terme \emph{perçu} car la plupart des systèmes étudiés en physique peuvent être décrits comme simple alors qu'ils sont intrinsèquement complexe et finalement mal compris~\cite{laughlin2006different}.} des objets étudiés : par exemple, les constructions mathématiques et par extension la physique théorique sont \emph{simples} au sens où elles sont généralement résolubles de manière analytique (ou au moins semi-analytique), tandis que les sujets des Sciences Sociales tels les humains ou la société (pour prendre un exemple préconçu) sont \emph{complexes} au sens de systèmes complexes\footnote{pour lesquels il n'existe pas de définition unifiée mais dont les champs d'application couvrent une étendue allant des neurosciences à la finance quantitative, en passant par exemple par la sociologie quantitative, la géographie quantitative, la biologie intégrative, etc.~\cite{newman2011complex}, et pour l'étude desquels diverses approches complémentaires peuvent être appliquées, comme les Systèmes Dynamiques, la Modélisation Basée Agent, la Théorie des Matrices Aléatoires}, d'où un besoin accru d'une construction théorique (qui se base généralement sur l'empirique) pour identifier et définir qui sont nécessairement plus arbitraires dans la définition de leur limites, relations et processus, de par la multitude des points de vue possibles : \noun{Pumain} suggère en effet dans~\cite{pumain2005cumulativite} une nouvelle approche de la complexité qui serait profondément ancrée dans les sciences sociales et qui serait ``mesurée par la diversité des disciplines nécessaires pour élaborer une notion''. Ces différences de fond sont naturellement bénéfiques pour la diversité scientifique, mais les choses peuvent se corser quand les terrains d'étude se chevauchent, typiquement dans le cas de problématiques liées aux systèmes complexes comme déjà détaillé, comme l'exemple géographique des systèmes urbains a récemment montré~\cite{dupuy2015sciences}. La Science des Systèmes Complexes\footnote{que nous appelons délibérément ainsi même si des débats existent sur le fait de considérer comme une science en elle-même ou comme une façon différente de faire de la science.} est présentée par certains comme ``un nouveau type de science''~\cite{wolfram2002new}, et serait au moins symptomatique d'un changement de paradigme des pratiques, des approches analytiques ``exactes'' vers des approches computationnelles et \emph{evidence-based}~\cite{arthur2015complexity}, mais il est certain que cela permet de faire émerger, conjointement avec de nouvelles méthodologies, des nouveaux champs scientifiques au sens d'intérêts convergents de disciplines variées sur des questions transversales ou d'approches intégrées d'un champ particulier~\cite{2009arXiv0907.2221B}.
}


\subsubsection*{Objectives}{Objectifs}


\bpar{
Within that scientific context, the study of what we will call \emph{Socio-technical Systems}, which we define in a rather broad way as hybrid complex systems including social agents or objects that interact with technical artifacts and/or a natural environment\footnote{geographical systems in the sense of \cite{dollfus1975some} are the archetype of such systems, but that definition may cover other type of systems such as an extended transportation system, social systems taken with an environmental context, complicated industrial systems taken with users, etc.}, lies precisely between social sciences and hard sciences. The example of Urban Systems is the best example, as already before the arrival of approaches claiming to be ``more exact'' than soft approaches (typically by physicists, see e.g. the positioning of~\cite{louf2014scaling}, but also by scientists coming from social sciences such as \noun{Batty}~\cite{batty2013new}), many aspects of urban systems were already in the field of exact sciences, such as urban hydrology, urban climatology or technical aspects of transportation systems, whereas the core of their study relied in social sciences such as geography, urbanism, sociology, economy. Therefore a necessary place of theory in their study: following~\cite{livet2010}, the study of complex systems in social science is an interaction between empirical analysis, theoretical constructions, and modeling.
}{
Dans ce contexte scientifique, l'étude de ce que nous désignons par \emph{Systèmes socio-techniques}, que nous définissons de manière assez large comme des systèmes complexes hybrides qui incluent des agents ou objets sociaux qui interagissent avec des artefacts techniques et/ou un environnement naturel\footnote{les systèmes géographiques au sens de \cite{dollfus1975some} sont l'archetype de tels systèmes, mais cette définition peut couvrir d'autres types de systèmes comme un système de transport étendu, des systèmes sociaux pris dans un contexte environnemental, des systèmes industriels compliqués considérés avec leur utilisateurs, etc.}, se situent précisément entre sciences sociales et sciences dures. L'exemple des systèmes urbains est parmi les meilleurs cas représentatifs, puisque même avant l'arrivée de nouvelles approches prétendant être ``plus exactes'' que les approches des sciences sociales (typiquement par des physiciens, voir e.g. le positionnement de~\cite{louf2014scaling}, mais aussi par des chercheurs venant des sciences sociales comme \noun{Batty}~\cite{batty2013new}), une multitude d'aspects des systèmes urbains étaient déjà traités dans des sciences dures, comme l'hydrologie urbaine, la climatologie urbaine ou les aspects techniques des systèmes de transport, tandis que le centre de leur attention se reposait sur des sciences sociales comme la géographie, l'urbanisme, la sociologie, l'économie. D'où une place nécessaire de la théorie dans leur étude : suivant~\cite{livet2010}, l'étude des systèmes complexes en sciences sociales est une interaction entre analyse empirique, construction théorique et modélisation.
}


\bpar{
We propose in this section to construct a theory, or rather a theoretical framework, that would ease some aspects of the study of such systems. Many theories already exist in all fields related to this kind of problems, and also at higher levels of abstraction concerning methods such as agent-based modeling e.g., but there is to our knowledge no theoretical framework including all of the following aspects that we consider as being crucial (and that can be understood as an informal basis of our theory):
\begin{enumerate}
\item a precise definition and emphasis on the notion of coupling between subsystems, in particular allowing to qualify or quantify a certain degree of coupling: dependence, interdependence, etc. between components.
\item a precise definition of scale, including timescale and scales for other dimensions.
\item as a consequence of the previous points, a precise definition of what is a system.
\item the inclusion of the notion of emergence in order to capture multi-scale aspects of systems.
\item a central place of ontology in the definition of systems, i.e. of the sense in the real world given to its objects\footnote{as already explained before, this positioning along with the importance of structure may be related to Ontic Structural Realism~\cite{frigg2011everything} in further developments.}.
\item taking into account heterogeneous aspects of the same system, that could be heterogeneous components but also complementary intersecting views.
\end{enumerate}
}{
Nous proposons dans cette section de construire une théorie, ou plutôt un cadre théorique, pour faciliter certains aspects de l'étude de tels systèmes. De nombreuses théories existent déjà dans l'ensemble des champs liés à ce type de questionnement, et aussi à de plus haut niveaux d'abstraction concernant des méthodes comme e.g. la modélisation basée agent, mais il n'existe à notre connaissance pas de cadre théorique qui incluraient l'ensemble des points suivants que nous jugeons cruciaux (et qui peuvent être compris comme une base informelle de notre théorie) :
\begin{enumerate}
\item une définition précise et une emphase particulière sur la notion de couplage entre sous-systèmes, en particulier permettant de qualifier ou quantifier un certain niveau de couplage : dépendance, interdépendance, etc. entre composantes.
\item une précise définition de l'échelle, incluant l'échelle temporelle et l'échelle pour d'autres dimensions.
\item en conséquence des points précédents, une définition précise de ce qu'est un système.
\item la prise en compte de la notion d'émergence pour capturer les aspects multi-scalaires des systèmes.
\item une place centrale de l'ontologie dans la définition des systèmes, i.e. du sens dans le monde réel donné à ses objets\footnote{comme déjà expliqué précedemment, ce positionnement combiné à l'importance de la structure pourrait être relié au \emph{Réalisme Structurel Ontologique} dans des approfondissements.}.
\item la prise en compte d'aspects hétérogènes du même système, qui peuvent être des composantes hétérogènes mais aussi des vues qui se croisent de manière complémentaire.
\end{enumerate}
}


\bpar{
The rest of this section is organized as follows: we construct the theory in the following subsection, staying at an abstract level, and propose a first application to the question of co-evolving subsystems. We then discuss positioning regarding existing theories, and possible developments and concrete applications.
}{
La suite de cette section est organisée de la façon suivante : nous construisons la théorie dans la sous-section suivante en restant à un niveau abstrait, et proposons une première application à la question des sous-systèmes co-évolutifs. % TODO : compatiblity with previous fwk ?
Nous discutons ensuite le positionnement au regard de théories existantes, ainsi que les développements possibles et des applications concrètes.
}



\subsection{Construction of the theory}

\subsubsection*{Perspectives and Ontologies}


\bpar{
The starting point of the theory construction is a perspectivist epistemological approach on systems introduced by \noun{Giere}~\cite{giere2010scientific}. To sum up, it interprets any scientific approach as a perspective, in which someone pursues some objective and uses what is called \emph{a model} to reach it. The model is nothing more than a scientific medium. \noun{Varenne} developed~\cite{varenne2010framework} a functional model typology that can be interpreted as a refinement of this theory. Let for now relax this possible precision and use perspectives as proxies of the undefined objects and concepts. Indeed, different views on the same object (being complementary or diverging) have the property to share at least the object in itself, thus the proposition to define objects (and more generally systems) from a set of perspectives on them, that verify some properties that we formalize in the following.
}{
Le point de départ pour construire la théorie est une approche épistémologique perspectiviste des systèmes introduite par \noun{Giere}~\cite{giere2010scientific}. Pour résumer, cette position interprète toute démarche scientifique comme une perspective, au sein de laquelle chacun poursuit certains objectifs et utilise ce qui est appelé \emph{un modèle} pour les atteindre. Le modèle n'est alors rien de plus qu'un medium scientifique. \noun{Varenne} a développé~\cite{varenne2010framework} une typologie fonctionnelle des modèles qui peut être interprété comme un raffinement de cette théorie. Relâchons dans un premier temps cette précision potentielle et utilisons les perspectives comme des approximations des objets et concepts indéfinis. En effet, diverses visions du même objet (pouvant être complémentaires ou divergentes) ont la propriété de partager au moins l'objet lui-même, d'où notre proposition de définir les objets (et plus généralement les systèmes) à partir d'un ensemble de perspectives sur ceux-ci, qui vérifient certaines propriétés que nous formalisons par la suite.
}


\bpar{
A perspective is defined in our case as a dataflow machine $M$ (that corresponds to the model as medium) in the sense of~\cite{golden2012modeling} that gives a convenient way to represent it and to introduce timescales and data, to which is associated an ontology $O$ in the sense of~\cite{livet2010}, i.e. a set of elements each corresponds to an entity (which can be an object, an agent, a process, etc.) of the real world. We include only two aspect (the model and the objects represented) of Giere's theory, making the assumption that purpose and producer of the perspective are indeed contained in the ontology if they make sense for studying the system. % TODO note here : ex méthodes quali, influence du sondeur.
}{
Une perspective est définie dans notre cas comme une \emph{Dataflow Machine} $M$ (qui correspond au model comme medium) au sens de~\cite{golden2012modeling} qui fournit un moyen adapté de représenter un modèle et d'y associer échelle de temps et données,  à laquelle on associe un ontologie $O$ au sens de~\cite{livet2010}, i.e. un ensemble d'éléments qui correspondent à une entité (qui peut être un objet, un agent, un processus, etc.) du monde réel. Nous incluons seulement ces deux aspects (le modèle et les objets représentés) de la théorie de Giere, en faisant l'hypothèse que le but et le producteur de la perspective sont en fait contenus dans l'ontologie s'ils font sens pour l'étude du système.
}


\bpar{
\begin{definition}
A \emph{perspective on a system} is given by a dataflow machine $M = (i,o,\mathbb{T})$ and an associated ontology $O$. We assume that the ontology can be decomposed into atomic elements $O=(O_j)_j$.
\end{definition}
}{
\begin{definition}
Une \emph{perspective sur un système} est donnée par une \emph{Dataflow Machine} $M = (i,o,\mathbb{T})$ et une Ontologie associée $O$. Nous supposons que l'ontologie peut être décomposée de manière discrete en éléments atomiques $O=(O_j)_j$.
\end{definition}
}


\bpar{
The atomic elements of the ontology can be particular elements such as agents or components of the system, but also processes, interactions, states, or concepts for example. The ontology can be seen as the exhaustive and rigorous description of the content of the perspective. The assumption of a dataflow machine implies that possible inputs and outputs can be quantified, what is not necessarily restrictive to quantitative perspectives, as most of qualitative approaches can be translated into discrete variables as soon as the set of possibles is known or assumed. 
}{
Les éléments atomiques de l'ontologie peuvent être des constituants particuliers du systèmes, comme des agents ou des composantes, mais aussi des processus, interactions, états ou concepts par exemple. L'ontologie peut être vue comme la description exhaustive et rigoureuse du contenu de la perspective. L'hypothèse d'une \emph{Dataflow Machine} implique que les entrées et sorties potentielles peuvent être quantifiées, ce qui n'est pas nécessairement restrictif aux perspectives quantitatives, puisque la plupart des approches qualitatives peuvent être traduites en variables discrètes à partir du moment où l'ensemble des possibles est connu ou supposé.
}


\bpar{
The system is then defined ``reversely'', i.e. from a set of perspectives on what would constitute then the system:

% def of a system as a set of perspectives.
\begin{definition}
A \emph{system} is a set of \emph{perspectives on a system}: $S = (M_i,O_i)_{I\in I}$, where $I$ may be finite or not.
\end{definition}

We denote by $\mathcal{O} = (O_{j,i})_{j,i\in I}$ the set of all elements within ontologies.
}{
Nous définissons alors le système de manière ``réciproque'', i.e. à partir d'un ensemble de perspectives sur ce qui constitue alors le système :

\begin{definition}
Un \emph{système} est un ensemble de \emph{perspectives sur un système}: $S = (M_i,O_i)_{I\in I}$, où $I$ n'est pas nécessairement fini.
\end{definition}

Nous désignons par $\mathcal{O} = (O_{j,i})_{j,i\in I}$ l'ensemble des elements dans les ontologies.
}


\bpar{
Note that at this level of construction, there is not necessarily any structural consistence in what we call a system, as given our broad definition could allow for example to consider as a system a perspective on a car together with a perspective on a system of cities what makes reasonably no sense at all. Further definitions and developments will allow to be closer from classical definition of a system (interacting entities, designed artifacts, etc.). The same way, the definition of a subsystem will be given further. The introduced elements of our approach help to tackle so far points three, five and six of the requirements.
}{
On peut noter qu'à ce stade de la construction, il n'existe pas nécessairement de cohérence structurelle sur ce qu'on appelle un système, puisque étant donné notre définition très large nous pourrions par exemple considérer un système comme une perspective sur un véhicule conjointement à une perspective sur un système de villes, ce qui ne fait pas raisonnablement sens. Des définitions approfondies et développements doivent permettre de se rapprocher des définitions classiques d'un système (entités en interaction, artefacts précisément définis, etc. ). De la même manière, la définition d'un sous-système sera donnée plus loin. Les éléments de l'approche déjà introduits permettent jusqu'ici de répondre aux points trois, cinq et six des recommandations.
}



\paragraph{Precision on the recursive aspect of the theory}{Précision sur l'aspect récursif de la théorie}


\bpar{
One direct consequence of these definitions must be detailed: the fact that they can be applied recursively. Indeed, one could imagine taking as perspective a system in our sense, therefore a set of perspectives on a system, and do that at any order. If ones takes a system in any classical sense, then the first order can be understood as an epistemology of the system, i.e. the study of diverse perspectives on a system. A set of perspectives on related systems may in some conditions be a domain or a field, thus a set of perspectives on various related systems the epistemology of a field. These are more analogies to give the idea behind the recursive character of the theory. It is indeed crucial for the meaning and consistence of the theory because of the following arguments:
\begin{itemize}
\item The choice of perspectives in which a system consists is necessarily subjective and therefore understood as a perspective, and a perspective on a system if we are able to build a general ontology.
\item We will use relations between ontologies in the following, which construction based on emergence is also subjective and seen as perspectives.
\end{itemize}
}{
Une conséquence directe de ces définitions doit être détaillée : le fait qu'elles peuvent être appliquées de manière récursive. En effet, on peut imaginer prendre comme perspective un système dans notre sens, c'est à dire un ensemble de perspectives sur un système, et le faire à tout ordre. Si on considère un système à n'importe quel sens classique, alors le premier ordre peut être interprété comme une épistémologie du système, i.e. l'étude de perspectives sur un système. Une ensemble de perspectives sur des systèmes en relation peut sous certaines conditions être un domaine ou un champ d'étude, et donc un ensemble de perspectives sur divers systèmes l'épistémologie d'un champ. % TODO phrase incompréhensible
On peut proposer des analogies supplémentaires pour traduire l'idée derrière le caractère récursif de la théorie. C'est en effet crucial pour la signification et la cohérence de la théorie, notamment pour les raisons suivantes :
\begin{itemize}
\item Le choix des perspectives qui constituent un système est nécessairement subjectif et peut donc être compris comme une perspective en lui-même, et ainsi une perspective sur un système si l'on est en mesure de construire une ontologie générale. % TODO second point unclear as fuck
\item Nous utiliserons des relations entre ontologies par la suite, dont la construction est basée sur l'émergence est également subjective et vue comme perspectives.
\end{itemize}
}




\subsubsection*{Ontological Graph}

% construction of the ontological graph / canonical tree decomposition
%  : pb with relation in the ontological graph : weak or strong emergence ?


\bpar{
We propose then to capture the structure of the system by linking ontologies. This approach could eventually be linked to structural realism epistemological positioning~\cite{frigg2011everything} as knowledge of the world is partly contained here in structure of models. % precise here epistemological positioning, we may be clearly within a structural realist positioning !!
 Therefore, we choose to emphasize the role of emergence as we believe that it may be one practical minimalist way to capture quite well complex systems structure\footnote{what of course can not been presented as a provable claim as it depends on system definition, etc.}. We follow on that point the approach of \noun{Bedau} on different type of emergences, in particular his definition of weak emergence given in~\cite{bedau2002downward}. Let recall briefly definitions we will use in the following. \noun{Bedau} starts from defining emerging properties and then extends it to phenomena, entities, etc. The same way, our framework is not restricted to objects or properties and wraps thus the generalized definitions into emergence between ontologies. We will apply the notion of emergence under the two following forms\footnote{the third form \noun{Bedau} recalls, \emph{Strong emergence} will not be used, as we need only to capture dependance and autonomy, and weak emergence is more satisfying in terms of complex systems, as it does not assume ``irreducible causal powers'' to objects of upper scales at a given level. Nominal emergence is used to capture inclusion between ontologies.}:
\begin{itemize}
\item \emph{Nominal emergence}: one ontology $O'$ is included in an other $O$ but the aspect of $O$ that is said to be nominally emergent regarding $O'$ does not depend on $O'$.
\item \emph{Weak emergence}: one part of an ontology $O$ can be \emph{computationnaly} derived by aggregation of elements and interactions between elements of an ontology $O'$.
\end{itemize}
}{
Nous proposons ensuite la structure du système en reliant les ontologies. Cette approche pourrait éventuellement être mise en perspective par rapport à un positionnement épistémologique de réalisme structurel~\cite{frigg2011everything} puisqu'une connaissance du monde est ici partiellement contenue dans la structure des modèles. Pour cette raison, nous faisons le choix d'appuyer le rôle de l'émergence, suivant l'intuition qu'il pourrait s'agir d'un outil pratique minimaliste pour capturer de façon raisonnable la structure d'un système complexe\footnote{ce qui bien sûr ne peut être formulé comme une affirmation prouvable car cela dépendra de la définition d'un système, etc.}. Nous prenons pour cet aspect le positionnement de \noun{Bedau} sur les différents types d'émergence, en particulier se définition de l'émergence faible donnée dans~\cite{bedau2002downward}. Rappelons brièvement les définitions que nous utiliserons par la suite. \noun{Bedau} commence par définir les propriétés émergentes puis étend le concept aux phénomènes, entités, etc. De la même manière, notre cadre n'est pas restreints aux objets ou propriétés et inclut ainsi les définitions généralisées comme lien entre ontologies. Nous appliquons la notion d'émergence sous les deux formes suivantes\footnote{la troisième forme rappelée par \noun{Bedau}, \emph{l'émergence forte}, ne sera pas utilisée, car nous avons besoin de capturer rien de plus des relations de dépendance et d'autonomie, et l'émergence faible est plus adéquate en termes de systèmes complexes, puisqu'elle n'assume pas ``des pouvoirs causaux irréductibles'' aux objets des échelles supérieures à un niveau donné. L'émergence nominale est utilisée pour capturer des relations d'inclusion entre les ontologies.} :
\begin{itemize}
\item \emph{Emergence nominale} : une ontologie $O'$ est inclue dans une autre ontologie $O$ mais l'aspect de $O$ qui est dit nominalement émergent en rapport à $O'$ ne dépend pas de $O'$.
\item \emph{Emergence faible} : une partie d'une ontologie $O$ peut être dérivée \emph{de manière computationnelle} par agrégation et interactions entre les éléments d'une ontologie $O'$.
\end{itemize}
}


\bpar{
As developed before, the presence of emergence, and especially weak emergence, will consist in itself in a perspective. It can be conceptual and postulated as an axiom within a thematic theory, but also experimental if clues of weak emergence are effectively measured between objects. In any case, the relation between ontologies must be encoded within an ontology, which was not necessarily introduced in the initial definition of the system.
}{
Comme développé précédemment, la présence d'émergence, et spécifiquement d'émergence faible, constitue une perspective en soi. Elle peut être conceptuelle et postulée comme un axiome dans une théorie thématique, mais aussi expérimentale si des traces d'émergence faible sont effectivement mesurées entre objets. Dans tous les cas, la relation entre ontologies doit être encodée dans une ontologie, ce qui n'était pas nécessairement introduit dans la définition initiale d'un système.
}

% Observation : -
%  - Ontologies are sets -> relation between subsets.
%  - include emergence relations between different perspectives of the system ? would imply a coupling ontology ? -> YES - system is not only a set but also a relation between its elements -> introduce the 'coupling ontology' ; or assumes there exists one ?


\bpar{
We make therefore the following assumption for next developments:
\begin{assumption}
A system can be partially structured by extending it with an ontology that contains (not necessarily only) relations between elements of ontologies of its perspectives. We name it the \emph{coupling ontology} and assume its existence in the following. We assume furthermore its atomicity, i.e. if $O$ is in relation with $O'$, then any subsets of $O,O'$ can not be in relation, what is not restrictive as a decomposition into several independent subsets ensures it if it is not the case.
\end{assumption}
}{
Nous faisons pour cette raison l'hypothèse suivante importante par la suite :
\begin{assumption}
Un système peut être partiellement structuré par son extension avec une ontologie qui contient (pas nécessairement uniquement) des relations entre les éléments des ontologies de ses perspectives. Nous la désignons \emph{ontologie de couplage} et supposons son existence par la suite. Nous postulons de plus son atomicité, i.e. si $O$ est en relation avec $O'$, alors tout sous-ensemble de $O,O'$ ne peuvent être en relation, ce qui n'est pas contraignant puisqu'une décomposition en des sous-ensembles indépendants assurera cette propriété si elle n'est pas vérifiée initialement. % TODO pas sûr d'avoir compris.
\end{assumption}
}

\bpar{
It allows to exhibit emergence relations not only within a perspective itself but also between elements of different perspectives. We define then pre-order relations between subsets of ontologies:
}{
Cela nous permet d'exhiber des relations d'émergence pas seulement au sein d'une perspective elle-même, mais également entre les éléments de différentes perspectives. Nous définissons ensuite des relations de pré-ordre entre les sous-ensemble des ontologies :
}


\bpar{
% order relations between ontologies
% first recall Bedau's paper. 
% check equivalence relation ; pre-order <- relations are only pre-order ?
\begin{proposition}
The following binary relationships are pre-orders on $\mathcal{P(O)}$:
\begin{itemize}
\item Emergence (based on Weak Emergence): $O' \preccurlyeq O$ if and only if $O$ weakly emerges from $O'$.
\item Inclusion (based on Nominal Emergence): $O' \Subset O$ if and only if $O$ nominally emerges from $O'$.
\end{itemize}
\end{proposition}
}{
\begin{proposition}
Les relations binaires suivantes sont des pré-ordres sur $\mathcal{P(O)}$ :
\begin{itemize}
\item Emergence (basée sur l'émergence faible) : $O' \preccurlyeq O$ si et seulement si $O$ émerge faiblement de $O'$.
\item Inclusion (basée sur l'émergence nominale) : $O' \Subset O$ si et seulement si $O$ émerge nominalement de $O'$.
\end{itemize}
\end{proposition}
}



\bpar{
\begin{proof}
With the convention that it can be said that an object emerges from itself, we have reflexivity (if such a convention seems absurd, we can define the relationships as \emph{$O$ emerges from $O'$ or $O=O'$ }). Transitivity is clearly contained in definitions of emergence.
\end{proof}
}{
Avec la convention qu'il peut être admis qu'un objet émerge de lui-même, on a réflexivité (si une telle convention parait absurde, on peut définir les relations comme \emph{$O$ émerge de $O'$ ou $O=O'$}). La transitivité est clairement contenue dans la définition de l'émergence.
}


\medskip

\bpar{
Note that the inclusion relation is more general than an inclusion between sets, as it translates an inclusion ``inside'' the elements of the ontology.
}{
Notons que la relation d'inclusion est plus général qu'une inclusion entre ensembles, puisqu'elle traduit une inclusion ``au sein'' des éléments de l'ontologie. \todo{give an example}
}



\bpar{
These relations are the basis for the construction of a graph called the \emph{ontological graph} :

% definition of the ontological graph
% ! beware to put only neighbor relations within the graph
% and to reconstruct by induction subsets at any level ?
\begin{definition}
The \emph{ontological graph} is constructed by induction the following way:
\begin{enumerate}
\item A graph is constricted, with vertices elements of $\mathcal{P(O)}$ and edges of two types: $E_W = \{(O,O') | O' \preccurlyeq O \}$ and $E_N = \{(O,O') | O' \Subset O \}$
\item Nodes are reduced\footnote{the reduction procedure aims to delete redundancy, keeping an entity at the higher level at which it exists.} by: if $o \in O,O'$ and ($O' \preccurlyeq O$ or $O' \Subset O$) but not ($O \preccurlyeq O'$ or $O \Subset O'$), then $O' \leftarrow O' \setminus o$
\item Nodes with intersecting sets are merged, keeping edges linking merged nodes. This step ensures non-overlapping nodes.
\end{enumerate}
\end{definition}
}{
\begin{definition}
Le \emph{graphe ontologique} est construit par induction de la manière suivante :
\begin{enumerate}
\item Un graphe est construit, avec pour noeuds des éléments de $\mathcal{P(O)}$ et des liens de deux types : $E_W = \{(O,O') | O' \preccurlyeq O \}$ et $E_N = \{(O,O') | O' \Subset O \}$
\item Les noeuds sont réduits\footnote{la procédure de réduction vise à supprimer la redondance, gardant une entité au plus haut niveau où elle existe.} par : si $o \in O,O'$ et ($O' \preccurlyeq O$ ou $O' \Subset O$) mais pas ($O \preccurlyeq O'$ or $O \Subset O'$), alors $O' \leftarrow O' \setminus o$
\item Les noeuds avec des ensemble se recoupant sont fusionnés, en gardant les liens liant des noeuds fusionnés. Cette étape assure des noeuds ne se recoupant pas. \todo{what if only partially overlapping ? this point is not clear}
\end{enumerate}
\end{definition}
}



% definition of a subsystem in the large sense ? -> done after treeing the graph
% can be done only if reconstruction is possible.



\subsubsection*{Minimal Ontological Tree}

% theorem : tree construction with the tree-bag decomposition theorem ; and show trivially that loops are at the same scale
% first get a connected component of the graph -> here the consistence is captured : if ontologies have nothing to do at all, then it cant be the same system.


\bpar{
The topological structure of the graph, that contains in a way the \emph{structure of the system}% positioning regarding structural realism.
, can be reduced into a minimal tree that captures hierarchical structure essential to the theory.
}{
La structure topologique du graphe, qui contient en un sens la \emph{structure du système}, peut être réduite en un arbre minimal qui capture la structure hiérarchique essentielle pour la théorie.
}



\bpar{
We need first to give consistence to the system:

\begin{definition}
A consistent part of the ontological graph is a weakly connected component of the graph. We assume for now to work on a consistent part.
\end{definition}
}{
Nous devons d'abord donner cohérence au système :
\begin{definition}
Une partie cohérente du graphe ontologique est une composante faiblement connectée du graphe. Nous assumons pour la suite travailler sur une partie cohérente.
\end{definition}
}



% rq : consistent system ? -> would need to reconstruct perspectives ? possible ? under certain assumption ? 'decoupling ontology' ? -> need to be worked harder.


\bpar{
The notion of consistent system, together with subsystem or nodes timescales that will be defined later, requires to reconstruct perspectives from ontological elements, i.e. the inverse operation of what was done in our deconstruction procedure.
}{
La notion de système cohérent, ainsi que de sous-système ou d'échelle de temps des noeuds qui seront définies par la suite, nécessite de reconstruire des perspectives à partir des éléments ontologiques, i.e. l'opération inverse de ce qui a été fait dans notre procédure de deconstruction. \todo{what is deconstructivism ; position our approach in regard ? Feyerabend as a confirmation ?}
}



\bpar{
\begin{assumption}
There exists $\mathcal{O}' \subset \mathcal{P(O)}$ such that for any $O \subset \mathcal{O}'$, there exists a corresponding dataflow machine $M$ such that the corresponding perspective is consistent with initial elements of the system (i.e. machines are equivalent on ontology overlaps). If $\Phi : M \mapsto O$ is the initial mapping, we denote this extended reciprocal construction by $M' = \Phi^{<-1>}(O)$.
\end{assumption}
}{
\begin{assumption}
Il existe $\mathcal{O}' \subset \mathcal{P(O)}$ tel que pour tout $O \subset \mathcal{O}'$, il existe une \emph{Dataflow Machine} $M$ correspondante telle que la perspective correspondante est cohérente avec les éléments initiaux du système (i.e. les machine sont équivalentes sur les parties communes des ontologies). Si $\Phi : M \mapsto O$ est la correspondance initiale, nous notons cette construction réciproque étendue par $M' = \Phi^{<-1>}(O)$.
\end{assumption}
% not clear also
}



\paragraph{Remark.}


\bpar{
This assumption could eventually be changed into a provable proposition, assuming that the coupling ontology indeed corresponds to a coupling perspective, which dataflow machine part is consistent with coupled entities. Therein, the decomposition postulate of~\cite{golden2012modeling} should allow to identify basic components corresponding to each element of the ontology, and then construct the new perspective by induction. We find however these assumptions too restrictive, as for example various ontological elements may be modeled by an irreducible machine, as a differential equations with aggregated variables. We prefer to be less restrictive and postulate the existence of the reverse mapping on some sub-ontologies, that should be in practice the ones where couplings can be effectively modeled.
}{
Cette hypothèse pourrait éventuellement être changée en une proposition prouvable, en supposant que l'ontologie de couplage correspond effectivement à une perspective de couplage, dont la composante \emph{Dataflow Machine} est cohérente avec les entités couplées. Ainsi, le postulat de décomposition de~\cite{golden2012modeling} devrait permettre d'identifier des composantes de base correspondantes à chaque élément de l'ontologie, et construire ainsi la nouvelle perspective par induction. Nous trouvons toutefois ces hypothèses trop restrictives, puisque par exemple divers éléments de l'arbre ontologique peuvent être modélisés par la même machine irréductible, à l'image d'une équation différentielle aux variables agrégées. Nous préférons être moins restrictifs et postuler l'existence de la correspondance inverse sur certaines sous-ontologies, qui devraient être en pratique celles sur lesquelles le couplage peut effectivement être modélisé.
}



\bpar{
Given this assumption, we can define the consistent system as the reciprocal image of the consistent part of the ontological graph. It ensures system connectivity what is a requirement for tree construction.
}{
Grace à l'hypothèse ci-dessus, on peut définir le système cohérent comme l'image réciproque de la partie cohérente du graphe ontologique. Cela permet la connectivité du système qui est un pré-requis pour la construction de l'arbre. 
}


\bpar{
\begin{proposition}
The tree decomposition of the ontological graph in which nodes contains strongly connected components is unique. The reduced tree, that corresponds to the ontological graph in which strongly connected components have been merged with edges kept, is called the \emph{Minimal Ontological Tree}.
\end{proposition}
}{
\begin{proposition}
La décomposition arborescente du graphe ontologique % not a tree decomposition in the sens of node bags ?
 dans laquelle les noeuds contiennent les composantes fortement connexes est unique. L'arbre réduit, qui correspond au graphe ontologique les composantes fortement connexes ont été fusionnées et les liens gardés, est nommé \emph{Arbre Ontologique Minimal}.
\end{proposition}
}


\bpar{
\begin{proof}
(sketch of) The unicity is obtained as nodes are fixed as strongly connected components. It is trivially a tree decomposition as in a directed graph, strongly connected components do not intersect, thus the consistence of the decomposition.
\end{proof}
}{
\begin{proof}
(esquisse) L'unicité découle de la définition univoque puisque les noeuds sont fixés comme les composantes fortement connexes. Il s'agit trivialement d'une décomposition en arbre puisque dans un graphe dirigé, les composantes fortement connexes ne se recoupent pas, d'où la cohérence de la décomposition. % argumentation bizarre 
\end{proof}
}



\bpar{
Any loop $O \rightarrow O' \rightarrow \ldots \rightarrow O$ in the ontological graph assumes that all its elements are equivalent in the sense of $\preccurlyeq$. This equivalence loops should help to define the notion of strong coupling as an application of the theory (see applications).
}{
Toute boucle $O \rightarrow O' \rightarrow \ldots \rightarrow O$ dans le graphe ontologique suppose que tous ses éléments sont équivalent au sens de $\preccurlyeq$. % def de la relation d'équivalence ?
Ces boucles d'équivalence devrait aider à définir la notion de couplage fort comme une application de la théorie (voir applications).
}

\todo{different approaches to coupling / coupling to a certain degree using Kolmogorov etc : specific section or insert here ?}


\medskip


\bpar{
The Minimal Ontological Tree (MOT) is a tree in the undirected sense but a forest in the directed sense. Its topology contains a sort of system hierarchy. Consistent subsystems are defined from the set $\mathcal{B}$ of branches of the forest, as $(\Phi^{<-1>}(\mathcal{B}),\mathcal{B})$. The timescale of a node, and by extension of a subsystem, is the union of timescales of corresponding machines. Levels of the tree are defined from root nodes, and the emergence relations between nodes implies a vertical inclusion between timescales.
}{
L'Arbre Minimal Ontologique (MOT) est un arbre au sens non-dirigé, mais une forêt au sens dirigé. Sa topologie contient une certaine représentation des hiérarchies du système. Les sous-systèmes cohérents sont définis à partir de l'ensemble $\mathcal{B}$ des branches de la forêt, comme $(\Phi^{<-1>}(\mathcal{B}),\mathcal{B})$. L'échelle de temps d'un noeud, et par extension d'un sous-système, est l'union est échelles de temps des machines correspondantes. Les niveaux de l'arbre sont définis à partir des noeuds racine, et les relations d'émergence entre les noeuds implique une inclusion verticale entre échelles de temps.
}


% check shortcuts reduction in graph construction : if O < O' < O'' , then O -> O'' is not an edge, must take the longest path.




\subsubsection*{Action on Data}

% this part should define the link with datasets
%  -> can use more the dataflow machine structure
%  -> try to set up a consistent algebraic structure with composition = coupling of models.
%  -> classes of action to define coupling strength // link with kolmogorov complexity
% -- clarify purpose of this part : define level of accurcay of perspectives ? define coupling ? use results thanks to algebraic structure ? --

% one dataset / dataspace or a set of potential datasets ?
%  -> using mappings, datasets defined ?

De la même manière que les actions de groupes permettent de donner structure à l'utilisation d'un groupe sur un ensemble (généralement de données), nous ajoutons à la théorie l'aspect essentiel de relation à la réalité par une action des noeuds de l'arbre ontologique sur des ensembles de données.





\subsubsection*{Scales}


\emph{
Finally, we propose to define scales associated to a system. Following~\cite{manson2008does}, an epistemological continuum of visions on scale is a consequence of differences between disciplines in the way we developed in the introduction. This proposition is indeed compatible with our framework, as the construction of scales for each level of the ontological tree results in a broad variety of scales.
}{
Enfin, nous proposons de définir les échelles associées à un système. Suivant~\cite{manson2008does}, un continuum épistémologique de visions sur l'échelle est une conséquence des différences propres à chaque discipline, comme nous avons développé en introduction. Cette proposition est en fait compatible avec notre cadre, puisque la construction d'échelles pour chaque niveau de l'arbre ontologique résulte en une grande variété d'échelles.
}



\bpar{
Let $(M,O)$ a subsystem and $\mathbb{T}$ the corresponding timescale. We propose to define the ``thematic scale'' (for example spatial scale) assuming a representation theorem, i.e. that an aspect (thematic aspect) of the machine can be represented as a dynamic state variable $\vec{X}(t)$. Assuming a scale operator\footnote{that can be of various nature: extent, probabilistic extent, spectral scales, stationarity scales, etc.} $\norm{\cdot}_{S}$ and that the state variable has a certain level of differentiability, the \emph{thematic scale} if defined as $\norm{(d^k \vec{X}(t))_k}_S$.
}{
Soit $(M,O)$ un sous-système et $\mathbb{T}$ l'échelle de temps correspondante. Nous proposons de définir ``l'échelle thématique'' (par exemple l'échelle spatiale) en supposant un théorème de représentation, i.e. qu'un aspect (aspect thématique) de la machine peut être représenté par une variable d'état dynamique $\vec{X}(t)$. Etant donné un opérateur d'échelle\footnote{qui peut être de nature variée : étendue, étendue probabiliste, échelles spectrales, échelles de stationnarité, etc.} $\norm{\cdot}_{S}$ et que la variable d'état est différentiable à un certain niveau, \emph{l'échelle thématique} est définie par $\norm{(d^k \vec{X}(t))_k}_S$.
}

\subsection{Application}

\subsubsection*{The particular case of geographical systems}


%%
%  Observation :
%   ontologies could evolve in time ? they do -> cf Lucie's framework on spatio-temporal functional decomposition : each atomic unit can have a different ontology. not an issue : connects ontologies through a superior layer representing the object through all temporal extent ; the object nominally emerges from each temporal ontology.


\bpar{
In~\cite{dollfus1975some} % ~ same def of system ; introduce structure in context ? link to explanation ?
 \noun{Durand-Dast{\`e}s} proposes a definition of geographical structure and system, structure would be the spatial container for systems viewed as complex open interacting systems (elements with attributes, relations between elements and inputs/outputs with external world). For a given system, its definition is a perspective, completed by structure to have a system in our sense. Depending on the way to define relations, it may be more or less easy to extract ontological structure.
 %\textit{Note : find typical emergence clues in standard relational formalizations ? would guide the application of the theory.}
}{
Dans~\cite{dollfus1975some}, \noun{Durand-Dast{\`e}s} introduit une définition des systèmes et structures géographiques, la structure étant le contenant spatial des systèmes vus comme des systèmes complexes ouverts en interaction (donné par ses éléments et leur attributs, les relations entre éléments et les entrée/sorties avec le monde extérieur). Pour un système donné, sa définition est une perspective, complété par la structure pour avoir un système selon notre sens. Selon la manière dont les relations sont définies, cela peut être plus ou moins aisé d'extraire la structure ontologique.
}

% TODO Reflexion :
%    - théorie de Levy geotypes and shit : have a look ?
%    - niveau de spécification/précision des systèmes formels : dans quel mesure est on "exact" dans la description, que veut dire une ambiguité, est-ce scientifique si ambiguité ?
%    - exemple d'application de ce fwk ? -> trouver un couplage quanti-quali


\subsubsection*{Modularity and co-evolving subsystems}


\bpar{
For the example of Urban Systems, urban evolutionary theory enters this framework using our previous thematic theory. The decomposition into uncorrelated subsystems yields precisely strongly coupled components as co-evolving components. The correlation between subsystems should be positively correlated with topological distance in the tree. If we define elements of a node before merging as \emph{strongly coupled elements}, in the case of dynamic ontologies, it provides a definition of \emph{co-evolution} and co-evolving subsystems equivalent to the thematic definition.
}{
Pour l'exemple des systèmes urbains, la théorie évolutive des villes entre dans ce cadre en utilisant notre théorie thématique développée dans la section précédente. La décomposition en sous-systèmes décorrélés fournit précisément des composantes fortement couplées comme des composantes en co-évolution.
}



\subsection{Discussion}


\paragraph{Link with existing frameworks}

A link with the Cottineau-Chapron framework for multi-modeling~\cite{10.1371/journal.pone.0138212} may be done in the case they add the bibliographical layer, which would correspond to the reconstruction of perspectives. \cite{reymond2013logique} proposes the notion of ``interdisciplinary coupling'' what is close to our notion of coupling perspectives. A correspondance with System of Systems approaches (see e.g. \cite{luzeaux2015formal} for a recent general framework englobing system modeling and system description) may be also possible as our perspectives are constructed as dataflow machines, but with the significant difference that the notion of emergence is central.


\paragraph{Contributions to the study of complex systems}

\begin{itemize}
\item We do not claim to provide a theory of systems (beware of cybernetics, systemics etc. that could not model everything), but more a framework to guide research questions (e.g. in our case the direct outcomes will be quantitative epistemology that comes from system construction as perspectives research ; empirical to construct robust ontologies for perspectives ; targeted thematic to unveil causal relationship/emergence for construction of ontological network ; study of coupling as possible processes containing co-evolution ; study of scales ; etc.). It may be understood as meta-theory which application gives a theory, the thematic theory developed before being a specific implementation to territorial networked systems.
\item We Emphasize the notion of socio-technical system, crossing a social complex system approach (ontologies) with a description of technical artifacts (dataflow machines), taking the ``best of both worlds''.
\end{itemize}




\subsection{Research Directions}

We can draw from the construction of this theoretical framework a set of research directions, that give a general line on how trying to answer to research questions asked after the thematic theory construction.

\begin{enumerate}
\item The perspectivist approach implies a broad understanding of existing perspectives on a system, and of possibility of coupling between them ; thus an emphasis on applied epistemology, i.e. \textbf{Algorithmic Systematic Review} (exploration of the knowledge space), \textbf{Disciplines Mapping}(extraction of its structure) and \textbf{Datamining for Content Analysis}(refinement at the atomic level in scientific knowledge) that correspond to the three sections of chapter~\ref{ch:quantepistemo}.
\item At a finer level of particularization, the knowledge of perspectives means \textbf{Knowledge of stylized facts}, i.e. empirical analysis of cases studies. These are the object of chapter~\ref{ch:empirical}.
\item The emphasis on coupled subsystems at different scales implies a deep understanding of coupling mechanisms, thus the need of methodological and technical developments : \textbf{Methods for Statistical Control}, \textbf{Methods for Model Exploration}, \textbf{Theoretical Study of Coupling}, \textbf{Multi-Modeling}, of which some are developed and other proposed in the methodological chapter~\ref{ch:methodology}.
\item Furthermore, the possibility of hidden elements within the ontology implies the test for causal relations and intermediate processes at the origin of emergence, thus e.g. the exploration of new paradigms such as role of governance within complex models as done in chapter~\ref{ch:complexmodels}.
\item Finally, the idea behind system structure contained within the ontological forest is a large set of coupled models for a given system : it means that a proper system definition (i.e. thematic problematization and exploration) and construction should yield to a structured family of models : parallel branches can be different implementations of the same process or various processes trying to explain the emerging ontology ; therefore the final objective of a family of models tackling the thematic question.
\end{enumerate}










