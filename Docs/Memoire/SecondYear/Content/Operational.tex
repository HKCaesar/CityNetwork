


% Chapter 

\chapter{A Roadmap for an Operational Family of Models of Coevolution}{Vers des Modèles Opérationnels de Coevolution} % Chapter title

\label{ch:operational}

%----------------------------------------------------------------------------------------

As previously stated, one of our principal aims is the validation of the network necessity assumption, that is the differentiating point with a classic evolutive urban theory. To do so, toy-model exploration and empirical analysis will not be enough as hybrid models are generally necessary to draw effective and well validated conclusions. We briefly give an overview of planned work in the following, that will be the conclusion of this Memoire.



%----------------------------------------------------------------------------------------



\section{Objectives}{Objectifs}


We expect to product \emph{models of coevolution}, \comment{(Florent) expliciter la différence avec ce que tu as fait jusque là}
 with the emphasis on processes of coevolution, to directly confront the theory. They will be necessary a flexible family because of the variety of scales and concrete cases we can include and we already began to explore in preliminary studies. Processes already studied can serve either as a thematic bases for a reuse as building bricks in a multi-modeling context, or as methodological tools such as synthetic data generator for synthetic control. Finally, we mean by operational models hybrid models, in the sense of semi-parametrized or semi-calibrated on real datasets or on precise stylized facts extracted from these same datasets. This point is a requirement to obtain a thematic feedback on geographical processes and on theory.


%----------------------------------------------------------------------------------------

\section{Case Studies}{Cas d'étude}

Currently we expect to work on the following case studies to build these hybrid models :

\begin{itemize}
\item Dynamical data for Bassin Parisien should allow to parametrize and calibrate a model at this temporal and spatial scale.
\item On larger scales, South African dataset of \noun{Baffi} will along empirical analysis also be used to parametrize hybrid co-evolution models.
\item A possibility that is not currently set up (and that may however be difficult because of a disturbing closed-data policy among a frightening large number of scientists !) is the exploitation of French railway growth dataset (with population dataset) used in~\cite{bretagnolle:tel-00459720}, that would also provide an interesting case study on other regimes, scales and transportation mode.
\end{itemize}



%----------------------------------------------------------------------------------------



\section{Roadmap}{Feuille de Route}


We give the following (non-exhaustive and provisory) roadmap for modeling explorations (theoretical and empirical domains being still explored conjointly) :

\begin{enumerate}
\item Complete the exploration of independent and weak coupled urban growth and network growth processes (all models presented in chapter~\ref{ch:modeling}), in order to know precisely involved mechanisms when they are virtually isolated, and to obtain morphogenesis scales.
\item Go further into the exploration of toy-model of non conventional processes such as governance network growth heuristic to pave the road for a possible integration of such modules in hybrid models.
\item Build a Marius-like generic infrastructure that implement the theory in a family of models that can be declined into diverse case studies.
\item Launch it and adapt it on these case studies.
\end{enumerate}

Next steps would be too hypothetical if formulated, we propose thus to proceed iteratively in our construction of knowledge and naturally update this roadmap constantly.

\bigskip
\bigskip
\bigskip

\textit{ - La route est longue mais la voie est libre.}







