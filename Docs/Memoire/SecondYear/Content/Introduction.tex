


%%%%%%%%%%%%%%%%
%%  Introduction
%%%%%%%%%%%%%%%%


%% Contents
%
%    - General considerations on Complex Systems, positioning etc (thesis in cs science etc)
%    - Thematic introduction, geographical introduction of the subject.
%
%   - precisions on v1 memoire : foreword ?
%
%    - reading precisions : organisation, interdependances etc 
%
%   - reflexive aspect : here ?  


%\chapter*{\bpar{Introduction}{Introduction}}
\chapter*{Introduction}

% to have header for non-numbered introduction
\markboth{Introduction}{Introduction}

%\headercit{We need to find Banos' tenth modeling law}{Ren{\'e} Doursat}{}
\headercit{It's when you shuffle the anthill that you get a touch of all its complexity.}{Arnaud Banos}{}
%\headercit{C'est quand on donne un coup de pied dans la fourmilière qu'on se rend compte de toute sa complexité.}{Arnaud Banos}{}



% citation self-consistent ? -> seems ok

\bigskip

\bpar{
``In consequence of a technical issue, traffic is interrupted on the line B of RER, for an unknown duration. More information will be given as soon as available''. There is a high probability that someone having lived or spent some time in the metropolitan region of Paris has already heard this frightening announce and endured the difficult consequences the rest of his day. But he might not be aware of the ramifications of causal cascades induced by this not-so-rare event. Territorial Systems, whatever the layers considered in their definitions, will always be extremely complex and interrelations at numerous temporal and spatial scales participate in the emergent behaviors observed at any levels of the system. Martin is a student who daily commutes from Paris to Palaiseau and will miss today a crucial exam, what will have a profound impact on his professional life : implications at a long time scale, small spatial scale and agent granularity. Yuangsi is connecting Orly and Roissy Airports, in his trip from London to Beijing, will miss his plane and his sister's wedding : large spatial scale, short time scale, agent granularity. A collective petition emerges from users, leading to new social organizational patterns and reaction from transportation authority that results in efforts to increase levels of service : mesoscopic temporal and spatial scale, swarm of agents granularity. Looking for causes of the event will also lead to intricate processes at various scales, none of which seems to be a better explication than others : historical railway network in Parisian region shaped further extensions and RER B followed the former \textit{Ligne de Sceaux}, \noun{Delouvrier}'s schema for regional development, and its subsequent partial execution, are elements of explanation of structural weaknesses of Parisian public transportation network~\cite{gleyze2005vulnerabilite} ; commuting patterns consequent to territorial organisation induce an overload of particular lines and thus a necessary increase in exploitation incidents. The list could be developed much longer and each approach related to an already mature scientific body of knowledge in different disciplines such as geography, urban economics, transportation. This amusing anecdote is enough to give a touch of the complexity of territorial systems. Our aim here is to dive into this complexity, and in particular to give an original insight into the study of relations between networks and territories. The choice of this reading position will be largely discussed in a further thematic part. Let for now concentrate on the originality of the point of view that we will take.
}{
``En conséquence d'un problème technique, le trafic est interrompu sur la ligne B du RER pour une durée indéterminée. Plus d'information seront fournies dès que possible''. Il y a des fortes chances pour que quiconque ayant vécu ou passé un peu de temps en région parisienne ait déjà entendu cette annonce glaçante et en ait subi les conséquences pour le reste de la journée. Mais il ne se doute sûrement pas des ramifications des cascades causales induites par cet évènement presque banal. Les systèmes territoriaux, quelles que soient les aspects considérés pour leur définition, seront toujours extrêmement complexes, les interrelations à de nombreuses échelles spatiales et temporelles participant à la production des comportements émergents observés à tout niveau du système. Martin est un étudiant qui fait l'aller-retour journalier entre Paris et Palaiseau and manquera un examen crucial, ce qui aura un impact profond sur sa vie professionnelle : implications à une longue échelle de temps, une petite échelle spatiale et à la granularité de l'agent. Yuangsi était en train de relier les aéroports d'Orly et Roissy dans son voyage de Londres à Pékin et va manquer son avion ainsi que le mariage de sa soeur : grande échelle spatiale, petite échelle de temps, granularité de l'agent. Une pétition collective émerge des voyageurs, conduisant à la création d'une organisation qui mettra la pression sur les autorités pour qu'elles augmentent le niveau de service : échelle temporelle et spatiales mesoscopique, granularité de l'aggregation d'agents. La recherche de cause possible à l'incident conduira à des processus intriqués à diverses échelles, parmi lesquels aucun ne semble être une meilleure explication ; le développement historique du réseau ferroviaire en région parisienne a conditionné les écolutions futures et le RER B a suivi l'ancienne Ligne de Sceaux, le plan de \noun{Delouvrier} pour le développement régional et son execution partielle, sont également des éléments d'explication des faiblesses structurelles du réseau parisien de transports en commun~\cite{gleyze2005vulnerabilite} ; le motifs pendulaires dus à l'organisation territoriale induisent une surcharge de certaines ligne et ainsi nécessairement une augmentation des incidents d'exploitation. La liste pourrait être ainsi continuée un certain temps, chaque approche apportant sa vision mature correspondant à un corpus de connaissances scientifiques dans des disciplines diverses comme la géographie, l'économie urbains, les transports. Cette anecdote amusante est suffisante pour faire ressentir la complexité des systèmes territoriaux. Notre but ici est de se plonger dans cette complexité, et en particulier donner un point de vue original sur l'étude des relations entre réseaux et territoires. Le choix de cette position sera largement discuté dans une partie thématique, nous nous concentrons à présent sur l'originalité du point de vue que nous allons prendre.
}


%\section*{\bpar{Scientific Context : Complexity Has Come of Age}{Contexte Scientifique : Paradigmes de la Complexité}}
\section*{Scientific Context : Complexity Has Come of Age}


\bpar{
To better introduce our subject, it is necessary to make the reader aware of the particular scientific context we are working in. It is necessary both to understand the general epistemology underlying research questions, and to be aware of the variety of methods and tools used. Contemporaneous science is progressively taking the shift of complexity in many fields. That also implies an epistemological revolution to abandon strict reductionism that failed in most of its synthesis attempts~\cite{anderson1972more}. Arthur recently recalled~\cite{arthur2015complexity} that a mutation of methods and paradigms was also at stake by the increasing role of computational approaches replacing purely analytical techniques generally self-limited in their modeling and resolution scope. Capturing \emph{emergent properties} in models of complex systems is one of the ways to understand the essence of these new approaches.
}{
Pour une meilleure introduction du sujet, il est nécessaire d'insister sur le cadre scientifique dans lequel nous nous positionnons. Ce contexte est crucial à la fois pour comprendre les concepts épistémologiques implicites dans nos questions de recherche, et aussi pour être conscient de la variété de méthodes et outils utilisés. La science contemporaine prend progressivement le tournant de la complexité dans de nombreux champs, ce qui implique une mutation épistémologique pour abandonner le réductionnisme strict qui a échoué dans la majorité de ses tentatives de synthèse~\cite{anderson1972more}. Arthur a rappelé récemment~\cite{arthur2015complexity} qu'une mutation des méthodes et paradigmes en était également un enjeu, de par la place grandissante prise par les approches computationnelles qui remplacent les résolutions purement analytiques généralement limité en possibilités de modélisation et de résolution. La capture des \emph{propriétés émergentes} par des modèles de systèmes complexes est une des façons d'interpréter la philosophie de ces approaches.
}


\bpar{
These considerations are well known in Social Science (both quantitative and qualitative), in which the complexity of studied agents and systems is the justification of their existence : if humans were particles a whole branch of fields may have never emerged as thermodynamics would have solved most of social issues. \footnote{even if it would probably not have been the case as classical physics also failed in their attempts to include irreversibility and evolutions of Complex Adaptive Systems as Prigogine points out in \cite{prigogine1997end}} 
They are however less known nor accepted in more ``hard'' sciences such as physics : Laughlin develops in~\cite{laughlin2006different} a view of the discipline at least as at a ``frontier of knowledge'' then other fields appearing as less mature. Most of knowledge is of classical nature although a majority of structures and systems would be \emph{self-organized}, what means that the single microscopic laws are not enough to determine macroscopic properties unless system evolution is simulated (more precisely this property can be taken as a definition of emergence on which we will come back further, and self-organization is intrinsically emergent). It corresponds to the first nightmare of Laplace's Deamon developed in~\cite{deffuant2015visions}.
}{
Ces considérations sont bien connus des Sciences Humaines (qualitatives et quantitatives) pour lesquelles la complexité des agents et systèmes étudiés est une des justifications de leur existence : si les humains étaient des particules, la majorité des disciplines les prenant comme objet d'étude n'auraient jamais émergé puisque la thermodynamique aurait alors résolu la majorité des problèmes sociaux\footnote{bien que cette affirmation soit elle-même discutable, les sciences physiques classiques ayant également échoué à prendre en compte l'irréversibilité et l'évolution de Systèmes Complexes Adaptatifs comme le souligne \noun{Prigogine} dans \cite{prigogine1997end}.}. Elles sont au contraire moins connues et acceptées en sciences ``dures'' comme la physique : \noun{Laughlin} développe dans~\cite{laughlin2006different} une vision de la discipline à la même position de ``frontière des connaissances'' que d'autre champs pouvant paraître moins matures. La plupart des connaissances actuelles concerne des structures classiques simples, alors qu'un grand nombre de système présentent des propriétés \emph{d'auto-organisation}, au sense ou les lois macroscopiques ne sont pas suffisantes pour inférer les propriétés macroscopiques du systèmes à moins que son évolution soit entièrement simulée (plus précisément cette vision peut être prise comme une définition de l'émergence sur laquelle nous reviendrons par la suite, or des propriétés auto-organisées sont par nature émergentes). Cela correspond au premier cauchemar du Démon de Laplace développé dans~\cite{deffuant2015visions}. 
} 


%-------------------------------------------------



\bpar{
As an informal mix of epistemological positions, methods, and fields of applications, \emph{Complexity Science} relies on typical paradigms such as the centrality of emergence and self-organization in most of phenomena of the real world, which make it lie on a frontier of knowledge closer of us than we can think (Laughlin, op.cit. ). Such concepts are indeed not new, as they were already enlighten by Anderson~\cite{anderson1972more}. Even cybernetics can be related to complexity by seing it as a bridge between technology and cognitive science~\cite{wiener1948cybernetics}. Later, synergetics~\cite{haken1980synergetics} paved the way for a theoretical approach of collective phenomena in physics. Reasons for the recent growth of works claiming a CS approach may be various. The explosion of computing power is surely one because of the central role of numerical simulations~\cite{varenne2010simulations}. They could also be the related epistemological progresses : apparition of the notion of perspectivism~\cite{giere2010scientific}, finer reflexions around the notion of model~\cite{varenne2013modeliser}\footnote{In that frame scientific and epistemological progress can not be dissociated and can be seen as coevolving}. The theoretical and empirical potentialities of such approaches play surely a role in their success\footnote{
Although the adoption of new scientific practices may be strongly biased by imitation and lack of originality~\cite{dirk1999measure}, or more ambivalent, by marketing strategies as the fight for funds is becoming a huge obstacle for research~\cite{bollen2014funding}.}, as confirmed in various domains of application (see~\cite{newman2011complex} for a general survey), as for example Network Science~\cite{barabasi2002linked} ; Neuroscience~\cite{koch1999complexity}; Social Sciences ; Geography~\cite{manson2001simplifying}\cite{pumain1997pour} ; Finance with the rising importance of econophysics~\cite{stanley1999econophysics} ; Ecology~\cite{grimm2005pattern}. The Complex Systems Roadmap~\cite{2009arXiv0907.2221B} proposed a double lecture of studies on Complex Systems : an horizontal approach connecting fields of study with transversal questions on theoretical foundations of complexity and empirical common stylized facts, and a vertical conceptions of disciplines, with the aim to construct integrated disciplines and corresponding multi-scale heterogeneous models. Interdisciplinarity is thus central in our scientific background.
}{
A la croisée de positionnements épistémologiques, de méthodes et de champs d'application, les \emph{Sciences de la complexité} se concentrent sur l'importance de l'émergence et de l'auto-organisation dans la plupart des phénomènes réel, ce qui les place plus proche de la frontière des connaissances que ce que l'on peut penser pour des disciplines classiques (\noun{Laughlin}, op. cit.). Ces concepts ne sont pas récents et avaient déjà été mis en valeur par \noun{Anderson}~\cite{anderson1972more}. On peut aussi interpréter la Cybernétique comme un précurseur des Sciences de la Complexité en la lisant comme un pont entre technologie et sciences cognitives~\cite{wiener1948cybernetics}. Plus tard, la Synergétique~\cite{haken1980synergetics} a posé les bases d'approches théoriques des phénomènes collectifs en physique. Reasons for the recent growth of works claiming a CS approach may be various. The explosion of computing power is surely one because of the central role of numerical simulations~\cite{varenne2010simulations}. They could also be the related epistemological progresses : apparition of the notion of perspectivism~\cite{giere2010scientific}, finer reflexions around the notion of model~\cite{varenne2013modeliser}\footnote{In that frame scientific and epistemological progress can not be dissociated and can be seen as coevolving}. The theoretical and empirical potentialities of such approaches play surely a role in their success\footnote{
Although the adoption of new scientific practices may be strongly biased by imitation and lack of originality~\cite{dirk1999measure}, or more ambivalent, by marketing strategies as the fight for funds is becoming a huge obstacle for research~\cite{bollen2014funding}.}, as confirmed in various domains of application (see~\cite{newman2011complex} for a general survey), as for example Network Science~\cite{barabasi2002linked} ; Neuroscience~\cite{koch1999complexity}; Social Sciences ; Geography~\cite{manson2001simplifying}\cite{pumain1997pour} ; Finance with the rising importance of econophysics~\cite{stanley1999econophysics} ; Ecology~\cite{grimm2005pattern}. The Complex Systems Roadmap~\cite{2009arXiv0907.2221B} proposed a double lecture of studies on Complex Systems : an horizontal approach connecting fields of study with transversal questions on theoretical foundations of complexity and empirical common stylized facts, and a vertical conceptions of disciplines, with the aim to construct integrated disciplines and corresponding multi-scale heterogeneous models. Interdisciplinarity is thus central in our scientific background.
}



%\section{\bpar{Interdisciplinarity}{Interdisciplinarité}}
\section*{Interdisciplinarity}

%\textit{Note : that term does not exist in english but is a rough translation from french \emph{interdisciplinarit{\'e}}, that we believe to better express }
%
%
%WHY and HOW is interdisciplinarity essential ?
%
% Q : quote Morin ?
%


We must further insist on the role of interdisciplinarity in the positions taken here. This is not a thesis in Geography nor in Complex Adaptive Systems Modeling, but in \emph{Complex Systems Science} that we claim as a proper discipline following \noun{Paul Bourgine}. It will naturally be seen with defiance by scholars of various concerned disciplines, as recent examples of misunderstandings and conflicts have illustrated~\cite{dupuy2015sciences}. The positioning of \noun{Batty} proposing \textit{A new Science of Cities}~\cite{batty2013new} (that he subtly presents as \textit{The} new science of cities) is directed towards an integration of disciplines and methods into a science defined by its object of study, cities. Its theoretical and epistemological weaknesses (no theoretical constructions of studied geographical objects on the one hand, approximative contextualization of complexity) combined with an overall impression of \emph{pot-pourri} of forgotten works (space syntax, land-use models), unfortunately avoid us to use it as we will use geographical theories (e.g. evolutive urban theory) in an appropriated epistemological complexity context. Yet our reading of this work may be the result of a misunderstanding due to different cultural backgrounds.


%\subsection*{Conflicting Complexities and Cultural Differences}

The scientific evolution of complexity that some see as a revolution~\cite{colander2003complexity}, or even as \emph{a new kind of science}~\cite{wolfram2002new}, could indeed face intrinsic difficulties due to behaviors and a-priori of researchers as human beings. More precisely, the need of interdisciplinarity that makes the strength of Complexity Science may be one of its greatest weaknesses, since the highly partitioned structure of science organization has sometimes negative impacts on works involving different disciplines. We do not tackle the issue of over-publication, competition, indexes, which is more linked to a question of open science and its ethics, also of high importance but of an other nature. That barrier we are dealing with and we might struggle to triumph of, is the impact of certains \emph{cultural disciplinary differences} and out-coming conflicts on views and approaches. 
The drama of scientific misunderstandings is that they can indeed annihilate progresses by interpreting as a falsification some work that answers to a totally different question. The example of a recent work on top-income inequalities given in~\cite{aghion2015innovation}, which conclusions are presented as opposed from the one obtained by Piketty~\cite{piketty2013capital}, follows such a scheme. Whereas Piketty focused on constructing long-time clean databases for income data and showed empirically a recent acceleration of income inequalities, his simple model aiming to link this stylized fact with the accumulation of capital has been criticized as oversimplified. On the other hand, Bergeaud \textit{et al.} prove by a model of innovation economics that \emph{under certain assumptions} income gaps may be beneficial to innovation and consequently a general utility. Thus diverging conclusions about the role of personal capitals in the economy. But diverging \emph{views} or \emph{interpretations} does not mean a scientific incompatibility, and one could imagine try to gather both approaches in an unified framework and model, yielding possibly similar or different interpretations. This integrated approach has chances to contain more information (depending on how coupling is done) and to be a further advance in Science. We shall now briefly develop other examples to give an overview how conflicts between disciplines can be damaging.


\paragraph{Physics reinvents geography.}

As already mentioned, \noun{Dupuy} and \noun{Benguigui} points out in \cite{dupuy2015sciences} the fact that urban sciences have recently known open conflicts between old tenants of the disciplines and new arrivants, especially physicists. The availability of large datasets of new type of data (social networks, ICT data) have drawn their attention towards the study of objects traditionally studied by human science, as analytical and computational methods of statistical physics became applicable. Although these studies are generally presented as the construction of a scientific approach to cities, implying that existing knowledge was not scientific because of their more qualitative aspect, they have not unveil specifically novel knowledge on urban systems : to give some examples, \cite{barthelemy2013self} concludes that Paris has followed a transition during Haussman period and that the evolution of a city is the combination of local transformations and global planning operations, what are facts known for a long time in urban history and urban geography. \cite{chen2009urban} rediscovers that the gravity model can be improved by adding lags in interactions and theoretically derives the expression of the force of interaction between cities, without any thematic theoretical background. Examples could be multiplied, confirming the current discomfort in communication between physicists and urban geographers. Significant benefices could results from a wise integration of disciplines~\cite{o2015physicists} but the road seems still long.

\paragraph{Economic Geography or Geographical Economics ?}

Similar conflict occurred in economics : as \cite{marchionni2004geographical} describes, the discipline of economic geography, traditionally close from geography, heavily criticized a new stream of thought named \emph{geographical economics}, which purposes is spatialization of mainstream economic techniques. Both do not have the same purposes and aims, and the conflict appears as a total misunderstanding for an external observer.


\paragraph{Agent-based Modeling in Economy}

Disciplinary conflicts may also manifest themselves as the reject of novel methods by mainstream currents. Following \cite{farmer2009economy}, the operational failure of most classic economic approaches could be compensated by a broader use of agent-based modeling and simulation practices. The lack of analytical framework that is natural in the study of complex adaptive systems seems to be rebutting for most of economists.


\paragraph{Finance}

In Quantitative Finance coexist various stream of research with a very few interactions. Let consider two examples. On the one hand, Statistics are highly advanced in theoretical mathematics, using stochastic calculus and probabilities to obtain very refined estimators of parameters for a given model (see e.g. \cite{barndorff2011multivariate}). On the other hand, Econophysics aims to study empirical stylized facts and infer empirical laws to explain complexity-related phenomena in financial systems~\cite{stanley1999econophysics}, such as e.g. cascades leading to market crashes, fractal properties of asset signals, complex structure of correlation networks. Both have their advantages in a particular context and each would benefit from increased interactions between the fields.


\bigskip

These diverse examples illustrate how interdisciplinarity is both crucial and difficult to achieve. We will try to follow that narrow path in our work, borrowing ideas, theories and methods from various disciplines, aiming for the construction of an integrated knowledge. Indeed, coupling heterogeneous approaches at different levels and scales will be a cornerstone of our thesis, skeleton of the underlying philosophy and building brick of the theory we will propose.







%-------------------------------------------------

\section*{Complexity in Geography}


Coming back to our introducing anecdote, we will focus on our thematic object of study that are territorial systems. More generally, we propose an overview of the role of complexity in geography. Geographers are familiar with complexity for a long time, as the study of spatial interactions is one of its purposes. The variety of fields in geography (geomorphology, physical geography, environmental geography, human geography, health geography to give a few) has certainly been important in the subtlety of the geographical thinking, that considers heterogeneous and multi-scalar processes.

\noun{Pumain} recalls in~\cite{pumain2003approche} a subjective history of the emergence of complexity paradigms in geography. Cybernetics yielded system theories as the one developed by Forrester. Later the shift to self-organized criticality and self-organisation concepts in physics conducted to corresponding developments in geography, as \cite{sanders1992systeme} witnesses the application of the concepts of synergetics for the dynamics of an urban system. Finally, Complex Systems paradigms as we currently know them appeared from various points of view. For example, the fractal nature of urban shape was introduced in~\cite{batty1994fractal} and had numerous application including more recent developments~\cite{keersmaecker2003using}. \noun{Batty} also introduced cellular automata in urban modeling and proposed a joint synthesis with agent-based modeling and fractals in~\cite{batty2007cities}. An other incursion of complexity in geography was for the case of urban systems through the evolutive urban theory of \noun{Pumain}. In close relation with modeling from the beginning (the first Simpop model described in~\cite{sanders1997simpop} enters the theoretical framework of \cite{pumain1997pour}), this theory aims to understand system of cities as systems of co-evolving adaptive agents, interacting in many ways, with particular features emphasized such as the diffusion of innovation. The series of Simpop models~\cite{pumain2012multi} focused in testing various assumptions of the theory. For example, different underlying mechanisms were revealed for european city systems and city system of the united states~\cite{bretagnolle2010comparer}. At other time scales and in other contexts, the SimpopLocal model~\cite{schmitt2014modelisation} aimed to investigate the conditions for the emergence of hierarchical urban systems from disparate settlements. A minimal model (in the sense of sufficient and necessary parameter) has been isolated thanks to the use of intensive computation with the model exploration software OpenMole~\cite{schmitt2014half}, what was a result analytically not derivable for this kind of complex model. The technical progresses of OpenMole~\cite{reuillon2013openmole} were done simultaneously with theoretical and empirical advancements. Epistemological advances were also essential to this framework, as \noun{Rey} develops in~\cite{rey2015plateforme}, and novel concepts such as incremental modeling~\cite{cottineau2015incremental} were found, with powerful concrete applications : \cite{cottineau2014evolution} implemented it on the soviet city system and isolated dominating socio-economic processes, by systematic testing of thematic assumptions and implementation functions. Directions for the development of such Modeling and Simulation practices in quantitative geography were recently introduced by \noun{Banos} in~\cite{banos2013pour}. He concludes with nine principles\footnote{I remember \noun{Ren{\'e} Doursat} insisting on the search of the last commandement of Banos}, from which we can cite the importance of intensive exploration of computational models and the importance of heterogeneous model coupling, that are among other principles such as reproducibility at the center of the study of complex geographical systems from the point of view described just before. Positioning in the legacy of this line of research, we will conjointly work in the theoretical, empirical, epistemological and modeling domains.




\section*{Research Question}


Research question and precise objects are deliberately fuzzy for now, as we postulate that the construction of a problematic can not be dissociated from the production of a corresponding theory. Reciprocally, it makes no sense to ask questions out of the blue, on objects that have been only partially or rapidly defined. Our preliminary question to enter the subject, that we can obtain from concrete cases such as our introducing anecdote or from preliminary literature review, is the following :

\bigskip

\textit{Is it possible to produce a definition of territorial systems, and corresponding scales and ontologies, that would yield a natural, consistent and informational view on processes ?}

\bigskip

Indeed, a necessary characteristic of territorial systems is their spatio-temporal nature, that is contained in spatio-temporal dynamics. The notion of \emph{process} in the sense of \cite{hypergeo} captures furthermore causal relationships in these dynamics, and is thus an interesting approach for an understanding of such systems. \emph{Scale} must be understood here in the operational sense (physical characteristic ) and \emph{ontology} as real-world studied objects\footnote{this use of ontology here naturally biaises our research towards modeling paradigms as it is close from the notion of ontology used in~\cite{livet2010}, but we take the position (largely developed further) to understand any scientific construction as \emph{models}, making the frontier between theory and modeling less relevant than in standard views. Any theory has to make choices on described objects, relations and processes, and therefore contains an ontology in that sense.}. Our question may be roughly viewed as a search for theories and models that would unveil some processes involved in complex systems containing at least human settlements, the last requirement being crucial for a convergent problematic construction rather than ending in non-realistic and non-constructive propositions to understand everything between the brain (that can be seen as one building brick of territorial systems as they emerge from human social constructions) and the ecosphere that includes territorial systems.  




%-------------------------------------------------

\section*{Contents}


This provisory Memoire is organized the following way. A first part with four chapters sets the thematic, theoretical and methodological background. The study of geographical systems implies, because of their complexity, a subtle combination of Theoretical constructions and Empirical Analysis, either in an inductive reasoning or in a didactic constitution of knowledge. The first part aims to approach our subject from the theoretical and methodological point of view, and rather as a \textit{necessary foundation} shall be understood as a body of knowledge \emph{coevolving} with Empirical and Modeling Parts. A linear reading is not necessarily the best way to deeply perceive the implications of theory on empirical and modeling experiments and reciprocally. Some methodological developments are necessary but explicit reference will be done when it will be the case. A first chapter starts from the provisory research question given above and frames from a thematic point of view geographical objects and processes to be studied, resulting in precise research questions. The scene is set up for the construction of our theoretical background in a second chapter, that consists in a geographical theory for territorial systems on the one hand and in an epistemological theory of socio-technical systems modeling that frames our approach at a meta-level. We then develop methodological considerations on diverse questions implied by theory and required for modeling. Finally, a chapter of quantitative epistemology finishes to pave the way for modeling directions, unveiling literature gaps precisely linked to our question. A second part develops results obtained from empirical analysis and modeling experiments, along with on-going and planned projects in these fields. It first present empirical analysis aimed at identifying stylized facts. Toy-models of urban growth are then proposed, followed by an example and propositions for more complex models. The third part constructs our research objective for the remaining part of our project and sets a corresponding roadmap. Appendices contain non-digest important parts of our work such as models implementation architecture and details and specific tools developed for a reproducible research workflow.













  



