

% Chapter 




\chapter{Interactions between Network and Territory} % Chapter title

\label{ch:quantepistemo} % For referencing the chapter elsewhere, use \autoref{ch:name} 


%%  Thematic chapter framing geographically the subject.
%%   and reviewing state of the art
%%   and why modeling : evolutive theory of urban systems etc ; multimodeling simfamily etc
%%  
%%   Q  : example to introduce theory ?
%
%   Modelography.  (non-exhaustive) : classification according to purpose, theme, scale, etc.
%   Why dynamic models of ``co-evolution''  ?
%   definition of terms, contextualisation, etc.  (le what/where d'Arnaud ; ontology de Anne)



%----------------------------------------------------------------------------------------




\section{Networks and Territories}



The notion of territory can be taken as a basis to explore the scope of geographical objects we will explore.

For Raffestin, 







%----------------------------------------------------------------------------------------

\section{Modeling Interactions}


\subsection{Modeling in Quantitative Geography}

% brief reference to the history of TQG ; history of modeling.
%  note : history of future of TQG, London september 2016




\subsection{Modeling Territories and Networks}

% here overview of different approaches
% TODO Q : do it here, not during quant epistemo part ?




\subsection{Sketch of a \emph{Modelography}}










