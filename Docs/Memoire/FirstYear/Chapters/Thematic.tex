

% Chapter 




\chapter{Interactions between Networks and Territories} % Chapter title

\label{ch:thematic} % For referencing the chapter elsewhere, use \autoref{ch:name} 


%%  Thematic chapter framing geographically the subject.
%%   and reviewing state of the art
%%   and why modeling : evolutive theory of urban systems etc ; multimodeling simfamily etc
%%  
%%   Q  : example to introduce theory ?
%
%   Modelography.  (non-exhaustive) : classification according to purpose, theme, scale, etc.
%   Why dynamic models of ``co-evolution''  ?
%   definition of terms, contextualisation, etc.  (le what/where d'Arnaud ; ontology de Anne)



%----------------------------------------------------------------------------------------

%\headercit{If you are embarrassed by the precedence of the chicken by the egg or of the egg by the chicken, it is because you are assuming that animals have always be the way they are}{Denis Diderot}{\cite{diderot1965entretien}}

\headercit{Si la question de la priorit{\'e} de l'\oe{}uf sur la poule ou de la poule sur l'\oe{}uf vous embarrasse, c'est que vous supposez que les animaux ont {\'e}t{\'e} originairement ce qu'ils sont {\`a} pr{\'e}sent.}{Denis Diderot}{\cite{diderot1965entretien}}

\bigskip


This analogy is ideal to evoke the questions of causality and processes in territorial systems. When trying to tackle naively our preliminary question, some observers have qualified the identification of causalities in complex systems as ``chicken and egg'' problems : if one effect appears to cause another and reciprocally, how can one disentangle effective processes ? This vision is often present in reductionist approaches that do not postulate an intrinsic complexity in studied systems. The idea that Diderot suggests is the notion of \emph{co-evolution} that is a central phenomenon in evolutive dynamics of Complex Adaptive Systems as \noun{Holland} develops in~\cite{holland2012signals}. He links the notion of emergence (that is ignored in a reductionist vision), in particular the emergence of structures at an upper scales from the interactions between agents at a given scale, materialized generally by boundaries, that become crucial in the coevolution of agents at any scales : the emergence of one structure will be simultaneous with one other, each exploiting their interrelations and generated environments conditioned by their boundaries. We shall explore these ideas in the case of territorial systems in the following.


This introductive chapter aims to set up the thematic scene, the geographical context in which further developments will root. It is not supposed to be understood as an exhaustive literature review nor the fundamental theoretical basement of our work (the first will be an object of chapter~\ref{ch:quantepistemo} whereas the second will be earlier tackled in chapter~\ref{ch:theory}), but more as narration aimed to introduce typical objects and views and construct naturally research questions.



%-------------------------------

\newpage

\section[Territories and Networks]{Territories and Networks}


\subsection{Territories and Networks : There and Back Again}

\paragraph{Human Territories}

The notion of territory can be taken as a basis to explore the scope of geographical objects we will study. In Ecology, a territory corresponds to a spatial extent occupied by a group of agents or more generally an ecosystem. \emph{Human Territories} are far more complex in the sense of semiotic representations of these that are a central part in the emergence of societies. For \noun{Raffestin} in~\cite{raffestin1988reperes}, the so-called \emph{Human Territoriality} is the ``conjonction of a territorial process with an informational process'', what means that physical occupation and exploitation of space by human societies is not dissociable from the representations (cognitive and material) of these territorial processes, driving in return its further evolutions. In other words, as soon as social constructions are assumed in the constitution of human settlements, concrete and abstract social structures will play a role in the evolution of the territorial system, through e.g. propagation of information and representations, political processes, conjonction or disjonction between lived and perceived territory. Although this approach does not explicitly give the condition for the emergence of a seminal system of aggregated settlements (i.e. the emergence of cities), it insists on the role of these that become places of power and of creation of wealth through exchange. But the city has no existence without its hinterland and the territorial system can not be summarized by its cities as a system of cities. There is however compatibility on this subsystem between \noun{Raffestin} approach to territories and \noun{Pumain}'s evolutive theory of urban systems~\cite{pumain2010theorie}, in which cities are viewed as an auto-organized complex dynamical systems, and act as mediators of social changes : for example, cycles of innovation occur within cities and propagate between them. Cities are thus competitive agents that co-evolve (in the sense given before). The territorial system can be understood as a spatially organized social structure, including its concrete and abstract artifacts. A imaginary free-of-man spatial extent with potential ressources will not be a territory if not inhabited, imagined, lived, and exploited, even if the same ressources would be part of the corresponding habited territorial system. Indeed, what is considered as a ressource (natural or artificial) will depend on the corresponding society (e.g. of its practices and technological potentialities). A crucial aspect of human settlements that were studied in geography for a long time, and that relate with the previous notion of territory, are \emph{networks}. Let see how we can switch from one to the other and how their definition may be indissociable.


\paragraph{A Territorial Theory of Networks}

We paraphrase \noun{Dupuy} in~\cite{dupuy1987vers} when he proposes elements for ``a territorial theory of networks'' based on the concrete case of Urban Transportation Networks. This theory sees \emph{real networks} (i.e. concrete networks, including transportation networks) as the materialization of \emph{virtual networks}. More precisely, a territory is characterized by strong spatio-temporal discontinuities induced by the non-uniform distribution of agents and ressources. These discontinuities naturally induce a network of ``transactional projects'' that can be understood as potential interactions between elements of the territorial system (agents and/or ressources). For example today, people need to access the ressource of employments, economic exchanges operate between specialized production territories. At any time period, potential interactions existed\footnote{even when nomadism was still the rule, spatially dynamic networks of potential interactions necessarily existed, but should have less chance to materialize into concrete routes.% bib on that ?
}. The potential interaction network is concretized as offer adapts to demand, and results of the combination of economic and geographical constraints with demand patterns, in a non-linear way through agents designed as \emph{operators}. This process is not immediate, leading to strong non-stationarity and path-dependance effects : the extension of an existing network will depend on previous configuration, and depending on involved time scales, the logic and even the nature of operators may have evolved. \noun{Raffestin} points out in his preface of~\cite{offner1996reseaux} that a geographical theory articulating space, network and territories had never been consistently formulated. It appears to still be the case today, but the theory developed just before is a good candidate, even if it stays at a conceptual level. The presence of a human territory necessarily imply the presence of abstract interaction networks and concrete networks used for transportation of people and ressources (including communication networks as information is a crucial ressource). Depending on regime in which the considered system is, the respective role of different networks may be radically different. Following \noun{Duranton} in \cite{duranton1999distance}, pre-industrial cities were limited in growth because of limitations of transportation networks. Technological progresses have lead to the end of these limitations and the preponderance of land markets in shaping cities (and thus a role of transportation network as shaping prices through accessibility), and recently to the rising importance of telecommunication networks that induce a ``tyranny of proximity'' as physical presence is not replaceable by virtual communication. This territorial approach to networks seems natural in geography, since networks are studied conjointly with geographical objects with an underlying theory, in opposition to network science that studies brutally spatial networks with few thematic background~\cite{ducruet2014spatial}.


\paragraph{Networks shaping territories ?}

% how do network shape territories : boundaries, scales, etc.
% example : \cite{l2012ville} bahn-ville, volontary coevol ? // idem villes nouvelles

However networks are not only a material manifestation of territorial processes, but play their part in these processes as they evolution may shape territories in return. In the case of \emph{technical networks}, an other designation of real networks given in~\cite{offner1996reseaux}, many examples of such feedbacks can be found : the interconnectivity of transportation networks allows multi-scalar mobility patterns, thus shaping the lived territory. At a smaller scale, changes in accessibility may result in an adaptation of a functional urban space. Here emerges again an intrinsic difficulty : it is far from evident to attribute territorial mutations to some network evolutions and reciprocally materialization of a network to precise territorial dynamics. Coming back to Diderot should help, in the sense that one must not consider network nor territories as independent systems that would have causal relationships but as strongly coupled components of a larger system. The confusion on possible simple causal relationships has fed a scientific debate that is still active. Methodologies to identify so-called \emph{structural effects} of transportation networks were proposed by planners in the seventies~\cite{bonnafous1974detection,bonnafous1974methodologies}. It took some time for a critical positioning on unreasoned and decontextualized use of these methods by planners and politics generally to technocratically justify transportation projects, that was first done by \noun{Offner} in~\cite{offner1993effets}. Recently the special issue~\cite{espacegeo2014effets} on that debate recalled that on the one hand misconceptions and misuses were still greatly present in operational and planning milieus as~\cite{crozet:halshs-01094554} confirmed, and on the other hand that a lot of scientific progresses still need to be made to understand relations between networks and territories as \noun{Pumain} highlights that recent works gave evidence of systematic effects on very long time scales (as e.g. the work of \noun{Bretagnolle} on railway evolution, that shows a kind of structural effect in the necessity of connectivity to the network for cities to ``stay in the game'', but that is not fully causal as not sufficient). At a macroscopic level typical patterns of interaction emerge, but microscopic trajectories of the system are essentially chaotic : the understanding of coupled dynamics strongly depends on the scale considered. At a small scale it seems indeed impossible to show systematic behavior, as \noun{Offner} pointed out. For example, on comparable French mountain territories, \cite{berne2008ouverture} shows that reactions to a same context of evolution of the transportation network can lead to very different reactions of territories, some finding a huge benefit in the new connectivity, whereas others become more closed. These potential retroactions of networks on territories does not necessarily act on concrete components : \noun{Claval} shows in~\cite{claval1987reseaux} that transportation and communication networks contribute to the collective representation of territories by acting on territorial belonging feeling.




\paragraph{Territorial Systems}

This detour from territories, to networks and back again, allows us to give a preliminary definition of a territorial system that will be the basis of our following theoretical considerations. As we emphasized the role of networks, the definition takes it into account.

\bigskip

\textbf{Preliminary Definition.} \textit{A territorial system is a human territory to which both interaction and real networks can be associated. Real networks are a component the system, involved in evolution processes, through multiples feedbacks with other components at various spatial and temporal scales.}


\bigskip


This reading of territorial systems is conditional to the existence of networks and may discard some human territories, but it is a deliberate choice that we justify by previous considerations, and that drives our subject towards the study of interactions between networks and territories.



\subsection{Transportation Networks}


\paragraph{The particularity of transportation networks}

Already evoked in relation to the question of structural effects of networks, transportation networks play a determining role in the evolution of territories. Although other types of networks are also strongly involved in the evolution of territorial systems (see e.g. the discussions of impacts of communication networks on economic activities), transportation networks shape many other networks (logistics, commercial exchanges, social concrete interactions to give a few) and are prominent in territorial evolution patterns, especially in our recent societies that has become dependent of transportation networks~\cite{bavoux2005geographie}. The development of French High Speed Rail network is a good illustration of the impact of transportation networks on territorial development policies. Presented as a new era of railway transportation, a top-down planning of totally novel lines was introduced as central for developments~\cite{zembri1997fondements}. The lack of integration of these new networks with existing ones and with local territories is now observed as a structural weakness and negative impacts on some territories have been shown~\cite{zembri2008contribution}. A review done in~\cite{bazin2011grande} confirms that no general conclusions on local effects of High Speed lines connection can be drawn although it keeps a strong place in imaginaries. These are examples of how transportation networks have both direct and indirect impacts on territorial dynamics. Integrated planning, in the sense of a joint planning of transportation infrasctures and urban development, considers the network as a determining component of the territorial system. Parisian \emph{Villes Nouvelles} are such a case, that witnesses of the complexity of such planning actions that generally do not lead to the desired effect~\cite{es119}. Recent projects as~\cite{l2012ville} have try to implement similar ideas but we have now not enough temporal scope to judge their success in effectively producing an integrated territory. Transportation networks are anyway at the center of these approaches of urban territories. We will focus in our work on transportation networks for the various reasons given here.




\paragraph{Deconstructing Accessibility}

% critic of accessibility as a planning tool : danger of not taking into account socio-eco dynamics and coupled dynamics (coevol) - cit Hadri mobility as a constructed notion.

The notion of accessibility comes rapidly when considering transportation networks. Based on the possibility to access a place through a transportation network (including transportation speed, difficulty of travel), it is generally described as a potential of spatial interaction\footnote{and often generalized as \emph{functional accessibility}, for example employments accessible for actives at a location. Spatial interaction potentials ruling gravity law can also been understood this way.}~\cite{bavoux2005geographie}. This object is often used as a planning tool or as an explicative variable of agents localisation for example. We must warn here on the potential dangers of its unconditional use. More precisely, it may be a construction that misses a consistent part of territorial dynamics. The mystification of the notion of \emph{mobility} was shown by \noun{Commenges} in~\cite{commenges:tel-00923682}, which proved than most of debates on modeling mobility and corresponding notions were mostly made-of by transportation administrators of \emph{Corps des Ponts} who roughly imported ideas from the United States without adaptation and reflexion fit to the totally different French context. Accessibility may be such a social construct and have no theoretical root since it is mostly a modeling and planning tool. Recent debates on the planification of \emph{Grand Paris Express}
% TODO cit. Mangin and/or Salat ?



\paragraph{Scales and Hierarchies}


% \cite{10.1371/journal.pone.0102007}

% \cite{Tsekeris20131} : congestion related to land-use




%----------------------------------------------------------------------------------------

\newpage

\section{Modeling Interactions}


\subsection{Modeling in Quantitative Geography}

% brief reference to the history of TQG ; history of modeling.
%  note : history of future of TQG, London september 2016




\subsection{Modeling Territories and Networks}

% here overview of different approaches
% TODO Q : do it here, not during quant epistemo part ?

\subsubsection{Land-Use Transportation Interaction Models}

A subsequent bunch of literature in modeling interaction between networks and territories can be found in the field of planning, with the so-called \emph{Land-use Transportation Interaction Models}. These works are difficult to be precisely bounded as they may be influenced by various disciplines. For example, from the point of view of Urban Economics, propositions for synthesizing models have existed for a relatively long term~\cite{putman1975urban}. The variety of possible models has lead to operational comparisons~\cite{paulley1991overview,wegener1991one}. More recently, the respective advantages of static and dynamic modeling was investigated in~\cite{kryvobokov2013comparison}. Generally these type of models operate at relatively small temporal and spatial scales. \cite{wegener2004land} reviewed state of the art in empirical and modeling studies on interactions between land-use and transportation. It is positioned in economic, planning and sociological theoretical contexts, and is relatively far from our geographical approach aiming to understand long-time processes. Seventeen models are compared and classified, none of which implements actually network endogenous evolution on the relatively small time scales of simulation. A complementary review done in \cite{chang2006models} broadens the scope with inclusion of more general classes of models, such as spatial interaction models (including traffic assignment and four steps models), operational research planning models (optimal localisations), micro-based random utility models, and urban market models. These techniques operate also at small scales and consider at most land-use evolution. \cite{iacono2008models} covers a similar scope with a further emphasis on cellular automata models of land-use change and agent-based models. These type of models are still largely developed and used today, as for example \cite{delons:hal-00319087} which is used for Parisian metropolitan region. The short-term range of application and their operational character makes them useful for planning, what is far from our preoccupation to obtain explicative models for geographical processes. 





\subsubsection{Network Growth}


% economic models

\cite{yerra2005emergence} % : cost-driven model of nw reinforcement (// slime mould)

\cite{louf2013emergence} % : trade-off between cost and benefits due to flows -> compare with lutecia rules ?
 
 \cite{yamins2003growing} strange model
 
 \cite{zhang2007economics}
 
 \cite{xie2009modeling} review of network growth economic appraoches
 
 
 \cite{bigotte2010integrated} % : planning network - similar to nw growth ? - hierarchy in cities and nw
 
 % geometric - local optimization
 
 \cite{barthelemy2008modeling}
 
 \cite{courtat2011mathematics} % measures and morphogenesis. Vision of morphogenesis as living organism : nuance that in theory.
 
 \cite{de2007netlogo} % geom rules in Tijuana model
 
 \cite{rui2013exploring} % local based optimisation morphogenesis model. RQ : quote only physicists work -> justification for extended quantitative epistemology ?
 
 
 
 

% biological nws

\cite{tero2010rules} % physarum : biological nw heuristics
\cite{tero2006physarum} % potentialities of physarum machines, here for routing.
\cite{adamatzky2010road} % planning absurdities
\cite{zhu2013amoeba} % TSP solving : long range correlations.


\subsubsection{Hybrid Modeling}

Models of simulation implementing a coupled dynamic between urban growth and transportation network growth are relatively rare, and always rather poor from a theoretical and thematic point of view. A generalization of the geometrical local optimization model described before was developed in~\cite{barthelemy2009co}. % pb of scales, def of coevol, thematic meaning of assumptions, etc.



\cite{levinson2007co} : economic model of coevolution. % TODO check timescales if are consistent.

\cite{levinson2005paving} % markov chain : not really modeling but more statistics.

\cite{raimbault2014hybrid}



\subsubsection{Urban Systems Modeling}

% differentiate to other hybrid models : SimpopNet and others ? (if exist ?)






\subsection{Sketch of a \emph{Modelography}}





%-------------------------

\newpage

\section{Research Question}







