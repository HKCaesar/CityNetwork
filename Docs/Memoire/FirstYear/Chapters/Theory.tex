


% Chapter 

\chapter{Theoretical Framework} % Chapter title

\label{ch:theory} % For referencing the chapter elsewhere, use \autoref{ch:name} 

%----------------------------------------------------------------------------------------



\headercit{Your theory is crazy, but not enough to be true.}{Niels Bohr}{}

\bigskip


Theory is a key element of any scientific construction, especially in Human Sciences in which object definition and questioning are more open but also determining for research directions. We develop in this chapter a self-consistent theoretical background. It naturally emerges from thematic considerations of previous chapter, empirical explorations done in chapter~\ref{ch:empirical} and modeling experiments conducted in chapter~\ref{ch:modeling}, as a linear structure of knowledge is not appropriate to translate the type of scientific entreprise we are conducting, typically in the spirit of \noun{Sanders} in~\cite{livet2010} for which the simultaneous conjonction of empirical, conceptual and modeling domains is necessary for the emergence of knowledge. This theoretical construction is however presented to be understood independently, and is used as a structuring skeleton for the rest of the thesis.

We propose first to construct the \emph{geographical theory} that will pose the studied objects and their meaning in the real world (their ontology), with their interrelations. This yields precise assumptions that will be sought to be confirmed or proven false in the following. Staying at a thematic level appears however to be not enough to obtain general guidelines on the type of methodologies and the approaches to use. More precisely, even if some theories imply an more natural use of some tools\footnote{to give a rough example, a theory emphasizing the complexity of relations between agents in a system will conduct generally to use agent-based modeling and simulation tools, whereas a theory based on macroscopic equilibrium will favorise the use of exact mathematical derivations.}, at the subtler level of contextualization in the sense of the approach taken to implement the theory (as models or empirical analysis), the freedom of choice may mislead into unappropriated techniques or questionings (see \cite{raimbault2016cautious} on the example of incautious use of big data and computation). We develop therefore in a second section a theoretical framework at a meta-level, aiming to give a vision and framing for modeling socio-technical systems.



%----------------------------------------------------------------------------------------

\newpage

% first section that develops elements of geographical theory : system of cities, territories, etc.
%  city morphogenesis ? -> link to other approaches of morphogenesis (Turing)
%  coevolution : give a precise theoretical meaning

% HERE precise exact definition, framed precisely (reference for the following).

\section{Geographical Theoretical Context}



%%%%%%%%%%%
\subsection{Foundation}



%%%%%%%%%%%%%%
\subsubsection{Networked Human Territories}

Our first pillar has already been constructed before in the thematic exploration of the research subject. We rely on the notion of \emph{Human Territory} elaborated by \noun{Raffestin} as the basis for a definition of territorial systems. It permits to capture complex human geographical systems in their concrete and abstract characteristics and representation. For example, a metropolitan territorial system can be apprehended simply by the functional extent of daily commuting, or by the perceived or lived space of different populations, the choice depending on the precise question asked. Note that this approach to territory is a position and that other (possibly compatible) entries could be taken~\cite{murphy2012entente}. The concrete of this pillar in reinforced by the territorial theory of networks of \noun{Dupuy}, yielding the notion of networked human territory, as a human territory in which a set of potential transactional networks have been realized, which is in accordance with vision of the territory as networked places~\cite{champollion:halshs-00999026}. We make therein the assumption that real networks are necessary elements of territorial systems.


%%%%%%%%%%%%%%%
\subsubsection{Evolutive Urban Theory}

% development of Denise theory
%  -> extension with precision on coevolution ? (read Holland on coevolution) -- beware of biological //

The second pillar of our theoretical construction is the Evolutive Urban Theory of \noun{Pumain}, closely linked to the complexity approach we take. This theory was first introduced in~\cite{pumain1997pour} which argues for a dynamical vision of city systems, in which self-organization is key. Cities are interdependent evolutive spatial entities whose interrelations produces the macroscopic behavior at the scale of city system. The city system is also designed as a network of city what emphasizes its view as a complex system. Each city is itself a complex system in the spirit of~\cite{berry1964cities}, the multi-scale aspect being essential in this theory, since microscopic agents convey system evolution through complex feedbacks between scales. The positioning within Complex System Sciences was later confirmed~\cite{pumain2003approche}. It was shown that this theory provide an interpretation for the origin of pervasive scaling laws, resulting from the diffusion of innovation cycles between cities~\cite{pumain2006evolutionary}. The aspect of resilience of system of cities, induced by the adaptive character of these complex systems, implies that cities are drivers and adapters of social change~\cite{pumain2010theorie}. Finally, path dependance yield non-ergodicity within these systems, making ``universal'' interpretations of scaling laws developed by physicists incompatible with evolutive urban theory~\cite{pumain2012urban}. We will interpret territorial systems following that idea of complex adaptive systems.




%%%%%%%%%%%
\subsubsection{Urban Morphogenesis}

% -> make a link between city systems and urban form/cityscape / territorial configurations

% Why morphogenesis is important : linked with modularity and scale -> if a submodule can be explained independantly (ie morphigeneis process is isolated), then we have the characteristic scale. then when size grows and interaction within city system -> can not explain alone (or with externalities ?) -> need a change in scale. ex. influecne of city system for size, activities; posiiton of an airport in a metropoltian region ; emergence of MCR.  ==> Assimptions to be tested with models ?.

% Alexander and Salingaros

% include transportation network, hierarchy and congestion in transport : Remy vs Benjamin (paper ? -> see with René)


The idea of morphogenesis was introduced by \noun{Turing} in~\cite{turing1952chemical} when trying to isolate simple chemical rules that could lead to the emergence of the embryo and its form. The morphogenesis of a system consists in self-consistent evolution rules that produce the emergence of its successives states, i.e. the precise definition of self-organization. Progresses towards the understanding of embryo morphogenesis (in particular the isolation of processes producing the differentiation of cells from an unique cell) has been made only recently with the use of Complexity Approaches in integrative biology~\cite{delile2016chapitre}. In the case of urban systems, the idea of urban morphogenesis, i.e. of self-consistent mechanisms that would produce the urban form, is more used in the field of architecture and urban design~\cite{hachi2013master} (as \noun{Alexander} generative grammar ``Pattern Language'' e.g.), in relation with theories of Urban Form~\cite{moudon1997urban}. This idea can be pushed into very small scales such as the building~\cite{whitehand1999urban} but we will use it more at a mesoscopic scale, in terms of land-use changes within an intermediate scale territorial system, in the same ontologies as Urban morphogenesis modeling literature (for example \cite{bonin2012modele} describes a model of urban morphogenesis with qualitative differentiation, whereas \cite{makse1998modeling} give a model of urban growth based on a mono-centric population distribution perturbed with correlated noises). The notion of morphogenesis will be important in our theory in link with modularity and scale. Modularity of a complex system consists in its decomposition into relatively independent sub-modules, and modular decomposition of a system can be seen as a way to disentangle non-intrinsic correlations~\cite{2015arXiv150904386K} (think of a block diagonalisation of a first order dynamical system). The isolation of a subsystem yields a corresponding characteristic scale. Isolating possible morphogenesis processes imply a controlled isolation (controlled boundary conditions e.g.) of the considered system, corresponding to a modularity level and thus a scale. When self-consistent processes are not enough to explain the evolution of the system (with reasonable action on boundary conditions), a change of scale is necessary, caused by an underlying phase transition in modularity. The example of metropolitan growth is a good example : complexity of interactions within the metropolitan region will grow with size and diversity of functions leading to a change in scale necessary to understand processes. The emergence of an international airport will strongly influence local development, what corresponds to the significant integration within a larger system. The characteristic scales and processes for which these change occur will be precise questions to be investigated through modeling. It is interesting to remark that a territorial subsystem in which morphogenesis has a sense can be seen as an \emph{autopoietic system} in the extended sense of \noun{Bourgine} in~\cite{bourgine2004autopoiesis}, as a network of auto-reproducing processes\footnote{which are however not cognitive, making this auto-organized systems fortunately not alive in the sense of autopoietic and cognitive systems} regulating their boundary conditions, what emphasizes boundaries on which we will last insist.


% transition : Bourgine autopoiesis -> importance of boundaries -> link to Holland.

%%%%%%%%%%%
\subsubsection{Co-evolution}

% other insight : Holland Signal and Boundaries, ecological niche etc. : contextualize within this framework, clarify definition of co-evolution

Our last pillar is a clarification of the notion of \emph{co-evolution}, on which \noun{Holland} shed light through an approach of complex adaptive systems by a theory of CAS as signal processing agents operating thanks to their boundaries~\cite{holland2012signals}. In this theory, complex adaptive systems form aggregates at diverse hierarchical levels, that correspond to different level of self-organization, and boundaries are vertically and horizontally intricate in a complex way. That approach introduces the notion of \emph{niche} as a relatively independent subsystem in which ressources circulate (the same way as network communities) : numerous illustrations are given such as economical niches or ecological niches. Agents within a niche are said to be \emph{co-evolving}. Co-evolution thus means strong interdependences (implying circular causal processes) and a certain independence regarding the exterior of the niche. The notion is naturally flexible as it will depend on ontologies, resolution, thresholds etc. considered to define the system. This concept is easily transmissible to the evolutive urban theory and converges with the notion of co-evolution described by \noun{Pumain} : co-evolving agents in a system of cities consist in a niche with its flows, signals and boundaries and thus co-evolving entities in the sense of \noun{Holland}. This notion will be important for us in the definition of territorial subsystems and their coupling.



%%%%%%%%%%%
%\subsection{Requirements}
% RQ : no requirements for the theory, contained within pillars : requirement is the presence of these pillars ?



%%%%%%%%%%%
\subsection{Synthesis : an theory of co-evolutive networked territorial systems}

% put different elements together and construct the geographical theory
% give here precise definitions

We synthesize our pillars as a short self-consistent geographical theory of territorial systems in which networks play a central role in the co-evolution of components of the system. See the foundation subsection for definitions and references. The formulation is intended to be minimalistic.

\medskip

\begin{definition}
\textbf{ - Territorial System.} A territorial system is a set of networked human territories, i.e. human territories in and between which real networks exist.
\end{definition}

\medskip

At this step complexity and dynamical evolutive characters of territorial systems are implied but not an explicit part of the theory. We will assume to simplify a discrete definition of temporal, spatial and ontological scales under modularity and local stationarity assumptions.

% definition of scale and stationarity
%\textit{Equivalence between existence of discrete scales and discrete stationarity levels ?}

\medskip

\begin{proposition}
\textbf{ - Discrete scales.} Assuming a discrete modular decomposition of a territorial system, the existence of a discrete set $(\tau_i,x_i)$ of temporal and functional scales for the territorial system is equivalent to the local temporal stationarity of a random dynamical system specification of the system.
\end{proposition}

\begin{proof}
\textbf{(Sketch of).} We underlie that any territorial system can be represented by random variables, what is equivalent to have well defined objects and states and use the Transfer Theorem on events of successive states. If $X=(X_j)$ is the modular decomposition, we have necessarily quasi-independence of components in the sense that $\Covb{dX_j}{dX_{j'}}\simeq 0$ at any time. General stationarity transitions induce modular transitions that are kept or not depending if they correspond to an effective transition within the subsystem, what provide temporal scales as characteristic times of sub-dynamics. Functional scales are the corresponding extent in the state space.\qed
\end{proof}

% assumption : existence of scales

\medskip

This proposition induce a discrete representation of system dynamics in time. Note that even in the case of no modular representation, the system as a whole will verify the property. This definition of scales allows to explicitly introduce feedback loops and thus emergence and complexity, making our theory compatible with the evolutive urban theory.



\begin{assumption}
\textbf{ - Scales and Subsystems intrication. } Complex networks of feedbacks exist both between and inside scales, what impose the existence of weak emergence~\cite{bedau2002downward}. Furthermore a horizontal and vertical hierarchical imbrication of boundaries is not the rule.
\end{assumption}

% co-evolution

Within these complex subsystems intrications we can isolate co-evolving components using morphogenesis. The following proposition is a consequence of the equivalence between the independence of a niche and its morphogenesis. Morphogenesis provides the modular decomposition (local stationarity assumed) needed for the existence of scale, giving minimal vertically (scale) and horizontally (space) independent subsystems.

\begin{proposition}
\textbf{ - Co-evolution of components. } Morphogenesis processes of a territorial system are an equivalent formulation of the existence of co-evolutive subsystems.
\end{proposition}


% importance of nws as necessary subcomponents
%  maybe where we diverge from Denise theory ?

Finally we make a key assumption putting real networks at the center of co-evolutive dynamics, introducing their necessity to explain dynamical processes of territorial systems.

\begin{assumption}
\textbf{ - Necessity of Networks. } Network evolution can not be explained only by the dynamics of other territorial components and reciprocally, i.e. co-evolving territorial subsystems include real networks. They can thus be at the origin of regime changes (transition between stationarity regimes) or more dramatic bifurcations in dynamics of the whole territorial system.
\end{assumption}

On long time scale, an overall co-evolution has been shown for the french railway network by~\cite{bretagnolle:tel-00459720}. At smaller scales it is less evident (debate on structural effects) but we postulate that co-evolution effects are present at any scale. Regional examples may illustrate that : Lyon has not the same dynamical relations with Clermont than with Saint-Etienne and network connectivity has necessarily a role in that (among intrinsic interaction dynamics and distance). At a smaller scale, we think that effects are even less observable, but precisely because of the fact that co-evolution is stronger and local bifurcations will occur with stronger amplitude ans greater frequency than in macroscopic systems where attractors are more stable and stationarity scales greater. We will try to identify bifurcation or phase transitions in toy models, hybrid models and empirical analysis, at different scales, on different case studies and with different ontologies.

One difficulty in our construction is the stationarity assumption. Even if it seems a reasonable assumptions on large scales and has already been observed in empirical data~\cite{sanders1992systeme}, we shall verify it in our empirical studies. Indeed, this question is at the center of current research efforts to apply deep learning techniques to geographical systems : \noun{Bourgine} has recently developed a framework to extract patterns of Complex Adaptive Systems\footnote{Using a representation theorem~\cite{knight1975predictive}, any discrete stationary process is a \emph{Hidden Markov Model}. Given the definition of a causal state as $\Pb{future | A} = \Pb{future | B}$, the partition of system states induced by the corresponding equivalence relations allows to derive a \emph{Recurrent Network} that is enough to determine next state of the system, as it is a \emph{deterministic} function of previous state and hidden states~\cite{shalizi2001computational} : $(x_{t+1},s_{t+1}) = F\left[(x_t,s_t)\right]$. The estimation of Hidden States and of the Recurrent Function thus captures through deep learning entirely dynamical patterns of the system, i.e. full information on its dynamics and internal processes.}. The issues are then if the stationarity assumption be tackled through augmentation of system states, and if heterogeneous and asynchronous data be used to bootstrap long time-series necessary for a correct estimation of the neural network. These issue are related to the stationarity assumption for the first and to non-ergodicity for the second.




%----------------------------------------------------------------------------------------

\newpage

\section{A theoretical Framework for the Study of Socio-technical Systems}

After having set up the thematic theoretical framework, we develop a more general framework in which the previous can enter. At an epistemological level, it is essential to frame generally our directions of research.


\subsection{Introduction}

\subsubsection*{Scientific Context}


% this part may rather be in introduction ? -> ok here as explains for why this meta-framework is needed.

The structural misunderstandings between Social Sciences and Humanities on one side, and so-called Exact Sciences on the other side, far from being a generality, seems to have however a significant impact on the structure of scientific knowledge~\cite{2015arXiv151103981H}. In particular, the place of theory (and indeed the signification of this term itself) in the elaboration of knowledge has a totally different place, partly because of the different \emph{perceived complexities}\footnote{We used the term \emph{perceived} as most of systems studied by physics might be described as simple whereas they are intrinsically complex and indeed not well understood~\cite{laughlin2006different}.} of studied objects : for example, mathematical constructions and by extent theoretical physics are \emph{simple} in the sense that they are mostly entirely analytically solvable, whereas Social Science subjects such as humans or society (to give a \emph{clich{\'e}} exemple) are \emph{complex} in the sense of complex systems\footnote{for which no unified definition exists but of which fields of application range broadly from neuroscience to quantitative finance, including e.g. quantitative sociology, quantitative geography, integrative biology, etc.~\cite{newman2011complex}, and for which study various complementary approaches may be applied, such as Dynamical Systems, Agent-based Modeling, Random Matrix Theory}, thus a stronger need of a constructed theoretical (generally empirically based) framework to identify and define the objects of research that are necessarily more arbitrary in the framing of their boundaries, relations and processes, because of the multitude of possible viewpoints : Pumain suggests indeed in~\cite{pumain2005cumulativite} a new approach to complexity deeply rooted in social sciences that ``would be measured by the diversity of disciplines needed to elaborate a notion''. These differences in backgrounds are naturally desirable in the spectrum of science, but things can get nasty when playing on ``common'' terrains, typically complex systems problematics as already detailed, as the exemple of geographical urban systems has recently shown~\cite{dupuy2015sciences}. Complex System Science\footnote{that we deliberately call that way although there is a running debate on wether it can be seen as a Science in itself or more as a different way to do Science.} is presented by some as a ``new kind of Science''~\cite{wolfram2002new}, and would at least be a symptom of a shift in scientific practices, from analytical and ``exact'' approaches to computational and evidence-based approaches~\cite{arthur2015complexity}, but what is sure is that it brings, together with new methodologies, new scientific fields in the sense of converging interests of various disciplines on transversal questions or of integrated approaches on a particular field~\cite{2009arXiv0907.2221B}.



\subsubsection*{Objectives}

Within that scientific context, the study of what we will call \emph{Socio-technical Systems}, which we define in a rather broad way as hybrid complex systems including social agents or objects that interact with technical artifacts and a natural environment\footnote{geographical systems in the sense of \cite{dollfus1975some} are the archetype of such systems, but that definition may cover other type of systems such as an extended transportation system, social systems taken with an environmental context, complicated industrial systems taken with users, etc.}, lies precisely between social sciences and hard sciences. The example of Urban Systems is the best example, as already before the arrival of approaches claiming to be ``more exact'' than soft approaches (typically by physicists, see e.g. the rather disturbing introduction of~\cite{louf2014scaling}, but also by scientists coming from social sciences such as Batty~\cite{batty2013new}), many aspects of urban systems were already in the field of exact sciences, such as urban hydrology, urban climatology or technical aspects of transportation systems, whereas the core of their study relied in social sciences such as geography, urbanism, sociology, economy. Therefore a necessary place of theory in their study : following~\cite{livet2010}, the study of complex systems in social science is an interaction between empirical analysis, theoretical constructions, and modeling.

We propose in this paper to construct a theory, or rather a theoretical framework, that would ease some aspects of the study of such systems. Many theories already exist in all fields related to this kind of problems, and also at higher levels of abstraction concerning methods such as agent-based modeling e.g., but there is to our knowledge no theoretical framework including all of the following aspects that we consider as being crucial (and that can be understood as an informal basis of our theory) :
% QUESTION : do we need to include here empirical examples to support these claims ? - or are there taken as granted, the theoretical basis (that indeed comes from empirical practice)?
\begin{enumerate}
\item a precise definition and emphasis on the notion of coupling between subsystems, in particular allowing to qualify or quantify a certain degree of coupling : dependence, interdependence, etc. between components.
\item a precise definition of scale, including timescale and scales for other dimensions.
\item as a consequence of the previous points, a precise definition of what is a system.
\item the inclusion of the notion of emergence in order to capture multi-scale aspects of systems.
\item a central place of ontology in the definition of systems, i.e. of the sense in the real world given to its objects\footnote{\textit{as already explained before, this positioning along with the importance of structure may be related to Ontic Structural Realism~\cite{frigg2011everything} in further in further developments.}
}.
\item taking into account heterogeneous aspects of the same system, that could be heterogeneous components but also complementary intersecting views.
\end{enumerate}


The rest of this section is organized as follows : we construct the theory in the following part, staying at an abstract level, and propose a first application to the question of co-evolving subsystems. We then discuss positioning regarding existing theories, and possible developments and concrete applications.


\subsection{Construction of the theory}

\subsubsection*{Perspectives and Ontologies}

The starting point of the theory construction is a perspectivist epistemological approach on systems introduced by Giere~\cite{giere2010scientific}.  To sum up, it interprets any scientific approach as a perspective, in which someone pursues some objective and uses what is called \emph{a model} to reach it. The model is nothing more than a scientific medium. Varenne developed~\cite{varenne2010framework} model typologies that can be interpreted as a refinement of this theory. Let for now relax this possible precision and use perspectives as proxies of the undefined objects and concepts. Indeed, different views on the same object (being complementary or diverging) have the property to share at least the object in itself, thus the proposition to define objects (and more generally systems) from a set of perspectives on them, that verify some properties that we formalize in the following.

A perspective is defined in our case as a dataflow machine $M$ (that corresponds to the model as medium) in the sense of~\cite{golden2012modeling} that gives a convenient way to represent it and to introduce timescales, to which is associated an ontology $O$ in the sense of~\cite{livet2010}, i.e. a set of elements each corresponds to a \emph{thing} (it can be an object, an agent, a process, etc.) % better word than ``thing'' ? seems appropriate as can be object, agent, process, etc.
 in the real world. We include only two aspect (the model and the objects represented) of Giere's theory, making the assumption that purpose and user of the perspective are indeed contained in the ontology.

\begin{definition}
A \emph{perspective on a system} is given by a dataflow machine $M = (i,o,\mathbb{T})$ and an associated ontology $O$. We assume that the ontology can be decomposed into atomic elements $O=(O_j)_j$.
\end{definition}

The atomic elements of the ontology can be particular elements such as agents or components of the system, but also processes, interactions, states, or concepts for example. The ontology can be seen as the rigorous description of the content of the perspective. The assumption of a dataflow machine implies that possible inputs and outputs can be quantified, what is not necessarily restrictive to quantitative perspectives, as most of qualitative approaches can be translated into discrete variables as long as the set of possibles is known or assumed. 

The system is then defined ``reversely'', i.e. from a set of perspectives on a system :

% def of a system as a set of perspectives.
\begin{definition}
A \emph{system} is a set of \emph{perspectives on a system} : $S = (M_i,O_i)_{I\in I}$, where $I$ may be finite or not.
\end{definition}

We denote by $\mathcal{O} = (O_{j,i})_{j,i\in I}$ the set of all elements within ontologies.

Note that at this level of construction, there is not necessarily any structural consistence in what we call a system, as given our broad definition could allow for example to consider as a system a perspective on a car together with a perspective on a system of cities what makes reasonably no sense at all. Further definitions and developments will allow to be closer from classical definition of a system (interacting entities, designed artifacts, etc.). The same way, the definition of a subsystem will be given further. The introduced elements of our approach help to tackle so far points three, five and six of the requirements.

\paragraph{Precision on the recursive aspect of the theory}

One direct consequence of these definitions must be detailed : the fact that they can be applied recursively. Indeed, one could imagine taking as perspective a system in our sense, therefore a set of perspectives on a system, and do that at any order. If ones takes a system in any classical sense, then the first order can be understood as an epistemology of the system, i.e. the study of diverse perspectives on a system. A set of perspectives on related systems may in some conditions be a domain or a field, thus a set of perspectives on various related systems the epistemology of a field. These are more analogies to give the idea behind the recursive character of the theory. It is indeed crucial for the meaning and consistence of the theory because of the following arguments :
\begin{itemize}
\item The choice of perspectives in which a system consists is necessarily subjective and therefore understood as a perspective, and a perspective on a system if we are able to build a general ontology.
\item We will use relations between ontologies in the following, which construction based on emergence is also subjective and seen as perspectives.
\end{itemize}



\subsubsection*{Ontological Graph}

% construction of the ontological graph / canonical tree decomposition
%  : pb with relation in the ontological graph : weak or strong emergence ?

We propose then to capture the structure of the system by linking ontologies. This approach could eventually be linked to structural realism epistemological positioning~\cite{frigg2011everything} as knowledge of the world is partly contained here in structure of models. % precise here epistemological positioning, we may be clearly within a structural realist positioning !!
 Therefore, we choose to emphasize the role of emergence as we believe that it may be one practical minimalist way to capture quite well complex systems structure\footnote{what of course can not been presented as a provable claim as it depends on system definition, etc.}. We follow on that point the approach of Bedau on different type of emergences, in particular his definition of weak emergence given in~\cite{bedau2002downward}. Let recall briefly definitions we will use in the following. Bedau starts from defining emerging properties and then extends it to phenomena, entities, etc. The same way, our framework is not restricted to objects or properties and wrapped thus the generalized definitions into emergence between ontologies. We will apply the notion of emergence under the two following forms\footnote{the third form Bedau recalls, \emph{Strong emergence} will not be used, as we need only to capture dependance and autonomy, and weak emergence is more satisfying in terms of complex systems, as it does not assume ``irreducible causal powers'' to the greater scale objects. Nominal emergence is used to capture inclusion between ontologies.} :
\begin{itemize}
\item \emph{Nominal emergence} : one ontology $O'$ is included in an other $O$ but the aspect of $O$ that is said to be nominally emergent regarding $O'$ does not depend on $O'$.
\item \emph{Weak emergence} : one part of an ontology $O$ can be derived by aggregation of elements and interactions between elements of an ontology $O'$.
\end{itemize}

As developed before, the presence of emergence, and especially weak emergence, will consist in itself in a perspective. It can be conceptual and postulated as an axiom within a thematic theory, but also experimental if clues of weak emergence are effectively measured between objects. In any case, the relation between ontologies must be encoded within an ontology, which was not necessarily introduced in the initial definition of the system.


% Observation : -
%  - Ontologies are sets -> relation between subsets.
%  - include emergence relations between different perspectives of the system ? would imply a coupling ontology ? -> YES - system is not only a set but also a relation between its elements -> introduce the 'coupling ontology' ; or assumes there exists one ?

We make therefore the following assumption for next developments :
\begin{assumption}
A system can be partially structured by extending it with an ontology that contains (not necessarily only) relations between elements of ontologies of its perspectives. We name it the \emph{coupling ontology} and assume its existence in the following. We assume furthermore its atomicity, i.e. if $O$ is in relation with $O'$, then any subsets of $O,O'$ can not be in relation, what is not restrictive as a decomposition into several independent subsets ensures it if it is not the case.
\end{assumption}

It allows to exhibit emergence relations not only within a perspective itself but also between elements of different perspectives. We define then pre-order relations between subsets of ontologies : % check pre-order and equivalence relations.

% order relations between ontologies
% first recall Bedau's paper. 
% check equivalence relation ; pre-order <- relations are only pre-order ?
\begin{proposition}
The following binary relationships are pre-orders on $\mathcal{P(O)}$ :
\begin{itemize}
\item Emergence (based on Weak Emergence) : $O' \preccurlyeq O$ if and only if $O$ weakly emerges from $O'$.
\item Inclusion (based on Nominal Emergence) : $O' \Subset O$ if and only if $O$ nominally emerges from $O'$.
\end{itemize}
\end{proposition}

\begin{proof}
With the convention that it can be said that an object emerges from itself, we have reflexivity (if such a convention seems absurd, we can define the relationships as \emph{$O$ emerges from $O'$ or $O=O'$ }). Transitivity is clearly contained in definitions of emergence.
\end{proof}

\medskip

Note that the inclusion relation is more than an inclusion between sets, as it translates an inclusion ``inside'' the elements of the ontology.

These relations are the basis for the construction of a graph called the \emph{ontological graph} :

% definition of the ontological graph
% ! beware to put only neighbor relations within the graph
% and to reconstruct by induction subsets at any level ?
\begin{definition}
The {ontological graph} is constructed by induction the following way :
\begin{enumerate}
\item A graph with vertices elements of $\mathcal{P(O)}$ and edges of two types : $E_W = \{(O,O') | O' \preccurlyeq O \}$ and $E_N = \{(O,O') | O' \Subset O \}$
\item Nodes are reduced\footnote{the reduction procedure aims to delete redundancy, keeping an entity at the higher level it exists.} by : if $o \in O,O'$ and ($O' \preccurlyeq O$ or $O' \Subset O$) but not ($O \preccurlyeq O'$ or $O \Subset O'$), then $O' \leftarrow O' \setminus o$
\item Nodes with intersecting sets are merged, keeping edges linking merged nodes. This step ensures non-overlapping nodes.
\end{enumerate}
\end{definition}


% definition of a subsystem in the large sense ? -> done after treeing the graph
% can be done only if reconstruction is possible.



\subsubsection*{Minimal Ontological Tree}

% theorem : tree construction with the tree-bag decomposition theorem ; and show trivially that loops are at the same scale
% first get a connected component of the graph -> here the consistence is captured : if ontologies have nothing to do at all, then it cant be the same system.

The topological structure of the graph, that contains in a way the \emph{structure of the system}% positioning regarding structural realism.
, can be reduced into a minimal tree that contains hierarchical structure essential to the theory.

We need first to give consistence to the system :

\begin{definition}
A consistent part of the ontological graph is a weakly connected component of the graph. We assume for now to work on a consistent part.
\end{definition}

% rq : consistent system ? -> would need to reconstruct perspectives ? possible ? under certain assumption ? 'decoupling ontology' ? -> need to be worked harder.

The notion of consistent system, together with subsystem or nodes timescales that will be defined later, requires to reconstruct perspectives from ontological elements, i.e. the inverse operation of what was done in our deconstruction procedure.

\begin{assumption}
There exists $\mathcal{O}' \subset \mathcal{P(O)}$ such that for any $O \subset \mathcal{O}'$, there exists a corresponding dataflow machine $M$ such that the corresponding perspective is consistent with initial elements of the system (i.e. machines on ontology overlaps are equivalent). If $\Phi : M \mapsto O$ is the initial mapping, we denote this extended reciprocal construction by $M' = \Phi^{<-1>}(O)$.
\end{assumption}

\paragraph{Remark.}

This assumption could eventually be changed into a provable proposition, assuming that the coupling ontology is indeed a coupling perspective, which dataflow machine part is consistent with coupled entities. Therein, the decomposition postulate of~\cite{golden2012modeling} should allow to identify basic components corresponding to each element of the ontology, and then construct the new perspective by induction. We find however these assumptions too restrictive, as for example various ontological elements may be modeled by an irreducible machine, as a differential equations with aggregated variables. We prefer to be less restrictive and postulate the existence of the reverse mapping on some sub-ontologies, that should be in practice the ones where couplings can be effectively modeled.

Given this assumption, we can define the consistent system as the reciprocal image of the consistent part of the ontological graph. It ensures system connectivity what is a requirement for tree construction.

\begin{proposition}
The tree decomposition of the ontological graph in which nodes contains strongly connected components is unique. The corresponding reduced tree, that corresponds to the ontological graph in which strongly connected components have been merged with edges kept, is called the \emph{Minimal Ontological Tree}.
\end{proposition}

\begin{proof}
(sketch of) The unicity is obtained as nodes are fixed as strongly connected components. It is trivially a tree decomposition (with no edges) as in a directed graph, strongly connected components do not intersect, thus the consistence of the decomposition.
\end{proof}

Any loop $O \rightarrow O' \rightarrow \ldots \rightarrow O$ in the ontological graph assumes that all its elements are equivalent in the sense of $\preccurlyeq$. This equivalence loops should help to define the notion of strong coupling as an application of the theory (see applications).

\medskip

The Minimal Ontological Tree (MOT) is a tree in the undirected sense but a forest in the directed sense. Its topology contains a sort of system hierarchy. Consistent subsystems are defined from the set $\mathcal{B}$ of branches of the forest, as $(\Phi^{<-1>}(\mathcal{B}),\mathcal{B})$. The timescale of a node, and by extension of a subsystem, is the union of timescales of corresponding machines. Levels of the tree are defined from root nodes, and the emergence relations between nodes implies a vertical inclusion between timescales.



% check shortcuts reduction in graph construction : if O < O' < O'' , then O -> O'' is not an edge, must take the longest path.



\subsubsection*{Scales}

Finally, we propose to define scales associated to a system. Following~\cite{manson2008does}, an epistemological continuum of visions on scale is a consequence of differences between disciplines in the way we developed in the introduction. This proposition is indeed compatible with our framework, as the construction of scales for each level of the ontological tree results in a broad variety of scales.

Let $(M,O)$ a subsystem and $\mathbb{T}$ the corresponding timescale. We propose to define the ``thematic scale'' (for example spatial scale) assuming a representation theorem, i.e. that an aspect (thematic aspect) of the machine can be represented as a dynamic state variable $\vec{X}(t)$. Assuming a scale operator\footnote{that can be of various nature : extent, probabilistic extent, spectral scales, stationarity scales, etc.} $\norm{\cdot}_{S}$ and that the state variable has a certain level of differentiability, the \emph{thematic scale} if defined as $\norm{(d^k \vec{X}(t))_k}_S$


\subsection{Application}

\subsubsection*{The particular case of geographical systems}


%%
%  Observation :
%   ontologies could evolve in time ? they do -> cf Lucie's framework on spatio-temporal functional decomposition : each atomic unit can have a different ontology. not an issue : connects ontologies through a superior layer representing the object through all temporal extent ; the object nominally emerges from each temporal ontology.


In \cite{dollfus1975some} % ~ same def of system ; introduce structure in context ? link to explanation ?
 \noun{Durand-Dast{\`e}s} proposes a definition of geographical structure and system, structure would be the spatial container for systems viewed as complex open interacting systems (elements with attributes, relations between elements and inputs/outputs with external world). For a given system, its definition is a perspective, completed by structure to have a system in our sense. Depending on the way to define relations, it may be easy to extract ontological structure.
 %\textit{Note : find typical emergence clues in standard relational formalizations ? would guide the application of the theory.}


\subsubsection*{Modularity and co-evolving subsystems}


For the example of Urban Systems, urban evolutionary theory enters this framework using our previous thematic theory? The decomposition into uncorrelated subsystems yields precisely strongly coupled components as co-evolving components. The correlation between subsystems should be positively correlated with topological distance in the tree. If we define elements of a node before merging as \emph{strongly coupled elements}, in the case of dynamic ontologies, it provides a definition of \emph{co-evolution} and co-evolving subsystems equivalent to the thematic definition.


\subsection{Discussion}


\paragraph{Link with existing frameworks}

A link with the Cottineau-Chapron framework for multi-modeling~\cite{10.1371/journal.pone.0138212} may be done in the case they add the bibliographical layer, which would correspond to the reconstruction of perspectives. \cite{reymond2013logique} proposes the notion of ``interdisciplinary coupling'' what is close to our notion of coupling perspectives. A correspondance with System of Systems approaches (see e.g. \cite{luzeaux2015formal} for a recent general framework englobing system modeling and system description) may be also possible as our perspectives are constructed as dataflow machines, but with the significant difference that the notion of emergence is central.


\paragraph{Contributions to the study of complex systems}

\begin{itemize}
\item We do not claim to provide a theory of systems (beware of cybernetics, systemics etc. that could not model everything), but more a framework to guide research questions (e.g. in our case the direct outcomes will be quantitative epistemology that comes from system construction as perspectives research ; empirical to construct robust ontologies for perspectives ; targeted thematic to unveil causal relationship/emergence for construction of ontological network ; study of coupling as possible processes containing co-evolution ; study of scales ; etc.). It may be understood as meta-theory which application gives a theory, the thematic theory developed before being a specific implementation to territorial networked systems.
\item We Emphasize the notion of socio-technical system, crossing a social complex system approach (ontologies) with a description of technical artifacts (dataflow machines), taking the ``best of both worlds''.
\end{itemize}




\subsection{Research Directions}

We can draw from the construction of this theoretical framework a set of research directions, that give a general line on how trying to answer to research questions asked after the thematic theory construction.

\begin{enumerate}
\item The perspectivist approach implies a broad understanding of existing perspectives on a system, and of possibility of coupling between them ; thus an emphasis on applied epistemology, i.e. \textbf{Algorithmic Systematic Review} (exploration of the knowledge space), \textbf{Disciplines Mapping}(extraction of its structure) and \textbf{Datamining for Content Analysis}(refinement at the atomic level in scientific knowledge) that correspond to the three sections of chapter~\ref{ch:quantepistemo}.
\item At a finer level of particularization, the knowledge of perspectives means \textbf{Knowledge of stylized facts}, i.e. empirical analysis of cases studies. These are the object of chapter~\ref{ch:empirical}.
\item The emphasis on coupled subsystems at different scales implies a deep understanding of coupling mechanisms, thus the need of methodological and technical developments : \textbf{Methods for Statistical Control}, \textbf{Methods for Model Exploration}, \textbf{Theoretical Study of Coupling}, \textbf{Multi-Modeling}, of which some are developed and other proposed in the methodological chapter~\ref{ch:methodology}.
\item Furthermore, the possibility of hidden elements within the ontology implies the test for causal relations and intermediate processes at the origin of emergence, thus e.g. the exploration of new paradigms such as role of governance within complex models as done in chapter~\ref{ch:complexmodels}.
\item Finally, the idea behind system structure contained within the ontological forest is a large set of coupled models for a given system : it means that a proper system definition (i.e. thematic problematization and exploration) and construction should yield to a structured family of models : parallel branches can be different implementations of the same process or various processes trying to explain the emerging ontology ; therefore the final objective of a family of models tackling the thematic question.
\end{enumerate}










