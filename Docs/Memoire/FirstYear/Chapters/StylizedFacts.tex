

% Chapter 

\chapter{Empirical Analysis : Insights from Stylized Facts} % Chapter title

\label{ch:methodology} % For referencing the chapter elsewhere, use \autoref{ch:name} 

%----------------------------------------------------------------------------------------


%  plan : 

%  1) static morphological analysis : requires a formal link between temporal and spatial correlations ?  -- typology etc can already be interesting --


%  2) presentation of BP case study

%  3) base Bien

%  4) Work with Solène




%----------------------------------------------------------------------------------------


%%%%%%%%
%  Section : static analysis
%%%%%%%%

\section{Static correlations of urban form and network shape for European territorial systems}


%%%%%%%%%%%%%%%%%%
\subsection{Morphological Measures of European Population Density}




%%%%%%%%%%%%%%%%%%
\subsection{Network Measures}



%%%%%%%%%%%%%%%%%%
\subsection{Effective static correlations}








%----------------------------------------------------------------------------------------

\section{Disentangling co-evolutions from causal relations : a case study on \emph{Bassin Parisien}}




\subsection{Context Formalization}

\subsubsection{Variables}

\paragraph{Description}

We assume a dynamic transportation network $n(\vec{x},t)$ within a dynamic territorial landscape $\vec{T}(\vec{x},t)$, which components are to simplify population $p(\vec{x},t)$ and employments $e(\vec{x},t)$. Data is structured the following way :
\begin{itemize}
\item Observation of territorial variables are discretized in space and in time, i.e. the spatial field $\vec{T}$ is summarized by $\mathbf{T} = \left(\vec{T}(\vec{x}_i,t_j^{(T)})\right)_{i,j}$ with $1\leq i \leq N$ and $1\leq j \leq T$. They concretely correspond to census on administrative units (\emph{communes} in our case) at different dates.
\item Network has a continuous spatial position but
\end{itemize}



\paragraph{Definitions}



\subsubsection{Accessibility}

% accessibility : need to introduce it ?
%  -> read Weibull

The notion of accessibility has been central to regional science since its introduction and systematization in planning around 1970. 

\paragraph{Existence of accessibility}

An elegant axiomatic definition is derived in~\cite{weibull1976axiomatic}. Starting from expected properties of an accessibility function $A$ that associate a value to \emph{attraction} $a$ and distance $d$, defined on the set of discrete spatial configurations $\mathcal{C} = \cup_{n\in \mathbb{N}}{(d_i,a_i)_{1\leq i \leq n}}$. These properties include (among technical others with no thematic meaning) :
\begin{enumerate}
\item $A$ is invariant regarding the order of the configuration
\item $A$ decrease with distance at fixed attraction and increase with attraction at fixed distance
\item $A$ is invariant when adding null attractions and constant configurations
\end{enumerate}

A canonical decomposition of any accessibility function 


\paragraph{Continuous approach and accessibility potential}

% Paul : Helmoltz-Hodge theorem to infer potential field from speed spatial field ?
%  Q : what are trajectories ? dirac field has no rotational -> continuous approach does not work ?






\subsection{Statistical Tests}


\subsubsection{Bivariate linear models}

\subsubsection{Autocorrelated univariate models}

\subsubsection{Autocorrelated multivariate models}

\subsubsection{Granger causality tests}

\cite{xie2009streetcars} use Granger causality to link transit with land-use changes.


\subsubsection{Autoregressive multivariate models}



\subsubsection{Autoregressive autocorrelated multivariate models}











%----------------------------------------------------------------------------------------

\section{Early warnings of Network Breakdowns : socio-economic and real-estate trajectories}

















%----------------------------------------------------------------------------------------

\section{South-African historical events as instruments to understand network-territory relations}








