

% Chapter 

\chapter{Empirical Analysis : Insights from Stylized Facts} % Chapter title

\label{ch:methodology} % For referencing the chapter elsewhere, use \autoref{ch:name} 

%----------------------------------------------------------------------------------------


%  plan : 

%  1) static morphological analysis : requires a formal link between temporal and spatial correlations ?  -- typology etc can already be interesting --


%  2) presentation of BP case study

%  3) base Bien

%  4) Work with Solène




%----------------------------------------------------------------------------------------


%%%%%%%%
%  Section : static analysis
%%%%%%%%

\section{Static correlations of urban form and network shape for European territorial systems}


%%%%%%%%%%%%%%%%%%
\subsection{Morphological Measures of European Population Density}




%%%%%%%%%%%%%%%%%%
\subsection{Network Measures}



%%%%%%%%%%%%%%%%%%
\subsection{Effective static correlations}








%----------------------------------------------------------------------------------------

\section{Disentangling co-evolutions from causal relations : a case study on \emph{Bassin Parisien}}




\subsection{Context Formalization}

\subsubsection{Variables}

\paragraph{Description}

We assume a dynamic transportation network $n(\vec{x},t)$ within a dynamic territorial landscape $\vec{T}(\vec{x},t)$, which components are to simplify population $p(\vec{x},t)$ and employments $e(\vec{x},t)$. Data is structured the following way :
\begin{itemize}
\item Observation of territorial variables are discretized in space and in time, i.e. the spatial field $\vec{T}$ is summarized by $\mathbf{T} = \left(\vec{T}(\vec{x}_i,t_j^{(T)})\right)_{i,j}$ with $1\leq i \leq N$ and $1\leq j \leq T$. They concretely correspond to census on administrative units (\emph{communes} in our case) at different dates.
\item Network has a continuous spatial position but
\end{itemize}



\paragraph{Definitions}



\subsection{On Accessibility}

% accessibility : need to introduce it ?
%  -> read Weibull

The notion of accessibility has been central to regional science since its introduction and systematization in planning around 1970. 

\paragraph{Existence of accessibility}

%An elegant axiomatic definition is derived in~\cite{weibull1976axiomatic}. Starting from expected properties of an accessibility function $A$ that associate a value to \emph{attraction} $a$ and distance $d$, defined on the set of discrete spatial configurations $\mathcal{C} = \cup_{n\in \mathbb{N}}{(d_i,a_i)_{1\leq i \leq n}}$. These properties include (among technical others with no thematic meaning) :
%\begin{enumerate}
%\item $A$ is invariant regarding the order of the configuration
%\item $A$ decrease with distance at fixed attraction and increase with attraction at fixed distance
%\item $A$ is invariant when adding null attractions and constant configurations
%\end{enumerate}

%A canonical decomposition of any accessibility function 




\textit{\textbf{Is the notion of accessibility crucial for statistical analysis ?}}

\medskip


Weibull has proposed an axiomatic approach to accessibility~\cite{weibull1976axiomatic}, deriving a canonical decomposition for any \emph{attraction-accessibility} function $A(a,d)$, assuming expected thematic axioms among others technical ones that are :
\begin{enumerate}
\item \footnotesize $A$ is invariant regarding the order of the configuration
\item \footnotesize $A$ decrease with distance at fixed attraction and increase with attraction at fixed distance
\item \footnotesize $A$ is invariant when adding null attractions and constant configurations
\end{enumerate}
Then $A$ verifies these \emph{iff} it is of the form
\[
A\left[(a_i,d_i)\right] = T\left(\bigoplus_i z(d_i,a_i)\right)
\]
where $T$ is increasing with null origin, $z$ is a \emph{distance substitution function} (i.e. verifying axiom 2) and $\oplus$ a \emph{standard composition} associating two attractions at zero distance to th corresponding unique one. 

$\rightarrow$ \textit{Well suited matrices of autocorrelation should capture accessibility in regressions ; or captured by non-linear regression on $\mathbf{N}$}

\medskip

{\normalsize\textit{\textbf{Accessibility as potential ?}}}

Given any stationary dynamic for $n,\vec{T}$, Helmoltz theorem states that it derives from a potential (can be adapted to non-stationary dynamics with time-varying potential).








\paragraph{Continuous approach and accessibility potential}

% Paul : Helmoltz-Hodge theorem to infer potential field from speed spatial field ?
%  Q : what are trajectories ? dirac field has no rotational -> continuous approach does not work ?






\subsection{Statistical Tests}



\textit{Large set of analysis to be tested (non exhaustive) :}
\begin{itemize}
\item On data :
\begin{itemize}
\item Multivariate models $\mathcal{L}\left[\mathbf{T},\mathbf{N}\right]\sim \varepsilon$
\item Autocorrelated univariate models $(\mathbf{I} - \Sigma R W) \mathbf{X} \sim \varepsilon$
\item Autocorrelated multivariate models $(\mathcal{L}' - \Sigma R W)\left[\mathbf{T}+\mathbf{N}\right] \sim \varepsilon$
\item Geographically Weighted Regression~\cite{brunsdon1998geographically}
\[
\mathcal{L}\left[\mathcal{G}\left(\mathbf{T},\mathbf{N}\right)\right] \sim \varepsilon
\]
\item Granger causality tests : \cite{xie2009streetcars} use Granger causality to link transit with land-use changes.
\end{itemize}
\item On data returns :
\begin{itemize}
\item Autoregressive multivariate models
\[\mathcal{L}\left[(\Delta \mathbf{T}(t_{j'}))_{j'\leq j},(\Delta \mathbf{N}(t_{j'}))_{j'\leq j}\right] \sim \varepsilon\]
\item Autoregressive autocorrelated multivariate models : idem with spatial autocorrelation term.
\item Synthetic Instrumental Variables : static territory and/or network ?
\end{itemize}
\end{itemize}



\subsubsection{Bivariate linear models}

\subsubsection{Autocorrelated univariate models}

\subsubsection{Autocorrelated multivariate models}

\subsubsection{Granger causality tests}

\cite{xie2009streetcars} use Granger causality to link transit with land-use changes.


\subsubsection{Autoregressive multivariate models}



\subsubsection{Autoregressive autocorrelated multivariate models}











%----------------------------------------------------------------------------------------

\section{Early warnings of Network Breakdowns : socio-economic and real-estate trajectories}

















%----------------------------------------------------------------------------------------

\section{South-African historical events as instruments to understand network-territory relations}








