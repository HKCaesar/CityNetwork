


%%%%%%%%%%%%%%%%
%%  Introduction
%%%%%%%%%%%%%%%%


%% Contents
%
%    - General considerations on Complex Systems, positioning etc (thesis in cs science etc)
%    - Thematic introduction, geographical introduction of the subject.
%
%   - precisions on v1 memoire : foreword ?
%
%    - reading precisions : organisation, interdependances etc 
%
%   - reflexive aspect : here ?  


\chapter*{Introduction}

\headercit{We need to find Banos' tenth modeling law}{Ren{\'e} Doursat}{}


``In consequence of a technical issue, traffic is interrupted on the line B of RER, for an unknown duration. More information will be given as soon as available''. There is a high probability that someone having lived or spent some time in the metropolitan region of Paris  has already heard this frightening announce and endured the difficult consequences the rest of his day.




%-------------------------------------------------

\section*{Scientific Context : Complexity Has Come of Age}


Contemporaneous science is progressively taking the shift of complexity in many fields. That also implies an epistemological revolution to abandon strict reductionism that failed in most of its synthesis attempts~\cite{anderson1972more}. Arthur recently recalled~\cite{arthur2015complexity} that a mutation of methods and paradigms was also at stake by the increasing role of computational approaches replacing purely analytical techniques generally self-limited in their modeling and resolution scope. Capturing \emph{emergent properties} in models of complex systems is one of the ways to understand the essence of these new approaches.

These considerations are well known in Social Science (both quantitative and qualitative), in which the complexity of studied agents and systems is the justification of their existence : if humans were particles a whole branch of fields may have never emerged as thermodynamics would have solved most of social issues. \marginpar{\textit{even if it would probably not have been the case as classical physics also failed in their attempts to include irreversibility and evolutions of Complex Adaptive Systems }} 
They are however less known nor accepted in more ``hard'' sciences such as physics : Laughlin develops in~\cite{laughlin2006different} a view of the discipline at least as at a ``frontier of knowledge'' then other fields appearing as less mature. Most of knowledge is of classical nature although a majority of structures and systems would be \emph{self-organized}, what means that the single microscopic laws are not enough to determine macroscopic properties unless system evolution is simulated (more precisely this property can be taken as a definition of emergence on which we will come back further, and self-organization is intrinsically emergent).


As an informal mix of epistemological positions, methods, and fields of applications, \emph{Complexity Science} relies on \emph{unconventional} paradigms such as the centrality of emergence and self-organization in most of phenomena of the real world, which make it lie on a frontier of knowledge closer of us than we can think, as Laughlin develops in~\cite{laughlin2006different}. Such concepts are indeed not new, as they were already enlighten by Anderson~\cite{anderson1972more}. Even cybernetics can be related to complexity by seing it as a bridge between technics and cognitive science~\cite{wiener1948cybernetics}. Later, synergetics~\cite{haken1980synergetics} paved the way for a theoretical approach of collective phenomena in physics. Reasons for the recent growth of works claiming a CS approach may be various. The explosion of computing power is surely one because of the central role of numerical simulations~\cite{varenne2010simulations}. They could also be the related epistemological progresses : apparition of the notion of perspectivism~\cite{giere2010scientific}, finer reflexions around the notion of model~\cite{varenne2013modeliser}\footnote{[note : beware of a chicken-egg type problem on the relation between scientific and epistemological progress]}. The theoretical and empirical potentialities of such approach play surely a role in their success, as confirmed in various domains of application (see~\cite{newman2011complex} for a general survey), as for example Network Science~\cite{barabasi2002linked} ; Neuroscience~\cite{koch1999complexity} ; Social Sciences  ; Geography~\cite{manson2001simplifying}\cite{pumain1997pour} ; Finance with the rising importance of econophysics~\cite{stanley1999econophysics}.





\section*{Interdisciplinarity}
%\textit{Note : that term does not exist in english but is a rough translation from french \emph{interdisciplinarit{\'e}}, that we believe to better express }


%WHY and HOW is interdisciplinarity essential ?


% Q : quote Morin ?

Beyond ``fashionable'' positions that can be the consequence of a blind following~\cite{dirk1999measure}, or more ambivalent, of a marketing strategy as the fight for funds is becoming a huge obstacle for research~\cite{bollen2014funding}, Science of Complexity is taking a hole new place in the academic landscape.
%Its success may have several origins such as unexpected approaches, theoretical and practical promising results, or the recent explosion of computational possibilities. % needed ? say the same further.




\paragraph{Conflicting Complexities and Cultural Differences}

Yet this scientific evolution that some see as a revolution~\cite{colander2003complexity}, or even as \emph{a new kind of science}~\cite{wolfram2002new}, could face intrinsic difficulties due to behaviors and a-priori of researchers as human beings. More precisely, the need of interdisciplinarity that makes the strength of Complexity Science may be one of its greatest weaknesses, since the highly partitioned structure of science organization has sometimes negative impacts on works involving different disciplines. We do not tackle the issue of overpublication, competition, indexes, which is more linked to a question of open science and its ethics, also of high importance but of an other nature. That barrier we are dealing with and we might struggle to triumph of, is the impact of certains \emph{cultural disciplinary differences} and outcoming conflicts on views and approaches. We shall now develop some concrete example that lead to such considerations when encoutered. They are of many different natures and concern different disciplines, such that it would not be honest to assume that the issue is not general. Each come from personal research experience


\textit{Physics reinvents geography.} 



\textit{Economic Geography or Geographical Economics ?}



\textit{Agent-based Modeling in Economy}


\textit{Finance}



The drama of scientific misunderstandings is that they can indeed annihilate progresses by interpreting as a falsification some work that answers to a totally different question. The example of a recent work on top-income inequalities given in~\cite{aghion2015innovation}, which conclusions are presented as opposed from the one obtained by Piketty~\cite{piketty2013capital}, follows such a scheme. Whereas Piketty focused on constructing long-time clean databases for income data and showed empirically a recent acceleration of income inequalities, his simple model aiming to link this stylized fact with the accumulation of capital has been criticized as oversimplified. On the other hand, Bergeaud \textit{et al.} prove by a model of innovation economics that \emph{under certain assumptions} income gaps may be beneficial to innovation and consequently a general utility. Thus diverging conclusions about the role of personal capitals in the economy. But diverging \emph{views} or \emph{interpretations} does not mean a scientific incompatibility, and one could imagine try to gather both approaches in an unified framework and model, yielding possibly similar or different interpretations. This integrated approach has chances to contain more information (depending on how coupling is done) and to be a further advance in Science. Indeed, coupling heterogeneous approaches at different levels and scales will be a cornerstone of our thesis, skeleton of the underlying philosophy and building brick of the theory we will propose.



%\paragraph{Intrinsic Obstacles}
%\paragraph{Keep It Complex, Stupid}

%\paragraph{Means Are Here, Let Use Them}



















%-------------------------------------------------

\section*{Complexity in Geography}







