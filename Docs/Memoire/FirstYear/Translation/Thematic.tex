


% Chapter 




\chapter{Réseaux et Territoires} % Chapter title

\label{ch:thematic} % For referencing the chapter elsewhere, use \autoref{ch:name} 


%%  Thematic chapter framing geographically the subject.
%%   and reviewing state of the art
%%   and why modeling : evolutive theory of urban systems etc ; multimodeling simfamily etc
%%  
%%   Q  : example to introduce theory ?
%
%   Modelography.  (non-exhaustive) : classification according to purpose, theme, scale, etc.
%   Why dynamic models of ``co-evolution''  ?
%   definition of terms, contextualisation, etc.  (le what/where d'Arnaud ; ontology de Anne)



%----------------------------------------------------------------------------------------

%\headercit{If you are embarrassed by the precedence of the chicken by the egg or of the egg by the chicken, it is because you are assuming that animals have always be the way they are}{Denis Diderot}{\cite{diderot1965entretien}}

\headercit{Si la question de la priorit{\'e} de l'\oe{}uf sur la poule ou de la poule sur l'\oe{}uf vous embarrasse, c'est que vous supposez que les animaux ont {\'e}t{\'e} originairement ce qu'ils sont {\`a} pr{\'e}sent.}{Denis Diderot}{\cite{diderot1965entretien}}

\bigskip


Cette analogie est idéale pour introduire les notions de causalité et de processus dans les systèmes territoriaux. En voulant traiter naïvement des questions similaires à notre question de recherche préliminaire, certains on qualifiés les causalités au sein de systèmes complexes comme un problème ``de poule et {\oe}uf'' : si un effet semble causer l'autre et réciproquement, comment est-il possible d'isoler les processus correspondants ? Cette vision est souvent présente dans les approches réductionnistes qui ne postulent pas une complexité intrinsèque au sein des systèmes étudiés. L'idée suggérée par \noun{Diderot} est celle de \emph{co-evolution} qui est un phénomène central dans les dynamiques évolutionnaires des Systèmes Complexes Adaptatifs comme \noun{Holland} élabore dans~\cite{holland2012signals}. Il fait le lien entre la notion d'émergence (ignorée dans les approches réductionnistes), en particulier l'émergence de structures à une plus grand échelle par les interactions entre agents à une échelle donnée, en général concrétisée par un systèmes de limites, qui devient cruciale pour la co-évolution des agents à toutes les échelles : l'émergence d'une structure sera simultanée avec une autre, chacune exploitant leur interrelations et environnements générés conditionnés par le système de limites. Nous explorerons ces idées pour le cas des systèmes territoriaux par la suite.


Ce chapitre introductif est destiné à poser le cadre thématique, le contexte géographique sur lesquels les développements suivants se baseront. Il n'est pas supposé être compris comme une revue de littérature exhaustive ni comme les fondations théoriques fondamentales de notre travail (le premier point étant l'objet du chapitre~\ref{ch:quantepistemo} tandis que le second sera traité plus tôt dans le chapitre~\ref{ch:theory}), mais plutôt comme une construction narrative ayant pour but d'introduire nos objets et positions d'étude, afin de construire naturellement des questions de recherche précises.



