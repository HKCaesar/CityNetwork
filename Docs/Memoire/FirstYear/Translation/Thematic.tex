


% Chapter 




\chapter{Réseaux et Territoires} % Chapter title

\label{ch:thematic} % For referencing the chapter elsewhere, use \autoref{ch:name} 


%%  Thematic chapter framing geographically the subject.
%%   and reviewing state of the art
%%   and why modeling : evolutive theory of urban systems etc ; multimodeling simfamily etc
%%  
%%   Q  : example to introduce theory ?
%
%   Modelography.  (non-exhaustive) : classification according to purpose, theme, scale, etc.
%   Why dynamic models of ``co-evolution''  ?
%   definition of terms, contextualisation, etc.  (le what/where d'Arnaud ; ontology de Anne)



%----------------------------------------------------------------------------------------

%\headercit{If you are embarrassed by the precedence of the chicken by the egg or of the egg by the chicken, it is because you are assuming that animals have always be the way they are}{Denis Diderot}{\cite{diderot1965entretien}}

\headercit{Si la question de la priorit{\'e} de l'\oe{}uf sur la poule ou de la poule sur l'\oe{}uf vous embarrasse, c'est que vous supposez que les animaux ont {\'e}t{\'e} originairement ce qu'ils sont {\`a} pr{\'e}sent.}{Denis Diderot}{\cite{diderot1965entretien}}

\bigskip


Cette analogie est idéale pour introduire les notions de causalité et de processus dans les systèmes territoriaux. En voulant traiter naïvement des questions similaires à notre question de recherche préliminaire, certains on qualifiés les causalités au sein de systèmes complexes comme un problème ``de poule et {\oe}uf'' : si un effet semble causer l'autre et réciproquement, comment est-il possible d'isoler les processus correspondants ? Cette vision est souvent présente dans les approches réductionnistes qui ne postulent pas une complexité intrinsèque au sein des systèmes étudiés. L'idée suggérée par \noun{Diderot} est celle de \emph{co-evolution} qui est un phénomène central dans les dynamiques évolutionnaires des Systèmes Complexes Adaptatifs comme \noun{Holland} élabore dans~\cite{holland2012signals}. Il fait le lien entre la notion d'émergence (ignorée dans les approches réductionnistes), en particulier l'émergence de structures à une plus grand échelle par les interactions entre agents à une échelle donnée, en général concrétisée par un systèmes de limites, qui devient cruciale pour la co-évolution des agents à toutes les échelles : l'émergence d'une structure sera simultanée avec une autre, chacune exploitant leur interrelations et environnements générés conditionnés par le système de limites. Nous explorerons ces idées pour le cas des systèmes territoriaux par la suite.


Ce chapitre introductif est destiné à poser le cadre thématique, le contexte géographique sur lesquels les développements suivants se baseront. Il n'est pas supposé être compris comme une revue de littérature exhaustive ni comme les fondations théoriques fondamentales de notre travail (le premier point étant l'objet du chapitre~\ref{ch:quantepistemo} tandis que le second sera traité plus tôt dans le chapitre~\ref{ch:theory}), mais plutôt comme une construction narrative ayant pour but d'introduire nos objets et positions d'étude, afin de construire naturellement des questions de recherche précises.





\newpage

\section[Réseaux et Territoires]{Réseaux et Territoires}


\subsection{Une circularité naturelle}

\paragraph{Territorialité Humaine}

Une entrée possible dans l'ensemble des objets géographiques que nous proposons d'étudier est la notion de territoire. En Ecologie, un territoire correspond à l'étendue spatiale occupée par un groupe d'agent ou plus généralement un écosystème. Les \emph{Territoires Humains} sont extrêmement plus complexes de par l'importance de leur représentations sémiotiques, qui jouent un rôle significatifs dans l'émergence des constructions sociétales. Selon \noun{Raffestin} dans~\cite{raffestin1988reperes}, la \emph{Territorialité Humaine} est ``la conjonction d'un processus territorial avec un processus informationnel'', ce qui implique que l'occupation physique et l'exploitation de l'espace par les sociétés humaines n'est pas dissociable des représentations (cognitives et matérielles) de ces processus territoriaux, qui influent en retour leur évolution. En d'autres termes, à partir de l'instant où les constructions sociales déterminent la constitution des établissements humains, les structures sociales abstraites et concrètes joueront un role dans l'évolution des systèmes territoriaux, par exemple à travers la propagation d'informations et de représentations, par des processus politiques, ou encore par la correspondance effective entre territoire vécu et territoire perçu. Bien que cette approche ne donne pas de conditions explicites pour l'émergence d'un système séminal d'établissements agrégés (c'est à dire l'émergence des villes), elle insiste sur leur role comme lieu de pouvoir et de création de richesse au travers des échanges. Mais la ville n'a pas d'existence sans son hinterland et le système territorial peut difficilement être résumé par ses villes, comme un système de villes. En se restreignant à ce sous-système, il y a toutefois compatibilité entre la théorie de territoires de \noun{Raffestin} et la théorie évolutive des villes de \noun{Pumain}~\cite{pumain2010theorie}, qui interprète les villes comme des systèmes complexes dynamiques auto-organisés, qui agissent comme des médiateurs du changement social : par exemple, les cycles d'innovation s'initialisent au sein des villes et se propagent entre elles. Les villes sont ainsi des agents compétitifs qui co-évoluent (au sens donné précédemment). Le système territorial peut ainsi être compris comme une structure sociale organisée dans l'espace, qui comprend ses artefacts concrets et abstraits. Une étendue spatiale imaginaire avec des ressources potentielles qui n'aurait jamais connu de contact avec l'humain ne pourra pas être un territoire si elle n'est pas habitée, imaginée, vécue, exploitée, même si ces ressources pourraient être potentiellement exploitée le cas échéant. En effet, ce qui est considéré comme une ressource (naturelle ou artificielle) dépendra de la société (par exemple de ses pratiques et de ses capacité technologiques). Un aspect central des établissements humains qui a une longue tradition d'étude en géographie, et qui est directement relié à la notion de territoire, est celui des \emph{réseaux}. Nous allons voir comment le passage de l'un à l'autre est inévitable et leur définition indissociable.


\paragraph{A Territorial Theory of Networks}

We paraphrase \noun{Dupuy} in~\cite{dupuy1987vers} when he proposes elements for ``a territorial theory of networks'' based on the concrete case of Urban Transportation Networks. This theory sees \emph{real networks} (i.e. concrete networks, including transportation networks) as the materialization of \emph{virtual networks}. More precisely, a territory is characterized by strong spatio-temporal discontinuities induced by the non-uniform distribution of agents and ressources. These discontinuities naturally induce a network of ``transactional projects'' that can be understood as potential interactions between elements of the territorial system (agents and/or ressources). For example today, people need to access the ressource of employments, economic exchanges operate between specialized production territories. At any time period, potential interactions existed\footnote{even when nomadism was still the rule, spatially dynamic networks of potential interactions necessarily existed, but should have less chance to materialize into concrete routes.% bib on that ?
}. The potential interaction network is concretized as offer adapts to demand, and results of the combination of economic and geographical constraints with demand patterns, in a non-linear way through agents designed as \emph{operators}. This process is not immediate, leading to strong non-stationarity and path-dependance effects : the extension of an existing network will depend on previous configuration, and depending on involved time scales, the logic and even the nature of operators may have evolved. \noun{Raffestin} points out in his preface of~\cite{offner1996reseaux} that a geographical theory articulating space, network and territories had never been consistently formulated. It appears to still be the case today, but the theory developed just before is a good candidate, even if it stays at a conceptual level. The presence of a human territory necessarily imply the presence of abstract interaction networks and concrete networks used for transportation of people and ressources (including communication networks as information is a crucial ressource). Depending on regime in which the considered system is, the respective role of different networks may be radically different. Following \noun{Duranton} in \cite{duranton1999distance}, pre-industrial cities were limited in growth because of limitations of transportation networks. Technological progresses have lead to the end of these limitations and the preponderance of land markets in shaping cities (and thus a role of transportation network as shaping prices through accessibility), and recently to the rising importance of telecommunication networks that induce a ``tyranny of proximity'' as physical presence is not replaceable by virtual communication. This territorial approach to networks seems natural in geography, since networks are studied conjointly with geographical objects with an underlying theory, in opposition to network science that studies brutally spatial networks with few thematic background~\cite{ducruet2014spatial}.


\paragraph{Networks shaping territories ?}

% how do network shape territories : boundaries, scales, etc.
% example : \cite{l2012ville} bahn-ville, volontary coevol ? // idem villes nouvelles

However networks are not only a material manifestation of territorial processes, but play their part in these processes as they evolution may shape territories in return. In the case of \emph{technical networks}, an other designation of real networks given in~\cite{offner1996reseaux}, many examples of such feedbacks can be found : the interconnectivity of transportation networks allows multi-scalar mobility patterns, thus shaping the lived territory. At a smaller scale, changes in accessibility may result in an adaptation of a functional urban space. Here emerges again an intrinsic difficulty : it is far from evident to attribute territorial mutations to some network evolutions and reciprocally materialization of a network to precise territorial dynamics. Coming back to Diderot should help, in the sense that one must not consider network nor territories as independent systems that would have causal relationships but as strongly coupled components of a larger system. The confusion on possible simple causal relationships has fed a scientific debate that is still active. Methodologies to identify so-called \emph{structural effects} of transportation networks were proposed by planners in the seventies~\cite{bonnafous1974detection,bonnafous1974methodologies}. It took some time for a critical positioning on unreasoned and decontextualized use of these methods by planners and politics generally to technocratically justify transportation projects, that was first done by \noun{Offner} in~\cite{offner1993effets}. Recently the special issue~\cite{espacegeo2014effets} on that debate recalled that on the one hand misconceptions and misuses were still greatly present in operational and planning milieus as~\cite{crozet:halshs-01094554} confirmed, and on the other hand that a lot of scientific progresses still need to be made to understand relations between networks and territories as \noun{Pumain} highlights that recent works gave evidence of systematic effects on very long time scales (as e.g. the work of \noun{Bretagnolle} on railway evolution, that shows a kind of structural effect in the necessity of connectivity to the network for cities to ``stay in the game'', but that is not fully causal as not sufficient). At a macroscopic level typical patterns of interaction emerge, but microscopic trajectories of the system are essentially chaotic : the understanding of coupled dynamics strongly depends on the scale considered. At a small scale it seems indeed impossible to show systematic behavior, as \noun{Offner} pointed out. For example, on comparable French mountain territories, \cite{berne2008ouverture} shows that reactions to a same context of evolution of the transportation network can lead to very different reactions of territories, some finding a huge benefit in the new connectivity, whereas others become more closed. These potential retroactions of networks on territories does not necessarily act on concrete components : \noun{Claval} shows in~\cite{claval1987reseaux} that transportation and communication networks contribute to the collective representation of territories by acting on territorial belonging feeling.




\paragraph{Territorial Systems}

This detour from territories, to networks and back again, allows us to give a preliminary definition of a territorial system that will be the basis of our following theoretical considerations. As we emphasized the role of networks, the definition takes it into account.

\bigskip

\textbf{Preliminary Definition.} \textit{A territorial system is a human territory to which both interaction and real networks can be associated. Real networks are a component of the system, involved in evolution processes, through multiples feedbacks with other components at various spatial and temporal scales.}


\bigskip


This reading of territorial systems is conditional to the existence of networks and may discard some human territories, but it is a deliberate choice that we justify by previous considerations, and that drives our subject towards the study of interactions between networks and territories.




