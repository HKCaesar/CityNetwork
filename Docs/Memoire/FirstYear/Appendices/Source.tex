

% Chapter 

\chapter{Architecture and Sources for Algorithms and Models of Simulation} % Chapter title

\label{app:code} % For referencing the chapter elsewhere, use \autoref{ch:name} 

%----------------------------------------------------------------------------------------


% do not list all codes, but roughly gives architectures overview
%   and links to git repo

% TODO : script that generates this directly from metadata files ? INCLUDING temporal statistics from git

% Q : current state of programs ? -> frozen state on specific branch for each model -> could use metafig that way also ?

\headercit{You must not be afraid of putting code in your thesis, code is not dirty}{Alexis Drogoul}{}



And yet it is. It makes no sense to put code listings in the core of the text if there is no particular algorithmic detail that requires attention. As soon as implementation biases are avoided, architecture and source for a computational model should be independent from its formal description.


%----------------------------------------------------------------------------------------

\newpage

\section{Algorithmic Systematic Review}

\paragraph{Objective}


\paragraph{Location}


\paragraph{Characteristics}

\begin{itemize}
\item Language : \texttt{Java}
\item Size : 7116
\end{itemize}


\paragraph{Particularities}

\paragraph{Architecture}

\paragraph{Additional scripts}


%----------------------------------------------------------------------------------------

\newpage

\section{Indirect Bibliometrics}

\paragraph{Objective}

\paragraph{Characteristics}

\begin{itemize}
\item Language : \texttt{Python} and \texttt{R}
\item Size :
\end{itemize}


\paragraph{Particularities}

\paragraph{Architecture}

\paragraph{Additional scripts}






%----------------------------------------------------------------------------------------

\newpage

\section{Lutecia Model}




\paragraph{Objective}

\paragraph{Characteristics}

\begin{itemize}
\item Language : \texttt{NetLogo}
\item Size : 4791
\end{itemize}


\paragraph{Particularities}

\paragraph{Architecture}

\paragraph{Additional scripts}



