%%%%%%%%%%%%%%%%%%%%%%%%%
%% Header for standard beamer presentation
%%
%%  PresentationHeader.tex
%%
%%%%%%%%%%%%%%%%%%%%%%%%%

\documentclass[english,10pt]{beamer}

%%%%%%%%%%%%%%%%%%%%
%% Include general header where common packages are defined
%%%%%%%%%%%%%%%%%%%%

% general packages without options
\usepackage{amsmath,amssymb,bbm}




%%%%%%%%%%%%%%%%%%%%
%% Idem general commands
%%%%%%%%%%%%%%%%%%%%

%%% Commands

\newcommand{\noun}[1]{\textsc{#1}}


%% Math

% Operators
\DeclareMathOperator{\Cov}{Cov}
\DeclareMathOperator{\Var}{Var}
\DeclareMathOperator{\E}{\mathbb{E}}
\DeclareMathOperator{\Proba}{\mathbb{P}}

\newcommand{\Covb}[2]{\ensuremath{\Cov\!\left[#1,#2\right]}}
\newcommand{\Eb}[1]{\ensuremath{\E\!\left[#1\right]}}
\newcommand{\Pb}[1]{\ensuremath{\Proba\!\left[#1\right]}}
\newcommand{\Varb}[1]{\ensuremath{\Var\!\left[#1\right]}}

% norm
\newcommand{\norm}[1]{\| #1 \|}


% amsthm environments
\newtheorem{definition}{Definition}



%% graphics

% renew graphics command for relative path providment only ?
%\renewcommand{\includegraphics[]{}}






\usetheme{Warsaw}

\setbeamertemplate{footline}[text line]{}
\setbeamercolor{structure}{fg=purple!50!blue, bg=purple!50!blue}

\setbeamercovered{transparent}


% shortened command for a justified frame
\newcommand{\jframe}[2]{\frame{\frametitle{#1}\justify{#2}}}



%%%%%%%%%%%%%%%%%%%%%
%% Begin doc
%%%%%%%%%%%%%%%%%%%%%

\begin{document}



\title{Thesis Progress Meeting}


\author{J.~Raimbault$^{1,2}$}

\institute{$^{1}$G{\'e}ographie-cit{\'e}s (UMR 8504 CNRS)\\
$^{2}$LVMT (UMR-T 9403 IFSTTAR)}


\date{March 10th 2016}


%%%%%%%%%%%%%%%%%%%%%%%%%%%%%%%%
\begin{frame}
\titlepage
\end{frame}

%\begin{frame}
%\tableofcontents
%\end{frame}
%%%%%%%%%%%%%%%%%%%%%%%%%%%%%%%%



\section{Achieved Work}


\jframe{Achieved Work (by projects)}{
\begin{itemize}
\item Biblio/Meetings/Organisation [0.5w]
\item Seminars : Cartha-g{\'e}o-prisme ; mandatory English course [0.8w]
\item Memoire [2.2w] (ETA 2w)
\item Cybergeo Project [1.8w] (ETA 0.5w)
\item Network-Density Statistics [1.2w] (ETA 0.5w)
\end{itemize}
}



\section{Theoretical Framework}
% chapter one and two of Memoire

\sframe{Subject Construction}{
\textit{Definition of Territorial Systems ?}
\bigskip

$\rightarrow$ Raffestin Human Territoriality~\cite{raffestin1988reperes} to introduce the subject

$\rightarrow$ Privileged role of Networks, following Dupuy \textit{Territorial Theory of Networks}~\cite{dupuy1987vers}

$\rightarrow$ Reciprocally, debate on Structural Effects of Transportation Networks still active today~\cite{espacegeo2014effets}

\bigskip

\textbf{Preliminary def. of Territorial Systems.} \textit{Human Territories in and between which real networks exist.}
}


\sframe{Subject Construction}{
\textit{Transportation Networks and Modeling}
\bigskip

$\rightarrow$ Necessary role of transportation networks : choice to concentrate on these.

$\rightarrow$ Modeling approach because of explicative potentialities and powerful to validate/invalidate a theory, among others.

$\rightarrow$ Consequent set of literature from multiple domains and on multiple points of view.


\bigskip

\textbf{General research Question.} \textit{To what extent a modeling approach to territorial systems as networked human territories can help disentangling complexly involved processes ?}

}


\sframe{Theory : First Pillar}{
\textit{Networked Human Territories}
\bigskip

$\rightarrow$ Raffestin approach to territory combined with Dupuy theory of networks.

\bigskip

Production and Organisation of human settlements and associated societies. Transactional networks induce real networks as soon as transactions are realized. Territories as networked places~\cite{champollion:halshs-00999026}.

}

\sframe{Theory : Second Pillar}{
\textit{Evolutive Urban Theory}
\bigskip

$\rightarrow$ City Systems as complex Adaptive systems, applied tu human settlements in general and thus territorial systems.

\bigskip

Cities as auto-organized complex systems~\cite{pumain1997pour} ; positioned within Complex Systems Science~\cite{pumain2003approche} ; provide explanation for scaling laws~\cite{pumain2006evolutionary} ; cities as adapters of social change (e.g. diffusion of innovation)~\cite{pumain2010theorie} ; importance of non-ergodicity
\cite{pumain2012urban}.

}

\sframe{Theory : Third Pillar}{
\textit{Urban Morphogenesis}
\bigskip

$\rightarrow$ Morphogenesis as autonomous rules to explain growth of urban form. Used as the provider of modular decompositions.

\bigskip

Imported from integrative biology~\cite{delile2016chapitre} ; used in diverse ways for cities~\cite{bonin2012modele},\cite{makse1998modeling}. System exhibiting morphogenesis as \textit{autopoietic systems}~\cite{bourgine2004autopoiesis} : importance of boundaries.

}



\sframe{Theory : Fourth Pillar}{
\textit{Boundaries and Co-evolution}
\bigskip

$\rightarrow$ Co-evolution as the existence of \textit{niche}, consequence of boundary patterns.

\bigskip

Signal and Boundary theory of complex adaptive systems by Holland \cite{holland2012signals}. Co-evolution as dynamic interdependence and relative independence to exterior. Compatible with Pumain's approach to co-evolution : city system as a niche with its internal flows.

}


\sframe{Theory : Specification}{
\begin{itemize}
\jitem{Previous def. of territorial systems}
\jitem{Modular decomposition and stationarity : existence of scales}
\jitem{Feedback loops between and inside scales yield weak emergence, thus complexity}
\jitem{Morphogenesis gives modular decomposition and co-evolution}
\jitem{\textbf{Main assumption.} Necessity of Networks : networks are necessary component of co-evolutive niches.}
\end{itemize}
}

\sframe{Theory : Application}{
\textit{Directions to validate, invalidate, modify the theory :}
\bigskip
\begin{itemize}
\item Disentangle circular causation, search for morphogenesis, etc. in empirical data
\item Show the necessity of networks in toy-models
\item In particular exhibit bifurcations due to networks (\textit{emergence of real networks as extension of SimpopLocal ?})
\item Same on hybrid models
\end{itemize}



}




\section{Methodology}
% detail osm nw simplification


\sframe{From Static Correlations to Dynamical Correlations}{
Assumptions on the spatio-temporal stochastic processes $Y_i\left[\vec{x},t\right]$ :
\begin{enumerate}
\item Local spatial autocorrelation is present on a maximal span of $l_{\rho}$ : for any $\vec{x}$ and $t$, $\left|\rho_{\norm{\Delta \vec{x}} < l_{\rho}}\left[Y_i (\vec{x}+\Delta \vec{x},t), Y_i (\vec{x},t) \right]\right| > 0$.
\item Processes are locally parametrized : $Y_i = Y_i\left[\alpha_i\right]$, where $\alpha_i (\vec{x})$ varies with $l_{\alpha}$, with $l_{\alpha} \gg l_{\rho}$.
\item Spatial correlations between processes have a sense at an intermediate scale $l$ such that $l_{\alpha}\gg l \gg l{\rho}$.
\item Processes covariance stationarity times scale as $\sqrt{l}$.
\item Local ergodicity is present at scale $l$ and dynamics are locally chaotic.
\end{enumerate}

Hyp. 1 yields sampling consistence ; 2 to 3 allow to compute intermediate spatial correlations comparable along large spatial scales ; 4 and 5 link temporal correlation to spatial correlation.
}


\sframe{Road Network Simplification}{
\textit{OpenStreetMap Road Network analysis : computation of local network indicators}
\bigskip

Europe OSM in Postgresql database (filtered on main roads only) : $\simeq$ 100Gb.

$\rightarrow$ must proceed to network simplification.

\medskip

At the scale of density raster, extract links between raster cells centroids, construct network, iteratively delete node of degree, stored in Postgresql. Parallelized, country by country (3 weeks estimated computation time).

}







%%%%%%%%%%%%%%%%%%%%%%%%%%%%%%%%
\jframe{Next steps (until April 15th 2016)}{
\begin{itemize}
\item Theoretical Paper if not crazy ? [1w]\medskip
\item Spatial Statistics / Case studies (Le Corre and Baffi data) [2w]\medskip
\item Cybergeo et al. [1w]\medskip\medskip
\end{itemize}
}






%%%%%%%%%%%%%%%%%%%%%%%%%%%%%%%%
\begin{frame}[allowframebreaks]
\frametitle{References}
\bibliographystyle{apalike}
\bibliography{/Users/Juste/Documents/ComplexSystems/CityNetwork/Biblio/Bibtex/CityNetwork}
\end{frame}
%%%%%%%%%%%%%%%%%%%%%%%%%%%%%%%%


\end{document}




