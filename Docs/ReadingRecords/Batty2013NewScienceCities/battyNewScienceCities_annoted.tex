%%%%%%%%%%%%%%%%%%%%%%%%%%%%%
% Standard header for working papers
%
% WPHeader.tex
%
%%%%%%%%%%%%%%%%%%%%%%%%%%%%%

\documentclass[11pt]{article}

%%%%%%%%%%%%%%%%%%%%
%% Include general header where common packages are defined
%%%%%%%%%%%%%%%%%%%%

% general packages without options
\usepackage{amsmath,amssymb,bbm}




%%%%%%%%%%%%%%%%%%%%
%% Idem general commands
%%%%%%%%%%%%%%%%%%%%
%% Commands

\newcommand{\noun}[1]{\textsc{#1}}


%% Math

% Operators
\DeclareMathOperator{\Cov}{Cov}
\DeclareMathOperator{\Var}{Var}
\DeclareMathOperator{\E}{\mathbb{E}}
\DeclareMathOperator{\Proba}{\mathbb{P}}

\newcommand{\Covb}[2]{\ensuremath{\Cov\!\left[#1,#2\right]}}
\newcommand{\Eb}[1]{\ensuremath{\E\!\left[#1\right]}}
\newcommand{\Pb}[1]{\ensuremath{\Proba\!\left[#1\right]}}
\newcommand{\Varb}[1]{\ensuremath{\Var\!\left[#1\right]}}

% norm
\newcommand{\norm}[1]{\| #1 \|}


% amsthm environments
\newtheorem{definition}{Definition}



%% graphics

% renew graphics command for relative path providment only ?
%\renewcommand{\includegraphics[]{}}








% geometry
\usepackage[margin=2cm]{geometry}

% layout : use fancyhdr package
\usepackage{fancyhdr}
\pagestyle{fancy}

\makeatletter

\renewcommand{\headrulewidth}{0.4pt}
\renewcommand{\footrulewidth}{0.4pt}
%\fancyhead[RO,RE]{\textit{Working Paper}}
\fancyhead[RO,RE]{\textit{ECTQG 2015}}
%\fancyhead[LO,LE]{G{\'e}ographie-Cit{\'e}s/LVMT}
\fancyhead[LO,LE]{An Algorithmic Systematic Review}
\fancyfoot[RO,RE] {\thepage}
\fancyfoot[LO,LE] {\noun{J. Raimbault}}
\fancyfoot[CO,CE] {}

\makeatother


%%%%%%%%%%%%%%%%%%%%%
%% Begin doc
%%%%%%%%%%%%%%%%%%%%%

\begin{document}







\title{Reading Record\bigskip\\
\cite{batty2013new}
}
\author{\noun{Juste Raimbault}}
\date{Date}


\maketitle

\textbf{\textit{Reading Record for \cite{batty2013new}}}
\textit{The New Science of Cities}, \noun{M. Batty}, 2013


\subsection*{General purpose}

It aims to develop various tools associated to what is presented as ``a new way'' of thinking in tackling urban systems. These can be technical tools or paradigms, and borrow from physics, geography, architecture, being typically ideas that can be classified as belonging to Complex System science in the sense of~\cite{chavalarias2009french} (transdisciplinary problems and methods associated to complex systems), but to which the author strangely does not refer, as well as some significant contributions in quantitative geography seem to be ignored (\textit{French school} ? find other examples ?).

Focus on flows on network, compared to previous book~\cite{Bat07} that was more on agent-based modeling paradigms.

\textit{Rq : } Confusion on the term complex network ? e.g. subway nws are not complex in the sense of nw science ? beware of definitions and terminologies !

\newgeometry{right=6cm}

\section{Linear reading}

\section*{Part I : Foundations}

\subsection{Chapter 1 : Contextualisation}

Vision on cities from Jacobs~\cite{Jac56}. Complexity theory/importance of scaling as a signature of complexity.

Thematization : Cerda.

System of cities, hierarchization.

Fractal form~\cite{FractalCities}, linked to organic growth of the city

\subsubsection*{CS notions}

Defines then essential notions central to see cities as CS :
\begin{itemize}
\item Equilibrium and Dynamics : non-stationnary, out-of-equilibrium systems. ``catastrophe, bifurcation, chaos'' : ok but bla, bla, no concrete link
\item Pattern and processe : Morphological analysis to reconstruct underlying processes.
\item Interactions, flows, nws : Nws attachment model ; gravitation model for flows.
\item Evolution and Emergence : typical CA examples ; Schelling model~\cite{schelling1969models}
\end{itemize}


\subsubsection*{Scaling laws}

Empirical scaling laws, including allometry, vonThunen law on fractal exponent, etc.

Power law on size, density, etc.

CL : not ``the'' but ``a'' new science.



\subsection{Chapter 2 : Gravity and Potential}

Flows and conservation laws. Desire lines, flow representations. Interesting history of flows representation.

Gravitation model $T_{ij}=\frac{KP_iP_j}{d_{ij}^{\phi}}$.

Discrete choices models for constants ?

Potentials / solving for particular cases.

\subsection{Chapter 3 : NW Science}

\textbf{Def} Thresholded connectivity.

Bipartite graphs : correlation on dual variables.

Hierarchical clustering ; clustering coef

Flow dynamic : basic markov matrix equation.

Spatial networks, planar graphs.

Centrality and Accessibility measures.

\textit{Note : } Apparent chapter structure until now : develops ``complicated'' maths tools, quite out-of-purpose (ex Markov chain for NW dynamics -> ?!!, wtf when argues to deal with CS !), then small thematic development on possible application/related thematic work - but no direct link !


\section*{Part II : The Science of Cities}

\subsection{Chapter 4 : Rank, Size}

Power law established for simplest growth model.

Scaling : rank clock for visualization of evolution of scaling.

Empirical results on Italian cities.

\textit{Nothing to break 3 duck legs, apart viz tool.}


\subsection{Chapter 5 : Hierarchies and NW}

Growth model where hierarchy emerges. (CA) more elaborated with diffusion.

Spatial NW (Bibrat Model)/spatial hierarchy.

Back on Central Place Theory : provides also a Zipf law.

Bla bla role of hierarchy in design


\subsection{Chapter 6 : Space Syntax}

Topological representation of space. (if spaces related ?). Back on Space Syntax Theory (linked spaces etc). Used as tool to compute accessibility ?

Topological distance, connectivity, same weighted $\rightarrow$ weighted accessibilities.

``steady-state acc'' : redoes sort of Wardrop equilibrium.

\subsection{Chapter 7 : Distance in Complex Networks}

Network measures, shortest path -- ¡¡ not complex nw !!

multilayer (two transportation modes, underground and street).

\subsection{Chapter 8 : Fractal growth and Form}

\textbf{Def} fractals (ex Koch) and fractal dimension. // link to models of aggregation-diffusion.

Morphogenesis models (CA), diverse CA, concrete applications (ex SLEUTH).


\subsection{Chapter 9 : Urban Simulation}

Review of Luti models.

Example of generic location model (entropy-based).

Role of visual interface in model exploration. Exploratory data analysis, scenarii.

Different time scales for various dynamics (slow and fast changes).


\section*{Part III : The Science of Design}


\subsection{Chapter 10 : Hierarchical Design}

Formal def of design problem as a function of diverse factors.

``Linear synthesis'' : just wieghted aggregation function for choice. Design network : explore strategies for weighting. (//decision-making literature). NW adjacency matrix defined as link between factors. Operations on matrix $\rightarrow$ weights (with some free dimensions arbitrary fixed).

``Sequantial averaging'' : convergence to fixed point. // link to Leurent algos etc.

\subsection{Chapter 11 : Markovian Design Machines}

New method to converge to a decision ? Reformulation of network design. Conv proof, Kemeny-Snell ¡¡ specific conditions for it to work !!

Classification of chains types.

Example of application. optimal solution ? what objectives ?



\subsection{Chapter 12 : A theory for collective action}

Interest and control matrices for the design pb $\rightarrow$ idem Markov chain formulation. existence of steady state. Interpretation as exchanges between actors. General formulation, computation of equilibrium // again a Leurent algo? externalities give out-of-equilibrium. Two concrete application to toy problems, ex ressource allocation.

\subsection{Chapter 13 : Urban Development as Exchange}

Model for land development, equilibrium. Based on Coleman's theory (as Markovian design machines).

Application to particular area in London. Reconstructs roughly actual events.

\subsection{Chapter 14 : Plan Design as Committee Decision Making}

RO computations for optimal solution for all actors. Budget problem/social exchange.


\subsection{Conclusion}

More positivist than normative approach. Propose solution to solve conflicts. Basis for one sci of cities.





\section{Synthesis}

\textit{TBW}











%%%%%%%%%%%%%%%%%%%%
%% Biblio
%%%%%%%%%%%%%%%%%%%%

\bibliographystyle{apalike}
\bibliography{/Users/Juste/Documents/ComplexSystems/CityNetwork/Biblio/Bibtex/CityNetwork,/Users/Juste/Documents/ComplexSystems/Biblio/BibTex/global}


\end{document}
