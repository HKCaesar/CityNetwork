%%%%%%%%%%%%%%%%%%%%%%%%%%%%%
% Standard header for working papers
%
% WPHeader.tex
%
%%%%%%%%%%%%%%%%%%%%%%%%%%%%%

\documentclass[11pt]{article}

%%%%%%%%%%%%%%%%%%%%
%% Include general header where common packages are defined
%%%%%%%%%%%%%%%%%%%%

% general packages without options
\usepackage{amsmath,amssymb,bbm}




%%%%%%%%%%%%%%%%%%%%
%% Idem general commands
%%%%%%%%%%%%%%%%%%%%
%% Commands

\newcommand{\noun}[1]{\textsc{#1}}


%% Math

% Operators
\DeclareMathOperator{\Cov}{Cov}
\DeclareMathOperator{\Var}{Var}
\DeclareMathOperator{\E}{\mathbb{E}}
\DeclareMathOperator{\Proba}{\mathbb{P}}

\newcommand{\Covb}[2]{\ensuremath{\Cov\!\left[#1,#2\right]}}
\newcommand{\Eb}[1]{\ensuremath{\E\!\left[#1\right]}}
\newcommand{\Pb}[1]{\ensuremath{\Proba\!\left[#1\right]}}
\newcommand{\Varb}[1]{\ensuremath{\Var\!\left[#1\right]}}

% norm
\newcommand{\norm}[1]{\| #1 \|}


% amsthm environments
\newtheorem{definition}{Definition}



%% graphics

% renew graphics command for relative path providment only ?
%\renewcommand{\includegraphics[]{}}








% geometry
\usepackage[margin=2cm]{geometry}

% layout : use fancyhdr package
\usepackage{fancyhdr}
\pagestyle{fancy}

\makeatletter

\renewcommand{\headrulewidth}{0.4pt}
\renewcommand{\footrulewidth}{0.4pt}
%\fancyhead[RO,RE]{\textit{Working Paper}}
\fancyhead[RO,RE]{\textit{ECTQG 2015}}
%\fancyhead[LO,LE]{G{\'e}ographie-Cit{\'e}s/LVMT}
\fancyhead[LO,LE]{An Algorithmic Systematic Review}
\fancyfoot[RO,RE] {\thepage}
\fancyfoot[LO,LE] {\noun{J. Raimbault}}
\fancyfoot[CO,CE] {}

\makeatother


%%%%%%%%%%%%%%%%%%%%%
%% Begin doc
%%%%%%%%%%%%%%%%%%%%%

\begin{document}







\title{Reading Record\bigskip\\
\cite{emangard2009transports}
}
\author{\noun{Juste Raimbault}}
\date{Date}


\maketitle

\textbf{\textit{Reading Record for \cite{emangard2009transports}}}




\section{Introduction}

Thematic book on recent developments in Transportation Geography - link with other disciplines.



\section{Linear Reading}

\subsection*{Introduction}

Not a full synthesis, but focus on some emerging themes. More recent and innovative fields.


\subsection{Position of Transportation Studies in Humanities}

\subsubsection{Borderline in Geography}

40-70 : circulation

70-00 : by mode ; and also links between transportation network and development, interfaces, environmental effects.

recently : networks, flows.

difficultly recognized today.

\subsubsection{Secondary Theme in economics, history, sociology}

Economics : Economics of transportation - 3 themes :
\begin{itemize}
\item relation between transportation and economic growth
\item regulation and concurrency
\item sustainibility
\end{itemize}

History : most rail. GDR R{\'e}seaux

Sociology : dynamics of transportation jobs ; cooperation ; mobility.

increasing relation between disciplines.

\subsubsection{Results and Limitations of Transportation Geography}

\paragraph{Models and Concepts}

\textbf{Distance and Accessibility} ; choices models ; network models ; impedance in raster

Today more oriented in graph theory, with use of new technologies (gps etc)

\paragraph{Stakeholder interplays}

role of states and geopolitics (organisations, treaties).

Intervention of the state, liberalisation. very few literature on articulation between different levels of power (ex. state and regional).

cities : public transportation and urbanism.

Study of production companies, and exploitants. 

\paragraph{Role of Transportation in contemporary issues}

Environmental risks. link between transportation and energy consumption. pollution and noise principally. social and spatial equity.

No global covering study in geography.


\subsection{Scales and their articulation}

Different transportation modes associated to corresponding travel time, and thus spatial scale.

Technological innovation has contracted space, but not in an isotropic way.

\subsubsection{Actors, Transportation Modes and Scale}

Spatial scale : different for fret and people ?

Modes have preferential distance (scale) regimes.

Organisation of transportation companies has influenced choice of study scales.

Regional scale : quite recent. studies on accessibility. Country scale : role of state. Continental scale : infrastructure projects helped by EU. Global scale : modal studies.

Link between scale and infrastructure hierarchy. example maritime and airplanes.

Temporal scale : thresholds of access time.

Concurrency : complexity of tarification, do not directly depend on scale.

Multi-dimensional scale (space, time and cost).

Spatial inequality in accessibility differentials (ex LGV).

Scale adapts to mode.

Dereglementation : partial dissociation of mode and scale.

Urban and peri-urban scales are particular : public transportation ; importance of congestion. Recently extension of study area : greater urban areas. Modal and thematic research.

Mobility studies : complex because of scales interdependance and interrelations.

\subsubsection{Relations between scales}

Superposition and co-existence of scales - connexion of scales in multimodal exchange nodes. Hybrid infrastructures : tram-train.

\textbf{Separation and Integration}

Conflicts of different scales and usages on common infrastructures ; but also integration between scales through operators : connectivity and intermodality.

\textbf{Local effects of infrastructures : } debates on structuring effects. infrastructure alone is not enough, but accessibility favors attractivity and growth.

Local negative effects : congestion ; environmental degradation.

Consequence of infra on territorial development ? transportation nodes at the intersection of scales. ; interconnection of railway types e.g.

Multimodal platforms : change in scale and generally mode. crucial on logistics also.

Recent studies generally limited to relation between two scales only : local and regional ; local and international.


\subsection{Effects of technological innovation : High speed trains}

High speed (greater than 160km.h$^{-1}$) major technological innovation ; after WWII. concurrence of other modes : survival of railway.

need of new line, problem of line curves.

First Japan (1964), later France and Germany.


\subsubsection{Research themes}

\begin{itemize}
\item insertion of new lines and stations in urban and natural environment
\item articulation with preexisting network
\item impact on mobility and other modes
\item impact on spatial organisation and territorial organisation
\end{itemize}

related question : really a technical innovation ? 

\textbf{Insertion : } implantation of the new line. Stations : only a few ; changes urban environment.

\textbf{Impact on other modes : } concurrency, economical impact.





\section{Synthesis}

\textit{TBW}





%%%%%%%%%%%%%%%%%%%%
%% Biblio
%%%%%%%%%%%%%%%%%%%%

\bibliographystyle{apalike}
\bibliography{/Users/Juste/Documents/ComplexSystems/CityNetwork/Biblio/Bibtex/CityNetwork}


\end{document}
