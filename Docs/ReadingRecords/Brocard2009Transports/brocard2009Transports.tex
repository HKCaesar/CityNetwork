%%%%%%%%%%%%%%%%%%%%%%%%%%%%%
% Standard header for working papers
%
% WPHeader.tex
%
%%%%%%%%%%%%%%%%%%%%%%%%%%%%%

\documentclass[11pt]{article}

%%%%%%%%%%%%%%%%%%%%
%% Include general header where common packages are defined
%%%%%%%%%%%%%%%%%%%%

% general packages without options
\usepackage{amsmath,amssymb,bbm}




%%%%%%%%%%%%%%%%%%%%
%% Idem general commands
%%%%%%%%%%%%%%%%%%%%
%% Commands

\newcommand{\noun}[1]{\textsc{#1}}


%% Math

% Operators
\DeclareMathOperator{\Cov}{Cov}
\DeclareMathOperator{\Var}{Var}
\DeclareMathOperator{\E}{\mathbb{E}}
\DeclareMathOperator{\Proba}{\mathbb{P}}

\newcommand{\Covb}[2]{\ensuremath{\Cov\!\left[#1,#2\right]}}
\newcommand{\Eb}[1]{\ensuremath{\E\!\left[#1\right]}}
\newcommand{\Pb}[1]{\ensuremath{\Proba\!\left[#1\right]}}
\newcommand{\Varb}[1]{\ensuremath{\Var\!\left[#1\right]}}

% norm
\newcommand{\norm}[1]{\| #1 \|}


% amsthm environments
\newtheorem{definition}{Definition}



%% graphics

% renew graphics command for relative path providment only ?
%\renewcommand{\includegraphics[]{}}








% geometry
\usepackage[margin=2cm]{geometry}

% layout : use fancyhdr package
\usepackage{fancyhdr}
\pagestyle{fancy}

\makeatletter

\renewcommand{\headrulewidth}{0.4pt}
\renewcommand{\footrulewidth}{0.4pt}
%\fancyhead[RO,RE]{\textit{Working Paper}}
\fancyhead[RO,RE]{\textit{ECTQG 2015}}
%\fancyhead[LO,LE]{G{\'e}ographie-Cit{\'e}s/LVMT}
\fancyhead[LO,LE]{An Algorithmic Systematic Review}
\fancyfoot[RO,RE] {\thepage}
\fancyfoot[LO,LE] {\noun{J. Raimbault}}
\fancyfoot[CO,CE] {}

\makeatother


%%%%%%%%%%%%%%%%%%%%%
%% Begin doc
%%%%%%%%%%%%%%%%%%%%%

\begin{document}







\title{Reading Record\bigskip\\
\cite{emangard2009transports}
}
\author{\noun{Juste Raimbault}}
\date{Date}


\maketitle

\textbf{\textit{Reading Record for \cite{emangard2009transports}}}




\section{Introduction}

Thematic book on recent developments in Transportation Geography - link with other disciplines.



\section{Linear Reading}

\subsection*{Introduction}

Not a full synthesis, but focus on some emerging themes. More recent and innovative fields.


\subsection{Position of Transportation Studies in Humanities}

\subsubsection{Borderline in Geography}

40-70 : circulation

70-00 : by mode ; and also links between transportation network and development, interfaces, environmental effects.

recently : networks, flows.

difficultly recognized today.

\subsubsection{Secondary Theme in economics, history, sociology}

Economics : Economics of transportation - 3 themes :
\begin{itemize}
\item relation between transportation and economic growth
\item regulation and concurrency
\item sustainibility
\end{itemize}

History : most rail. GDR R{\'e}seaux

Sociology : dynamics of transportation jobs ; cooperation ; mobility.

increasing relation between disciplines.

\subsubsection{Results and Limitations of Transportation Geography}

\paragraph{Models and Concepts}

\textbf{Distance and Accessibility} ; choices models ; network models ; impedance in raster

Today more oriented in graph theory, with use of new technologies (gps etc)

\paragraph{Stakeholder interplays}

role of states and geopolitics (organisations, treaties).

Intervention of the state, liberalisation. very few literature on articulation between different levels of power (ex. state and regional).

cities : public transportation and urbanism.

Study of production companies, and exploitants. 

\paragraph{Role of Transportation in contemporary issues}

Environmental risks. link between transportation and energy consumption. pollution and noise principally. social and spatial equity.

No global covering study in geography.


\subsection{Scales and their articulation}

Different transportation modes associated to corresponding travel time, and thus spatial scale.

Technological innovation has contracted space, but not in an isotropic way.

\subsubsection{Actors, Transportation Modes and Scale}

Spatial scale : different for fret and people ?

Modes have preferential distance (scale) regimes.

Organisation of transportation companies has influenced choice of study scales.

Regional scale : quite recent. studies on accessibility. Country scale : role of state. Continental scale : infrastructure projects helped by EU. Global scale : modal studies.

Link between scale and infrastructure hierarchy. example maritime and airplanes.

Temporal scale : thresholds of access time.

Concurrency : complexity of tarification, do not directly depend on scale.

Multi-dimensional scale (space, time and cost).

Spatial inequality in accessibility differentials (ex LGV).

Scale adapts to mode.

Dereglementation : partial dissociation of mode and scale.

Urban and peri-urban scales are particular : public transportation ; importance of congestion. Recently extension of study area : greater urban areas. Modal and thematic research.

Mobility studies : complex because of scales interdependance and interrelations.

\subsubsection{Relations between scales}

Superposition and co-existence of scales - connexion of scales in multimodal exchange nodes. Hybrid infrastructures : tram-train.

\textbf{Separation and Integration}

Conflicts of different scales and usages on common infrastructures ; but also integration between scales through operators : connectivity and intermodality.

\textbf{Local effects of infrastructures : } debates on structuring effects. infrastructure alone is not enough, but accessibility favors attractivity and growth.

Local negative effects : congestion ; environmental degradation.

Consequence of infra on territorial development ? transportation nodes at the intersection of scales. ; interconnection of railway types e.g.

Multimodal platforms : change in scale and generally mode. crucial on logistics also.

Recent studies generally limited to relation between two scales only : local and regional ; local and international.


\subsection{Effects of technological innovation : High speed trains}

High speed (greater than 160km.h$^{-1}$) major technological innovation ; after WWII. concurrence of other modes : survival of railway.

need of new line, problem of line curves.

First Japan (1964), later France and Germany.


\subsubsection{Research themes}

\begin{itemize}
\item insertion of new lines and stations in urban and natural environment
\item articulation with preexisting network
\item impact on mobility and other modes
\item impact on spatial organisation and territorial organisation
\end{itemize}

related question : really a technical innovation ? 

\textbf{Insertion : } implantation of the new line. Stations : only a few ; changes urban environment.

\textbf{Impact on other modes : } concurrency, economical impact.


\textbf{Impact on spatial organisation and territorial dynamics : } accessibility ; organisation of the urban system ; development ; impact of localization of economic activities ; compatibility with existing network

\subsubsection{Results}

$\rightarrow$ High Speed Rail is a powerful innovation

$\rightarrow$ Importance and value of time in societies ; anisotropy of space-time. 2h travel time as a threshold for work commuting through high speed.

$\rightarrow$ environment : few results, not enough temporal span

$\rightarrow$ territorial orga and station implantation : french specificity. Renewal of station districts

$\rightarrow$ commercial success and concurrency with airplane ; few impact on car ; failure of new guided transportation modes (aerotrain etc).

structural effects : no clear link, complex and variable impacts. (impact of accessibility increase on territorial development ?)

most impact of high speed on tourism ; on city development : some short-circuit negatives, example of Lille : improvment.

p.76 : ``No automatic nor regular effect in territorial impact of high speed. Openings insert within a territorial evolution process that they influence, sometimes importantly, but without any general conclusion that can be drawn''.

\subsubsection{Current debates}

\paragraph{debate on speed}

is it necessarily a progress ? - cost advantages analyses only ? philosophical and anthropological approaches

\paragraph{Accessibility increase}

Plassard : ill-posed question : too much political underlyings at the beginning, and forgot spatial scales and actors. not simple as direct causality. Importance of new infrastructures within transformations, but complex part of it.

\paragraph{Impact on existing network}

techical lit. on railway itself ; other aspects very few studies.

need international research on open questions.

\subsubsection{Future research}

\begin{itemize}
\item broaden territorial contexts.
\item need more comparative studies - synthesis
\end{itemize}

\paragraph{Open questions}

\begin{itemize}
\item impact of geography on implementation of high speed
\item why failure in anglo-saxon countries
\item why not worldwide propagation ?
\item inventory of potential development corridors
\item concurrency airlines
\item impact on interurban mobility
\item study of tourism impacts
\item co-effect of new line and dereglementation
\item multimodal studies of spatial impacts
\item impact of lack of accessibility
\end{itemize}




\subsection{TIC in Transportation Geography}

\subsubsection{Difficult intersection between transportation and tic}

\paragraph{70-90 : myths and illusions}

deterritorialisation ? opposition of French geographers to this thesis.

\paragraph{95 : fail of information highways}

no more structuring effects than transportation infrastructures.

\paragraph{00 : converging questions}

impact on functional organisation of transportation systems. (examples of impacts on transportation and mobility)

\subsubsection{Places Space against Flows Space}

cf Castells.

impact on trade.

more safety/security.

eco-friendly : ``teletravail'' ?

more individual mobility. (cf Dupuy 95)

Urbanisation of societies.

Virtual realities.


\subsection{GIS, Transportation and Geography}

\subsubsection{Transportation Analysis}

GIS Modeling : observable world. structure of geographical perception.

Systematic analysis of transportation networks. 

GIS-T = combination of GIS and TIS.

larger than only info system within the transportation system : coupling transportation models and gis.

\subsubsection{Spatialization of exchanges}

very bad flow data before - now new granularities.

\subsubsection{Types of approaches}

GIS, gestion, computer science, geography.

Thematic modeling : territorial knowledge.

\subsubsection{Potentialities}

Scales of congestion : open question.

Spatial ergonomy : optimization of access to ressources.



\subsection{Transportation and Development}

Development different from growth : human and qualitative. transportation/territories too much seen from that point of view.

\subsubsection{Transportation : indicator and driver of economic growth}

Transportation is a major economic activity.

For economics, space is just a distance to cross, with a given cost. An economical space is much more than that : complex interdependencies space.

$\rightarrow$ importance of scales : complexity of scales interaction. importance of integration by geographers into a territoriality as a complex system.

\subsubsection{Back on theory of Structuring Effects}

First a political question : infrastructure as a tool of power. does not change with new geopolitical issues : network and borders (network and boundaries ? -> Holland) are complementary, in a complex interplay (remark : crucial for morphogenesis etc ?)

Developed from an economical point of view ; infrastructure as a necessary point for development ; structures public life. Conviction that transportation is structuring : accessibility as a key factor in hedonic models, under perfect concurrence between modes, and perfect rationality of agents. Von Thunen, Capozza.

Indeed, infrastructure can trigger increase in interactions (answer a mobility demand). 

Quoting~\cite{offner1996reseaux} :

\textbf{``It is difficult to picture a territory without transportation network, since these constitute both the support, the condition and the realization of exchanges they generate. Transportation networks are more than a functional tool, they are a factor of their development, since they progressively induce on the spaces were they are organized, territorial and social solidarities.''}

what is contested in structuring effects theory is the direct deterministic causal link. $\rightarrow$ \textbf{Need to come to the whole territory}

congruence between multiple factors : economical, social, political, technical, cultural.

Complex causal chains, that can superpose, branch etc. \textbf{Interfering causal chains.} \textit{Examples : } Improved circulation $\implies$ logistics $\rightarrow$ mobility $\rightarrow$ increase of traffic $\rightarrow$ new activities $\rightarrow$ attractivity $\rightarrow$ growth $\rightarrow$ prosperity ; change in life frame $\rightarrow$ pollution $\rightarrow$ risks $\rightarrow$ congestion $\rightarrow$ degraded activity $\rightarrow$ economic slow down $\rightarrow$ territorial destruction ; Environmental awareness $\rightarrow$ security $\rightarrow$ need of regulation $\rightarrow$ reglementation $\rightarrow$ innovation $\rightarrow$ amenagement politics $\rightarrow$ change in practices $\rightarrow$ new values.


Necessity of an interdisicplinary research to capture all aspects ; still a lot to do on congruence, role of space. Evaluation of effects of an infrastructure.

Change of perspective on accessibility. desenclavement can lead to territorial isolation : potential accessibility but still closed territories ; can also go in the direction of territorial emptying. Tunnel effect ; new territorial polarizations. France : distance to Paris is crucial.

Also does not tackle socio-spatial justice, only economic performance. discontinuities created by infrastructures. By connecting places, make other farther. General models on colonisation (Taaffe, Debrie, Vance, Rimmer). Stays influencing in many South states. \textbf{Q : see with Sol{\`e}ne how it is true ?}

When a territory is discovered, strong structuring effect (// conclusion Anne, before co-evolution). Infrastructure as a factor to territorialisation.


\subsubsection{Transportation and Sustainable Development}

Energy, security.

Transportation and Mobility. Sustainibility of new urban forms ? (cf These Florent)

Sustainable : transportation at the service of humanity.

\subsubsection{Conclusion : transportation and development, a question of liberty ?}

Transportation as a component to diffuse liberty. Opening not necessarily way to liberty. Liberty to move : not understood well in its collective dimension.




\subsection{Public and Provate stakeholders}

\subsubsection{Redefinition of relations between public and private : separation of functions and dereglementation}

Liberal evolutions : spatial consequences. Territorial consequences of private interests ? (cf Raffestin)

Development and integration of transportation networks. Maritime : impact of concurrencies ? Airways. Understand role of space in various typologies.

Public : too old political frameworks ? adaptation of institutional spaces.

\subsubsection{Variety of modes and domains of terrestrial transportation}

opposition of research on fret and people. few geography against economics. some privilegied time periods. choices influenced by opportunities.

Conclusions : impact of dereglementation : increase in concurrence, then simplification and new domining positions. Concurrency kills monopoly and oligopoly kills concurrence. impact on level of service (negative). intervention of state on reglementation. Typology : north america (antitrust) ; britain (neoliberal, privatisation) ; european (opening to concurrency, no more monopoly to exploit infrastructure)

On fret, more impact, desequilibrium on market.

Theories to justify dereglementation : economic/math theories, but with wrong assumptions.

Research directions : typology of various processes. pluridisciplinary research.

\subsubsection{Geographical diversity}

States vs Europe. fail of Eurotunnel.

\subsection{Conclusion}

Opening on possible research directions. need interdisciplinary colloquium. Need more general studies. \textbf{``How do general processes on the mondial space, shifted in time, yield strong geographical differenciations within this same space ?''} $\rightarrow$ \textit{lagged spatio-temporal complex processes ?}

[... ??]

\section{Synthesis}

\textit{TBW}





%%%%%%%%%%%%%%%%%%%%
%% Biblio
%%%%%%%%%%%%%%%%%%%%

\bibliographystyle{apalike}
\bibliography{/Users/Juste/Documents/ComplexSystems/CityNetwork/Biblio/Bibtex/CityNetwork}


\end{document}
