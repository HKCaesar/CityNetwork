%%%%%%%%%%%%%%%%%%%%%%%%%%%%%
% Standard header for working papers
%
% WPHeader.tex
%
%%%%%%%%%%%%%%%%%%%%%%%%%%%%%

\documentclass[11pt]{article}

%%%%%%%%%%%%%%%%%%%%
%% Include general header where common packages are defined
%%%%%%%%%%%%%%%%%%%%

% general packages without options
\usepackage{amsmath,amssymb,bbm}




%%%%%%%%%%%%%%%%%%%%
%% Idem general commands
%%%%%%%%%%%%%%%%%%%%
%% Commands

\newcommand{\noun}[1]{\textsc{#1}}


%% Math

% Operators
\DeclareMathOperator{\Cov}{Cov}
\DeclareMathOperator{\Var}{Var}
\DeclareMathOperator{\E}{\mathbb{E}}
\DeclareMathOperator{\Proba}{\mathbb{P}}

\newcommand{\Covb}[2]{\ensuremath{\Cov\!\left[#1,#2\right]}}
\newcommand{\Eb}[1]{\ensuremath{\E\!\left[#1\right]}}
\newcommand{\Pb}[1]{\ensuremath{\Proba\!\left[#1\right]}}
\newcommand{\Varb}[1]{\ensuremath{\Var\!\left[#1\right]}}

% norm
\newcommand{\norm}[1]{\| #1 \|}


% amsthm environments
\newtheorem{definition}{Definition}



%% graphics

% renew graphics command for relative path providment only ?
%\renewcommand{\includegraphics[]{}}








% geometry
\usepackage[margin=2cm]{geometry}

% layout : use fancyhdr package
\usepackage{fancyhdr}
\pagestyle{fancy}

\makeatletter

\renewcommand{\headrulewidth}{0.4pt}
\renewcommand{\footrulewidth}{0.4pt}
%\fancyhead[RO,RE]{\textit{Working Paper}}
\fancyhead[RO,RE]{\textit{ECTQG 2015}}
%\fancyhead[LO,LE]{G{\'e}ographie-Cit{\'e}s/LVMT}
\fancyhead[LO,LE]{An Algorithmic Systematic Review}
\fancyfoot[RO,RE] {\thepage}
\fancyfoot[LO,LE] {\noun{J. Raimbault}}
\fancyfoot[CO,CE] {}

\makeatother


%%%%%%%%%%%%%%%%%%%%%
%% Begin doc
%%%%%%%%%%%%%%%%%%%%%

\begin{document}







\title{Reading Record\bigskip\\
\cite{pan2003croissance}
}
\author{\noun{Juste Raimbault}}
\date{Date}


\maketitle

\textbf{\textit{Reading Record for \cite{pan2003croissance}}}


\section{Introduction}

Close to an operational work in urbanism/mobility. Franco-Chinese colloquium (october 2001)

\paragraph{Reflexion on evolving cities} 

Mobility crucial for modernity ; urban growth linked to intermodality/multimodality ; exchange experiences and ideas ; joint seminar.

\paragraph{Benefits of scientific/cultural exchange}

\paragraph{City as a preferred field of exchange}

International and interdisciplinary (practitioners and scientists). 



\section{Linear Reading}

\subsection{Part 1 : Places of Intermodality}

Importance of intermodality hubs in its concretisation.

\subsubsection{Perspective on Urban Mobility}

Twentieth century was all about transitions ; latest is urban mobility.

\paragraph{Pedestrian cities}

Size of former cities determined by walking radius (around 3km) : pedestrian cities

\paragraph{Multimodal cities}

Industrial revolution : new transportation modes.

\paragraph{Car cities}

From the States, progressive diffusion of the all-for-cars model.

\paragraph{Reasoned multimodal cities}

Negative externalities of car have rapidly induced limitation of its share. Cities become multi-scalar. Requires a general political action for mobility.

\paragraph{Only one model ?}

Diverse trajectories across the globe. Bus can be before car. Or train for density (Japan) or political (URSS) reasons. China : bike.


\subsubsection{Problems and Solutions in Development of Urban Transportation in China}

\paragraph{Context}

\begin{itemize}
\item Very fast economic growth and mutation of industrial structure induced an accelerated urban growth.
\item Combined to an exponential growth of automobile ownership.
\item Strong real estate and environmental constraints.
\end{itemize}

\paragraph{Current issues}

\begin{enumerate}
\item Congestion for bikes as no dedicated road system.
\item Small share of public transportation
\item Increase of taxis share
\item Increase of private car
\item Specific strategy for lightweight rail
\item High construction density
\item Center redevelopment induce functional mutation and population migrations from center to suburbs
\item High pollution impact
\end{enumerate}

\paragraph{Solutions}

Criteria : economic feasibility ; financial equilibrium ; social acceptation ; enviromnental sustainibility.

Solutions : coupled development ({\'a}-la-TOD) ; concrete measures for public transportation ; increase investments in infrastructures ; specialization of road network ; projects of circulation ; increase regulation role of central government ; NTIC.


\subsubsection{Transportation and Intermodality in Ile-de-France}

\paragraph{What is intermodality}

Transfer between two modes. In practice implies a collective mode.

\paragraph{Requirements}

\begin{itemize}
\item Physical organization of transfer space
\item Coordination of timetables
\item Coordination of tarification
\item Information of travelers
\end{itemize}

Places of intermodality : exchange poles, multimodal poles, intermodal poles, multimodal platform.

\paragraph{Why encourage it ?}

Negative externalities of private car ; desillusion of Modern Urbanism.

Reduce use of car, preserve health, urban quality of life reduce infrastructure cost, energy consumption. One way to create a desirable city.

\paragraph{Organisation in the past (70-95)}

Concentration on commuting, radial trajectories, peak hours, between public transportation. Creation of ``Carte Orange''.

Later understood need to encourage car-public transportation intermodality. (P+R). fail, because no significant improvement for the user. : larger span, in suburban stations. Then larger multimodal poles : La Defense, and 3 new high speed stations.

\paragraph{Recent directions}

New multimodal poles, densification around stations, quality improvement for median poles. Improvement of insertion in the city : less urban cuts e.g., better accesses, quality of exchange, services in stations, etc.

\paragraph{Quantification}

552 P+R. 11\% TC-multimodal.

Strong potential but geographical inequalities (long and very long travels have the strongest intermodal share).

\paragraph{Current issues and perspectives}

Weak penetration rate. Strong dispersion of responsabilities (governance of poles). Logistics and fret.

\paragraph{Conclusion}

Necessity of very large poles, but need of smaller as car is generalized. Answer for an efficient, agreable and sustainable city.



\subsubsection{Project presentation of Beijing new transportation hub}

In Beijing : 50 years ago, ``cyclopousses'', 30 : bike, 20 : subway, 10 : car, 5 : public transportation (bus, subway, suburban rail). Transportation hub as a new urban component. Not only superposition, importance of organization (taking into account flows evolution on long time scales). Also tertiary hotspot.

Conception : Chinese construction minister, in partnership with AREP.

In growth context.


\subsubsection{Invent the city of ``mechanical transportation'' (Duthilleul)}

\paragraph{Still to invent}

Organisation of networks (modernism) was not enough, everything still to write.

\paragraph{New spaces to construct}

Hub as a concentrated urban space, convergence of flows.

Modern and agreable city invented around all transportation modes. For the user, the space between infrastructures is crucial. The project, the expression of an ideal.

\paragraph{Examples}

CDG2 : new place of life ; Valence ; Aix : optimal exchange space ; Lille : generate an economic development thanks to dynamism - the station as a new street ; Paris (RERE) : new parisian spaces, no ``over'' and ``under'', all levels of the city are noble (\textbf{\textit{reflexion on that point : which implication for the superposition of networks. the multi-layer city is physically realized. the technical performance allows the complexity of network-territories interactions - reread Dupuy on that - on the importance of technical networks as veins of the city ?}}) ; Marseille : keep identity ; Shangai : Circular space, identity of this city far East, from which the sun rises on all other Chinese cities.


\subsubsection{Analysis of modal structure of mobility}

Crucial step for planification. Frequency and distance of commuting depending on economical regime (planified or market economy).

Factors of modal choices ? formula of multimodal travel times : analysis of concurrency between two modes, as a function of $\Delta t$ between modes (roughly).

Implications for policies to change its sign : examples : increase density of public transportation ; better info (GPS) ; improve traffic conditions ; coordinate timetables ; incentives for users.

Impact of congestion if use of individual modes. Change in income : change in transportation mode. Taxis and motorbikes also strongly problematic. Cars : income transition, corresponds to when subway was built in European cities. Each mode has its corresponding spatial span. Depending on modal share structure for a given city, corresponding policies. 




\subsubsection{Political Management of Urban Mobility Places in Nantes - Parking as a regulation tool}

\paragraph{Situation}

History strongly influenced by geographical localization : port, importance of water, economic hub. Urban innovation recently.

\paragraph{Mobility}

First city to implement a modern tramway line. Requalification of public spaces. Voluntary action for complementarity of transportation modes : coordinated networks ; common tarification with regional trains.

\paragraph{Experiment of parking regulation}

Creation of P+R to follow evolving practices. 

\begin{itemize}
\item Positive aspects : increasing occupation rates ; choice motivated by travel time and cost.
\item Issues : Modal report, some (20\%) use car now.
\item Favorable factors : Localization, transportation system performance, ambiance, services. General governance : reduction of parking in the city, beneficial tarification.
\end{itemize}

\paragraph{Perspectives}

Equilibrium between modes ; new parking offer ; modulation in some areas ; easying for residents.

Train : same transportation ticket.

Actions : modernisation of infrastructures ; new stations ; organisation around stations ; new intra-urban station ; new tarifications.

Objective : also increase inter-urban train modal share.





%%%%%%%%%%%%%%%%%%
\subsection{Flows management}

Contemporaneous city as a space a flows.

\subsubsection{Life modes and evolution of cities : a new Urbanism ?}

Third urban revolution (Renaissance, then capitalist city)

\paragraph{A new urbanism}

\textit{New Urbanism} : smaller Urban Projects, more flexible. Rodgers in UK. In France, Urban renewal (against modernism). But also increase in demand of individual housing. Organize urban sprawl. Countries where urban sprawl is reasonable. Crisis due to forms of modernism, and of lifestyle discrepancy between urban and rural.

\paragraph{New urbanism as a consequence of lifestyle evolution}

Increase in individual autonomy : change in housing demand (greater lodging). faster mobility. Increase in variety of demand, trajectories, etc. Issue of suburban mobility, at any time. Urban life less and less regular. Mobility and Information at the heart of contemporary urban life, make success of cities. Accelerating pace of life. Time is what misses most : implies use of urban services to compensate. Shopping in stations. Decrease of work time across lifespan. Life less determined by work. Mobility for leisure, weekend, also with family. Increase in interest for natural and cultural heritage (consequence of population fears) : influence on archi. But also regret old cities : urban tourism. European city, alternative to american cities. Soft transportation modes (tram, bike). Also risk society : increase in safety requirements (techno), but risks also increase with technology. Consequences : fears and individuality, everything must be debated, can have negative impacts on projects, but also beneficial if helps to understand needs. Generally, quality of projects is consequence of difficulties during conception. As Urbanist, better, as stimulate creativity and requires more.










%%%%%%%%%%%%%%%%%%%%
%% Biblio
%%%%%%%%%%%%%%%%%%%%

\bibliographystyle{apalike}
\bibliography{/Users/Juste/Documents/ComplexSystems/CityNetwork/Biblio/Bibtex/CityNetwork}


\end{document}
