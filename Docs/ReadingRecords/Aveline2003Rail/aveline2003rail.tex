%%%%%%%%%%%%%%%%%%%%%%%%%%%%%
% Standard header for working papers
%
% WPHeader.tex
%
%%%%%%%%%%%%%%%%%%%%%%%%%%%%%

\documentclass[11pt]{article}

%%%%%%%%%%%%%%%%%%%%
%% Include general header where common packages are defined
%%%%%%%%%%%%%%%%%%%%

% general packages without options
\usepackage{amsmath,amssymb,bbm}




%%%%%%%%%%%%%%%%%%%%
%% Idem general commands
%%%%%%%%%%%%%%%%%%%%
%% Commands

\newcommand{\noun}[1]{\textsc{#1}}


%% Math

% Operators
\DeclareMathOperator{\Cov}{Cov}
\DeclareMathOperator{\Var}{Var}
\DeclareMathOperator{\E}{\mathbb{E}}
\DeclareMathOperator{\Proba}{\mathbb{P}}

\newcommand{\Covb}[2]{\ensuremath{\Cov\!\left[#1,#2\right]}}
\newcommand{\Eb}[1]{\ensuremath{\E\!\left[#1\right]}}
\newcommand{\Pb}[1]{\ensuremath{\Proba\!\left[#1\right]}}
\newcommand{\Varb}[1]{\ensuremath{\Var\!\left[#1\right]}}

% norm
\newcommand{\norm}[1]{\| #1 \|}


% amsthm environments
\newtheorem{definition}{Definition}



%% graphics

% renew graphics command for relative path providment only ?
%\renewcommand{\includegraphics[]{}}








% geometry
\usepackage[margin=2cm]{geometry}

% layout : use fancyhdr package
\usepackage{fancyhdr}
\pagestyle{fancy}

\makeatletter

\renewcommand{\headrulewidth}{0.4pt}
\renewcommand{\footrulewidth}{0.4pt}
%\fancyhead[RO,RE]{\textit{Working Paper}}
\fancyhead[RO,RE]{\textit{ECTQG 2015}}
%\fancyhead[LO,LE]{G{\'e}ographie-Cit{\'e}s/LVMT}
\fancyhead[LO,LE]{An Algorithmic Systematic Review}
\fancyfoot[RO,RE] {\thepage}
\fancyfoot[LO,LE] {\noun{J. Raimbault}}
\fancyfoot[CO,CE] {}

\makeatother


%%%%%%%%%%%%%%%%%%%%%
%% Begin doc
%%%%%%%%%%%%%%%%%%%%%

\begin{document}







\title{Reading Record\bigskip\\
\cite{aveline2003ville}
}
\author{\noun{Juste Raimbault}}
\date{Date}


\maketitle

\textbf{\textit{Short Reading Record for \cite{aveline2003ville}}}



\section{Linear Reading}

\subsection*{Introduction}

Recent mutations of mobility consequence of societal ; in Japan good mobility conditions but less evolution. Good intermodality ; based on private companies.

Otemintetsu : train only ; Kanto and Kansai.

\subsection{Train transportation in Japan}

In Japan, very low car penetration rate. 1950 : rail 90\% travels. Tokyo and Osaka urbanized long before motorization. Car taxation disadvantaged.

\paragraph{Dense Rail NW}

Group JR : national network. Otemintetsu : 15 small companies. Tokyo : radial lines. 3 categories : territorial monopoly ; big capital but modest network ; small companies. Other companies (regional) : 2\% traffic. Metropolitan networks : public.

\paragraph{Transportation service quality}

fragmented but high level of service. high security and capacity. good confort and congestion decreased (compared to 1960). Tarification variable.

\subsection{Rail network origin}

\paragraph{National railway}

From 1860 real development. first private company 1881. Nationalisation 1900.

\paragraph{Urbanisation}

very rapid urban growth in Tokyo. Osaka : periurban.

\paragraph{War}

reconfiguration for optimization. after the war, unification of metropolitan nw. JNR in 1949.

\paragraph{Growth}

1955->1973. high increase of demand. concurrence of motorization. JNR : technology ; otemintetsu : diversification.

JNR : Shinkansen. Otemintetsu : activity diversification : commercial, amenagement, parcs, sport etc.

\paragraph{From 1973 to today}

No more strong land value market for otemintetsu. Growth of third sector (other private companies). privatisation fo JNR : JR. lets debt : increase in lines exploitation ; opening to other activities.

Complexity of rail network due to simple historical path ; separation of activities among scales. today roles less clear.

\subsection{Diversification of otemintetsu}

Numerous subsocieties. even transportation core is diversified, only 58\% train : real estate, hotel, tourism, etc.

\paragraph{Diversification of transport}

Car complementary to rail. Freight. Maritime and aerial transportation. Technical expertise in train.

\paragraph{First stages of diversification}

K Ichizo : company Mino-arima. real estate promotion, zoos, thermal. Not directly profitable but increase traffic on the line. At terminal : mall to capture economies done on transportation costs. 

Garden cities : 1923 ; special company absorbed by Meguro-Kamata Dentetsu (connection of garden city to train network).

After the war : projects with new acquisition mode. organized around station. Keio university.

Sport and parcs : also increase traffic.

Differences Kanto and Kansai : in Kanto endogeneous pendular migrations. today difference in level of service.

Seibu empire : illustrates well diversification, along generations.

\paragraph{Real estate expansion}

New towns - garden cities. campus-cities. special subcompanies to manage real estate during high growth. Tokyu Dentetsu : 60\% of urbanized surface in Yokohama.

Also renting (huge capital). bails ; real estate renting.

\paragraph{Leisures}

first in Kansai : pelerinage. then tourism. mountains,thermal.

Restaurants and Hotels : Seibu has largest positioning. profits from empire fall. Franchized fast-foods also.

Sport : early sport equipements. companies owned base-ball teams.

Yuenchi : parcs. back to 1987. zoo, natural parcs, amusement parks. 

Culture : numerous cultural activities. Seibo.

\paragraph{Diverse scales}

Terminal station : real urban centers ; generally coupled to JR stations. new centralities. also at extremities of lines (bus station). well organized public spaces (fluid space).

Local version of the mall : important role because of low motorization rate ; variety of service and numerous employees. Also supermarkets and specialized commerces. Convenience stores. Specialized station commerces. Commercial galeries.

\paragraph{New services}

Advertissement ; telecoms ; finance.

\paragraph{Gestion of urban space}

initial structuration around urban centers.

public/private relations : very complex. company executives play a role in state decisions. Weak investment of state in transportation. in housing it has negative impact. suburban space mainly shaped by private companies.




\subsection{Rail Network and Land value profits}

\paragraph{Economical insight}

back Von Thunen. Alonso Muth and neoclassical economy. Totally inappropriate for Japanese stations case.

\paragraph{Empirical insight}

Direct effect on real estate values ; but accessibility not the only factor (especially for extremes : richest and poorest districts). other factors : zoning changes etc.

\paragraph{Spatial equilibrium with capital}

works on ressources allocation in transportation for japanese case. abuse of position by private companies.

\paragraph{Spatial distribution of land values}

real coevolution ; difficulty to distinguish. local station accessibility shape prices.


\paragraph{Reglementary frame}

Regional planification of urabn transportation. Mixed commissions : influence of private. Mixed economy companies for subventions.

POS/PLU : local, whereas transportation national. no preemption nor expropriation. Today : resplitting of terrains and change of use, no change of property.

\paragraph{Land properties of otemintetsu}

not large surfaces but high prices. Some sold. Seibu company success based on land value.

\paragraph{Strategies}

Example of acquisition before project : Denentoshi ; Hankyu at Osaka. 

Recycling. ex golf and base-ball.

\paragraph{Example of new line}

Keihanna Shinsen. land already owned. conflicts with municipality. However first obstacle landowners more than administration.

\subsection{Private Transportation and General Interest}

Control of the state on transportation activities, but right to territorial monopoly and liberty in extra activities.

Complicated procedure to change tarification. Balance between activities allow to have fixed tarifications. recently more complicated (land value crisis).

\paragraph{Public subsidies}

Low rate lends. Provisory owning by national construction company. Special cities development : reserve fund.

\subsection{Perspective for otemintetsu ?}

\paragraph{Fall of traditionnal activities}

financial crisis. difficulty to balance transportation only. increase in level of service, confort etc to gain parts of mode.

Difficulty because of land value crisis. new strategy : share investments with other operator. New real estate companies for investments (on stock market).

Leisure : more impact of economic crisis ; too old against innovative projects.

Adaptation of consommation. 

\paragraph{New activities}

NTIC : perspectives.

Funeral offices. all stages of life accompanied.

\subsection*{Conclusion}

Genesis of otemintetsu ; advantage state and them. Strong personalities of executives. 

Very high level of service.

However urban sprawl not wanted.

Today adaptation. large land possessions and notoriety. Not all same response capacity.

Rail should keep its structuring role. but mutations (change of life rythms : adaptation of needs). Mobility as a strategy : continuity of space, of transportation, of services, of management. restructurations do not avoid it, objective of sustaining mobility keeps a priority.


\section{Biblio}

Dupuy : Urbanisme des réseaux théories et méthodes.

Alcaly 1976 : transportation landuse values.

Coffman Gregson 1998  : railroad development and land use value.

Damm Lerman Youg 1980 Response urban real estate anticipation Washington metro


Deboulet Agnes 92-93

Derycke 96 Equilibre spatial urban.

Goldberg 72 : evaluation interaction between urban transport land use systems.

Knight Trygg 1977: evidence land use impacts of rapid transit systems.

{\centering

\vspace{2cm}

$\ast$ \hspace{1.5cm} $\ast$

\vspace{0.5cm}

$\ast$

}








%%%%%%%%%%%%%%%%%%%%
%% Biblio
%%%%%%%%%%%%%%%%%%%%

\bibliographystyle{apalike}
\bibliography{/Users/Juste/Documents/ComplexSystems/CityNetwork/Biblio/Bibtex/CityNetwork}


\end{document}
