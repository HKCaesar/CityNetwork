%%%%%%%%%%%%%%%%%%%%%%%%%%%%%
% Standard header for working papers
%
% WPHeader.tex
%
%%%%%%%%%%%%%%%%%%%%%%%%%%%%%

\documentclass[11pt]{article}

%%%%%%%%%%%%%%%%%%%%
%% Include general header where common packages are defined
%%%%%%%%%%%%%%%%%%%%

% general packages without options
\usepackage{amsmath,amssymb,bbm}




%%%%%%%%%%%%%%%%%%%%
%% Idem general commands
%%%%%%%%%%%%%%%%%%%%
%% Commands

\newcommand{\noun}[1]{\textsc{#1}}


%% Math

% Operators
\DeclareMathOperator{\Cov}{Cov}
\DeclareMathOperator{\Var}{Var}
\DeclareMathOperator{\E}{\mathbb{E}}
\DeclareMathOperator{\Proba}{\mathbb{P}}

\newcommand{\Covb}[2]{\ensuremath{\Cov\!\left[#1,#2\right]}}
\newcommand{\Eb}[1]{\ensuremath{\E\!\left[#1\right]}}
\newcommand{\Pb}[1]{\ensuremath{\Proba\!\left[#1\right]}}
\newcommand{\Varb}[1]{\ensuremath{\Var\!\left[#1\right]}}

% norm
\newcommand{\norm}[1]{\| #1 \|}


% amsthm environments
\newtheorem{definition}{Definition}



%% graphics

% renew graphics command for relative path providment only ?
%\renewcommand{\includegraphics[]{}}








% geometry
\usepackage[margin=2cm]{geometry}

% layout : use fancyhdr package
\usepackage{fancyhdr}
\pagestyle{fancy}

\makeatletter

\renewcommand{\headrulewidth}{0.4pt}
\renewcommand{\footrulewidth}{0.4pt}
%\fancyhead[RO,RE]{\textit{Working Paper}}
\fancyhead[RO,RE]{\textit{ECTQG 2015}}
%\fancyhead[LO,LE]{G{\'e}ographie-Cit{\'e}s/LVMT}
\fancyhead[LO,LE]{An Algorithmic Systematic Review}
\fancyfoot[RO,RE] {\thepage}
\fancyfoot[LO,LE] {\noun{J. Raimbault}}
\fancyfoot[CO,CE] {}

\makeatother


%%%%%%%%%%%%%%%%%%%%%
%% Begin doc
%%%%%%%%%%%%%%%%%%%%%

\begin{document}







\title{Reading Record\bigskip\\
\cite{knowles2016sir}
}
\author{\noun{Juste Raimbault}}
\date{Date}


\maketitle

\textbf{\textit{Reading Record for \cite{knowles2016sir}}}




\section{Introduction}

Collaborative memory book for Peter Hall. 


\section{Linear Reading}

\subsection{Introduction}

Peter Hall began Pr. of geography, later Urban Planning. Thesis (1962) in urban economics and historical geography. Very critical on Planning. Many research themes, from history of planning, London growth, spatial planning, mobility to globalized urbanisation ; combined all in a complementary way. Many professional activities.

Book reflect on various research themes.

\subsection{A Polymath in City studies}

\subsubsection{Being a Polymath}

Across disciplines but uses all to study one object, cities. Importance of not taking disciplines seriously, bridges between disciplines (economics $\rightarrow$ geography ; turned planning from practice to discipline)

\subsubsection{Five intersections}

\paragraph{Geography and Planning}

Formerly, planning seen as applied geo. Hall did not import economics into geo, but made planning more theoretical, provided biography of many practitioners and thinkers.

\paragraph{Cities and State}

Critical view on State actions : actions against urban sprawl ; large projects : London third airport as planning disaster ; divergence of planning purposes and planning outcomes.

\paragraph{Times and Spaces}

Relation between time and space. Waves of innovation : spatial agglomeration produces innovation. Space of flows ; impredictability of patterns. Unique period/unique place : \textbf{cf non-ergodicity in evolutive urban theory}.

\paragraph{Town/Country and City/Region}

Question of regional spatial organisation.

\textbf{Biblio : } \cite{hall2006polycentric}

Functional Urban regions : city regions ; now coming back as global city regions. Entities between national and local policy.

\paragraph{London and Globalization}

Largely studied cities ; wrote on globalization before the trend. Identifying world cities. Multi-nodal regions. Focus on intra-regional transport and communication.

Some kind of intersection Geography/Planning.

\subsubsection{An Optimistic Legacy}

Study of cities in a political context : eye on the future. Parallel with Gottman and Jacob. All three were optimistic. Hall did engage in policy.


\subsection{Location and Innovation}

\subsubsection{Introduction}

Importance of firm relocation dynamics in Hall understanding of regional development : chronological perspective on the subject.

\subsubsection{Importance of History 1}

large work on industrial and urban history. ``economic geo, treated historically''. On London : always be a center.

\subsubsection{Importance of Chance}

topics that may seem details but witness large scale transitions. Data but merely historical knowledge. memory for detail : appreciation of happenstance (time and chance). 

\subsubsection{Importance of History 2}

not convinced by econometrics, micro and macroeconomics models. Theory crucial but not everything.

Hierarchy and Christaller : compatible with polycentric MCR. Marshallian vision, but also close to Weber : locational optimization (shipping cost etc). Ubiquitous commodities ? difficult, even for telecoms : non homogeneous networks. Information/knowledge in human : world cities ?

\subsubsection{Importance of clusters}

against death of distance arg ; example of the library : informational gradient. need of open environments for innovation. reproduction of silicon valley ? crucial role of universities and gov.

\subsubsection{Importance of infrastructure} 

role of accessibility, but also negative network effects (decrease for smaller places). \textbf{cit}

generic functions and concentrated deconcentration. adaptability of older industrial places : revitalization of inner city economy : link between industry and transportation and the built environment.

\subsubsection{Role of the State}

policy for urban and regional development. for large scale infrastructures, nationalization useful. more recently Europe and telecoms. Military in britain. difficult for the state to cope with successive states of creative destruction.

\subsubsection{Conclusion}

growth business of 21th ? eco-tech ? suited to ``real'', historical cities.

complex feedback effects : \textbf{\textit{technology and locational preferences interact with cities and infrastructures to generate complex feedback effects}}. 

From Hall 91, captures essences of regional development : \textit{The fortune of these world cities, and those that depend on them at lower levels in their local hierarchy, will clearly be determined by both their traditional and new roles as hubs or central nodes shaping the trunk lines of these networks... Cities themselves will become increasingly multi-centred around information-based industries, and the development of local centres will help to shorten travel generally and commuting distances in particular. And recreational facilities may be expected to be increasingly important components in (and between) these centres, as the quality of life increases in important(ce?) in the hierarchy of human need.}

study of the past to understand future.


\subsection{Transport and Place-Making : A Long View}





\section{Synthesis}


\textbf{Citation : } p30 (Location and Innovation - The Importance of Infrastructure, J. Reades). ``What Peter's lifelong advocacy of transport - and particularly of High Speed Rail - pointed towards is the idea that infrastructure investment could ``bend the curve'', \textbf{but only if it came in the right place, at the right time, and in the right amount}''. (rq : says a lot and nothing, quite fuzzy indeed. points however in the sense of co-evolution).









%%%%%%%%%%%%%%%%%%%%
%% Biblio
%%%%%%%%%%%%%%%%%%%%

\bibliographystyle{apalike}
\bibliography{/Users/Juste/Documents/ComplexSystems/CityNetwork/Biblio/Bibtex/CityNetwork}


\end{document}
