%%%%%%%%%%%%%%%%%%%%%%%%%%%%%
% Standard header for working papers
%
% WPHeader.tex
%
%%%%%%%%%%%%%%%%%%%%%%%%%%%%%

\documentclass[11pt]{article}

%%%%%%%%%%%%%%%%%%%%
%% Include general header where common packages are defined
%%%%%%%%%%%%%%%%%%%%

% general packages without options
\usepackage{amsmath,amssymb,bbm}




%%%%%%%%%%%%%%%%%%%%
%% Idem general commands
%%%%%%%%%%%%%%%%%%%%
%% Commands

\newcommand{\noun}[1]{\textsc{#1}}


%% Math

% Operators
\DeclareMathOperator{\Cov}{Cov}
\DeclareMathOperator{\Var}{Var}
\DeclareMathOperator{\E}{\mathbb{E}}
\DeclareMathOperator{\Proba}{\mathbb{P}}

\newcommand{\Covb}[2]{\ensuremath{\Cov\!\left[#1,#2\right]}}
\newcommand{\Eb}[1]{\ensuremath{\E\!\left[#1\right]}}
\newcommand{\Pb}[1]{\ensuremath{\Proba\!\left[#1\right]}}
\newcommand{\Varb}[1]{\ensuremath{\Var\!\left[#1\right]}}

% norm
\newcommand{\norm}[1]{\| #1 \|}


% amsthm environments
\newtheorem{definition}{Definition}



%% graphics

% renew graphics command for relative path providment only ?
%\renewcommand{\includegraphics[]{}}








% geometry
\usepackage[margin=2cm]{geometry}

% layout : use fancyhdr package
\usepackage{fancyhdr}
\pagestyle{fancy}

\makeatletter

\renewcommand{\headrulewidth}{0.4pt}
\renewcommand{\footrulewidth}{0.4pt}
%\fancyhead[RO,RE]{\textit{Working Paper}}
\fancyhead[RO,RE]{\textit{ECTQG 2015}}
%\fancyhead[LO,LE]{G{\'e}ographie-Cit{\'e}s/LVMT}
\fancyhead[LO,LE]{An Algorithmic Systematic Review}
\fancyfoot[RO,RE] {\thepage}
\fancyfoot[LO,LE] {\noun{J. Raimbault}}
\fancyfoot[CO,CE] {}

\makeatother


%%%%%%%%%%%%%%%%%%%%%
%% Begin doc
%%%%%%%%%%%%%%%%%%%%%

\begin{document}







\title{Reading Record\bigskip\\
\cite{espacegeo2014effets}
}
\author{\noun{Juste Raimbault}}
\date{Date}


\maketitle

\textbf{\textit{Reading Record for \cite{espacegeo2014effets}}}


Controversies on possible causal relations of transportation infrastructures on territorial development still regularly come back in the spotlight, as the example of the 2013 book published by the \emph{DATAR} (french agency for territorial development) that pretended to study ``territorial effects of high speed lines in France''.

\noun{Offner} : why is the myth still strong although most of works since have demonstrated the absence of structural effects ? Amplification and reinforcement of existing trends. Easy for politics to believe that. Today : no scientific misunderstanding but still in political field.


\noun{Beaucire} Some cases of influence, such as immediate relocations, do exist, but not systematically and many times accompanied by \emph{destructuring} effects in other parts of the territory. Offner proposed the notion of territorial \emph{congruence} : not used so much, but rather a ``succession of linear causations''. What changed ? First thing is a change in scale for territorial considerations : European frame, reduction of territorial inequalities : opportunities to be exploited or not, structuring effects depend on local public and private actors. Secondly most projects are now planned in an \emph{integrated} way (cf Bahn-ville).

\noun{Delaplace} Recent work~\cite{bazin2011grande} compared academic perspectives with ``grey literature'' corresponding to planning projects and showed a huge gap. Problem of decontextualization of local case studies. Take specific territorial context into account ?


\noun{Fremont} Better way to ask the question : possible contribution of an infrastructure project to a territorial project ? multi-scale frame to question need of a new infrastructure. Necessity to take territory into account in projects.

\noun{Ninot} Relations between Networks and Territories in developing countries. Various examples that should be understood differently depending on temporal and spatial scale. Importance of empirical considerations to understand that ``black box''.


\noun{Bretagnolle} Particular historical context for the emergence of such debates. Beware of linear causation assumption : complex interactions between heterogeneous actors at different scales ; need of long time period and large spatial scale to identify clear mechanisms. Example of France : positive retroaction between accessibility and centrality level. Other example of postal routes. On very long time scales, \emph{structuring effects} can be observed.

\noun{Pumain} Importance of spatial and temporal context. Particularity of explication is indeed not particular to social sciences. Constructivist approach : scientific knowledge can only be understood in its social context and current state of knowledge. Offner affirmation does not hold as non-contextualized. Attractors in dynamic : indices of irreversible evolution (Q : link to Prigogine), if identified, causal relation exists at a macro-geographical scale. At smaller scales, causality are circular. $\rightarrow$ \textbf{rq : circular causation : close to our theoretical framework ?}













%%%%%%%%%%%%%%%%%%%%
%% Biblio
%%%%%%%%%%%%%%%%%%%%

\bibliographystyle{apalike}
\bibliography{/Users/Juste/Documents/ComplexSystems/CityNetwork/Biblio/Bibtex/CityNetwork}


\end{document}
