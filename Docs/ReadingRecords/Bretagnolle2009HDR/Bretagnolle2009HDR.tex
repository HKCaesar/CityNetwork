%%%%%%%%%%%%%%%%%%%%%%%%%%%%%
% Standard header for working papers
%
% WPHeader.tex
%
%%%%%%%%%%%%%%%%%%%%%%%%%%%%%

\documentclass[11pt]{article}

%%%%%%%%%%%%%%%%%%%%
%% Include general header where common packages are defined
%%%%%%%%%%%%%%%%%%%%

% general packages without options
\usepackage{amsmath,amssymb,bbm}




%%%%%%%%%%%%%%%%%%%%
%% Idem general commands
%%%%%%%%%%%%%%%%%%%%
%% Commands

\newcommand{\noun}[1]{\textsc{#1}}


%% Math

% Operators
\DeclareMathOperator{\Cov}{Cov}
\DeclareMathOperator{\Var}{Var}
\DeclareMathOperator{\E}{\mathbb{E}}
\DeclareMathOperator{\Proba}{\mathbb{P}}

\newcommand{\Covb}[2]{\ensuremath{\Cov\!\left[#1,#2\right]}}
\newcommand{\Eb}[1]{\ensuremath{\E\!\left[#1\right]}}
\newcommand{\Pb}[1]{\ensuremath{\Proba\!\left[#1\right]}}
\newcommand{\Varb}[1]{\ensuremath{\Var\!\left[#1\right]}}

% norm
\newcommand{\norm}[1]{\| #1 \|}


% amsthm environments
\newtheorem{definition}{Definition}



%% graphics

% renew graphics command for relative path providment only ?
%\renewcommand{\includegraphics[]{}}








% geometry
\usepackage[margin=2cm]{geometry}

% layout : use fancyhdr package
\usepackage{fancyhdr}
\pagestyle{fancy}

\makeatletter

\renewcommand{\headrulewidth}{0.4pt}
\renewcommand{\footrulewidth}{0.4pt}
%\fancyhead[RO,RE]{\textit{Working Paper}}
\fancyhead[RO,RE]{\textit{ECTQG 2015}}
%\fancyhead[LO,LE]{G{\'e}ographie-Cit{\'e}s/LVMT}
\fancyhead[LO,LE]{An Algorithmic Systematic Review}
\fancyfoot[RO,RE] {\thepage}
\fancyfoot[LO,LE] {\noun{J. Raimbault}}
\fancyfoot[CO,CE] {}

\makeatother


%%%%%%%%%%%%%%%%%%%%%
%% Begin doc
%%%%%%%%%%%%%%%%%%%%%

\begin{document}







\title{Reading Record\bigskip\\
\textit{\cite{bretagnolle:tel-00459720}}
}
\author{\noun{Juste Raimbault}}
\date{Tuesday August 11th}


\maketitle

\justify


\textbf{Reading Record for \cite{bretagnolle:tel-00459720}} \textit{Villes et R{\'e}seaux de Transports : Des interactions dans la longue dur{\'e}e (France, Europe, Etats-Unis)}


\section*{General Purpose}

Focus on relations between transportation networks and cities, at different spatial and temporal scales. Network evolution do deeply influence the structure of cities and system of cities, but network shape has also its proper evolution. The industrial revolution and the introduction of faster transportation modes lead to differentiated conceptions and definitions of the city accross Europe. Therefore, a consistent study of the coevolution between transportation and urban frame must rely on \emph{harmonized urban database}, for which contruction methods are proposed, based on an ontology of the city on large time scales. These database finally allow to understand that major transitions in city systems occured simultaneously with technological innovations in transportation. Evidences for coevolution and mutual adaptation processes are extracted from the study of accessibility and network shape measures.



\section{Linear Reading}


\subsection*{Introduction}


Assumption of structuring effect of intra- and inter-urban transportation network~\cite{bavoux2005geographie}. \textit{Rq :} Opposed to \emph{the myth of structuring effects}~\cite{offner1993effets}. -- precised here that long time-scale are considered, thus local approximation are made. One can not demonstrate the direct impact of a new line on the economy or urban development because feedbacks do not occur at short term, and effects may appear ``random'' at non-appropriate time scales. \textbf{Q : } references verifying this fact ? $\rightarrow$ Good question, that should be tested statistically and through \emph{modeling} : good for our purpose.

Urban hierarchy is remarquably stable in mong time, whereas accessibility patterns radically changed, thus the central question of this work, which is the exploration of links between the evolution of urban hierarchy and the stucture of exchange networks.


\subsection{A long-time ontology of the city (?): Evolution of urban transportation, city morphology.}

After the industrial revolution, limit of juridic limitations of cities (what was practiced before). No consensus on city definition across countries, still not harmonized today.

\subsubsection{Transportation Revolution and Urban Agglomeration (19th)}

First railways : dilatation of territorries (cf Reclus). Direct effect on urban sprawl : e.g. Boston, associated to enormous growth rates (x3 $\simeq 30$ years).

Juridic definitions of the city try to adapt (cf Portugal). In England, pragmatic adaptation (health issues etc).

Other countries propose a statistical defintion (population threshold) : France, Italy, Germany, US.

Later (1880), first definition of agglomerations by continuity.

$\rightarrow$ list of city definition in Europe in 1900. Only 4 use morphological criteria.

\subsubsection{Temporal or Morphological definition of cities ? 1910-1950}

New generation of public transportation (electricity) : up to x5 in 30 years of pendular mobility. Isochrone maps : temporal referential.

City size determined by an 1-hour time-budget equivalent. Isochrone maps 1919 G. Bonnier (Paris) : natural delimitation of a city.

// morphological definitions, by continuity of constructions. Importance of aerial views (1950).

Insee 1960 : construction continuity implies other criteria. Later diffusion into other countries.

\subsubsection{High Speed Transportation and Territorial Discontinuities : the city as a functional area (2 half of 20th century)}

US : federal-aid Highway 1956 $\rightarrow$ more city sprawl. [Example highway networks 2008 Europe/USA] ; new morphology, more diffuse frontiers between urban and rural.

Even more difficult to give a definition of the city (ex Florence, totally different among 2 databases).

Stability of time-budget (Zahavi's law) : new structuration of urban systems through speed

Definition of urban areas by commuters threshold : US 1950. 1962 Frce, 1967 GB.

Great heterogeneity in LUZ definition accross Europe.

\subsubsection{Intermediate conclusion}

Importance of links between transportation modes and city morphology.

On the long time, (juridic, morphologic, functional) definition <=> (municipality, agglomeration, functional area) views, corresponds to actual historical evolution of cities.

Question of hierarchy of spatial scale (inclusion) : interesting, correponds to the manifestation of different processes : municipality = political ; agglo = ``characteristics'' (comparable cities) ; functional area = economic functions.

\textbf{Q : } What about temporal scales ? and city systems ?

\textit{Rq :} Link with Duranton Tyrannies ?~\cite{duranton1999distance} maybe not so far from the long-time city ontology ?



\subsection{Harmonized urban databases for dynamical and international comparisons}

Proposition of methods to reconstruct the three levels of definition from various sources.

Difficult to handle because data models are necessarily multi-level (as city is agglomeration of smaller entities) and dynamical.

\subsubsection{From ontology to harmonization}

Harmonization = find the most consistent definition of the city depending on temporal and spatial context.

\textit{Temporal situation} Evolution in time of time-budget. Ex : Boston 1830-20000.

\textit{Longitudinal harmonization} trajectories of population for different defs for a city : allows to detect change of status.

\textit{Transversal Harmonization} At a given date, best definition to compare cities accross the world.


\subsubsection{Methods for data harmonization}

\textit{Aggregation of finer statistical data} Ex Frce INED, US.

\textit{Retropolation of past data} Reconstruction of unknown past data ; done on Frce ; method proposed for US.

\textit{Temporal Harmonization} Correction of criteria changes in time. Ex of New York : 11Mio $\rightarrow$ 9 Mio. Case by case modification.

Harmonization method implies relation between objects of the data model.


\subsubsection{Studying urbanization through harmonized databases}

Importance of population threshold. On the long time threshold must be evolutive.

Estimation of threshold from local data ; or automatized with population growth rate. In recent periods, less problems with threshold.

Comparison with official definitions : overestimation at the beginning (threshold), underestimation after 1940 (functional areas).

\textit{Measure of cities growth rates} Evaluation of growth rate in a constant perimeter.

\textit{Concentration of urban population} Importance of number of cities, cut-off for Gibrat etc parameter estimations. Harmonized database gives more consistent results (increase in hierarchy) than municipality definition.



\subsubsection{European Urban Databases}

Base Bairoch : before 1800. realitively biased.

Between 1850 and 1950, no harmonized database. 

Problem of reproducibility for Geopolis database.

\textit{Use of satelite imaging to determine european agglomerations}

Project Grump : various biases, seems quite shitty for Europe. therefore use of UMZ (corine land cover. Morpho maths (closing) to obtain zones. \textbf{Q : } which kernel size ? Combination with Nuts5 to reconstruct population grid. Validation ($\sigma = 5\%$) on Danemark.


\textit{ESPON/Urban Audit} $\rightarrow$ definition of LUZ. Incompatibility with other databases, ex : UMZ/MUAS ; difficult at european level.

CL : integrated vision (theoretical and thematic) ; comparison of databases is essential, through semantic and empirical methods.




\subsection{Exchange Networks and City Dynamics : Theory, Experiments, Modeling}

Non-linear growth of urban systems are not well explained by e.g. pure economic theory, since they neglect the role of space in the hierarchisation process. It has been shown in examples (American East Coast 19th) that accessibility is a crucial variable in the competition between cities.

Influence of network is strongly different depending on the period : before industrial revolution, importance of ``position rents''. Then apparition of transportation networks by extension, reinforcement and new network apparitions. After 1950, path-dependance impose the hierarchy and it is therefore difficult to know wether networks still play a role in the hierarchisation process.

Empirical Data Analysis of network and population \emph{is not sufficient} to answer this question (too much noise ?) and a theoretical approach is needed. \textbf{implies modeling ?}.

``We will show that thoughts of specialists making space aimed to give definitions of city systems, since 1830, are closely linked [to the historical transformations of communication networks].'' :: confirms exactly what Florent proposed on the lack of vision of modelers/disciplines ? $\rightarrow$ linked to multidisciplinarity / scope of models / scale of models / purpose of models :: put that in a perspectivist framework ? 


\subsubsection{Networks and changes in dimensions of systems of cities}

A long time range is necessary to understand the range of interaction in city systems at a given time, i.e. an historical perspective is mandatory.

\textit{Before 1900 : regional urban networks} stable and very slow exchanges in the Middle Age. Role of waterways in some places. Regional range of exchange for most of products.

Regular positions of cities : linked to daily mobility and time-budget ; or as Reclus explains, because cities are regular stops in long journeys.

Later Central Place Christaller : travel time shapes city system by positioning central places.

\textit{Network creation in National States context (1800-1950)} Importance of road network : extension and renovation around 1800 in GB, France.

Later development of rail network. Importance of planification and construction within a global view (in US, many problems due to unconcerted private initiatives).
In France, big implication of the State. Legrand Star adopted in 1842 (against other projects uch as a coupling with main waterways, or a rebalancing of national inequalities).

Whatever the way networks are developed, the mutation of space inducted is huge. (see isochrone maps). Birth of the mondial metropolis. After 1900, global cities are interdependant.

Study of systems at this scale leads to first systemic approaches ; Zipf law ; Berry (largely criticized later as no geographical ontology but ony mathematical similarity in his vision).

\textit{After 1950 : International Opening and exchanges complexification}

After 1950, creation of first airlines. travel speed up to x10. Corresponding Shrinking maps.

Later, decoupling of transportation networks and telecomunication networks. Corresponding approaches, considering city systems as \emph{open} and \emph{complex} $\rightarrow$ Complex Systems Theory seems to be crucial in the development of new frameworks.

\bigskip
\textit{Modeling change of scale for centers of global economy}  

Application of potential model (derived from gravital models), using interaction distances : includes transportation networks.

Potential of interaction for a city is

\[
P_i = \lambda_i(t) m_i(t) + \sum_{j}{\frac{\lambda_{ij}(t)m_j(t)}{d_{ij}}}
\]

where $\lambda_i$ is an intercept parameter (that one would interpret as the intrinsic ``dynamism'' of a city) and $k_{ij}(t)$ allows to take real accessibility into account (computed through historical data, for 1200-1990).

Comparison at different dates of $P_i(t)/\max{P_i(t)}$ : relative importance of a city at date $t$. Cities with maximal potential correspond realtively well to centers of global economy defined by qualitative research.




%%%%%%%%%%%%%%%%%%%%
%% Biblio
%%%%%%%%%%%%%%%%%%%%

\bibliographystyle{apalike}
\bibliography{/Users/Juste/Documents/ComplexSystems/CityNetwork/Biblio/Bibtex/CityNetwork}


\end{document}
