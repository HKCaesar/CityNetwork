%%%%%%%%%%%%%%%%%%%%%%%%%%%%%
% Standard header for working papers
%
% WPHeader.tex
%
%%%%%%%%%%%%%%%%%%%%%%%%%%%%%

\documentclass[11pt]{article}

%%%%%%%%%%%%%%%%%%%%
%% Include general header where common packages are defined
%%%%%%%%%%%%%%%%%%%%

% general packages without options
\usepackage{amsmath,amssymb,bbm}




%%%%%%%%%%%%%%%%%%%%
%% Idem general commands
%%%%%%%%%%%%%%%%%%%%
%% Commands

\newcommand{\noun}[1]{\textsc{#1}}


%% Math

% Operators
\DeclareMathOperator{\Cov}{Cov}
\DeclareMathOperator{\Var}{Var}
\DeclareMathOperator{\E}{\mathbb{E}}
\DeclareMathOperator{\Proba}{\mathbb{P}}

\newcommand{\Covb}[2]{\ensuremath{\Cov\!\left[#1,#2\right]}}
\newcommand{\Eb}[1]{\ensuremath{\E\!\left[#1\right]}}
\newcommand{\Pb}[1]{\ensuremath{\Proba\!\left[#1\right]}}
\newcommand{\Varb}[1]{\ensuremath{\Var\!\left[#1\right]}}

% norm
\newcommand{\norm}[1]{\| #1 \|}


% amsthm environments
\newtheorem{definition}{Definition}



%% graphics

% renew graphics command for relative path providment only ?
%\renewcommand{\includegraphics[]{}}








% geometry
\usepackage[margin=2cm]{geometry}

% layout : use fancyhdr package
\usepackage{fancyhdr}
\pagestyle{fancy}

\makeatletter

\renewcommand{\headrulewidth}{0.4pt}
\renewcommand{\footrulewidth}{0.4pt}
%\fancyhead[RO,RE]{\textit{Working Paper}}
\fancyhead[RO,RE]{\textit{ECTQG 2015}}
%\fancyhead[LO,LE]{G{\'e}ographie-Cit{\'e}s/LVMT}
\fancyhead[LO,LE]{An Algorithmic Systematic Review}
\fancyfoot[RO,RE] {\thepage}
\fancyfoot[LO,LE] {\noun{J. Raimbault}}
\fancyfoot[CO,CE] {}

\makeatother


%%%%%%%%%%%%%%%%%%%%%
%% Begin doc
%%%%%%%%%%%%%%%%%%%%%

\begin{document}







\title{Reading Record\bigskip\\
\textit{\cite{bretagnolle:tel-00459720}}
}
\author{\noun{Juste Raimbault}}
\date{Tuesday August 11th}


\maketitle

\justify


\textbf{Reading Record for \cite{bretagnolle:tel-00459720}} \textit{Villes et R{\'e}seaux de Transports : Des interactions dans la longue dur{\'e}e (France, Europe, Etats-Unis)}


\section*{General Purpose}

Focus on relations between transportation networks and cities, at different spatial and temporal scales. Network evolution do deeply influence the structure of cities and system of cities, but network shape has also its proper evolution. The industrial revolution and the introduction of faster transportation modes lead to differentiated conceptions and definitions of the city accross Europe. Therefore, a consistent study of the coevolution between transportation and urban frame must rely on \emph{harmonized urban database}, for which contruction methods are proposed, based on an ontology of the city on large time scales. These database finally allow to understand that major transitions in city systems occured simultaneously with technological innovations in transportation. Evidences for coevolution and mutual adaptation processes are extracted from the study of accessibility and network shape measures.



\section{Linear Reading}


\subsection*{Introduction}


Assumption of structuring effect of intra- and inter-urban transportation network~\cite{bavoux2005geographie}. \textit{Rq :} Opposed to \emph{the myth of structuring effects}~\cite{offner1993effets}. -- precised here that long time-scale are considered, thus local approximation are made. One can not demonstrate the direct impact of a new line on the economy or urban development because feedbacks do not occur at short term, and effects may appear ``random'' at non-appropriate time scales. \textbf{Q : } references verifying this fact ? $\rightarrow$ Good question, that should be tested statistically and through \emph{modeling} : good for our purpose.

Urban hierarchy is remarquably stable in mong time, whereas accessibility patterns radically changed, thus the central question of this work, which is the exploration of links between the evolution of urban hierarchy and the stucture of exchange networks.






%%%%%%%%%%%%%%%%%%%%
%% Biblio
%%%%%%%%%%%%%%%%%%%%

\bibliographystyle{apalike}
\bibliography{/Users/Juste/Documents/ComplexSystems/CityNetwork/Biblio/Bibtex/CityNetwork}


\end{document}
