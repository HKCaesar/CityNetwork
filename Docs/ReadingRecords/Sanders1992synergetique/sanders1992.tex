%%%%%%%%%%%%%%%%%%%%%%%%%%%%%
% Standard header for working papers
%
% WPHeader.tex
%
%%%%%%%%%%%%%%%%%%%%%%%%%%%%%

\documentclass[11pt]{article}

%%%%%%%%%%%%%%%%%%%%
%% Include general header where common packages are defined
%%%%%%%%%%%%%%%%%%%%

% general packages without options
\usepackage{amsmath,amssymb,bbm}




%%%%%%%%%%%%%%%%%%%%
%% Idem general commands
%%%%%%%%%%%%%%%%%%%%
%% Commands

\newcommand{\noun}[1]{\textsc{#1}}


%% Math

% Operators
\DeclareMathOperator{\Cov}{Cov}
\DeclareMathOperator{\Var}{Var}
\DeclareMathOperator{\E}{\mathbb{E}}
\DeclareMathOperator{\Proba}{\mathbb{P}}

\newcommand{\Covb}[2]{\ensuremath{\Cov\!\left[#1,#2\right]}}
\newcommand{\Eb}[1]{\ensuremath{\E\!\left[#1\right]}}
\newcommand{\Pb}[1]{\ensuremath{\Proba\!\left[#1\right]}}
\newcommand{\Varb}[1]{\ensuremath{\Var\!\left[#1\right]}}

% norm
\newcommand{\norm}[1]{\| #1 \|}


% amsthm environments
\newtheorem{definition}{Definition}



%% graphics

% renew graphics command for relative path providment only ?
%\renewcommand{\includegraphics[]{}}








% geometry
\usepackage[margin=2cm]{geometry}

% layout : use fancyhdr package
\usepackage{fancyhdr}
\pagestyle{fancy}

\makeatletter

\renewcommand{\headrulewidth}{0.4pt}
\renewcommand{\footrulewidth}{0.4pt}
%\fancyhead[RO,RE]{\textit{Working Paper}}
\fancyhead[RO,RE]{\textit{ECTQG 2015}}
%\fancyhead[LO,LE]{G{\'e}ographie-Cit{\'e}s/LVMT}
\fancyhead[LO,LE]{An Algorithmic Systematic Review}
\fancyfoot[RO,RE] {\thepage}
\fancyfoot[LO,LE] {\noun{J. Raimbault}}
\fancyfoot[CO,CE] {}

\makeatother


%%%%%%%%%%%%%%%%%%%%%
%% Begin doc
%%%%%%%%%%%%%%%%%%%%%

\begin{document}







\title{Reading Record\bigskip\\
\cite{sanders1992systeme}
}
\author{\noun{Juste Raimbault}}
\date{22th December 2015}


\maketitle

\textbf{\textit{Reading Record for \cite{sanders1992systeme}}}



\section{Introduction}

Synergetics introduced for the study of physical complex systems by Haken around 1984.

\textit{TODO : detail Haken approach}

$\rightarrow$ particular application to systems of cities ; part of a larger corpus of theoretical and quantitative geography developed since ? at Geocites.




\section{Linear Reading}

\subsection*{Introduction}

Particularity of systems of cities : exchange networks and strong interdependencies. ``Breakdowns and continuity''. Since beginning of 19th, diffusion of social and technical innovations ; stable and auto-reproducing system.

Regularities and fluctuations in such systems. Studied period : 1954-1982.

Concepts of complex systems : non-linearity, centrality of interactions, self-organization far from equilibrium (SOC ?), determinism and stochasticity. Not only conceptual parallelism, but demonstration of application.

\subsection{The system of Cities}

The city as a spatial entity, coarse-graining at this level. Delimitation of the system ? technical constraints more than theoretical.

\subsubsection{City size dynamics}

\paragraph{Theories of city growth}

Classical models ; growth by steps (specialization then diversification) ; agglomeration economies (in both sense, with negative feedbacks)

\paragraph{Evolution of the system of cities}

More qualitative (geographical ?) approaches : cf. Berry~\cite{berry1964cities} ; Pumain. For French Urban System : before 19th, quite independent evolutions (Gibrat ?), later ``temporal autocorrelation of growth rates''. Thesis of Guerin-Pace : coupling macro (stable, trend) with micro fluctuations.

\subsubsection{Cities and cycles}

Urban life cycles : more for intra-urban characteristics. At a greater scale : cycles of innovation ; link with role of technical innovations in Schumpeterian theory. Economic cycles have deeply shaped French urban system. rq: product cycles more localized and precise than economic cycles, more complex to analyze.

\textbf{Combination of two diffusion processes :}
\begin{itemize}
\item spatial diffusion (core-periphery)
\item hierarchical diffusion (schematically, more refined diffusion at different level occurs in reality)
\end{itemize}


\subsubsection{Innovation cycles and evolution of the system of cities}

Understanding link between relative growth and diffusion of innovation is complex. Interferences between cycles, spatial and temporal. ``Multi-dimensional diffusion analysis''.

Various scales can be taken :
\begin{itemize}
\item scale of the process
\item scale of the firm
\item scale of city evolution, innovation as a driver of urban growth
\end{itemize}

\paragraph{Long-time approach}

Marchetti technological substitution model : $f\in [0,1]$ fraction of city in urban system, then $\log{\frac{f}{1-f}} = a\cdot t + b \implies f = \frac{e^{at+b}}{1+e^{at+b}} = \frac{e^{\frac{t-t_0}{\tau}}}{1+e^{\frac{t-t_0}{\tau}}}$. $\rightarrow$ different growth factors for different spatial entities ? fraction is then normalized fraction. PB : assumes stationarity of processes ? or assumed $\tau (t)$ with $\tau \sim_{- \infty} \tau_{-} > 0 $ and $\tau \sim_{\infty} \tau_{+} < 0 $. [quite strange here]. Idea : Growth/decline cycles.


\paragraph{Short and Middle temporal scales}

more precise data at these scales. tertiary/secondary substitution, position of agglomeration on a logistic curve. (cf Marchetti model ?). 

ex tertiarization french agglos : variety of situations. approach however too general ? (sector, etc particularities). typology by sectors : constant repartition between cities ; homogeneisation ; growth and reinforcment of inequalities.


\subsubsection{Dynamical models applied to urban systems evolution [dynamics ?]}

Bruxelles, Prigogine and Synergetics, Haken.

particularity of models :
\begin{itemize}
\item designed for complex systems ; spatio-temporal
\item differential equations : continuity of urban change
\item non-linear equations, taking feedbacks into account
\item classical assumptions can easily be integrated
\item out-of equilibrium, non-unicity of eq. ; ``evolve in time'' : non-stationarity ?
\item qualitative bifurcations possible because of equilibrium multiplicity
\end{itemize}


With $\mathbf{P}$ populations, $\mathbf{X}$ state variables, $\mathbf{\mu}$ parameters, most general equation is

\[
\frac{\partial \mathbf{P}}{\partial t} = \mathbf{F}(\mathbf{\mu},\mathbf{X},\mathbf{P})
\]

\textit{here cross terms between state variables vanish, we have with less generality,} $\frac{\partial P_i}{\partial t} = F_i(\mu, \mathbf{X}_i,\mathbf{P})$.

various formulations have been proposed ; models not tractable analytically if take multiplicity of interactions into account.

Noise term $\varepsilon$ can be added. example of ``deterministic'' trend plus noise curve.

Other formulation : Master equation approach, from synergetics.


\subsection{Multiscalarity : from individuals to city systems}

Emergence of city dynamics from its micro components.

Two visions :
\begin{itemize}
\item micro description of behavior
\item statistical distribution
\end{itemize}

what about coupling both ? example of migration process : discriminating character is crucial.

\subsubsection{Difficulty of migration models typology}

Variety of applications (explanation of behavior, flows distribution, planning, embedding into larger model) ; choice of explanatory variables depends on application ; variety of formalizations (from Markov chains to econometric, log-linear, micro-utility, gravitation, etc) ; different role of time : all these contribute to difficulty of typology.

\subsubsection{Individual behavior : micro-geographical models}

log-linear models : explain the indicator of move for one household.

Logit and Nested logit models. [// discrete choices]

examples and biblio.

\subsubsection{Meso and Macro scales}

Meso : flows between entities ; Macro : global organisation of system.

\paragraph{Econometric models}

explanatory vars ; utility etc : works at an interregional level.

\paragraph{Spatial interaction models : from gravitation to entropy maximization}

importance of distance for interactions. 

Most general gravitation model : $M_{ij} = k P_i P_j \exp{(-\frac{d_{ij}}{d_0})}(1+\alpha c_{ij})$ (or power law instead of exponential), where $c_ij$ captures ratio surfaces / common frontier (to include travel possibility). Rq : thematic based ok, but could fit anything ? pb of equifinality again.

Wilson model, based on entropy maximisation, generalizes gravitation model.

Tobler : not explain migrations but include them accurately.

ex application for France.


\paragraph{More general models}

Lowry : gravitation plus incomes and unemployment.

Batty 1983 dynamic model of simulation.

\subsubsection{Attempts for micro-macro integrations}

Different variables ; temporal scales. According to Weidlich, reductionism vs holistic approach - close to autonomy of weak emergence.

Probabilistic interpretation of gravitation model, extension with conditional probas.

Leeds : synthetic population from census data, simulation : too heavy computationnaly [rq : different today ? cf all city systems simulations ?] : top-down approach.

vs bottom-up : use micro-geo data. example of difference between expectations and reality for individuals. Other study : importance of macro factors.

$\rightarrow$ various scales, sometimes contradictory results ; logic for each scale and difficulty of explicit link between scales.


\subsection{Answers from Synergetics}

Interdependance and cooperation between subsystems : auto-organisation.

Typical characteristics of complex systems (recall from dyn. models ?)

\begin{itemize}
\item ``Hierarchy between scales, each level of aggregation is well defined, perceptible as an entity at a given scale''. \textbf{RQ : very close to ontological decomposition, furthermore here detailed as ``perceptible''} $\rightarrow$ \textbf{TO BE DETAILED}
\item complex interactions, non-linearity between elements at a given level, can influence upper level (RQ : links \emph{between} levels ?)
\item Combination of deterministic trends and stochastic fluctuations.
\item Possibility of bifurcation because of non-linearity (oscillating or chaotic behaviors) - ex. phase transition.
\end{itemize}


For city systems : behavior of macro depending on micro constituents ? Weidlich and Haag have first used synergetics to study the evolution of an urban system.

\subsubsection{Master Equations system}

$\mathbf{n}$ cities populations, $\sum_i n_i (t) = N(t)$, evolution of $\Pb{\mathbf{n},t}$ ? Conservation of probas. Master equation : describes dynamics $\frac{\partial \Pb{\mathbf{n},t}}{\partial t}$.

Integration of migration process : exchanges with external world ; exchanges between cities of the system.

Derivation : proba of jump $\mathbf{n}(t) \rightarrow \mathbf{n}(t + dt)$ at micro level, developed at the first order, so-called ``individual transition rate'' is derivative ; at level of city ``configurational transition rate'' - both are linked simply.

$\rightarrow$ Master eq. with transition matrix, classic definition.

\[
\frac{\partial \Pb{\mathbf{n},t}}{\partial t} = \sum_{ij}{w_{ij}(n'_{ij}) \Pb{n'_{ij},t}} - \sum_{ij}{w_{ij}(\mathbf{n}\Pb{\mathbf{n},t}}
\]


\subsubsection{From master equation to mean values}


expectancy taken in master eq to have deterministic eqs., because single realization observed. \textbf{RQ : loose interest of master eq, take only mean on all possible configurations. ISSUE as indeed linear aggregation : ignore possible emergence from stochastic fluctuations and chaotic } $\rightarrow$ chapter VI details implications of that assumption, which seems reasonable for migratory flows at interregional and interurban scales. RQ : means that the realized config. is the most probable $\sim$ the mean/median for a symmetric unimodal distribution. If everything is gaussian, works ? ok but does not allows for fat tails/amplification by chaos ? at this level of aggregation linearity for aggregation is acceptable ? then what purpose in detailing master eq. ? [$\ldots$ ??]

Need of specification for individual rates.

\subsubsection{Individual transition rates}

$p_{ij}(\mathbf{n})$ migration between $i$ and $j$ for global state $\mathbf{n}$, proposed as

\[
p_{ij}(t) = v_{ij}(t)\exp{(A_i(t) - A_j(t))}
\]

where $v_{ij}$ is a symmetrical component, capturing proximity between the two cities, assumed to have a canonical decomposition as $v_{ij}(t) = <v_{ij}(t)>\cdot f_{ij}$ (independence of temporal evolution and exchange potential) ; and $A_i$ measures city ``attractivity''.

injection into aggregated master eq. : adjustment of $L(L-1)/2 + T(L+1)$ parameters ($L$ number of cities).

originality of model : independence between model construction and model interpretation/generalization : test of hypotheses : allows both operational character and theoretical investigations. Does not presuppose on ``causes'' of processes, but reproduces flows and allows to identify a posteriori these.




\subsection{Application to French Urban System Dynamics}

studies for 1954-1982 differentials of population induced by migrations.

\subsubsection{Database}

Population, socio-economic variables, flow matrices. Issues : MAUP in time, evolving activities typology.

\paragraph{Population}

delimitation of agglomerations to be carefully tackled.

\paragraph{Migration flows matrices}

reconstructed from residential mobility from census data. contraint of data availability : 78 cities only.

\paragraph{Socio-economic}

actives per CSP and category of economic activity. wealth/unemployment not used as incomplete in time. csp simplified.

\subsubsection{Evolution of interurban mobility}

computation of $<v_{ij}(t)>$ at 4 dates (54-62 : 0.54 ; 62-68 : 0.82 ; 68-75 : 1.23 ; 75-82 : 1.01).

\subsubsection{Proximity between cities : role of distance}

many formulations : exponential, Pareto (inverse power law), modified exponential $f_{ij}^{(0)} = a\exp{(-b\cdot d_{ij}/(1+c d_{ij}))}$ ; third explains better [RQ : because of supplementary degree of freedom ?] - big cities always underestimated.

solution : add size influence by $f_{ij} = \left(\frac{n_i}{N}\cdot\frac{n_j}{N}\right)^{\alpha} f_{ij}^{(0)}$. RQ : why not a more general function $g\left(\frac{n_i}{N},\frac{n_j}{N}\right)$, ex different elasticities : add more parameters so should fit better, maybe add small number ?

\textbf{TODO : link with generic machine learning formulation/approach}

20\% of supplementary explained variance, still deviations from expected behavior for couple of cities.

from 3003 parameters to only 2,3,4 ; but operationally not convenient, as ``analytical'' values are more precise than geographical derivation -- RQ : confusion on the sense of analytical ?

\subsubsection{Role of hierarchy in agglomeration attractiveness}

clear non-linear relation between size and attractiveness. RQ : not well detailed how it is defined/computed : seems to be from raw data by

\[
A_i(t) - A_j(t) = \log{\frac{M_{ij}}{v_{ij}(t)}}
\]

$\rightarrow$ system is not determined, need to fix a baseline $A_{i_0} = A_0$ to obtain all $A_i$, i.e. they are not determined given an additive constant ? but then consistency pb ? matrix must be antisymmetric !

Then specify a linear model to estimate size positive (linear) and negative effects (square)

\[
A_i(t) = K\cdot n_i(t) - \sigma\cdot n_i(t)^2 + \delta_i(t)
\]

residuals are called ``preference'', capture intrinsic city attractiveness. so model captures both size effects (positive and negative) and intrinsic attractiveness.

applied to each 4 periods ; maps and regression results.

66\% of differentiations explained. Paris is most impacted by negative effects. sort of optimal size should exist for that model. \textbf{Q : } as estimated from real migrations, could be not only undesirable effects of size, but also saturation of dwellings/employment = saturation of the agglo ?

size threshold under which size is not significant ; increase in time (e.g. 125300 in 1954 to 210000 in 1975). agglomeration effects favors bigger and bigger cities and threshold at which saturation appears also increases. Paris still most attractive despite negative effects.


\subsubsection{Preference indices}

old industrial regions : switch in trend, from positive to negative ; coastal and alpine cities on the contrary gain preference // third industrial revolution.

regression of preference against socio-eco vars : typology of profiles.


\subsection{Attractiveness, Preferences and socio-economic profiles}

stylized facts : city baseline profile and tertiary preferential attachment (concentration) after 1968. best way to integrate into models ? typical trajectories ? substitution effects ?

\subsubsection{Interurban differentiation}

pca to isolate main drivers of interurban differentiation. strong stability in profiles.

\paragraph{Inequalities between cities : primary driver}

tertiarisation level very important, correlates highly with socio-economic composition. some specificities : mediteranean/atlantic.

\paragraph{Secondary differentiations}

e.g. type of service ; instable in time (whereas for CSP pca, relatively stable in time).

correlations attractiveness and preference to socio-eco vars, for each period.
$\rightarrow$ quite correlated when significant.

\subsubsection{Trajectories of cities in socio-eco space}

three bases of projection for different point of views for comparison.

\paragraph{Trajectories}

stability of secondary/tertiary ratio, illustrates social segregation ; general trend with various cycles for small modifications of a given sector for example.

typology in 5 classes on stable components :
\begin{itemize}
\item high proportion of workers (North and North-East)
\item lot of workers (different transition)
\item middle profile
\item independent professions (South and West)
\item ``Framing'' cities
\end{itemize}

also typology depending on dynamical profile : breakdown, slow change, dynamical, late but accelerating growth ; no link with attractiveness/preference.

\paragraph{Economics trajectories}

more regularity. typology according to economic profile ; idem dynamical typology.

Conclusions : good correspondance between economic and social ; however, economic more homogeneously distributed than social ; and particular cases.

\subsubsection{Rythm of economic change}

def of variation rates.

irregular and cyclic evolution. interurban balancing processes VS concentration and accentuation of differences.

interferences of various cycles of innovation.

description of economic differentiation for each period.

in multi-dimensional space of pca : cycles revealed by positive/negative coordinates.

Conclusion of chapter : analyses reveal stability of system, global trend, auto-reproduction ; but also short time fluctuations, various differentiation rythmes. (both included in synergetics).



\subsection{Equilibrium States}


condition for eq. : $d\mathbf{n} = 0$. stationary state for flows : equilibrium.

\subsubsection{Theory}

\paragraph{Stability}

stable and instable equilibria.

trajectories around eq. : phase space representation. oscillations around attractor (Lotka-Volterra e.g.), or straigth convergence (resp. divergence) (logistic evol. e.g.). Limit cycle. example birth rates in Sweden.

Stable/chaotic behavior. strange attractor.

Bifurcation : attractor of different nature leads to qualitative bifurcations // link to aggregation in master eq : bimodal distrib correspond to two attractors.

bifurcation in individual very fast ; for cities $\sim$ 30 years ; system never observed (Pumain : rank-size relation as a very stable attractor).

well introduced in model.

\subsubsection{Stationary state for French cities}

Population given by $dn_i(t) = 0$. Unique solution for migration trends.

illustration of two stationary states (first and last period).

typology according to distance to stationary state.

stabilisation time of 1000years, only virtual state.

\subsection{Future scenarii for the urban system}

exploratory model ; can be used for prediction.

\subsubsection{Four scenarios}

\begin{itemize}
\item Hierarchy only : only size effects taken into account.
\item Most recent trend : preferences added as constants.
\item Preferences as function of economical dynamism : application of regression model. Emphasize role of innovation.
\item Frontier effect after European market : increase preference of cities less than 60km from the frontier. 
\end{itemize}

evaluation : stability of hierarchy ; trajectories of cities.

\subsubsection{Application}

Hierarchy stable along scenarii.

Scenario A : Big cities advantage.

Scenario B : growth of south cities.

Scenario C : role of innovation.

Scenario D : reaction of system to external perturbation.

\subsubsection{Particular trajectories}

Cities that always decline ; cities growing ; more complex trajectories ; Paris and Lyon strong variation e.g.


\subsection*{Conclusion}


Synergetics as conceptual framework to study urban systems dynamics. Model of Allen. Model of Weidlich and Haag : not explicative, not evolutive.

\textbf{TODO : find } \cite{allen1991evolutionary}

role of socio-eco in attractiveness, cycles of innovation, interferences. 

Towards a more general model coupling population, economic, social. pb of endogeneous/exogeneous character ? need of empirical knowledge : intensity and length of innovation cycles.

conceptual and applied model.

to be developed : empirical knowledge, multiscalarity, more urban theory. Importance for planning (counter-intuitive effects), global view necessary.

\subsection*{PCA}

description of pca.

\section{Synthesis}

rq : does not really apply synergetics.

\textit{TBW}



%%%%%%%%%%%%%%%%%%%%
%% Biblio
%%%%%%%%%%%%%%%%%%%%

\bibliographystyle{apalike}
\bibliography{/Users/Juste/Documents/ComplexSystems/CityNetwork/Biblio/Bibtex/CityNetwork,biblio}


\end{document}

