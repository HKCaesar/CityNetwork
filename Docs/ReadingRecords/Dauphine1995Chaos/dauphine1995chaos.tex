%%%%%%%%%%%%%%%%%%%%%%%%%%%%%
% Standard header for working papers
%
% WPHeader.tex
%
%%%%%%%%%%%%%%%%%%%%%%%%%%%%%

\documentclass[11pt]{article}

%%%%%%%%%%%%%%%%%%%%
%% Include general header where common packages are defined
%%%%%%%%%%%%%%%%%%%%

% general packages without options
\usepackage{amsmath,amssymb,bbm}




%%%%%%%%%%%%%%%%%%%%
%% Idem general commands
%%%%%%%%%%%%%%%%%%%%
%% Commands

\newcommand{\noun}[1]{\textsc{#1}}


%% Math

% Operators
\DeclareMathOperator{\Cov}{Cov}
\DeclareMathOperator{\Var}{Var}
\DeclareMathOperator{\E}{\mathbb{E}}
\DeclareMathOperator{\Proba}{\mathbb{P}}

\newcommand{\Covb}[2]{\ensuremath{\Cov\!\left[#1,#2\right]}}
\newcommand{\Eb}[1]{\ensuremath{\E\!\left[#1\right]}}
\newcommand{\Pb}[1]{\ensuremath{\Proba\!\left[#1\right]}}
\newcommand{\Varb}[1]{\ensuremath{\Var\!\left[#1\right]}}

% norm
\newcommand{\norm}[1]{\| #1 \|}


% amsthm environments
\newtheorem{definition}{Definition}



%% graphics

% renew graphics command for relative path providment only ?
%\renewcommand{\includegraphics[]{}}








% geometry
\usepackage[margin=2cm]{geometry}

% layout : use fancyhdr package
\usepackage{fancyhdr}
\pagestyle{fancy}

\makeatletter

\renewcommand{\headrulewidth}{0.4pt}
\renewcommand{\footrulewidth}{0.4pt}
%\fancyhead[RO,RE]{\textit{Working Paper}}
\fancyhead[RO,RE]{\textit{ECTQG 2015}}
%\fancyhead[LO,LE]{G{\'e}ographie-Cit{\'e}s/LVMT}
\fancyhead[LO,LE]{An Algorithmic Systematic Review}
\fancyfoot[RO,RE] {\thepage}
\fancyfoot[LO,LE] {\noun{J. Raimbault}}
\fancyfoot[CO,CE] {}

\makeatother


%%%%%%%%%%%%%%%%%%%%%
%% Begin doc
%%%%%%%%%%%%%%%%%%%%%

\begin{document}







\title{Reading Record\bigskip\\
\cite{dauphine1995chaos}
}
\author{\noun{Juste Raimbault}}
\date{Date}


\maketitle

\textbf{\textit{Reading Record for \cite{dauphine1995chaos}}}



\section{Linear Reading}

\subsection*{Introduction}

Semi-stationarity of geographical systems. New thematics : non-linear dynmics, chaos, fractals. Logical theoretical compatibility.

\subsection{New principles}

\subsubsection{Out of reductionism}

Burgess and Von Thunen models are reductionist.

Systemic approach : Forrester. Also multivariate statistics. Pb : simplify reality. simulation of complex phenomena from simple rules (cf Prigogine). simple example of Julia sample.

\subsubsection{Non-predictible determinism}

classical determinism : random fluctuations due to unobserved variable. But chaos exists : cf Lorenz attractor. Henon attractor : strange attractor.

non-linearity and sensitivity to initial conditions. feedbacks and self-regulations are non-linear. necessary (but not sufficient) to have chaotic behavior. sensitivity to initial conditions : Lorenz ; billard dynamics. René Thom : deterministic chaos is not disorder.

\subsubsection{Non-linear dissipative systems yield fractal structures}

(back on fractal generation). scale invariance. multifractality : different levels in nature.

In dissipative systems, strange attractors are fractals. dissipative : far from equilibrium. Reciprocally, dynamical system non-linear yield fractal structures. chaos : dynamics ; fractals : form.

\subsubsection{From order to chaos and chaos to order}

Parametrized dynamics : from stability to chaos. bifurcation (Hopf bifurcation). Sub-harmonic cascade : progressi ve switch from equilibrium, to periodicity and chaos. Intermitent chaos. (economic crisis ? Floquet matrix). Quasi periodicity.

order to chaos : example KdV. (waves) solitons.

chaos and catastrophes. R Thom. exogenous bifurcation, contrary to chaos where it is endogeneous.

\subsection{Techniques}

\subsubsection{Deductives}

Equa dif ; iterative resolution. Visualisation of derivative field. : stable or not. Iterated maps : convergence in $(x_n,x_{n+1})$.

Phase space representations ; time-series ; spectrum for periodic systems.

Attractors in phase space.

Poincarré projection : projection of phase space in 2d.

Liapounov. classification of attractors depending on Liapounov values.

\subsubsection{Inductives}

Complex systems "random" and disordered -> ? . Separate deterministic chaos from random disorder. \textit{do not agree, complex system has not random but self-organized behavior. depends on scale. but yes great number of variables or parameters.}

Reconstruction of strange attractor : $x_{t+1} = f(x_{t})$. : phase space for unidimensional series. 

Random vs chaotic :  correlations between prediction. \textit{Wrong ? gives stationarity of random process}

Spectrum of chaos. 

dynamically finite generated systems (?)

\subsubsection{Fractal Dimensions}

Def of fractal dimension. ; concrete computation.

Other dimensions : Hausdorff, etc. Kolmogorov entropy.

Multifractals.

\subsection{Functional chaos}

\subsubsection{Physical geography}

ecological models : malthus. Verlhust with carrying capacity. Generalisation with chaos. (arima at diffeent orders). works for ecology (experimental cases) not for human populations. Chaos in epidemiology empirically proven : due to network propagation ?

Prey-predator models.

chaos in climatology : fractals in space and time ; chaotic dynamics of atmosphere.

Physical geography : terrestrial shapes.

\subsubsection{Human geography}

non-linearity of economic time-series. Growth model with different scales and agents (Day).

Logistic map : 49 different behaviors.

example of eco-energetic model : PACA. simulation of catastrophes, chaos.

activation-inhibition : biology.

\subsection{Chaos and Geographical fractals}

\subsubsection{Diffusion}

back to Hägerstrand. spatial diffusion

Linear, continuous, hierarchical, random diffusions.

Macro and micro diffusion : brownian motion. Refined models.

Macrodiffusion : Partial diff eq. (Fick, Fourier, etc). Ohm model for migratory flows.

\subsubsection{Various models}

DLA and DBM. DBM better (Batty). city growth and network growth. Zipf law.

Morphogenesis model : polarisation and aggregation.

Percolation models. explosive percolation. diffusion of innovations.

CA : ex Game of Life.

Macrodiffusion : Hotelling. Meinhardt : polarized structures (physical geography).

\subsection*{Conclusion}

Pertinence of geo knowledge ? gravitation models etc have given insight.

Lack of theoretical support in geographical knowledge : advantage of chaos and fractals. Qualitative argument invalid.

chaos : non-linearity, non-predictalibility.









%%%%%%%%%%%%%%%%%%%%
%% Biblio
%%%%%%%%%%%%%%%%%%%%

\bibliographystyle{apalike}
\bibliography{/Users/Juste/Documents/ComplexSystems/CityNetwork/Biblio/Bibtex/CityNetwork}


\end{document}
