%%%%%%%%%%%%%%%%%%%%%%%%%%%%%
% Standard header for working papers
%
% WPHeader.tex
%
%%%%%%%%%%%%%%%%%%%%%%%%%%%%%

\documentclass[11pt]{article}

%%%%%%%%%%%%%%%%%%%%
%% Include general header where common packages are defined
%%%%%%%%%%%%%%%%%%%%

% general packages without options
\usepackage{amsmath,amssymb,bbm}




%%%%%%%%%%%%%%%%%%%%
%% Idem general commands
%%%%%%%%%%%%%%%%%%%%
%% Commands

\newcommand{\noun}[1]{\textsc{#1}}


%% Math

% Operators
\DeclareMathOperator{\Cov}{Cov}
\DeclareMathOperator{\Var}{Var}
\DeclareMathOperator{\E}{\mathbb{E}}
\DeclareMathOperator{\Proba}{\mathbb{P}}

\newcommand{\Covb}[2]{\ensuremath{\Cov\!\left[#1,#2\right]}}
\newcommand{\Eb}[1]{\ensuremath{\E\!\left[#1\right]}}
\newcommand{\Pb}[1]{\ensuremath{\Proba\!\left[#1\right]}}
\newcommand{\Varb}[1]{\ensuremath{\Var\!\left[#1\right]}}

% norm
\newcommand{\norm}[1]{\| #1 \|}


% amsthm environments
\newtheorem{definition}{Definition}



%% graphics

% renew graphics command for relative path providment only ?
%\renewcommand{\includegraphics[]{}}








% geometry
\usepackage[margin=2cm]{geometry}

% layout : use fancyhdr package
\usepackage{fancyhdr}
\pagestyle{fancy}

\makeatletter

\renewcommand{\headrulewidth}{0.4pt}
\renewcommand{\footrulewidth}{0.4pt}
%\fancyhead[RO,RE]{\textit{Working Paper}}
\fancyhead[RO,RE]{\textit{ECTQG 2015}}
%\fancyhead[LO,LE]{G{\'e}ographie-Cit{\'e}s/LVMT}
\fancyhead[LO,LE]{An Algorithmic Systematic Review}
\fancyfoot[RO,RE] {\thepage}
\fancyfoot[LO,LE] {\noun{J. Raimbault}}
\fancyfoot[CO,CE] {}

\makeatother


%%%%%%%%%%%%%%%%%%%%%
%% Begin doc
%%%%%%%%%%%%%%%%%%%%%

\begin{document}







\title{Reading Record\bigskip\\
\cite{offner1996reseaux}
}
\author{\noun{Juste Raimbault}}
\date{Date}


\maketitle

\textbf{\textit{Reading Record for \cite{offner1996reseaux}}}


\section{Introduction}

\textit{Preface by Raffestin : }

Network and Territories : two notions that can represent both geographical materiality and theoretical domain.

Crucial notion : articulation space-network-territory. Need for a theory, some ``laws of progressive composition''.

Sketch of a theory (from Nicolas Curien) : territoriality = projection of a system of intentions into space ; systems of intention are expressed within material and social networks. Successive temporal imbrications [dynamical causalities ? ]. Importance of power in the process (network operators) : role of governance. 

Role of information : modification of mobility but no decrease. Role of social networks and company networks. Network of cities : also conceptual indetermination.


\section{Linear Reading}

\subsection{Introduction}

Various questions from crossing of different disciplines : do networks contribute to territorial fabric ? correspondance between network and territory types ? Deconstruction of territories by increasing networking ?

$\rightarrow$ different significations and notions across disciplines.

First technical networks : materialisation ; II : network operators ; III : communication networks and IV social networks ; V and VI : companies and city networks.  \textit{Overview of semantic variations linking networks and territories}

\subsection{Technical Networks}

\subsubsection*{A contradictory discourse}

Functionality of technical networks. Simple to define but complex to qualify : various agents involved.

Structuration and destructuration of territories : cf Curien. Three layers in network : support (infrastructure), service and information.

\subsubsection*{Geographical Scales of Technical Networks}

Dupuy : technical network as a functional entity of territorial systems. 

\paragraph{Morphogenesis of Technical Networks}

Morphogenesis = transformation of networks. Material complementarity in first phase of network development, more free later (territorial coverage). No general rule for territorial limits or contribution to hierarchy of nodes.

\paragraph{Scale formatting, or connection against boundaries}

Historical role of networks in structuration of national spaces. Dialogue between scales of network and territories. Network fixes reference scale. Territories disappear while other are born ; homogenization only in rare cases of a covering network. Technical network contribute to scale definition. Correspondance between \textbf{Institutional territory} and \textbf{functional space} not necessarily direct.

\paragraph{Interconnections and scale skipping}

Example of RERA : difficult to define a reference scale (multiplicity of usages : \emph{multi-scale integration} ?)

$\rightarrow$ so-called scale skipping, not strictly hierarchical interaction between scales. (// rq : different to hierarchy in more simple CAS ? ). phenomenon induced by network interconnection.

In this multi-scale context, network becomes \textit{relay} between territories at different levels. Furthermore, no more correspondance between network scale and territory scale. Not to be confused to a ``disparition'' of scales.

\paragraph{Network vertices, between nodes and poles}

Importance of network vertices. Increasing role because of evolution of network structure (hub and spokes). Nodes becomes a ``pole'', structuring space around it. Difficult question : what is required for a vertex to become a pole (in many complementary ways).

\subsubsection*{Network Material Insertion : Street Network example}

Definition of \textit{voirie} (no english term) : public domain.

\paragraph{Spaces and flows}

Different public administrators ; collective functions. role of public space : more than support (transportation) and services. \textit{Genie Urbain} : ensures multi-functionality of public space.

\paragraph{Conflicts of use and heritage management}

Conflicts between actors of \textit{amenagement}, accentuated by variety of scales.

\textit{Voirie} as a concretization of interactions between territories and networks : has characteristics of an heritage.

\textit{Voirie} as road network participates to construction of individual territories, but \textit{voirie} as territory on which administrator can control uses.


\subsubsection*{Transportation Networks, Cornerstones for Territorial Dynamics}

Transportation Network can be infrastructure or services using infrastructure.

\paragraph{Effects on accessibility}

Accessibility : link between places and resistance to exchange flows (as a function of distance, time, cost).

Transportation Network as links and vertices, through which one accesses the service.

Transportation Network Morphogenesis : decision processes of implantation and local realization. Morphogenesis is decision processes of construction and localization.

Morphology of transportation network ensures more or less territorial connectivity.

Evolution in time of the role of infrastructures : apparition of ``tunnel effect'' (Bonnafous). High speed networks (rail and motorways).

Different homogeneity and isotropy parameters depending on network type (ex : road nw vs TGV) ; loss of spatial equality through tunnel effect. Hubs and interconnection : increase of polarisation effects. Impact of TGV on importance of nodes of the old network. Also impact on local reorganisation of urban functions.

\paragraph{Socio-economic impact of transportation networks}

Direct rapid effects : choices of mode, mobility modifications, local economy. Long term : indirect effects (structuring effects).

Difficult question of structuring effects : no determinism. Defendors of this retoric base their discourse on technical determinism, mechanical metaphor, pure rationality economic model, positivism. Position linked to economical context ; in stagnation apparent causality does not exist anymore. Transportation networks as product and support of socio-economic activity. TGV can have very different effects : Le Creusot vs TGV Atlantique.

Socio-economic effects of transportation networks translate territorial dynamics. Congruence of both evolution at a macro-economic level (offer of innovation satisfy needs generated by socio-economic evolution). Innovation are locally appropriated, yielding local impact ; linked to representation of networks: opening, power, efficacity. Local adaptation can take the form of opportunities or resistance (nimby). Strategy of local and global actors are determining and can be illustrated in different ways.

\paragraph{High Speed Transportation}

Territories become closer ; spatial divisino of work increases, increasing mobility and concentration of transportation networks in high population density areas. Issue of too strong differentiation of territories, condemning some. Coordination problems between institutional territories yield difficulties for solidarity between territories.


\subsubsection*{Temporalities}

\begin{itemize}
\item Very different temporalities in interaction between networks and territories
\item Geographical scales as object of study and instruments
\item potentialities, opportunities and strategies
\item other approaches than dynamical : network as public territory
\item relations between different types of networks : operators as mediators ?
\end{itemize}



\subsection{Network Operators}

\subsubsection*{From Infrastructure Management to Territorial Service Offer}


def of network operators depends on how def of network is broad : can be manager, but also a set of actors in the case of a network of transactional projects. Operator is necessarily territorial.

Territory following Roncayolo : belonging, power, globality, representations. how do operator change different dimensions ?

\subsubsection*{Technical networks : expression of a system of actions}

Insertion of networks in institutional territories is understood through specification of relations between actors.

Public service : unique and monopolistic exploitant.

\paragraph{French specificity}

in France, cooperation between public and private sectors ; equilibrium between productive and political functions.

Historically, appears in technical transfer for local public service ; later regulated and nationalized.

different evolution of local and centralized, local closer to users.

Deregulation against monopolistic situations.

Network operator is the key for public service, has a composite role : technical, economic and juridicial.

\paragraph{Scales of territorial regulation of networks}

depending on scale, very different role of regulation and relation between operator and political.

French regions :local scale. deterritorialisation because of high speed network. consistence of intermediate scale ?

european networks : greater level of governance.

\paragraph{Network management, spatial and territorial implications}

manager change space through technical infrastucture ; and socio-economic relations through offer ; creates territory.


\paragraph{Different roles of network manager}

$\rightarrow$ technical, commercial, economic, investor, etc. - in relation for each with institutional territory and social territory

\paragraph{Morphogenesis and Extension}

\textbf{Morphogenesis of networks : extension and space, beyond territorial boundaries}.

Network externalities : production costs and scale economies depends on which network.

Functional space of nw ; its functionalities are essential in its morphogenesis.

\paragraph{Equipements and Functional Logics}

Territory of operator : depends on function.

adaptation of institutions to functional territories.

Dichotomy between functional and institutional territories.

\subsubsection*{Network Public Policy : Regulation and Opening}

\paragraph{Tarification}

Monopolistic public situation : budget equilibrium. segmentation of demand : adpated tarification.

no more spatial perequation with dynamic tarification. can be spatially dynamic (tel network)

\paragraph{Opening}

infrastructure and services coupled exploitation deregulated.

separation between infrastructures and services. base service and added value services.

intelligent demand : IT and electrical networks.

support network, command network and service network.

\subsubsection*{Utility and social norm} 

From offer to demand-based organisation of networks.

\textbf{Network create territory of flows} : not structuring effects but \textbf{congruence}.

nature of demand is important, sometimes difficult to fullfill as decisions taken by not concerned actors.


\subsection{Telecommunications Networks}

\subsubsection*{Telecommunication Networks and Territorial Production}

\paragraph{From the complex object}

Development of ICT. - linked with deregulation.

obstacle to convergence ? spatial equality. question of future usage.

\paragraph{to instrumentalized object}

can ITC networks shape territories ? problem of opposition between local actors and centralized.

\paragraph{and reconstructed territory}

further layer in territory ; economically real territories.

\subsubsection*{Space and Territory seen through telecomunication networks}

\paragraph{Emergence of real space-time}

Real time, instantaneity, induce spatial ubiquity. Economy submitted to real-time regime because of high-frequency finance. structuring effects of communication networks ?

\paragraph{Simplifying myths}

Spatial transparency : deterritorialisation ? not so easy as geographical space is not distance only.

innovation not linear nor systematic. cybernetics, technocracy and complexity (Morin) $\rightarrow$ technology can not be isolated. auto-organisation of living machines.

linearity of progress : beware to distinguish information from decision-making.


\paragraph{Territorial Production and human territorialities}

Raffestin : territorial as social production ; territorial system as set of power relations needed to construct the territory. territoriality os reappropriation of territory by populations.

\subsubsection*{State and Telecommunication networks}

ICT : new means for state to control, power, etc. historically not new. strategical resssource.

\paragraph{Time of Castles}

metaphor of centralized network star : center is the castle.

all telecom means under strict control of state. same as in other fields of infrastructure.

\paragraph{Time of opening}

deregulation : reevaluation of role of state ; increase of private needs. in France less deregulated than anglo-saxon countries.

new relations between actors : concurrence between state actors ; regions, datar, etc.

Political factor still present, not totally deregulated.

\subsubsection*{Telecommunication networks and economic activities localisation}

wrong visions of planners, simplified space.

\paragraph{Site localisation}

homogeneous offer, not difference for standard companies.

\paragraph{Spatial organisation of companies}

can favor delocalisations and spatial spread.

\paragraph{Working at distance}

concerns a few number of jobs - corresponds to imaginary of networks (myths).

\subsubsection*{Physical proximity in media ubiquity society}

No total technological determinism nor total social reproduction, but appropriation and production. $\rightarrow$ telecom as production tool ; innovation produces social.

physical proximity is essential ; closely interlaced with telecomunication.


\subsection{Social Networks and Territories}


\subsubsection*{Preliminary definitions and research question}

social network :relations. ; opposed to organisation as frame of exchanges.

territory : social production.

recent mutations may induce a replacement of territory by social network : in fact an other expression of the territory.

\subsubsection*{Social Networks and Territories : a difficult meeting}

\paragraph{Social networks : formal approach to social link}

back to Simmel, social circles.  network analysis. Granovetter : weak and strong links. importance of social significations ?

\paragraph{Space in network analysis}

transversality of network vision : works also with space. influence of space on social relation : ex city vs rural. bullshit maffesoli.

\paragraph{From place to social links}

interdisciplinary : geography not only to link territory and social networks.

Raffestin as entry ? anthropology to understand relation of individual and groups with space.

difficult : no integrated definition : or territory presupposed, or social network only. (// with transportation networks ?)

\subsubsection*{Social links and movement}


Defs : social networks : individual in interrelation ; territory : construction and appropriation.

\paragraph{Networks and social relations}

top-down social link neglected in network analysis.

typology of social link construction : modern (self-constructed) vs patrimonial (context)

\paragraph{Proximity links}

overdeterminated social relations because of geographical proximity. 

Territory is also an element of social proximity.

\paragraph{Link creates territory : spatial dispersion and social cohesion}

collective memory : identity construction.

typology : village community, diaspora, migrants.

\paragraph{Networks and territorial institutions}


institutionalisation of social links. diffusion of innovation when legitimized.

network dynamics coupled with territorial inscription.

network deploy social links, territorialized or not.


\subsubsection*{Between research of sense and efficacity}

efficacy of flow of ressources in networks. Epistemological opposition between networks and territory ? different temporalities. different regulation regimaes.

Territorial and network processes ; contradiction between institutional and functional.

new research perspectives.


\subsection{Networks and Companies} 




%%%%%%%%%%%%%%%%%%%%
%% Biblio
%%%%%%%%%%%%%%%%%%%%

\bibliographystyle{apalike}
\bibliography{/Users/Juste/Documents/ComplexSystems/CityNetwork/Biblio/Bibtex/CityNetwork}


\end{document}
