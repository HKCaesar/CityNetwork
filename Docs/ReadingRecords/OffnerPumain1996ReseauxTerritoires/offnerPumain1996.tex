%%%%%%%%%%%%%%%%%%%%%%%%%%%%%
% Standard header for working papers
%
% WPHeader.tex
%
%%%%%%%%%%%%%%%%%%%%%%%%%%%%%

\documentclass[11pt]{article}

%%%%%%%%%%%%%%%%%%%%
%% Include general header where common packages are defined
%%%%%%%%%%%%%%%%%%%%

% general packages without options
\usepackage{amsmath,amssymb,bbm}




%%%%%%%%%%%%%%%%%%%%
%% Idem general commands
%%%%%%%%%%%%%%%%%%%%
%% Commands

\newcommand{\noun}[1]{\textsc{#1}}


%% Math

% Operators
\DeclareMathOperator{\Cov}{Cov}
\DeclareMathOperator{\Var}{Var}
\DeclareMathOperator{\E}{\mathbb{E}}
\DeclareMathOperator{\Proba}{\mathbb{P}}

\newcommand{\Covb}[2]{\ensuremath{\Cov\!\left[#1,#2\right]}}
\newcommand{\Eb}[1]{\ensuremath{\E\!\left[#1\right]}}
\newcommand{\Pb}[1]{\ensuremath{\Proba\!\left[#1\right]}}
\newcommand{\Varb}[1]{\ensuremath{\Var\!\left[#1\right]}}

% norm
\newcommand{\norm}[1]{\| #1 \|}


% amsthm environments
\newtheorem{definition}{Definition}



%% graphics

% renew graphics command for relative path providment only ?
%\renewcommand{\includegraphics[]{}}








% geometry
\usepackage[margin=2cm]{geometry}

% layout : use fancyhdr package
\usepackage{fancyhdr}
\pagestyle{fancy}

\makeatletter

\renewcommand{\headrulewidth}{0.4pt}
\renewcommand{\footrulewidth}{0.4pt}
%\fancyhead[RO,RE]{\textit{Working Paper}}
\fancyhead[RO,RE]{\textit{ECTQG 2015}}
%\fancyhead[LO,LE]{G{\'e}ographie-Cit{\'e}s/LVMT}
\fancyhead[LO,LE]{An Algorithmic Systematic Review}
\fancyfoot[RO,RE] {\thepage}
\fancyfoot[LO,LE] {\noun{J. Raimbault}}
\fancyfoot[CO,CE] {}

\makeatother


%%%%%%%%%%%%%%%%%%%%%
%% Begin doc
%%%%%%%%%%%%%%%%%%%%%

\begin{document}







\title{Reading Record\bigskip\\
\cite{offner1996reseaux}
}
\author{\noun{Juste Raimbault}}
\date{Date}


\maketitle

\textbf{\textit{Reading Record for \cite{offner1996reseaux}}}


\section{Introduction}

\textit{Preface by Raffestin : }

Network and Territories : two notions that can represent both geographical materiality and theoretical domain.

Crucial notion : articulation space-network-territory. Need for a theory, some ``laws of progressive composition''.

Sketch of a theory (from Nicolas Curien) : territoriality = projection of a system of intentions into space ; systems of intention are expressed within material and social networks. Successive temporal imbrications [dynamical causalities ? ]. Importance of power in the process (network operators) : role of governance. 

Role of information : modification of mobility but no decrease. Role of social networks and company networks. Network of cities : also conceptual indetermination.


\section{Linear Reading}

\subsection{Introduction}

Various questions from crossing of different disciplines : do networks contribute to territorial fabric ? correspondance between network and territory types ? Deconstruction of territories by increasing networking ?

$\rightarrow$ different significations and notions across disciplines.

First technical networks : materialisation ; II : network operators ; III : communication networks and IV social networks ; V and VI : companies and city networks.  \textit{Overview of semantic variations linking networks and territories}

\subsection{Technical Networks}

\subsubsection*{A contradictory discourse}

Functionality of technical networks. Simple to define but complex to qualify : various agents involved.

Structuration and destructuration of territories : cf Curien. Three layers in network : support (infrastructure), service and information.

\subsubsection*{Geographical Scales of Technical Networks}

Dupuy : technical network as a functional entity of territorial systems. 

\paragraph{Morphogenesis of Technical Networks}

Morphogenesis = transformation of networks. Material complementarity in first phase of network development, more free later (territorial coverage). No general rule for territorial limits or contribution to hierarchy of nodes.

\paragraph{Scale formatting, or connection against boundaries}

Historical role of networks in structuration of national spaces. Dialogue between scales of network and territories. Network fixes reference scale. Territories disappear while other are born ; homogenization only in rare cases of a covering network. Technical network contribute to scale definition. Correspondance between \textbf{Institutional territory} and \textbf{functional space} not necessarily direct.

\paragraph{Interconnections and scale skipping}

Example of RERA : difficult to define a reference scale (multiplicity of usages : \emph{multi-scale integration} ?)

$\rightarrow$ so-called scale skipping, not strictly hierarchical interaction between scales. (// rq : different to hierarchy in more simple CAS ? ). phenomenon induced by network interconnection.

In this multi-scale context, network becomes \textit{relay} between territories at different levels. Furthermore, no more correspondance between network scale and territory scale. Not to be confused to a ``disparition'' of scales.

\paragraph{Network vertices, between nodes and poles}

Importance of network vertices. Increasing role because of evolution of network structure (hub and spokes). Nodes becomes a ``pole'', structuring space around it. Difficult question : what is required for a vertex to become a pole (in many complementary ways).

\subsubsection*{Network Material Insertion : Street Network example}

Definition of \textit{voirie} (no english term) : public domain.

\paragraph{Spaces and flows}

Different public administrators ; collective functions. role of public space : more than support (transportation) and services. \textit{Genie Urbain} : ensures multi-functionality of public space.

\paragraph{Conflicts of use and heritage management}

Conflicts between actors of \textit{amenagement}, accentuated by variety of scales.

\textit{Voirie} as a concretization of interactions between territories and networks : has characteristics of an heritage.

\textit{Voirie} as road network participates to construction of individual territories, but \textit{voirie} as territory on which administrator can control uses.


\subsubsection*{Transportation Networks, Cornerstones for Territorial Dynamics}

Transportation Network can be infrastructure or services using infrastructure.

\paragraph{Effects on accessibility}

Accessibility : link between places and resistance to exchange flows (as a function of distance, time, cost).

Transportation Network as links and vertices, through which one accesses the service.

Transportation Network Morphogenesis : decision processes of implantation and local realization. Morphogenesis is decision processes of construction and localization.

Morphology of transportation network ensures more or less territorial connectivity.

Evolution in time of the role of infrastructures : apparition of ``tunnel effect'' (Bonnafous). High speed networks (rail and motorways).

Different homogeneity and isotropy parameters depending on network type (ex : road nw vs TGV). 




%%%%%%%%%%%%%%%%%%%%
%% Biblio
%%%%%%%%%%%%%%%%%%%%

\bibliographystyle{apalike}
\bibliography{/Users/Juste/Documents/ComplexSystems/CityNetwork/Biblio/Bibtex/CityNetwork}


\end{document}
