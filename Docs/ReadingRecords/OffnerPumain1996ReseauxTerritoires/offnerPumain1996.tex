%%%%%%%%%%%%%%%%%%%%%%%%%%%%%
% Standard header for working papers
%
% WPHeader.tex
%
%%%%%%%%%%%%%%%%%%%%%%%%%%%%%

\documentclass[11pt]{article}

%%%%%%%%%%%%%%%%%%%%
%% Include general header where common packages are defined
%%%%%%%%%%%%%%%%%%%%

% general packages without options
\usepackage{amsmath,amssymb,bbm}




%%%%%%%%%%%%%%%%%%%%
%% Idem general commands
%%%%%%%%%%%%%%%%%%%%
%% Commands

\newcommand{\noun}[1]{\textsc{#1}}


%% Math

% Operators
\DeclareMathOperator{\Cov}{Cov}
\DeclareMathOperator{\Var}{Var}
\DeclareMathOperator{\E}{\mathbb{E}}
\DeclareMathOperator{\Proba}{\mathbb{P}}

\newcommand{\Covb}[2]{\ensuremath{\Cov\!\left[#1,#2\right]}}
\newcommand{\Eb}[1]{\ensuremath{\E\!\left[#1\right]}}
\newcommand{\Pb}[1]{\ensuremath{\Proba\!\left[#1\right]}}
\newcommand{\Varb}[1]{\ensuremath{\Var\!\left[#1\right]}}

% norm
\newcommand{\norm}[1]{\| #1 \|}


% amsthm environments
\newtheorem{definition}{Definition}



%% graphics

% renew graphics command for relative path providment only ?
%\renewcommand{\includegraphics[]{}}








% geometry
\usepackage[margin=2cm]{geometry}

% layout : use fancyhdr package
\usepackage{fancyhdr}
\pagestyle{fancy}

\makeatletter

\renewcommand{\headrulewidth}{0.4pt}
\renewcommand{\footrulewidth}{0.4pt}
%\fancyhead[RO,RE]{\textit{Working Paper}}
\fancyhead[RO,RE]{\textit{ECTQG 2015}}
%\fancyhead[LO,LE]{G{\'e}ographie-Cit{\'e}s/LVMT}
\fancyhead[LO,LE]{An Algorithmic Systematic Review}
\fancyfoot[RO,RE] {\thepage}
\fancyfoot[LO,LE] {\noun{J. Raimbault}}
\fancyfoot[CO,CE] {}

\makeatother


%%%%%%%%%%%%%%%%%%%%%
%% Begin doc
%%%%%%%%%%%%%%%%%%%%%

\begin{document}







\title{Reading Record\bigskip\\
\cite{offner1996reseaux}
}
\author{\noun{Juste Raimbault}}
\date{Date}


\maketitle

\textbf{\textit{Reading Record for \cite{offner1996reseaux}}}


\section{Introduction}

\textit{Preface by Raffestin : }

Network and Territories : two notions that can represent both geographical materiality and theoretical domain.

Crucial notion : articulation space-network-territory. Need for a theory, some ``laws of progressive composition''.

Sketch of a theory (from Nicolas Curien) : territoriality = projection of a system of intentions into space ; systems of intention are expressed within material and social networks. Successive temporal imbrications [dynamical causalities ? ]. Importance of power in the process (network operators) : role of governance. 

Role of information : modification of mobility but no decrease. Role of social networks and company networks. Network of cities : also conceptual indetermination.


\section{Linear Reading}

\subsection{Introduction}

Various questions from crossing of different disciplines : do networks contribute to territorial fabric ? correspondance between network and territory types ? Deconstruction of territories by increasing networking ?

$\rightarrow$ different significations and notions across disciplines.

First technical networks : materialisation ; II : network operators ; III : communication networks and IV social networks ; V and VI : companies and city networks.  \textit{Overview of semantic variations linking networks and territories}

\subsection{Technical Networks}

\subsubsection*{A contradictory discourse}

Functionality of technical networks. Simple to define but complex to qualify : various agents involved.

Structuration and destructuration of territories : cf Curien. Three layers in network : support (infrastructure), service and information.

\subsubsection*{Geographical Scales of Technical Networks}

Dupuy : technical network as a functional entity of territorial systems. 

\paragraph{Morphogenesis of Technical Networks}

Morphogenesis = transformation of networks. Material complementarity in first phase of network development, more free later (territorial coverage). No general rule for territorial limits or contribution to hierarchy of nodes.

\paragraph{Scale formatting, or connection against boundaries}

Historical role of networks in structuration of national spaces. Dialogue between scales of network and territories. Network fixes reference scale. Territories disappear while other are born ; homogenization only in rare cases of a covering network. Technical network contribute to scale definition. Correspondance between \textbf{Institutional territory} and \textbf{functional space} not necessarily direct.

\paragraph{Interconnections and scale skipping}

Example of RERA : difficult to define a reference scale (multiplicity of usages : \emph{multi-scale integration} ?)

$\rightarrow$ so-called scale skipping, not strictly hierarchical interaction between scales. (// rq : different to hierarchy in more simple CAS ? ). phenomenon induced by network interconnection.

In this multi-scale context, network becomes \textit{relay} between territories at different levels. Furthermore, no more correspondance between network scale and territory scale. Not to be confused to a ``disparition'' of scales.

\paragraph{Network vertices, between nodes and poles}

Importance of network vertices. Increasing role because of evolution of network structure (hub and spokes). Nodes becomes a ``pole'', structuring space around it. Difficult question : what is required for a vertex to become a pole (in many complementary ways).

\subsubsection*{Network Material Insertion : Street Network example}

Definition of \textit{voirie} (no english term) : public domain.

\paragraph{Spaces and flows}

Different public administrators ; collective functions. role of public space : more than support (transportation) and services. \textit{Genie Urbain} : ensures multi-functionality of public space.

\paragraph{Conflicts of use and heritage management}

Conflicts between actors of \textit{amenagement}, accentuated by variety of scales.

\textit{Voirie} as a concretization of interactions between territories and networks : has characteristics of an heritage.

\textit{Voirie} as road network participates to construction of individual territories, but \textit{voirie} as territory on which administrator can control uses.


\subsubsection*{Transportation Networks, Cornerstones for Territorial Dynamics}

Transportation Network can be infrastructure or services using infrastructure.

\paragraph{Effects on accessibility}

Accessibility : link between places and resistance to exchange flows (as a function of distance, time, cost).

Transportation Network as links and vertices, through which one accesses the service.

Transportation Network Morphogenesis : decision processes of implantation and local realization. Morphogenesis is decision processes of construction and localization.

Morphology of transportation network ensures more or less territorial connectivity.

Evolution in time of the role of infrastructures : apparition of ``tunnel effect'' (Bonnafous). High speed networks (rail and motorways).

Different homogeneity and isotropy parameters depending on network type (ex : road nw vs TGV) ; loss of spatial equality through tunnel effect. Hubs and interconnection : increase of polarisation effects. Impact of TGV on importance of nodes of the old network. Also impact on local reorganisation of urban functions.

\paragraph{Socio-economic impact of transportation networks}

Direct rapid effects : choices of mode, mobility modifications, local economy. Long term : indirect effects (structuring effects).

Difficult question of structuring effects : no determinism. Defendors of this retoric base their discourse on technical determinism, mechanical metaphor, pure rationality economic model, positivism. Position linked to economical context ; in stagnation apparent causality does not exist anymore. Transportation networks as product and support of socio-economic activity. TGV can have very different effects : Le Creusot vs TGV Atlantique.

Socio-economic effects of transportation networks translate territorial dynamics. Congruence of both evolution at a macro-economic level (offer of innovation satisfy needs generated by socio-economic evolution). Innovation are locally appropriated, yielding local impact ; linked to representation of networks: opening, power, efficacity. Local adaptation can take the form of opportunities or resistance (nimby). Strategy of local and global actors are determining and can be illustrated in different ways.

\paragraph{High Speed Transportation}

Territories become closer ; spatial divisino of work increases, increasing mobility and concentration of transportation networks in high population density areas. Issue of too strong differentiation of territories, condemning some. Coordination problems between institutional territories yield difficulties for solidarity between territories.


\subsubsection*{Temporalities}

\begin{itemize}
\item Very different temporalities in interaction between networks and territories
\item Geographical scales as object of study and instruments
\item potentialities, opportunities and strategies
\item other approaches than dynamical : network as public territory
\item relations between different types of networks : operators as mediators ?
\end{itemize}



\subsection{Network Operators}

\subsubsection*{From Infrastructure Management to Territorial Service Offer}


def of network operators depends on how def of network is broad : can be manager, but also a set of actors in the case of a network of transactional projects. Operator is necessarily territorial.

Territory following Roncayolo : belonging, power, globality, representations. how do operator change different dimensions ?

\subsubsection*{Technical networks : expression of a system of actions}

Insertion of networks in institutional territories is understood through specification of relations between actors.

Public service : unique and monopolistic exploitant.

\paragraph{French specificity}

in France, cooperation between public and private sectors ; equilibrium between productive and political functions.

Historically, appears in technical transfer for local public service ; later regulated and nationalized.

different evolution of local and centralized, local closer to users.

Deregulation against monopolistic situations.

Network operator is the key for public service, has a composite role : technical, economic and juridicial.

\paragraph{Scales of territorial regulation of networks}

depending on scale, very different role of regulation and relation between operator and political.

French regions :local scale. deterritorialisation because of high speed network. consistence of intermediate scale ?

european networks : greater level of governance.

\paragraph{Network management, spatial and territorial implications}

manager change space through technical infrastucture ; and socio-economic relations through offer ; creates territory.


\paragraph{Different roles of network manager}

$\rightarrow$ technical, commercial, economic, investor, etc. - in relation for each with institutional territory and social territory

\paragraph{Morphogenesis and Extension}

\textbf{Morphogenesis of networks : extension and space, beyond territorial boundaries}.

Network externalities : production costs and scale economies depends on which network.

Functional space of nw ; its functionalities are essential in its morphogenesis.

\paragraph{Equipements and Functional Logics}

Territory of operator : depends on function.

adaptation of institutions to functional territories.

Dichotomy between functional and institutional territories.

\subsubsection*{Network Public Policy : Regulation and Opening}

\paragraph{Tarification}

Monopolistic public situation : budget equilibrium. segmentation of demand : adpated tarification.

no more spatial perequation with dynamic tarification. can be spatially dynamic (tel network)

\paragraph{Opening}

infrastructure and services coupled exploitation deregulated.

separation between infrastructures and services. base service and added value services.

intelligent demand : IT and electrical networks.

support network, command network and service network.

\subsubsection*{Utility and social norm} 

From offer to demand-based organisation of networks.

\textbf{Network create territory of flows} : not structuring effects but \textbf{congruence}.

nature of demand is important, sometimes difficult to fullfill as decisions taken by not concerned actors.


\subsection{Telecommunications Networks}

\subsubsection*{Telecommunication Networks and Territorial Production}

\paragraph{From the complex object}

Development of ICT. - linked with deregulation.

obstacle to convergence ? spatial equality. question of future usage.

\paragraph{to instrumentalized object}

can ITC networks shape territories ? problem of opposition between local actors and centralized.

\paragraph{and reconstructed territory}

further layer in territory ; economically real territories.

\subsubsection*{Space and Territory seen through telecomunication networks}

\paragraph{Emergence of real space-time}

Real time, instantaneity, induce spatial ubiquity. Economy submitted to real-time regime because of high-frequency finance. structuring effects of communication networks ?

\paragraph{Simplifying myths}

Spatial transparency : deterritorialisation ? not so easy as geographical space is not distance only.

innovation not linear nor systematic. cybernetics, technocracy and complexity (Morin) $\rightarrow$ technology can not be isolated. auto-organisation of living machines.

linearity of progress : beware to distinguish information from decision-making.


\paragraph{Territorial Production and human territorialities}

Raffestin : territorial as social production ; territorial system as set of power relations needed to construct the territory. territoriality os reappropriation of territory by populations.

\subsubsection*{State and Telecommunication networks}

ICT : new means for state to control, power, etc. historically not new. strategical resssource.

\paragraph{Time of Castles}

metaphor of centralized network star : center is the castle.

all telecom means under strict control of state. same as in other fields of infrastructure.

\paragraph{Time of opening}

deregulation : reevaluation of role of state ; increase of private needs. in France less deregulated than anglo-saxon countries.

new relations between actors : concurrence between state actors ; regions, datar, etc.

Political factor still present, not totally deregulated.

\subsubsection*{Telecommunication networks and economic activities localisation}

wrong visions of planners, simplified space.

\paragraph{Site localisation}

homogeneous offer, not difference for standard companies.

\paragraph{Spatial organisation of companies}

can favor delocalisations and spatial spread.

\paragraph{Working at distance}

concerns a few number of jobs - corresponds to imaginary of networks (myths).

\subsubsection*{Physical proximity in media ubiquity society}

No total technological determinism nor total social reproduction, but appropriation and production. $\rightarrow$ telecom as production tool ; innovation produces social.

physical proximity is essential ; closely interlaced with telecomunication.


\subsection{Social Networks and Territories}


\subsubsection*{Preliminary definitions and research question}

social network :relations. ; opposed to organisation as frame of exchanges.

territory : social production.

recent mutations may induce a replacement of territory by social network : in fact an other expression of the territory.

\subsubsection*{Social Networks and Territories : a difficult meeting}

\paragraph{Social networks : formal approach to social link}

back to Simmel, social circles.  network analysis. Granovetter : weak and strong links. importance of social significations ?

\paragraph{Space in network analysis}

transversality of network vision : works also with space. influence of space on social relation : ex city vs rural. bullshit maffesoli.

\paragraph{From place to social links}

interdisciplinary : geography not only to link territory and social networks.

Raffestin as entry ? anthropology to understand relation of individual and groups with space.

difficult : no integrated definition : or territory presupposed, or social network only. (// with transportation networks ?)

\subsubsection*{Social links and movement}


Defs : social networks : individual in interrelation ; territory : construction and appropriation.

\paragraph{Networks and social relations}

top-down social link neglected in network analysis.

typology of social link construction : modern (self-constructed) vs patrimonial (context)

\paragraph{Proximity links}

overdeterminated social relations because of geographical proximity. 

Territory is also an element of social proximity.

\paragraph{Link creates territory : spatial dispersion and social cohesion}

collective memory : identity construction.

typology : village community, diaspora, migrants.

\paragraph{Networks and territorial institutions}


institutionalisation of social links. diffusion of innovation when legitimized.

network dynamics coupled with territorial inscription.

network deploy social links, territorialized or not.


\subsubsection*{Between research of sense and efficacity}

efficacy of flow of ressources in networks. Epistemological opposition between networks and territory ? different temporalities. different regulation regimes.

Territorial and network processes ; contradiction between institutional and functional.

new research perspectives.


\subsection{Networks and Companies} 

\subsubsection*{What link between company spaces and territory ?}

Company network : internal and external ; developement due to recent profond mutations.

Two entries on networks : networks as efficiency tools (new organisation paradigms) ; institutional and social networks produced by territory (territory as social production).

\subsubsection*{Companies and Networks : between market and cooperation}

Classical competition : networks as coordination tools for economic agents. With new means to increase productivity, such as cooperation, nws take a different sense. Organisation as trade-off between technical productivity and social creation of ressources.

\paragraph{Company as autonomous unit}

network opposed to hierarchical organisation ; materializes coordination.

Transaction costs for cooperation : bilateral relations? example of delegation.

vertical vs horizontal integration. Partenarial relations in horizontal quasi-integration (ex Japon Keiretsu) ; less hierarchical  :network of interacting autonomous entities. Complementarity of production processes is a possible explanation to coordination.

\paragraph{Network as global form}

organisation is a way for the company to adapt to temporal dynamics. various organisational changes :
\begin{itemize}
\item increase firm autonomy
\item conventional environment regularization
\item network economies and scale economies : partenarial organisation (coordination). network being only way to achieve scale economies.
\end{itemize}


\subsubsection*{Territorialized Networks for the Company}

Territory seen as result of territorialisation process in which firms play a role : construction of ressources, learning. Not a static vision of territory with static ressources. Territorialisation as the coupling of company spaces with juridic, social, historical spaces.

\paragraph{Networks and Territorialized learning}

territory implied in learning, ex localized cooperation processes. (learning as cumulative social processes).

\paragraph{Environment and territorial conventions}

Innovation environments are partially produced by territory through structural and organizational choices. These processes can be designated as \emph{territorial convention}.

\paragraph{Industrial organisation as territorial construction}

Localized industrial systems strongly linked to territory. Industrial districts as emerging from coordination, on long times through integration into technical industrial system, generally after technological jump.

\paragraph{Institutional forms and territorial forms}

importance of institutions on localized territorial dynamics : public policies reinforce partenarial dynamics.

\paragraph{Company and Urban Territories}

Cities as privileged territories by companies through history. Insertion in city systems through networks, participate in coevolution ; strongly influence urban territories in particular metropolitan territories. Diffusion of growth through networks? Which networks favor more reinforcement of urban hierarchy ?

\subsubsection*{Dimensions of coordination}

Territory either as a mode of productive integration, or as a dynamical organisation. Network is a metaphor. Network is representation of the firm but also contains concrete relations. link physical and social proximity is not clear.


\subsection{Network of Cities}

\subsubsection*{Cities and Relations}

City Network as related cities, shape the territory (in administrative sense or social appropriation sense). Network of Cities implies technical networks.

From Network of cities to system of cities as first sense. Second sense : voluntary organisation between cities.

\subsubsection*{Urban Network and Systems of Cities}

Urban Network : interdependencies of cities.

\paragraph{Central place theory}

Reynaud then Christaller : Central Place Theory :
\begin{itemize}
\item urban aggregation because of agglomeration economies and sociability ; urban centrality depending on accessibility
\item hierarchical organisation (Losch : non-imbricated hierarchies)
\item each city is at the center of its influence area (hexagonal for Christaller) ; probabilistic formulation of frequentation dynamics.
\end{itemize}

Role of cities in structuring territory : territorial planning.

\paragraph{Specialization}

economic specialization as driver of economic growth : assumes distribution network. city system between totally networked cities and isolated cities.

\paragraph{From Urban Network to System of Cities}

Centrality of interdependencies in system of cities. Spatial diffusion of innovation can unveil them. Auto-organized city system and cycles of innovation - path-dep and bifurcations.

Network implicitely present in these notions : interaction between system of cities and territory. Scales to consider networks have change : local relations are more of ecological order ; economic and social at the scale of system of cities.

\subsubsection{Networks of Cities}


Other vision of networks of cities : some only connected, through hierarchical or specialization criteria.

Three types of networks : city alliances, city size similarity, city specialized in same activity. Also city clubs as an additional type.

\paragraph{City alliances}

Complementarity of functions and activities. (ex Reims, Troyes, Chalons). Medium cities, territories without metropolitan areas. parallel between datar and old organisations. Objectives to increase in hierarchy. No real implementation, generally discursive only.

\paragraph{Network of large cities}

Over threshold of 200000 inhabitants, qualitative differenciation (in work qualification and spatial influence). Network of large cities disconnected from territory in the sense of administrative territory. Supranational networks : not only the superposition of national networks. Def of networks comes from selection on relations between cities.

\paragraph{Specialized cities networks}

specialization in comparison to regional or national reference. Network of cities with same activities. In closer geographical proximity, complementarity of specialization. single city integrated within global context (ex Cognac) is not relevant example.

\subsubsection*{Cities and Territorial recomposition}

Interogations on link between hierarchy and networks. Cities play different roles at different scales in territorial structuration.

\paragraph{Metropolisation}

Connection relations become more important than proximity. Metropolisation related to globalisation, induce a need for speed.

\paragraph{Various forms of hierarchy}

multiple semantics : in graph, hierarchy = centrality ; in organisation, depth in the tree. Urban hierarchy : size and function.

with metropolisation, new sense in urban hierarchy : flows hierarchy (centrality) and territorial hierarchies to be added. Very different from company networks. Complex coevolutions. Hierarchical network organisation : network of global cities ; network of specialized national cities ; network of specialized regional cities.

\paragraph{Complexity and Diversity of Urban}

Very different temporality between ICT nws, company nws and technical/cities networks.

Also network restructuration within urban entities.

Complexity of international insertion.

\paragraph{Towards new territorialities ?}

Beware of not using network at any sauce : dependance relations in spatial diffusion of innovation are crucial. Interdpeendence : superposition of various networks. Ntework of cities induce new territorialities : tunnel effect and ICT. class inequality : spatial inequality in diffusion of innovation and social inequality in use of innovations.


\subsection{Conclusion : Distanciations}

(Offner)

Demystification (ex structuring effects) and discussion of concepts.

Non exhaustive : 
\begin{itemize}
\item what about articulation between different types of networks ?
\item reflexion focused more on offer than on demand : what about network usage ?
\end{itemize}

new nomads and cocooning : Attali vs Finkielkraut. $\rightarrow$ political questioning. what governance ? What governement when new territories multiply ?


\textit{Postface (Neuschwander)}

Interdisciplinary collaboration.

Network crucial in history at any period.

Still necessary today :
\begin{itemize}
\item ``mutations in mutation'' (non-stationary dynamics) ; systemic responses of individuals. to stay efficient, hierarchical structures need networks.
\item interconnexion of knowledge
\item new tools of communications and networking (ICT)
\end{itemize}

Clarification and demystification of networks, consequences on various aspects : communications, territories, companies, cities. Still to be continued on societal organisation and governments.



\section{Synthesis}

\textit{TBW}

Biblio : \cite{dupuy1993geographie}





%%%%%%%%%%%%%%%%%%%%
%% Biblio
%%%%%%%%%%%%%%%%%%%%

\bibliographystyle{apalike}
\bibliography{/Users/Juste/Documents/ComplexSystems/CityNetwork/Biblio/Bibtex/CityNetwork}


\end{document}
