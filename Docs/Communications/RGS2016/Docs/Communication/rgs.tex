\documentclass[french,11pt]{beamer}

\DeclareMathOperator{\Cov}{Cov}
\DeclareMathOperator{\Var}{Var}
\DeclareMathOperator{\E}{\mathbb{E}}
\DeclareMathOperator{\Proba}{\mathbb{P}}

\newcommand{\Covb}[2]{\ensuremath{\Cov\!\left[#1,#2\right]}}
\newcommand{\Eb}[1]{\ensuremath{\E\!\left[#1\right]}}
\newcommand{\Pb}[1]{\ensuremath{\Proba\!\left[#1\right]}}
\newcommand{\Varb}[1]{\ensuremath{\Var\!\left[#1\right]}}

% norm
\newcommand{\norm}[1]{\| #1 \|}

\newcommand{\indep}{\rotatebox[origin=c]{90}{$\models$}}





\usepackage{mathptmx,amsmath,amssymb,graphicx,bibentry,bbm,babel,ragged2e}

\makeatletter

\newcommand{\noun}[1]{\textsc{#1}}
\newcommand{\jitem}[1]{\item \begin{justify} #1 \end{justify} \vfill{}}
\newcommand{\sframe}[2]{\frame{\frametitle{#1} #2}}

\newenvironment{centercolumns}{\begin{columns}[c]}{\end{columns}}
%\newenvironment{jitem}{\begin{justify}\begin{itemize}}{\end{itemize}\end{justify}}

\usetheme{Warsaw}
\setbeamertemplate{footline}[text line]{}
\setbeamercolor{structure}{fg=purple!50!blue, bg=purple!50!blue}

\setbeamersize{text margin left=15pt,text margin right=15pt}

\setbeamercovered{transparent}


\@ifundefined{showcaptionsetup}{}{%
 \PassOptionsToPackage{caption=false}{subfig}}
\usepackage{subfig}

\usepackage[utf8]{inputenc}
\usepackage[T1]{fontenc}



\makeatother

\begin{document}


\title{For a Cautious Use of Big Data and Computation}

\author{J.~Raimbault$^{1,2}$\\
\texttt{juste.raimbault@parisgeo.cnrs.fr}
}


\institute{$^{1}$UMR CNRS 8504 G{\'e}ographie-cit{\'e}s\\
$^{2}$UMR-T IFSTTAR 9403 LVMT\\
}


\date{RGS 2016\\\smallskip
\textit{Session Geocomputation : The Next 20 years}\\\smallskip
September 2016
}

\frame{\maketitle}







%%%%%%%%%%%%%%%%
%%
%% IDEAS
%%
%%  -> ex of synthetic data - interpolation, extrapolation.
%%  -> ex of Pok.Go (cf geotamtam mail) : direct on data without thinking ?
%%  -> Claims : make a stronger link between computational practices and maths/stats ? (if possible) - need more interdisciplinarity to do that.
%%     Geography in the center of that dynamic, but not all for granted ! (eg not simulate Gibrat as the fool example !)
%%     It implies the need for elaborated theory : here trash West/Bettencourt.
%%  -> Mention Cybergeo/CybergeoNetworks
%%
%%  - example with journal/paper title : what ?



%%%%%%%%%
%% Abstract

%\textbf{Keywords : }\textit{Data Deluge, Computational Science, City-transportation Interactions, Spatio-temporal Correlations}

%The so-called \emph{big data revolution} resides as much in the availability of large datasets of novel and various types as in the always increasing available computational power. Although the \emph{computational shift} (\cite{arthur2015complexity}) is central for a science aware of complexity and is undeniably the basis of future modeling practices in geography as \cite{banos2013pour} points out, we argue that both \emph{data deluge} and \emph{computational potentialities} are dangerous if not framed into a proper theoretical and formal framework. The first may bias research directions towards available datasets (as e.g. numerous twitter mobility studies) with the risk to disconnect from a theoretical background, whereas the second may overshadow preliminaries analytical resolutions essential for a consistent use of simulations. We illustrate this idea with an example on well-studied geographical objects that are interactions between networks and territories. We compute static correlations between indicators of urban form and indicators of road network topology, using open datasets of european population density (\cite{eurostat}) and OpenStreetMap. A mathematical derivation of the link between spatial covariance at fixed time and dynamical covariance for spatio-temporal stochastic processes, combined with a theory of city-transportation interactions within evolutive urban systems on long times (\cite{bretagnolle:tel-00459720}), allows to infer knowledge on involved \emph{geographical processes} from empirical static correlations. In particular we show the regional nature of dynamic interactions, confirming the non-ergodicity of urban systems (\cite{pumain2012urban}). We argue that the two conditions for this result are indeed the ones endangered by incautious big-data enthusiasm, concluding that a main challenge for future Geocomputation is a wise integration of novel practices within the existing body of knowledge.












%%%%%%%%%%%%%%%%%
\section{Introduction}
%%%%%%%%%%%%%%%%%

% introduce question with anecdote
%  idea : These Gleyze ultra-limited ; 5 years later These Claire exponential growth in computational power used ?

\sframe{Computational power : an exponential use}{

}


\sframe{But to what purpose ?}{
% here trash Gibrat simulation

}


\sframe{Theories and Computation}{
% state the main idea

}




%%%%%%%%%%%%%%%%%
\section{Case study}
%%%%%%%%%%%%%%%%%

% develop concrete example


\sframe{Case study : Context and Rationale}{
% interactions between networks and territories

}


\sframe{Dataset construction}{
% brief description of nw simplification ; 

}




%%%%%%%%%%%%%%%%%
\section{Discussion}
%%%%%%%%%%%%%%%%%

% explain why example justify our point of view






%%%%%%%%%%%%%%%%%
\sframe{Conclusion}{

}



%%%%%%%%%%%%%%%%%
\sframe{Reserve}{

\textbf{Reserve Slides}

}
%%%%%%%%%%%%%%%%%%%%%%%%%%%%





%%%%%%%%%%%%%%%%%%%%%
\begin{frame}[allowframebreaks]
\frametitle{References}
\bibliographystyle{apalike}
\bibliography{/Users/Juste/Documents/ComplexSystems/CityNetwork/Biblio/Bibtex/CityNetwork}
\end{frame}
%%%%%%%%%%%%%%%%%%%%%%%%%%%%












\end{document}















