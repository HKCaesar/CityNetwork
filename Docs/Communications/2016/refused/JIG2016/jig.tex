%%%%%%%%%%%%%%%%%%%%%%%%%%%%%
% Standard header for working papers
%
% WPHeader.tex
%
%%%%%%%%%%%%%%%%%%%%%%%%%%%%%

\documentclass[11pt]{article}

%%%%%%%%%%%%%%%%%%%%
%% Include general header where common packages are defined
%%%%%%%%%%%%%%%%%%%%

% general packages without options
\usepackage{amsmath,amssymb,bbm}




%%%%%%%%%%%%%%%%%%%%
%% Idem general commands
%%%%%%%%%%%%%%%%%%%%
%% Commands

\newcommand{\noun}[1]{\textsc{#1}}


%% Math

% Operators
\DeclareMathOperator{\Cov}{Cov}
\DeclareMathOperator{\Var}{Var}
\DeclareMathOperator{\E}{\mathbb{E}}
\DeclareMathOperator{\Proba}{\mathbb{P}}

\newcommand{\Covb}[2]{\ensuremath{\Cov\!\left[#1,#2\right]}}
\newcommand{\Eb}[1]{\ensuremath{\E\!\left[#1\right]}}
\newcommand{\Pb}[1]{\ensuremath{\Proba\!\left[#1\right]}}
\newcommand{\Varb}[1]{\ensuremath{\Var\!\left[#1\right]}}

% norm
\newcommand{\norm}[1]{\| #1 \|}


% amsthm environments
\newtheorem{definition}{Definition}



%% graphics

% renew graphics command for relative path providment only ?
%\renewcommand{\includegraphics[]{}}








% geometry
\usepackage[margin=2cm]{geometry}

% layout : use fancyhdr package
\usepackage{fancyhdr}
\pagestyle{fancy}

\makeatletter

\renewcommand{\headrulewidth}{0.4pt}
\renewcommand{\footrulewidth}{0.4pt}
%\fancyhead[RO,RE]{\textit{Working Paper}}
\fancyhead[RO,RE]{\textit{ECTQG 2015}}
%\fancyhead[LO,LE]{G{\'e}ographie-Cit{\'e}s/LVMT}
\fancyhead[LO,LE]{An Algorithmic Systematic Review}
\fancyfoot[RO,RE] {\thepage}
\fancyfoot[LO,LE] {\noun{J. Raimbault}}
\fancyfoot[CO,CE] {}

\makeatother


%%%%%%%%%%%%%%%%%%%%%
%% Begin doc
%%%%%%%%%%%%%%%%%%%%%

\begin{document}







\title{\vspace{-1.8cm}Coupled Territorial and Network Dynamics : from Modeling to Theory\\
\textit{Communication Proposal, JIG 2016}
}
\author{\small\noun{Juste Raimbault}$^{1,2}$\\
\small(1) UMR CNRS 8504 Géographie-cités and (2) UMR-T IFSTTAR 9403 LVMT
}
%\date{15 octobre 2015}
\date{}

\maketitle

\justify

\pagenumbering{gobble}

\vspace{-0.5cm}
\textbf{Keywords : }\textit{Transportation Network, Territorial Systems, Co-evolution, Models of Simulation, Systems Theory}

\medskip

The relation between transportation networks and territorial development is central to most approaches in territorial sciences, but remains poorly understood in a quantitative way : the example of \emph{the myth of structural effects of transportation infrastructures}~\citep{offner1993effets} shows that simple causal assumptions do not hold when trying to explain the \emph{co-evolution} between transportation networks and localized components of territorial systems. Dynamical modeling including both evolutions has been emphasized as a cornerstone for a better understanding of involved processes in the case of cities systems on long time scales (\citep{bretagnolle:tel-00459720}, p. 162-163). However, a broad interdisciplinary state-of-the-art, based on algorithmic systematic review, done in~\citep{raimbault2015models} shows the quasi-absence of models of that type (e.g. LUTI models at a middle scale where networks are considered static~\citep{iacono2008models}, whereas network growth models on a longer time scale~\citep{xie2009modeling}). The purpose of this communication is in a first part to develop results obtained by such agent-based toy-models of simulation, and then to propose a theoretical framework constructed from conclusions of models explorations.

A first family of models explores a weak coupling between a population density generation model generalizing the diffusion-limited aggregation model~\citep{batty2006hierarchy}, and network generation heuristics for which biological network generation~\citep{TeroAl10} and generalized gravity potential rupture are tested. Density model is explored and calibrated alone for morphological objectives through intensive computation, against real data from European density grid, which patterns are well reproduced by the calibrated model. Generated density grids are then used as a basis for network generation, which provide a broad spectrum of values for network measures and correlations between morphological and network measures. It shows that the inclusion of transportation network \emph{is not necessary} to reproduce typical patterns of urban growth, but that explanation of processes and interdependence mechanisms can only be done through more complex models. We develop then a model at the metropolitan scale aiming to involve stronger coupling and more complex mechanisms, in particular a game-theory-based governance process capturing the feedback of the territory on the network. The partial validation on stylized facts at different scales (e.g. land-use evolution patterns, network shape) and the exploration of parameter space suggest targeted experiments such as the comparison of governance systems, and pave the way to more operational similar dynamic models.


The conclusions of these first modeling experiments unveil or confirm requirements of a theoretical framework for territorial systems modeling. They include in particular a framing of the notion of coupling between subsystems, a precise definition of scale and an emphasis on emergence to take into account multi-scale aspects of systems, the superposition of heterogeneous views and components of a system. Starting from a perspectivist point of view~\citep{giere2010scientific}, we consider a system as a set of perspectives consisting in ontological sets~\citep{livet2010} associated with dataflow machines~\citep{golden2012modeling}. Formal pre-orders between subsets of ontologies, constructed from emergence relations~\citep{bedau2002downward}, yield after a canonical reduction an unique forest representing the structure of the system. Strong coupled components reside within nodes, whereas a temporal scale and ``thematic'' scale (scale for a state variable) can be associated to each node of the forest by construction from dataflow machines timescales. This framework is formally self-consistent and meets our requirements. Its application should in future work guide the construction of operational models of co-evolution.





%%%%%%%%%%%%%%%%%%%%
%% Biblio
%%%%%%%%%%%%%%%%%%%%
\tiny

\begin{multicols}{2}

\setstretch{0.3}
%\setlength{\parskip}{-0.4em}


\bibliographystyle{apalike}
\bibliography{/Users/Juste/Documents/ComplexSystems/CityNetwork/Biblio/Bibtex/CityNetwork}

\end{multicols}

\end{document}
