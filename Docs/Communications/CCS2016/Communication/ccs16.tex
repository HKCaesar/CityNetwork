\documentclass[english,11pt]{beamer}

\DeclareMathOperator{\Cov}{Cov}
\DeclareMathOperator{\Var}{Var}
\DeclareMathOperator{\E}{\mathbb{E}}
\DeclareMathOperator{\Proba}{\mathbb{P}}

\newcommand{\Covb}[2]{\ensuremath{\Cov\!\left[#1,#2\right]}}
\newcommand{\Eb}[1]{\ensuremath{\E\!\left[#1\right]}}
\newcommand{\Pb}[1]{\ensuremath{\Proba\!\left[#1\right]}}
\newcommand{\Varb}[1]{\ensuremath{\Var\!\left[#1\right]}}

% norm
\newcommand{\norm}[1]{\| #1 \|}

\newcommand{\indep}{\rotatebox[origin=c]{90}{$\models$}}





\usepackage{mathptmx,amsmath,amssymb,graphicx,bibentry,bbm,babel,ragged2e}

\makeatletter

\newcommand{\noun}[1]{\textsc{#1}}
\newcommand{\jitem}[1]{\item \begin{justify} #1 \end{justify} \vfill{}}
\newcommand{\sframe}[2]{\frame{\frametitle{#1} #2}}

\newenvironment{centercolumns}{\begin{columns}[c]}{\end{columns}}
%\newenvironment{jitem}{\begin{justify}\begin{itemize}}{\end{itemize}\end{justify}}

\usetheme{Warsaw}
\setbeamertemplate{footline}[text line]{}
\setbeamercolor{structure}{fg=purple!50!blue, bg=purple!50!blue}

\setbeamersize{text margin left=15pt,text margin right=15pt}

\setbeamercovered{transparent}


\@ifundefined{showcaptionsetup}{}{%
 \PassOptionsToPackage{caption=false}{subfig}}
\usepackage{subfig}

\usepackage[utf8]{inputenc}
\usepackage[T1]{fontenc}



\makeatother

\begin{document}


\title{Models of growth for system of cities : Back to the simple}

\author{J.~Raimbault$^{1,2}$\\
\texttt{juste.raimbault@parisgeo.cnrs.fr}
}


\institute{$^{1}$UMR CNRS 8504 G{\'e}ographie-cit{\'e}s\\
$^{2}$UMR-T IFSTTAR 9403 LVMT\\
}


\date{CCS 2016 - Amsterdam\\\smallskip
\textit{Session Urban 3}\\\smallskip 22th September 2016
}

\frame{\maketitle}





%%%%%%%%%%%%%%%%%%%
%% ABSTRACT

%Understanding growth patterns in complex systems of cities through modeling is an intensive branch of quantitative geography. Complex agent-based models have been recently provided promising results by multi- modeling and intensive computation for pattern discovery and calibration. However simple interaction-based extensions of seminal models of growth (such as the Gibrat model) have not yet been tested and calibrated against real datasets.
%We propose a spatial model of urban growth extending the Gibrat model by adding the contributions of gravity-based interactions to expected growth rates. Moments derivation for the stochastic model allows to implement a deterministic version on expectancies. Working with the Pumain-INED harmonized database for French cities (population of urban areas for 1831-1999), the 4-parameter interaction model is calibrated through intensive computation on grid, using the OpenMole software, yielding e.g. the characteristic interaction distance at different periods. We then add a second order term aimed at integrating interactions between physical transportation networks and cities, through a feedback of physical flows on traversed cities.It allows to obtain better fits and reproduce stylized facts such as hierarchy inversions and apparition of the “tunnel effect” with the development of railway network. We furthermore introduce a novel method to assess the impact of adding parameters to a simulation model on the effectively gained information, as an extension of Akaike Information Criterion to simulation models. This empirical AIC is estimated by comparing AICs for statistical models, with same parameter number, fitting best behavior space obtained by exploration. It confirms that our extension provide a gain of information on the French city system.
%This contribution provides a renewing insight on simple models of urban growth for system of cities, that proves to have good explicative potentialities. It also introduce a methodology to tackle the open question of quantifying overfitting in simulation models.








%%%%%%%%%%%%%%%%%
\section{Introduction}
%%%%%%%%%%%%%%%%%



\sframe{Modeling Urban Growth}{

%\justify


%\textit{}

%\bigskip

%\textbf{Diverse developments :}

%\medskip

%$\rightarrow$

%\medskip

}



%%%%%%%%%%%%%%%%%
\section{Methods and Results}
%%%%%%%%%%%%%%%%%









%%%%%%%%%%%%%%%%%
\section{Discussion}
%%%%%%%%%%%%%%%%%


\sframe{Conclusion}{


\bigskip
\bigskip
\bigskip


\footnotesize{ - All code available at \texttt{https://github.com/JusteRaimbault/}

}

}






%%%%%%%%%%%%%%%%%%%%%
\begin{frame}[allowframebreaks]
\frametitle{References}
\bibliographystyle{apalike}
\bibliography{/Users/Juste/Documents/ComplexSystems/CityNetwork/Biblio/Bibtex/CityNetwork,biblio}
\end{frame}
%%%%%%%%%%%%%%%%%%%%%%%%%%%%












\end{document}







