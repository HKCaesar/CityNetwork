%%%%%%%%%%%%%%%%%%%%%%%%%%%%%
% Standard header for working papers
%
% WPHeader.tex
%
%%%%%%%%%%%%%%%%%%%%%%%%%%%%%

\documentclass[11pt]{article}

%%%%%%%%%%%%%%%%%%%%
%% Include general header where common packages are defined
%%%%%%%%%%%%%%%%%%%%

% general packages without options
\usepackage{amsmath,amssymb,bbm}




%%%%%%%%%%%%%%%%%%%%
%% Idem general commands
%%%%%%%%%%%%%%%%%%%%
%% Commands

\newcommand{\noun}[1]{\textsc{#1}}


%% Math

% Operators
\DeclareMathOperator{\Cov}{Cov}
\DeclareMathOperator{\Var}{Var}
\DeclareMathOperator{\E}{\mathbb{E}}
\DeclareMathOperator{\Proba}{\mathbb{P}}

\newcommand{\Covb}[2]{\ensuremath{\Cov\!\left[#1,#2\right]}}
\newcommand{\Eb}[1]{\ensuremath{\E\!\left[#1\right]}}
\newcommand{\Pb}[1]{\ensuremath{\Proba\!\left[#1\right]}}
\newcommand{\Varb}[1]{\ensuremath{\Var\!\left[#1\right]}}

% norm
\newcommand{\norm}[1]{\| #1 \|}


% amsthm environments
\newtheorem{definition}{Definition}



%% graphics

% renew graphics command for relative path providment only ?
%\renewcommand{\includegraphics[]{}}








% geometry
\usepackage[margin=2cm]{geometry}

% layout : use fancyhdr package
\usepackage{fancyhdr}
\pagestyle{fancy}

\makeatletter

\renewcommand{\headrulewidth}{0.4pt}
\renewcommand{\footrulewidth}{0.4pt}
%\fancyhead[RO,RE]{\textit{Working Paper}}
\fancyhead[RO,RE]{\textit{ECTQG 2015}}
%\fancyhead[LO,LE]{G{\'e}ographie-Cit{\'e}s/LVMT}
\fancyhead[LO,LE]{An Algorithmic Systematic Review}
\fancyfoot[RO,RE] {\thepage}
\fancyfoot[LO,LE] {\noun{J. Raimbault}}
\fancyfoot[CO,CE] {}

\makeatother


%%%%%%%%%%%%%%%%%%%%%
%% Begin doc
%%%%%%%%%%%%%%%%%%%%%

\begin{document}







%\pagecolor{black}

%\color{white}


% Mixed nash equilibrium formula

\[
p_i = \frac{J}{\Delta X_{\bar{i}}{Z^{\star}_{C}} - \Delta X_{\bar{i}}{Z^{\star}_{\bar{i}}}}
\]


% Discrete choices model

\[
U_i(C) - U_i(NC) = p_{\bar{i}} \left( \Delta X_{i}{Z^{\star}_{C}} - \Delta X_{i}{Z^{\star}_{i}}\right) - J
\]

\[
p_i = \frac{1}{1 + \exp{\left(-\beta_{DC}\cdot \left(\frac{\Delta X_{i}{Z^{\star}_{C}} - \Delta X_{i}{Z^{\star}_{i}}}{1 + \exp{\left(- \beta_{DC}(p_i \cdot (\Delta X_{\bar{i}}{Z^{\star}_{C}} - \Delta X_{\bar{i}}{Z^{\star}_{\bar{i}}}) - J)\right)}} - J \right)\right)}}
\]


\bigskip
\bigskip


% default param values
$A_{max} = E_{max} = 500 ; r_A = 1 ; r_E = 0.8 ; \gamma_E = 0.9 ; \gamma_A = 0.65 ; \beta_{l} = 1.8 ; \lambda = 0.005 ; r_0 = 2$

$N_{expl} = 25 ; I = 0.001 ; J = 0.0001 ; \nu = 5 ; E_{ext}(t_0) = 3E_{max} ; t_f = 4$

\bigskip
\bigskip
\bigskip
\bigskip


%%%%%%%%%%%%%%
%% Reserve slides : model precise description
%%%%%%%%%%%%%%

Initial distribution of Actives and Employments around governance centers at positions $\vec{x}_i$ by
\[
A(\vec{x}) = A_{max} \cdot \exp{\left(\frac{\norm{\vec{x}-\vec{x}_i}}{r_A}\right)} ; 
E(\vec{x}) = E_{max} \cdot \exp{\left(\frac{\norm{\vec{x}-\vec{x}_i}}{r_E}\right)}
\]

\bigskip
\bigskip

For facility patches, employments are added by $E(\vec{x}) = E(\vec{x})+\frac{k_{ext}\cdot E_{max}}{n_{ext}}$.


\bigskip
\bigskip


Transportation module : computation of flows $\phi_{ij}$ by solving on $p_i,q_j$ by a fixed point method (Furness algorithm), the system of gravital flows
\[
\begin{cases}
\phi_{ij} = p_i q_j A_i E_j \exp{\left(-\lambda_{tr} d_{ij}\right)}\\
\sum_k \phi_{kj} = E_j ; \sum_k \phi_{ik} = A_i\\
p_i = \frac{1}{\sum_k{q_k E_k \exp{(-\lambda_{tr}d_{ik})}}} ; q_j = \frac{1}{\sum_k{p_k A_k \exp{(-\lambda_{tr}d_{kj})}}} 
\end{cases}
\]

Trajectories then attributed by effective shortest path, and corresponding congestion $c$ obtained (no Wardrop equilibrium). 

Speed of network given by BPR function $v(c) = v_0 \left(1 - \frac{c}{\kappa}\right)^{\gamma_c}$. Congestion not used in current studies (infinite capacity $\kappa$).



\bigskip
\bigskip

\newpage

Land-Use module : we assume that residential/employments relocations are at equilibrium at the time scale of a tick, that corresponds to transportation infrastructure evolution time scale which is much larger (Bretagnolle, 2009).

We take a Cobb-douglas function for utilities of actives/employments at a given cell
\[
U_i (A) = X_i(A)^{\gamma_A}\cdot {F_i(A)}^{1-\gamma_A} ; F_i(A) = \frac{1}{A_i E_i}
\]
\[
U_j (E) = X_j(E)^{\gamma_E}\cdot {F_j(E)}^{1-\gamma_E} ; F_j(E) = 1
\]

where $X_i(A) = A_i\cdot \sum_j{E_j \exp{\left(-\lambda\cdot d_{ij}\right)}}$ and $X_j(E) = E_j\cdot \sum_i{A_i \exp{\left(-\lambda\cdot d_{ij}\right)}}$.

Relocations are then done deterministically following a discrete choice model :
\[
A_i(t+1) = \sum_i{A_i(t)}\cdot\frac{\exp{(\beta U_i(A))}}{\sum_i{\exp{(\beta U_i(A))}}}
\]
\[
E_j(t+1) = \sum_j{E_j(t)}\cdot\frac{\exp{(\beta U_j(E))}}{\sum_j{\exp{(\beta U_j(E))}}}
\]



%%%%%%%%%%%%%%%%%%%%%
%% Dynamic programming
%%%%%%%%%%%%%%%%%%%%%

\newpage

Effective distances computation

\begin{itemize}
\item Euclidian distance matrix $d(i,j)$ computed analytically
\item Network shortest paths between network intersections (rasterized network) updated in a dynamic way (addition of new paths and update/change of old paths if needed when a link is added), correspondance between network patches and closest intersection also updated dynamically ; $O(N_{inters}^3)$
\item Weak component clusters and distance between clusters updated ; $O(N_{nw}^2)$
\item Network distances between network patches updated, through the heuristic of only minimal connexions between clusters ; $O(N_{nw}^2)$
\item Effective distances (taking paces/congestion into account) updated as minimum between euclidian time and $\min_{C,C'}{d(i,C)+d_{nw}(p_C(i),p_C'(j))+d(C',j)}$ ; $O(N_{clusters}^2\cdot N^2)$ [Approximed with $\min_C$ only in the implementation, consistent within the interaction ranges $\sim$ 5 patches taken in the model]. 
\end{itemize}



%%%%%%%%%%%%%%%%%%
%% Indicators
%%%%%%%%%%%%%%%%%%


Indicators :









\end{document}
