\documentclass[english,11pt]{beamer}

\DeclareMathOperator{\Cov}{Cov}
\DeclareMathOperator{\Var}{Var}
\DeclareMathOperator{\E}{\mathbb{E}}
\DeclareMathOperator{\Proba}{\mathbb{P}}

\newcommand{\Covb}[2]{\ensuremath{\Cov\!\left[#1,#2\right]}}
\newcommand{\Eb}[1]{\ensuremath{\E\!\left[#1\right]}}
\newcommand{\Pb}[1]{\ensuremath{\Proba\!\left[#1\right]}}
\newcommand{\Varb}[1]{\ensuremath{\Var\!\left[#1\right]}}

% norm
\newcommand{\norm}[1]{\| #1 \|}

\newcommand{\indep}{\rotatebox[origin=c]{90}{$\models$}}





\usepackage{mathptmx,amsmath,amssymb,graphicx,bibentry,bbm,babel,ragged2e}

\makeatletter

\newcommand{\noun}[1]{\textsc{#1}}
\newcommand{\jitem}[1]{\item \begin{justify} #1 \end{justify} \vfill{}}
\newcommand{\sframe}[2]{\frame{\frametitle{#1} #2}}

\newenvironment{centercolumns}{\begin{columns}[c]}{\end{columns}}
%\newenvironment{jitem}{\begin{justify}\begin{itemize}}{\end{itemize}\end{justify}}

\usetheme{Warsaw}
\setbeamertemplate{footline}[text line]{}
\setbeamercolor{structure}{fg=purple!50!blue, bg=purple!50!blue}

\setbeamersize{text margin left=15pt,text margin right=15pt}

\setbeamercovered{transparent}


\@ifundefined{showcaptionsetup}{}{%
 \PassOptionsToPackage{caption=false}{subfig}}
\usepackage{subfig}

\usepackage[utf8]{inputenc}
\usepackage[T1]{fontenc}



\makeatother

\begin{document}


\title{Towards a Theory of Co-evolutive Networked Territorial Systems: Insights from Transportation Governance Modeling in Pearl River Delta, China}

\author{J.~Raimbault$^{1,2}$\\
\texttt{juste.raimbault@polytechnique.edu}
}


\institute{$^{1}$UMR CNRS 8504 G{\'e}ographie-cit{\'e}s\\
$^{2}$UMR-T IFSTTAR 9403 LVMT\\
}


\date{Medium Project Seminar\\\smallskip
4th December 2016
}

\frame{\maketitle}





%%%%%%%%%%%%%%%%%%%
%% ABSTRACT

%Recent advances in Theoretical and Quantitative Geography have witnessed the co-construction of geographical theories, complex models of simulation and Empirical analysis. We build on this legacy to shed light on relations between territories and networks. Our contribution consists in two distincts parts. First we develop a new theory of territorial systems, that emphasizes on the role of networks in co-evolutive processes. More precisely, we speculate, building on several modeling and empirical previous contributions, that within Pumain’s Evolutive Urban Theory [Pumain, 1997], the existence of co-evolutive subsystems is equivalent to the existence of morphogenetic processes in which networks are significant drivers. Implications include the necessity of networks in explaining territorial systems dynamics, but also a modular decomposition of these systems into local stationary processes in space or time.
%In a second part, we discuss practical implications of the theory through the adaptation of a agent- based model introduced in [Le Néchet and Raimbault, 2015], which simulates the co-evolution of land-use and transportation infrastructure within a Mega-city Region. In particular, it includes game theoretical modeling of decision making processes for transportation governance. The model is partially validated on synthetic data by retrieving expected stylized facts on emergent infrastructure. We propose then to apply it to the Mega-city Region of Pearl River Delta, China, which is very typical of included characteristics such as the presence of a strong competition between inner cities and the recent realization, construction or planning of numerous large-scale transportation infrastructures. Calibration of the model will allow first to infer information on potentially hidden governance processes when applied on real or planned infrastructure, and secondly to compare among different possible governance frameworks when applied on hypothetical optimized infrastructure.








%%%%%%%%%%%%%%%%%
\section{Introduction}
%%%%%%%%%%%%%%%%%



\sframe{Complex Urban Systems}{

\centering

\includegraphics[width=0.45\textwidth,height=0.6\textheight]{}
\hspace{0.1cm}
\includegraphics[width=0.45\textwidth,height=0.6\textheight]{}

{\tiny Source : Wikipedia}

}





\sframe{Conclusion}{


\bigskip
\bigskip
\bigskip


\footnotesize{ - All code and data available at \texttt{https://github.com/JusteRaimbault/CityNetwork/tree/master/Models\\
/Governance}

}

}




\sframe{Reserve slides}{

\centering

\Large

\textbf{Reserve Slides}

}




%%%%%%%%%%%%%%%%%%%%%
\begin{frame}[allowframebreaks]
\frametitle{References}
\bibliographystyle{apalike}
\bibliography{/Users/Juste/Documents/ComplexSystems/CityNetwork/Biblio/Bibtex/CityNetwork}
\end{frame}
%%%%%%%%%%%%%%%%%%%%%%%%%%%%




\sframe{Reserve slides}{


}









\end{document}







