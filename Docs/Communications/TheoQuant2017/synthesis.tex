%% Commands

\newcommand{\noun}[1]{\textsc{#1}}

% command fort head of chapter citation
\newcommand{\headercit}[3]{
\begin{multicols}{2}
\phantom{}
\columnbreak
\textit{#1}

 - \noun{#2}~#3
\end{multicols}
}



%% Math

% Operators
\DeclareMathOperator{\Cov}{Cov}
\DeclareMathOperator{\Var}{Var}
\DeclareMathOperator{\E}{\mathbb{E}}
\DeclareMathOperator{\Proba}{\mathbb{P}}

\newcommand{\Covb}[2]{\ensuremath{\Cov\!\left[#1,#2\right]}}
\newcommand{\Eb}[1]{\ensuremath{\E\!\left[#1\right]}}
\newcommand{\Pb}[1]{\ensuremath{\Proba\!\left[#1\right]}}
\newcommand{\Varb}[1]{\ensuremath{\Var\!\left[#1\right]}}

% norm
\newcommand{\norm}[1]{\| #1 \|}

% independent
\newcommand{\indep}{\rotatebox[origin=c]{90}{$\models$}}


% amsthm environments
\newtheorem{definition}{Definition}
\newtheorem{proposition}{Proposition}
\newtheorem{assumption}{Assumption}
\newtheorem{lemma}{Lemma}

\newenvironment{proof}[1][Proof]{\begin{trivlist}
\item[\hskip \labelsep {\bfseries #1}]}{\end{trivlist}}




\newcommand{\qed}{\nobreak \ifvmode \relax \else
      \ifdim\lastskip<1.5em \hskip-\lastskip
      \hskip1.5em plus0em minus0.5em \fi \nobreak
      \vrule height0.75em width0.5em depth0.25em\fi}



%%%%%%%%%%%%%%%%%%%
%%  Additional packages
%%%%%%%%%%%%%%%%%%%

%\usepackage{subcaption}

\usepackage{amssymb}

\usepackage{multicol}

\usepackage{bbm}

% chinese
%\usepackage{ctex}
%\setCJKmainfont[BoldFont=FandolSong-Bold.otf,ItalicFont=FandolKai-Regular.otf]{FandolSong-Regular.otf}
%\setCJKsansfont[BoldFont=FandolHei-Bold.otf]{FandolHei-Regular.otf}
%\setCJKmonofont{FandolFang-Regular.otf}

%  --> needs XeLateX ? then needs to comment pdflatex options in class definition.

\usepackage{mdframed}

%%%

\renewcommand{\PrelimText}{%
  \footnotesize[\,\today\ at \thistime\ -- \texttt{Thesis}~\myVersion\,]}


%%%%%%%%
% bilingual version
\usepackage{xparse}
\usepackage{ifthen}


% biling paragraph

\newcommand{\bpar}[2]{
    \ifthenelse{\thelanguage=0}{#1}{}
    \ifthenelse{\thelanguage=1}{#2}{}
}


% Sectioning commands
% from http://tex.stackexchange.com/questions/109557/renewdefine-section-with-additional-argument


%\let\oldchapter\chapter
%
%\RenewDocumentCommand\chapter {s o o m m}
%   {
%    \IfBooleanTF{#1}                   
%        {
%        \ifthenelse{\thelanguage=0}{\oldchapter*{#4}}{}
%        \ifthenelse{\thelanguage=1}{\oldchapter*{#5}}{}
%          \IfNoValueF{#2}            % if TOC arg is given create a TOC entry
%            {          
%              \ifthenelse{\thelanguage=0}{\addcontentsline{toc}{chapter}{#2}}{}
%              \ifthenelse{\thelanguage=1}{\addcontentsline{toc}{chapter}{#3}}{}
%            }
%        }  
%      {                              % no star given 
%        \IfNoValueTF{#2}
%          {
%            \ifthenelse{\thelanguage=0}{\oldchapter{#4}}{}
%            \ifthenelse{\thelanguage=1}{\oldchapter{#5}}{}
%           }       % no TOC arg
%          { 
%            \ifthenelse{\thelanguage=0}{\oldchapter[#2]{#4}}{}
%            \ifthenelse{\thelanguage=1}{\oldchapter[#3]{#5}}{}
%           }
%      }   
%  }
%
%



\let\oldsection\section

\RenewDocumentCommand\section {s o o m m}
   {
    \IfBooleanTF{#1}                   
        {
        \ifthenelse{\thelanguage=0}{\oldsection*{#4}}{}
        \ifthenelse{\thelanguage=1}{\oldsection*{#5}}{}
          \IfNoValueF{#2}            % if TOC arg is given create a TOC entry
            {          
              \ifthenelse{\thelanguage=0}{\addcontentsline{toc}{section}{#2}}{}
              \ifthenelse{\thelanguage=1}{\addcontentsline{toc}{section}{#3}}{}
            }
        }  
      {                              % no star given 
        \IfNoValueTF{#2}
          {
            \ifthenelse{\thelanguage=0}{\oldsection{#4}}{}
            \ifthenelse{\thelanguage=1}{\oldsection{#5}}{}
           }       % no TOC arg
          { 
            \ifthenelse{\thelanguage=0}{\oldsection[#2]{#4}}{}
            \ifthenelse{\thelanguage=1}{\oldsection[#3]{#5}}{}
           }
      }   
  }

\let\oldsubsection\subsection

\RenewDocumentCommand\subsection {s o o m m}
   {
    \IfBooleanTF{#1}                   
        {
        \ifthenelse{\thelanguage=0}{\oldsubsection*{#4}}{}
        \ifthenelse{\thelanguage=1}{\oldsubsection*{#5}}{}
          \IfNoValueF{#2}            % if TOC arg is given create a TOC entry
             {          
              \ifthenelse{\thelanguage=0}{\addcontentsline{toc}{subsection}{#2}}{}
              \ifthenelse{\thelanguage=1}{\addcontentsline{toc}{subsection}{#3}}{}
            }
        }  
      {                              % no star given 
        \IfNoValueTF{#2}
          {
           \ifthenelse{\thelanguage=0}{\oldsubsection{#4}}{}
           \ifthenelse{\thelanguage=1}{\oldsubsection{#5}}{}
           }       % no TOC arg
          { 
           \ifthenelse{\thelanguage=0}{\oldsubsection[#2]{#4}}{}
           \ifthenelse{\thelanguage=1}{\oldsubsection[#3]{#5}}{}
           }
      }   
  }


\let\oldsubsubsection\subsubsection

\RenewDocumentCommand\subsubsection {s o o m m}
   {
    \IfBooleanTF{#1}   
        {
        \ifthenelse{\thelanguage=0}{\oldsubsubsection*{#4}}{}
        \ifthenelse{\thelanguage=1}{\oldsubsubsection*{#5}}{}
          \IfNoValueF{#2}            % if TOC arg is given create a TOC entry
             {          
              \ifthenelse{\thelanguage=0}{\addcontentsline{toc}{subsubsection}{#2}}{}
              \ifthenelse{\thelanguage=1}{\addcontentsline{toc}{subsubsection}{#3}}{}
            }
        }  
      {                              % no star given 
        \IfNoValueTF{#2}
          {
           \ifthenelse{\thelanguage=0}{\oldsubsubsection{#4}}{}
           \ifthenelse{\thelanguage=1}{\oldsubsubsection{#5}}{}
           }       % no TOC arg
          { 
           \ifthenelse{\thelanguage=0}{\oldsubsubsection[#2]{#4}}{}
           \ifthenelse{\thelanguage=1}{\oldsubsubsection[#3]{#5}}{}
           }
      }   
  }
  
  
  
\let\oldparagraph\paragraph

\RenewDocumentCommand\paragraph {s m m}
   {
    \IfBooleanTF{#1}
        {
        \ifthenelse{\thelanguage=0}{\oldparagraph*{#2}}{}
        \ifthenelse{\thelanguage=1}{\oldparagraph*{#3}}{}
        }  
      { 
           \ifthenelse{\thelanguage=0}{\oldparagraph{#2}}{}
           \ifthenelse{\thelanguage=1}{\oldparagraph{#3}}{}   
      }   
  }


\let\oldcaption\caption

\RenewDocumentCommand\caption {o o m m}
   {
        \IfNoValueTF{#1}
          {
           \ifthenelse{\thelanguage=0}{\oldcaption{#3}}{}
           \ifthenelse{\thelanguage=1}{\oldcaption{#4}}{}
           }       % no TOC arg
          { 
           \ifthenelse{\thelanguage=0}{\oldcaption[#1]{#3}}{}
           \ifthenelse{\thelanguage=1}{\oldcaption[#2]{#4}}{}
           }
      } 
      
      
        
%%%%%%%%%%
%  Citation

\let\oldcite\cite
\renewcommand{\cite}[1]{[\oldcite{#1}]}




%%%%%%%%%%
%  Drafting

% writing utilities

% comments	 and responses
%  -> use this comment to ask questions on what other wrote/answer questions with optional arguments (up to 4 answers)


\DeclareDocumentCommand{\comment}{m o o o o}
{\ifthenelse{\draft=1}{
    \textcolor{red}{\textbf{C : }#1}
    \IfValueT{#2}{\textcolor{blue}{\textbf{A1 : }#2}}
    \IfValueT{#3}{\textcolor{ForestGreen}{\textbf{A2 : }#3}}
    \IfValueT{#4}{\textcolor{red!50!blue}{\textbf{A3 : }#4}}
    \IfValueT{#5}{\textcolor{Aquamarine}{\textbf{A4 : }#5}}
 }{}
}


% todo
\newcommand{\todo}[1]{
    \ifthenelse{\draft=1}{\textcolor{red!50!blue}{\textbf{TODO : \textit{#1}}}}{}
}



% provisory part, removed if not draft


\newcommand{\provisory}[1]{
    \ifthenelse{\draft=1}{
    \color{blue} \textbf{\textit{PROVISORY}} #1 \color{black}
    }{}
}

%\newenvironment{provisory}{\par\color{blue}}{\par}














\title{Co-construire Modèles, Etudes Empiriques et Théories en Géographie Théorique et Quantitative : le cas des Interactions entre Réseaux et Territoires\\\bigskip
\textit{Proposition de Communication\\ 13èmes Rencontres Théo Quant 2017}\\\bigskip
}
\author{\noun{Juste Raimbault}$^{1,2}$\\
$^1$ UMR CNRS 8504 Géographie-cités\\
$^2$ UMR-T 9403 IFSTTAR LVMT
}
%\date{Novembre 2016}
\date{}

\maketitle

\justify


\begin{abstract}
\end{abstract}


\textbf{Mots-clés : } Co-construction des Connaissances ; Méthodologie de la Géographie Théorique et Quantitative ; Interaction entre Réseaux et Territoires

\bigskip


La construction de théories géographiques, dans le cadre d'une Géographie Théorique et Quantitative, s'effectue par itérations dans une dynamique de co-évolution avec les efforts empiriques et de modélisation~\cite{livet2010}. Parmi les nombreux exemples, on peut citer la théorie évolutive des villes (co-construite par un spectre de travaux s'étendant par exemple des premières propositions de \cite{pumain1997pour} jusqu'aux résultats matures présentés dans~\cite{pumain2012multi}), l'étude du caractère fractal des structures urbaines % TODO cit dedicace
ou plus récemment le projet Transmondyn visant à enrichir la notion de transition des systèmes de peuplement (ouvrage à paraître). Cette communication propose un format original en s'inscrivant dans cette lignée, par la synthèse différents travaux empiriques et de modélisation menés conjointement avec l'élaboration d'appareils théoriques visant à mieux comprendre les relations entre territoires et réseaux de transports. L'originalité de cette contribution réside à la fois dans la synthèse de travaux très divers pourtant reliés en filigrane, et dans la proposition d'une théorie géographique spécifique s'appuyant sur cette synthèse en seconde partie.



\paragraph{Pourquoi une théorie et des modèles de co-évolution}


Notre première entrée prend un point de vue d'épistémologie quantitative pour tenter d'expliquer le fait que, si la co-évolution entre territoires et réseaux a par exemple été prouvée par~\cite{bretagnolle:tel-00459720}, la littérature est très pauvre en modèles de simulation endogénéisant cette co-évolution. Une exploration algorithmique de la littérature a été menée dans \cite{raimbault2015models}, suggérant un cloisonnement des domaines scientifiques s'intéressant à ce sujet. Des méthodes plus élaborées ainsi que les outils correspondants (collecte et analyse des données), couplant une analyse sémantique au réseau de citations, ont été développées pour renforcer ces conclusions préliminaires~\cite{raimbault2016indirect}, et les premiers résultats au second ordre semblent confirmer l'hypothèse d'un domaine peu défriché car à l'intersection de champs ne dialoguant pas nécessairement aisément. Ces premiers résultats d'épistémologie quantitative confirment l'intérêt d'une modélisation couplant des processus relevant de différentes échelles et domaines d'études, et surtout l'intérêt de l'élaboration d'une théorie propre.


\paragraph{Etudes empiriques}

Le premier axe pour les développements en eux-mêmes consiste en des analyses empiriques. Une étude des corrélations spatiales statiques entre mesures de forme urbaine (indicateurs morphologiques calculés sur la grille de population eurostat) et mesures de forme de réseau (topologie du réseau routier issu d'OpenStreetMap), sur l'ensemble de l'Europe à différentes échelles, a pu révéler la non-stationnarité et la multi-scalarité spatiale de leurs interactions~\cite{raimbault2016cautious}. Cet aspect a aussi été mis en évidence dans l'espace et le temps à une échelle microscopique lors de l'étude des dynamiques d'un système de transport~\cite{raimbault2016investigating}, conjointement avec l'hétérogénéité des processus pour un autre type de système~\cite{raimbault2015hybrid}. Ces faits stylisés valident pour l'instant l'utilisation de modèles de simulation complexes, pour lesquels des premiers efforts de modélisation ont ouvert la voie vers des modèles plus élaborés.


\paragraph{Modélisation}

A l'échelle mesoscopique, des processus d'agrégation-diffusion ont été prouvés suffisant pour reproduire un grand nombre de formes urbaines avec un faible nombre de paramètres, calibrés sur l'ensemble du spectre des valeurs réelles des indicateurs de forme urbaine pour l'Europe. Ce modèle simple a pu, à l'occasion d'un exercice méthodologique explorant le possibilité de contrôle au second ordre de la structure de données synthétiques~\cite{raimbault2016generation}, être couplé faiblement à un modèle de génération de réseau, démontrant une grande latitude de configurations potentiellement générées. L'exploration de différentes heuristiques autonomes de génération de réseau a par ailleurs été entamée~\cite{raimbault2015labex}, pour comparer par exemple des modèles de croissance de réseau routier basés sur l'optimisation locale à des modèles inspirés des réseaux biologiques : chacun présente une très grande variété de topologies générées. A l'échelle macroscopique, un modèle simple de croissance urbaine calibré dynamiquement sur les villes françaises de 1830 à 2000 (base Pumain-Ined) a permis de démontrer l'existence d'un effet réseau de par l'augmentation de pouvoir explicatif du modèle lors de l'ajout d'un effet des flux transitant par un réseau physique, tout en corrigeant le gain dû à l'ajout de paramètres par la construction d'un Critère d'Information d'Akaike empirique~\cite{raimbault2016models}. Cet ensemble de modèles se positionne avec un objectif de parcimonie et dans une perspective d'application en multi-modélisation. Dans une démarche basée-agent plus descriptive et donc d'un modèle plus complexe, \cite{le2015modeling} décrit un modèle de co-évolution à l'échelle métropolitaine (modèle Lutecia) qui inclut en particulier des processus de gouvernance pour le développement des infrastructures de transport. Même si ce dernier modèle est toujours en exploration, les premières études de la dynamique montre l'importance du caractère multi-niveau du développement du réseau de transport pour obtenir des motifs complexes de réseaux et de collaboration entre agents. L'ensemble de ces premiers efforts de modélisation, bien qu'ils ne soient pas majoritairement centrés sur des modèles de co-évolution à proprement parler, supportent les premiers fondements théoriques que nous proposons par la suite.



\paragraph{Construction d'une Théorie Géographique}

En se basant sur les travaux précédents, nous proposons de joindre deux entrées pour la construction d'une théorie géographique ayant un focus privilégié sur les interactions entre territoires et réseaux. La première est par la notion de \emph{morphogénèse}, qui a été explorée d'un point de vue interdisciplinaire dans~\cite{antelope2016interdisciplinary}. Pour notre part, la morphogénèse consiste en l'émergence de la forme et de la fonction, via des processus locaux autonomes dans un système qui exhibe alors une architecture auto-organisée. La présence d'une fonction et donc d'une architecture distingue les systèmes morphogénétiques de systèmes simplement auto-organisés (voir~\cite{doursat2012morphogenetic}). De plus, les notions d'autonomie et de localité s'appliquent bien à des systèmes territoriaux, pour lesquels on essaye d'isoler les sous-systèmes et les échelles pertinentes. Les travaux sur la génération de forme urbaine calibrée par des processus autonomes, les premiers travaux sur la génération de réseaux par de multiples processus également autonomes, et des travaux plus anciens étudiant un modèle simple de morphogénèse urbaine qui suffisait à reproduire des motifs de forme stylisés~\cite{raimbault2014hybrid}, nous suggèrent la possible existence de tels processus au sein des systèmes territoriaux. D'autre part, le cadre d'un théorie évolutive des villes est plébiscité par nos résultats empiriques, qui montrent le caractère non-stationnaire, hétérogène, multi-scalaire des systèmes urbains. Pour rester le plus général possible, et comme nos résultats à la fois empiriques et de modélisation (génération de formes quelconques par le modèle d'agrégation-diffusion par exemple) s'appliquent aux systèmes territoriaux en général, nous nous plaçons dans le cadres de territoires humains de Raffestin~\cite{raffestin1988reperes}, c'est à dire ``la conjonction d'un processus territorial avec un processus informationnel'', qui peut être interprété dans notre cas comme le système complexe socio-techno-environmental que constitue un territoire et les agents et artefacts qui y interagissent. L'importance des réseaux est soulignée par nos résultats sur la nécessité du réseau dans le modèle de croissance macroscopique : nous proposons alors de parler de \emph{Systèmes Territoriaux Complexes en Réseaux}, en ajoutant au plongement du territoire dans la théorie évolutive la particularité qu'il existe des composantes cruciales qui sont les réseaux (de transport en l'occurrence), dont l'origine peut être expliquée par la théorie territoriale des réseaux de Dupuy~\cite{dupuy1987vers}. Nous spéculons alors l'hypothèse suivante afin de réconcilier nos deux approches : \textbf{l'existence de processus morphogénétiques dans lesquels les réseaux ont un rôle crucial est équivalente à la présence de sous-systèmes dans les systèmes territoriaux complexes en réseaux, qu'on définit alors comme co-évolutifs.} Cette proposition a de multiples implications, mais devrait guider notamment les travaux de modélisation vers une méthodologie modulaire et de multi-modélisation afin d'essayer d'exhiber des processus morphogénétiques, et les travaux empiriques vers une étude plus poussée des correlations, causalités (dans le cas de séries temporelles) et recherche de décompositions modulaires des systèmes.





\paragraph{Futurs Développements}

Cet exercice permet d'illustrer la co-construction d'un matériel quantitatif (études empiriques et modélisation) et d'un matériel théorique, puisqu'il est clair que les directions de recherche dans chacun des points ci-dessus ont été construites progressivement, par itérations et allers-retours entre les trois domaines : la conception de modèles de simulation complexes résulte d'un cadre théorique englobant les paradigmes de la complexité, la recherche empirique de propriétés de non-stationnarité également, tandis que les modèles hybrides sont construits et calibrés par les faits stylisés et données empiriques. Notre théorie est bien sûr bien loin d'être mature, mais son existence permet déjà de guider les analyses quantitatives suivantes (modèles fortement couplés ; tests de causalités dans les données spatio-temporelles; etc.) qui seront alors crucial pour la co-évolution de l'ensemble par la suite. L'essence de la Géographie Théorique et Quantitative réside dans cette co-production qui transcende les oppositions classiques entre quantitatif et qualitatif, et à laquelle nous prétendons que participent également les outils et les méthodes : toute perspective scientifique est une combinaison de chacune des dimensions, qui à chaque fois jouent un rôle différent dans la co-évolution. L'illustration de cette proposition de manière plus précise fera aussi l'objet de travaux futurs.



%\paragraph{Vers une Meta-théorie des Systèmes Socio-techniques}






%\paragraph{}

%Enfin, divers travaux ont été menés dans des perspectives méthodologiques : \cite{raimbault2016discrepancy} développe une méthode de quantification de la robustesse des évaluation multi-objectifs de systèmes complexes ; \cite{raimbault2016techno} applique les outils d'analyse sémantique à un corpus d'un autre type, démontrant leur potentialité comme outils plus généraux d'épistémologie quantitative ; \cite{serra2016game} propose un exercice de médiation scientifique et questionne des outils de communication d'enjeux environnementaux à un public large, pouvant trouver par exemple application sur nos sujets aux enjeux également sociétaux tel le développement urbain et les infrastructures de transport.







%\paragraph{Du positionnement général}

%\emph{L'ambition de cette thèse est de ne pas avoir d'ambition.} Cette entrée en matière, rude en apparence, contient à différents niveaux les logiques sous-jacentes à notre processus de recherche. Au sens propre, nous nous plaçons tant que possible dans une démarche constructive et exploratoire, autant sur les plans théoriques et méthodologiques que thématique, mais encore proto-méthodologique (outils appliquant la méthode) : si des ambitions unidimensionnelles ou intégrées devaient émerger, elles seraient conditionnées par l'arbitraire choix d'un échantillon temporel parmi la continuité de la dynamique qui structure tout projet de recherche. Au sens structurel, l'auto-référence qui soulève une contradiction apparente met en exergue l'aspect central de la réflexivité dans notre démarche constructive, autant au sens de la récursivité des appareils théoriques, de celui de l'application des outils et méthodes développés au travail lui-même ou que de celui de la co-construction des différentes approches et des différents axes thématiques. Le processus de production de connaissance pourra ainsi être lu comme une métaphore des processus étudiés. Enfin, sur un plan plus enclin à l'interprétation, cela suggérera la volonté d'une position délicate liant un positionnement politique dont la nécessité est intrinsèque aux sciences humaines (par exemple ici contre l'application technocratique des modèles, ou pour le développement d'outils luttant pour une science ouverte) à une rigueur d'objectivité plus propre aux autres champs abordés, position forçant à une prudence accrue.


%\paragraph{Des objectifs scientifiques}

% ambition ≠ objectif

%\emph{L'objectif d'une variété.} Un objectif est différent d'une ambition, et ceux-ci sont ainsi pour nous clairement fixés sur différents aspects et à différents niveaux. L'objectif principal du point de vue du géographe est d'enrichir l'état de la connaissance sur les processus co-évolutifs entre territoires et réseaux (la définition de ces termes et l'appareil théorique associé faisant parties intégrantes des sous-objectifs), par l'entrée particulière des réseaux de transports et dans une perspective axée premièrement sur la modélisation. Les aspects géographiques peuvent se décliner en sous-objectifs sur des plans variés :
%\begin{enumerate}
%\item Etablir par une étude d'épistémologie qualitative et quantitative le paysage scientifique associé à notre objectif principal, notamment sa diversité lié aux disciplines variées y étant associées.
%\item Extraire des faits stylisés empiriques sur les processus liant territoires et réseaux, à différentes échelles temporelles et spatiales et sur différents cas d'étude.
%\item Construire des modèles de croissance urbaine et/ou de croissance des réseaux, pouvant aller du modèle jouet au modèle semi-paramétrisé, dans le but d'être soit des outils exploratoires soit des briques élémentaires d'une famille de modèles de co-évolution des réseaux et des territoires.
%\item Par émergence issue de l'interaction des objectifs précédents, élaborer une théorie géographique des \emph{systèmes territoriaux réticulaires co-évolutifs}.
%\end{enumerate} 

%Des objectifs dont les aspects pouvant être classifiés à dominante plutôt méthodologique ou proto-méthodologique (même s'il est clair que dans la pratique l'ensemble des objectifs est complémentaire et entrelacé de manière \emph{complexe}) viennent ensuite s'ajouter : 

%\begin{enumerate}\setcounter{enumi}{4}
%\item Exercices de style sur différentes questions horizontales fondamentales à l'étude des système complexes, liés de près ou de loin au sujet thématique, dans le but d'un apport méthodologique.
%\item Développement d'outils (libres et ouverts) et de techniques, que ce soit au niveau de problèmes précis ou au niveau de l'organisation générale du travail de recherche.
%\end{enumerate}

%Enfin, l'articulation de ces différents objectifs devrait servir un objectif plus large :
%\begin{enumerate}\setcounter{enumi}{6}
%\item Construction d'un programme de recherche par la mise en évidence et en cohérence de grands axes structurants restant à explorer pour l'étude des systèmes territoriaux complexes.
%\end{enumerate}








%\paragraph{Du contenu courant}

%\emph{L'auto-organisation prend souvent l'architecture de court.} Une grande partie du travail résumé ci-dessous est organisé sous forme provisoire dans \cite{raimbault2016memoire} qui peut être lu comme complément à cette synthèse, mais dont le plan, suivant l'adage précédent, ne témoigne pas d'une éventuelle architecture finale. Les résultats présentés par la suite sont à mettre chacun en relation avec l'un ou plusieurs des objectifs précédents. 







%\paragraph{Du contenu final}


%\emph{La route est longue mais la voie est libre.} Il n'est pas raisonnable de prétendre donner à ce point une image précise du contenu final et de son architecture, tant la dynamique de co-construction des différentes composantes est intégrée. Il est cependant possible de lister des développements nécessaires afin de donner un peu de chair et plus de forme au squelette brossé ci-dessus.

%\begin{itemize}
%\item Sur le plan empirique, élaboration de tests de causalités sur séries temporelles, qui seront raffinés sur des données synthétiques (notamment des formes urbaines et réseaux associés issus de \cite{raimbault2014hybrid}, puis appliqués à différents cas d'étude pour lesquels les pré-traitement des données est effectué ou en cours : Bassin Parisien de 1950 à 2000 ; Villes françaises et réseaux ferré et autoroutier de 1830 à 2000 ; Transaction immobilières et réseau du Grand Paris de 1995 à 2010 (\textit{collaboration avec T. Le Corre}) ; Villes et réseau ferré en Afrique du Sud de 1900 à 2000 (\textit{collaboration avec S. Baffi}).
%\item Sur le plan de la modélisation, en se basant sur les différentes briques déjà introduites :
%\begin{itemize}
%\item Une extension dynamique de~\cite{raimbault2016generation} pourra être aisément calibrée en utilisant les données de~\cite{raimbault2016cautious}, possiblement en multi-modélisation pour les heuristiques de morphogénèse de réseau.
%\item De même, le modèle de croissance à l'échelle macroscopique est directement généralisable à un réseau dynamique, et sera appliqué en priorité aux villes chinoises, mais aussi sur les systèmes de villes français et sud-africains.
%\item Le modèle Lutecia sera appliqué à l'étude de cas de la région du delta du Zhujiang, à l'occasion d'un séjour de terrain (6mois) dans le cadre du projet européen Medium.
%\end{itemize}
%\item Les divers outils mis en place seront pour les plus utiles distribués sous forme de packages ou de bibliothèques, ainsi que les bases de données développées.
%\item Un travail de réflexivité sera mené, par l'application d'une part des méthodes d'épistémologie quantitative développées à notre propre travail (analyse du corpus cité, réseaux des concepts et des sous-projets), et d'autre part par l'explicitation du caractère implémentatif des différentes parties vis à vis de la théorie, et de la théorie elle-même par rapport à une méta-théorie des systèmes socio-techniques (en cours de développement, voir \cite{raimbault2016memoire}). Celui-ci sera essentiel pour les ajustements théoriques, mais aussi pour offrir au lecteur des clés de lecture non-linéaires.
%\end{itemize}


%Notre positionnement scientifique général développé en introduction est tel le medium dans lequel se manifestent les différents domaines (empirique, théorique, modélisation, méthodologique et proto-méthodologique) et rendant possible leur co-évolution, et surtout l'émergence de questionnement transversaux et verticaux, la toute première ébauche d'un projet de recherche : cette voie qui est libre.

%\newpage

%%%%%%%%%%%%%%%%%%%%
%% Biblio
%%%%%%%%%%%%%%%%%%%%

\small

\bibliographystyle{apalike}
\bibliography{biblio,/Users/Juste/Documents/ComplexSystems/CityNetwork/Biblio/Bibtex/CityNetwork}


\end{document}
