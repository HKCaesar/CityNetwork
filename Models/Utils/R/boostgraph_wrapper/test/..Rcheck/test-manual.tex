\nonstopmode{}
\documentclass[a4paper]{book}
\usepackage[times,inconsolata,hyper]{Rd}
\usepackage{makeidx}
\usepackage[utf8,latin1]{inputenc}
% \usepackage{graphicx} % @USE GRAPHICX@
\makeindex{}
\begin{document}
\chapter*{}
\begin{center}
{\textbf{\huge Package `test'}}
\par\bigskip{\large \today}
\end{center}
\begin{description}
\raggedright{}
\item[Type]\AsIs{Package}
\item[Title]\AsIs{What the package does (short line)}
\item[Version]\AsIs{1.0}
\item[Date]\AsIs{2015-09-17}
\item[Author]\AsIs{Your Name}
\item[Maintainer]\AsIs{Your Name }\email{your@email.com}\AsIs{}
\item[Description]\AsIs{More about what it does (maybe more than one line)}
\item[License]\AsIs{GPL (>= 2)}
\item[Imports]\AsIs{Rcpp (>= 0.11.5)}
\item[LinkingTo]\AsIs{Rcpp}
\end{description}
\Rdcontents{\R{} topics documented:}
\inputencoding{utf8}
\HeaderA{test-package}{What the package does (short line)}{test.Rdash.package}
\aliasA{test}{test-package}{test}
\keyword{package}{test-package}
%
\begin{Description}\relax
More about what it does (maybe more than one line)
\textasciitilde{}\textasciitilde{} A concise (1-5 lines) description of the package \textasciitilde{}\textasciitilde{}
\end{Description}
%
\begin{Details}\relax

\Tabular{ll}{
Package: & test\\{}
Type: & Package\\{}
Version: & 1.0\\{}
Date: & 2015-09-17\\{}
License: & GPL (>= 2)\\{}
}
\textasciitilde{}\textasciitilde{} An overview of how to use the package, including the most important functions \textasciitilde{}\textasciitilde{}
\end{Details}
%
\begin{Author}\relax
Your Name

Maintainer: Your Name <your@email.com>
\end{Author}
%
\begin{References}\relax
\textasciitilde{}\textasciitilde{} Literature or other references for background information \textasciitilde{}\textasciitilde{}
\end{References}
%
\begin{SeeAlso}\relax
\textasciitilde{}\textasciitilde{} Optional links to other man pages, e.g. \textasciitilde{}\textasciitilde{}
\textasciitilde{}\textasciitilde{} \code{\LinkA{<pkg>}{<pkg>}} \textasciitilde{}\textasciitilde{}
\end{SeeAlso}
\inputencoding{utf8}
\HeaderA{rcpp\_hello\_world}{Simple function using Rcpp}{rcpp.Rul.hello.Rul.world}
%
\begin{Description}\relax
Simple function using Rcpp
\end{Description}
%
\begin{Usage}
\begin{verbatim}
rcpp_hello_world()	
\end{verbatim}
\end{Usage}
%
\begin{Examples}
\begin{ExampleCode}
## Not run: 
rcpp_hello_world()

## End(Not run)
\end{ExampleCode}
\end{Examples}
\printindex{}
\end{document}
